\documentclass{myclass}
\usepackage[polish]{babel}

% TODO

% * Rewrite the abstract theory to deal explicitly only with finite dimensional case (?). 
% * Focus on spectral theorem and question-projection operator duality as a model of quantum
%   measurement (this seems elegant). 
% * Add some continuous example of operator-measured Stieltjes integral (maybe construction of
%   position operator x and momentum operator -ih d/dx using Plancherel theorem)
%
% * Rewrite 2-state theory (trim the obvious things like matrix representation and so on).
% * Add solved examples NMR, Rabi, rotating em wave, emission-absorption. 
% * Change notation to square brackets and X,Y,Z for Pauli matrices.
%
% * Composite systems made of 2-state systems (2 x 2 & later n x 2 or maybe n x 2 immediately(?))
% * Composite systems: entanglement, density operator, partial trace, no-cloning theorem,
%   decoherence, Bell's theory
%
% * later (after extensively studying composite systems, entanglement, Bell's theory) begin
%   description of quantum registers, gates, idea of computation using those devices (Deutsch,
%   Grover, QFTrans, Shor), applications to number theory, connection with Church-Turing hypothesis.
%   Examples
%
% * possible physical realizations of quantum computing (gates + adiabatic/topological(?)), I'll
%   probably focus on Josephson junctions and charge qubits as they seem most elegant (maybe
%   NMR/qdots as well)
%
% * A short overview of classical and quantum information theory (I don't want to go very deep there
%   at the moment): Shannon/Von Neumann entropy, quantum information transfer/cryptography
%   ------


\author{Bartosz Hanc}
% My notes regarding abstract foundations of quantum theory and introduction to quantum computing

\begin{document}

\subsection{TEORIA KWANTÓW}

\subsubsection{Elementy przestrzeni Hilberta}

\begin{itemize}

    \item Przez \(\mathbb{V} := (V,\mathbb{C},+,\cdot)\) będziemy oznaczać przestrzeń wektorową nad
    ciałem liczb zespolonych.

    \item \textbf{Def.} Odwzorowanie \(d: V \times V \mapsto \mathbb{R}\) będziemy nazywać
    \textit{metryką} w zbiorze \(V \neq \emptyset\) iff
    \begin{itemize}
    
        \item \(\forall u,v \in V : d(u,v) \geq 0\), przy czym równość zachodzi iff \(u = v\)
        (\textit{nieujemność})

        \item \(\forall u,v \in V : d(u,v) = d(v,u)\) (\textit{symetria})

        \item \(\forall u,v,w \in V : d(u,v) + d(v,w) \geq d(u,w)\) (\textit{nierówność trójkąta})

    \end{itemize}
    Parę \((V,d(\cdot,\cdot))\) będziemy nazywać \textit{przestrzenią metryczną}.

    \item \textbf{Def.} Niech \((V,d)\) będzie przestrzenią metryczną. Mówimy, iż dany ciąg
    \((u_n)\) elementów zbioru \(V\) jest zbieżny do \(g\in V\) tj. \(\lim_{n\to\infty}u_n = g\) iff
    \(\lim_{n\to\infty} d(u_n,g) = 0\).

    \item \textbf{Def.} Ciąg \((u_n)\) elementów \(u_n \in V\) będziemy nazywać \textit{ciągiem
    Cauchy'ego} w przestrzeni metrycznej \((V,d(\cdot,\cdot))\) iff spełnia on kryterium Cauchy'ego
    tj.
    \begin{equation*}
        \forall \epsilon > 0 : \exists N : \forall n,m > N : d(u_n,u_m) < \epsilon\,.
    \end{equation*}

    \item \textbf{Tw.} Każdy ciąg zbieżny w przestrzeni metrycznej \((V,d)\) jest ciągiem Cauchy'ego
    w tej przestrzeni.

    \item \textbf{Def.} Przestrzeń metryczną \((V,d(\cdot,\cdot))\) nazwiemy \textit{zupełną} iff
    każdy ciąg Cauchy'ego \((u_n)\) elementów \(u_n \in V\) jest zbieżny do granicy \(g \in V\).

    \item \textbf{Def.} Niech \(\mathbb{V}\) będzie przestrzenią wektorową. Odwzorowanie
    \(\braket{\cdot} : V \times V \mapsto \mathbb{C}\) nazwiemy \textit{iloczynem wewnętrznym}
    wektorów iff
    \begin{itemize}
        
        \item \(\forall u,v \in V : \braket{u}{v}^* = \braket{v}{u}\)
        
        \item \(\forall u,v_1,v_2 \in V : \forall \alpha, \beta \in \mathbb{C} : \braket{u}{\alpha
        v_1 + \beta v_2} = \alpha \braket{u}{v_1} + \beta \braket{u}{v_2}\)

        \item \(\forall u \in V : \braket{u} \geq 0\), przy czym równość zachodzi iff \(u =
        \mathsf{0}\). Zauważmy tutaj, iż z pierwszego aksjomatu \(\braket{u} \in \mathbb{R}\), gdyż
        \(\braket{u} = \braket{u}^* \implies \Im{\braket{u}} = 0\).

    \end{itemize}
    
    Parę \((\mathbb{V},\braket{\cdot})\) będziemy nazywać \textit{przestrzenią unitarną}.

    \item \textbf{Tw.} Każda przestrzeń unitarna jest metryczna z metryką indukowaną przez iloczyn
    wewnętrzny \(d(u,v) := \sqrt{\braket{u-v}}\).

    \item \textbf{Tw.} (\textit{Nierówność Cauchy'ego--Schwarza}) Niech
    \((\mathbb{V},\braket{\cdot})\) -- przestrzeń metryczna. Wówczas
    \begin{equation*}
        \forall u,v \in V : |\braket{u}{v}|^2 \leq \braket{u}\braket{v}\,.
    \end{equation*}  

    \item \textbf{Def.} Przeliczalny zbiór wektorów \(\{v_1,...,v_n\}\) nazwiemy
    \textit{ortogonalnym} iff 
    \begin{equation*}
        \forall i\neq j; i,j\in\{1,...,n\} : \braket{v_i}{v_j} = 0\,.
    \end{equation*}
    Ten sam zbiór wektorów nazwiemy \textit{ortonormalnym} iff 
    \begin{equation*}
        \forall i,j\in\{1,...,n\} : \braket{v_i}{v_j} = \delta_{ij}\,,
    \end{equation*}
    gdzie \(\delta_{ij}\) jest deltą Kroneckera.

    \item \textbf{Tw.} Każda przestrzeń unitarna \((\mathbb{V}, \braket{\cdot})\) posiada bazę
    ortonormalną, tj. bazę, której wektory bazowe tworzą zbiór ortonormalny.

    \item \textbf{Def.} \textit{Przestrzenią Hilberta} \(\mathscr{H}=(\mathbb{V},\braket{\cdot})\)
    nazwiemy zupełną przestrzeń unitarną.

    \item \textbf{Def.} Niech \(\mathscr{H} = (\mathbb{V},\braket{\cdot})\) będzie przestrzenią
    Hilberta. Odwzorowanie \(F:V\mapsto\mathbb{C}\) nazwiemy \textit{funkcjonałem liniowym} w
    przestrzeni \(\mathscr{H}\) iff
    \begin{equation*}
        \begin{split}
            &\forall u,v \in V : \forall \alpha, \beta \in \mathbb{C} : \\
            &F(\alpha u+\beta v) = \alpha F(u) + \beta F(v)\,.
        \end{split}  
    \end{equation*}

    \item \textbf{Tw.} Niech \(V^*\) oznacza zbiór wszystkich funkcjonałów liniowych
    \(F:V\mapsto\mathbb{C}\). Wówczas \(\mathbb{V}^*:=(V^*,\mathbb{C},+,\cdot)\), gdzie
    \begin{itemize}
        \item \(\forall F_1,F_2 \in V^* : \forall v \in V : (F_1+F_2)(v) = F_1(v) + F_2(v)\)

        \item \(\forall F \in V^* : \forall \alpha \in \mathbb{C} : \forall v \in V : (\alpha \cdot
        F)(v) = \alpha F(v)\)
    \end{itemize}
    jest przestrzenią wektorową, którą nazywamy \textit{przestrzenią dualną}.

    \item \textbf{Tw.} (\textit{Riesza}) Niech \(\mathscr{H}=(\mathbb{V},\braket{\cdot})\) będzie
    przestrzenią Hilberta, a \(\mathbb{V}^*\) jej przestrzenią dualną. Wówczas istnieje wzajemnie
    jednoznaczne odwzorowanie wektorów \(v \in V\) na funkcjonały liniowe \(F \in V^*\). Dodatkowo
    dla każdego funkcjonału \(F\) istnieje dokładnie jeden wektor \(u \in V\) taki, że
    \begin{equation*}
        \forall v \in V : F(v) = \braket{u}{v}\,.
    \end{equation*}

    \item \textbf{Def.} \textit{Iloczynem Kroneckera} macierzy \(\oper{A} = [a_{ij}]_{n\times n}\) i
    \(\oper{B} = [b_{ij}]_{m\times m}\) nazywamy macierz
    \begin{equation*}
        \begin{split}
            \oper{A} \otimes \oper{B} &= \mqty(a_{11} & \cdots & a_{1n} \\ \vdots & \ddots & \vdots \\ a_{n1} & \cdots & a_{nn}) \otimes \mqty(b_{11} & \cdots & b_{1m} \\ \vdots & \ddots & \vdots \\ b_{m1} & \cdots & b_{mm})\\
            &= \mqty(a_{11}\oper{B} & \cdots & a_{1n}\oper{B} \\ \vdots & \ddots & \vdots \\ a_{n1}\oper{B} & \cdots & a_{nn}\oper{B})
        \end{split}
    \end{equation*}

    \item \textbf{Def.} \textit{Iloczynem tensorowym} przestrzeni Hilberta \(\mathscr{H}_1\) i
    \(\mathscr{H}_2\) o bazach ortonoromalnych odpowiednio \(\{\phi_i^{(1)}\}\) i
    \(\{\phi_i^{(2)}\}\) nazywamy przestrzeń Hilberta \(\mathscr{H} = \mathscr{H}_1 \otimes
    \mathscr{H}_2\) taką, że
    \begin{itemize}
        
        \item Jej bazą ortonormalną jest zbiór \(\{\phi_i^{(1)} \otimes \phi_j^{(2)}\}\), gdzie
        \(\psi \otimes \phi\) to iloczyn Kroneckera wektorów dla reprezentacji tych wektorów jako
        macierzy kolumnowych o elementach będących współrzędnymi wektora w danej bazie.

        \item Iloczyn wewnętrzny w przestrzeni \(\mathscr{H}_1 \otimes \mathscr{H}_2\) jest
        zdefiniowany jako
        \begin{equation*}
            \braket{\phi_1 \otimes \phi_2}{\psi_1 \otimes \psi_2} := \braket{\phi_1}{\psi_1}_1 \cdot \braket{\phi_2}{\psi_2}_2\,,
        \end{equation*}
        gdzie \(\phi_i,\psi_i \in \mathscr{H}_i\) to pewne wektory, a \(\braket{\cdot}_i\) to
        iloczyn wewnętrzny w \(\mathscr{H}_i\).
    \end{itemize}

\end{itemize}

\subsubsection{Notacja Diraca}

Niech \(\mathscr{H}\) będzie przestrzenią Hilberta. Wprowadzimy teraz kompaktową notacją wektorów i
funkcjonałów liniowych wymyśloną przez Diraca. Aby uprościć zapis, będziemy mówili o wektorach
należących do przestrzeni \(\mathscr{H}\) (używając nawet symbolu należenia \(\in \mathscr{H}\)),
mając oczywiście formalnie na myśli wektory należące do zbioru \(V\).

Wektory należące do \(\mathscr{H}\) będziemy oznaczać jako
\begin{equation*}
    \ket{\psi},\, \ket{\phi},\, ...\,,
\end{equation*}
przy czym \(\ket{\cdot}\) to tzw. \textit{ket} i formalnie jest to odwzorowanie \(\ket{\cdot} :
\mathsf{S} \mapsto V\), gdzie \(\mathsf{S}\) jest zbiorem znaków, których używamy do oznaczenia
konkretnych wektorów ze zbioru \(V\). Nie będziemy jednak przestrzegali tego formalnego znaczenia,
utożsamiając dla wygody również  sam symbol z wektorem.

Funkcjonały liniowe należące do przestrzeni dualnej będziemy oznaczać jako
\begin{equation*}
    \bra{\psi},\, \bra{\phi},\, ...\,,
\end{equation*}
przy czym \(\bra{\cdot}\) to tzw. \textit{bra} i formalnie jest to odwzorowanie \(\bra{\cdot} :
\mathsf{S}^* \mapsto V^*\), gdzie \(\mathsf{S}^*\) jest zbiorem znaków, których używamy do
oznaczenia konkretnych funkcjonałów ze zbioru \(V^*\). Ponieważ z tw. Riesza istnieje wzajemnie
jednoznaczne odwzorowanie funkcjonałów liniowych na wektory, więc możemy utożsamić \(\mathsf{S}^* =
\mathsf{S}\). 

Iloczyn wewnętrzny wektorów \(\psi\), \(\phi\) tj. \(\braket{\psi}{\phi}\) będziemy teraz
interpretować jako wartość funkcjonału liniowego \(\bra{\psi}\) (tj. funkcjonału liniowego
odpowiadającego wektorowi \(\psi\)) dla argumentu \(\ket{\phi}\).

\subsubsection{Elementy operatorów liniowych}

\begin{itemize}
    
    \item \textbf{Def.} \textit{Operatorem liniowym} \(\oper{A}\) w przestrzeni \(\mathscr{H}\)
    nazywamy odwzorowanie liniowe 
    \begin{equation*}
        \oper{A} : D(\oper{A}) \mapsto D(\oper{A})\,,
    \end{equation*}
     gdzie \(D(\oper{A})\) jest podprzestrzenią wektorową przestrzeni \(\mathbb{V}\).

    \item \textbf{Def.} \textit{Sprzężeniem} operatora \(\oper{A}\) nazywamy operator
    \(\oper{A}^\dagger : D(\oper{A}^\dagger) \mapsto D(\oper{A}^\dagger)\) taki, że 
    \begin{equation*}
        \forall \phi \in D(\oper{A}^\dagger) : \forall \psi \in D(\oper{A}) : \braket{\phi}{\oper{A}\psi} = \braket{\oper{A}^\dagger\phi}{\psi}
    \end{equation*}
    co możemy również zapisać jako
    \begin{equation*}
        (\bra{\psi}\oper{A}^\dagger\ket{\phi})^* = \bra{\phi}\oper{A}\ket{\psi}\,.
    \end{equation*}

    \item \textbf{Def.} \textit{Komutatorem} operatorów \(\oper{A}\), \(\oper{B}\) o równych
    dziedzinach nazywamy operator \([\oper{A},\oper{B}]\) zdefiniowany jako
    \begin{equation*}
        \forall \psi \in D : [\oper{A},\oper{B}]\psi = \oper{A}(\oper{B}\psi) - \oper{B}(\oper{A}\psi)\,. 
    \end{equation*}
    Jeśli \([\oper{A},\oper{B}] = \oper{\mathsf{0}}\) (gdzie \(\oper{\mathsf{0}}\) oznacza
    \textit{operator zerowy} \(\oper{\mathsf{0}}\psi=\mathsf{0}\)), to mówimy, że operatory
    \(\oper{A}\), \(\oper{B}\) \textit{komutują}.
    
    \item \textbf{Def.} \textit{Antykomutatorem} operatorów \(\oper{A}\), \(\oper{B}\) nazywamy
    operator \(\{\oper{A},\oper{B}\}\) zdefiniowany jako
    \begin{equation*}
        \{\oper{A},\oper{B}\}\psi := \oper{A}\oper{B}\psi + \oper{B}\oper{A}\psi\,.
    \end{equation*}

    \item \textbf{Tw.} Dla dowolnych operatorów \(\oper{A}\), \(\oper{B}\), \(\oper{C}\) zakładając
    odpowiednie dziedziny, zachodzi:
    \begin{itemize}
        \item \([\oper{A} + \oper{B},\oper{C}] = [\oper{A},\oper{C}] + [\oper{B},\oper{C}]\) ;
        \item \([\oper{A}\oper{B},\oper{C}] = \oper{A}[\oper{B},\oper{C}] +
        [\oper{A},\oper{C}]\oper{B}\) .
    \end{itemize}

    \item \textbf{Def.} \textit{Ślad operatora \(\oper{A}\)} definiujemy jako liczbę \(\Trace
    \oper{A} \in \mathbb{C}\) równą
    \begin{equation*}
        \Trace{\oper{A}} := \sum_{i} \bra{\phi_i}\oper{A}\ket{\phi_i}\,,
    \end{equation*}
    gdzie \(\{\phi_i\}\) jest dowolną ortonormalną bazą przestrzeni \(\mathscr{H}\).

    \item \textbf{Tw.} Ślad operatora nie zależy od wyboru ortonormalnej bazy przestrzeni Hilberta.

    \item \textbf{Def.} Niech \(f(x):\mathbb{R}\mapsto\mathbb{R}\) będzie funkcją zmiennej
    rzeczywistej taką, że istnieje szereg potęgowy
    \begin{equation*}
        \sum_{n=0}^\infty\frac{a_n}{n!}x^n\,,
    \end{equation*}
    który jest zbieżny jednostajnie na \(\mathbb{R}\) do \(f\). Wówczas funkcję operatora
    \(f(\oper{A})\) definiujemy jako
    \begin{equation*}
        f(\oper{A}) := \sum_{n=0}^\infty \frac{a_n}{n!}\oper{A}^n\,,
    \end{equation*}
    gdzie przyjmujemy \(\oper{A}^0 := \oper{I}\). W szczególności mamy
    \begin{itemize}
        \item \(\exp(\oper{A}) := \sum_{n=0}^\infty \frac{1}{n!}\oper{A}^n\)
    
        \item \(\sin(\oper{A}) := \sum_{n=0}^\infty \frac{(-1)^n}{(2n+1)!}\oper{A}^{2n+1}\)
    
        \item \(\cos(\oper{A}) := \sum_{n=0}^\infty \frac{(-1)^n}{(2n)!}\oper{A}^{2n}\)
    
    \end{itemize}

\end{itemize}

W teorii kwantowej główną rolę odgrywają trzy rodziny operatorów: operatory samosprzężone, rzutowe i
unitarne.

\begin{itemize}

    \item \textbf{Def.} \textit{Operatorem symetrycznym} nazywamy operator \(\oper{A}\) taki, że
    \begin{equation*}
        \forall \psi,\phi \in D(\oper{A}) : \braket{\psi}{\oper{A}\phi} = \braket{\oper{A}\psi}{\phi}\,.
    \end{equation*}

    \item \textbf{Def.} \textit{Operatorem samosprzężonym} nazywamy operator \(\oper{A}\) taki, że
    \(\oper{A} = \oper{A}^\dagger\). Samosprzężoność nie pokrywa się z symetrycznością, gdyż równość
    operatorów oznacza z definicji równość ich dziedzin, co nie jest w ogólności prawdziwe dla
    dowolnego operatora symetrycznego. Oczywiście, jeśli operator jest samosprzężony, to jest
    symetryczny.

\end{itemize}

Operatory samosprzężone odgrywają wyróżnioną rolę w teorii kwantowej ze względu na trzy twierdzenia,
które są dla nich spełnione.
\begin{itemize}
    
    \item \textbf{Tw.} Wartości własne operatora samosprzężonego są liczbami rzeczywistymi.

    \item \textbf{Tw.} Wektory własne operatora samosprzężonego tworzą zbiór ortogonalny.

    \item \textbf{Tw.} Jeśli zbiór wartości własnych operatora samosprzężonego jest zbiorem
    przeliczalnym, to zbiór wektorów własnych rozpina przestrzeń \(\mathscr{H}\).

\end{itemize}

\begin{itemize}

    \item \textbf{Def.} \textit{Operatorem rzutowym} nazywamy operator \(\oper{P}\) taki, że
    \(\oper{P} = \oper{P}^\dagger\) (samosprzężoność) i \(\oper{P}^2 = \oper{P}\) (idempotentność).

\end{itemize}

Ważnym przykładem operatora rzutowego jest operator rzutowania na jednowymiarową podprzestrzeń
rozpiętą na unormowanym wektorze \(\ket{\phi}\) (rzutowanie na kierunek wektora \(\phi\)), który w
notacji Diraca możemy zapisać jako \(\oper{P} = \ket{\phi}\bra{\phi}\) tj. \(\forall \psi :
\oper{P}(\psi) = \braket{\phi}{\psi}\phi\). Jest to oczywiście operator liniowy, gdyż dla dowolnych
wektorów \(\ket{\psi_1}\), \(\ket{\psi_2}\) i skalarów \(\alpha\), \(\beta\) mamy
\begin{equation*}
    \begin{split}
        &\ket{\phi}\bra{\phi}(\alpha\ket{\psi_1} + \beta\ket{\psi_2}) = \ket{\phi}\braket{\phi}{\alpha\psi_1 + \beta\psi_2} \\
        &= \alpha\ket{\phi} \braket{\phi}{\psi_1} + \beta\ket{\phi}\braket{\phi}{\psi_2}\,.
    \end{split}
\end{equation*}
Jest również idempotentny, gdyż 
\begin{equation*}
    \ket{\phi}\bra{\phi}(\ket{\phi}\braket{\phi}{\psi}) = \ket{\phi}\braket{\phi}\braket{\phi}{\psi} = \ket{\phi}\braket{\phi}{\psi}
\end{equation*}
z założenia \(\braket{\phi} = 1\) oraz samosprzężony
\begin{equation*}
    (\braket{\psi_1}{\phi}\braket{\phi}{\psi_2})^* = \braket{\psi_2}{\phi}\braket{\phi}{\psi_1}\,.
\end{equation*}
Łatwo pokazać również, iż jeśli \(\{\phi_i\}\) jest ortonormalnym zbiorem wektorów, to
\begin{equation*}
    \oper{P} = \sum_i \ket{\phi_i}\bra{\phi_i}
\end{equation*}
jest operatorem rzutowym. W szczególności, jeśli \(\{\phi_i\}\) jest ortonormalną bazą przestrzeni
\(\mathscr{H}\), to
\begin{equation*}
    \sum_i \ket{\phi_i}\bra{\phi_i} = \oper{I}\,.
\end{equation*}

\begin{itemize}
    
    \item \textbf{Def.} \textit{Operatorem unitarnym} nazywamy operator \(\oper{U}\) taki, że
    \(\oper{U}\oper{U}^\dagger = \oper{U}^\dagger\oper{U} = \oper{I}\).

\end{itemize}

Przekształcenia unitarne reprezentowane przez operatory unitarne mają ważną własność polegającą na
zachowywaniu wartości iloczynu wewnętrznego dwóch wektorów, a zatem w szczególności normy wektora
\begin{equation*}
    \braket{\oper{U}\psi}{\oper{U}\phi} = \braket{\oper{U}^\dagger\oper{U}\psi}{\phi} = \braket{\psi}{\phi}\,.
\end{equation*}
    
\begin{itemize}
    
    \item \textbf{Tw.} (\textit{spektralne}) Niech \(\mathscr{H}\) będzie przestrzenią Hilberta. Dla
    każdego samosprzężonego operatora liniowego \(\oper{A}\) w \(\mathscr{H}\) istnieje unikalna
    rodzina operatorów rzutowych \(\oper{P}(\lambda)\) indeksowanych ciągłym parametrem \(\lambda
    \in \mathbb{R}\) taka, że
    \begin{itemize}
        
        \item Jeśli \(\lambda_1 < \lambda_2\) to
        \begin{equation*}
            \oper{P}(\lambda_1)\oper{P}(\lambda_2) =
        \oper{P}(\lambda_2)\oper{P}(\lambda_1) = \oper{P}(\lambda_1)\,.
        \end{equation*}

        \item Dla każdego \(\lambda\) \(\lim_{\epsilon\to0^+}\oper{P}(\lambda+\epsilon) =
        \oper{P}(\lambda)\).
        
        \item \(\lim_{\lambda\to-\infty}\oper{P}(\lambda) = \oper{\mathsf{0}}\)

        \item \(\lim_{\lambda\to+\infty}\oper{P}(\lambda) = \oper{I}\)

        \item \(\oper{A} = \int\limits_{-\infty}^{+\infty}\lambda \dd{\oper{P}(\lambda)}\)

    \end{itemize}
    gdzie ostatnia całka to tzw. \textit{całka Riemanna--Stieltjesa} względem miary operatorowej
    zdefiniowana jako
    \begin{equation*}
            \int\limits_a^b f(x) \dd{\sigma(x)} := \lim_{n\to\infty}\sum_{k=1}^nf(x_k)\left[\sigma(x_k) - \sigma(x_{k-1})\right]\,,
    \end{equation*}
    dla 
    \begin{equation*}
        f: \mathbb{R} \mapsto \mathbb{R}\,,\quad \sigma: \mathbb{R} \mapsto X\,,
    \end{equation*}
    gdzie \([a;b] = \bigcup_{k=1}^{n}[x_{k-1}; x_k]\) jest podziałem normalnym odcinka \([a;b]\).

\end{itemize}

W szczególnym przypadku, gdy operator samosprzężony \(\oper{A}\) ma widmo \(\{\lambda_i\}\) będące
zbiorem przeliczalnym, wiemy, że zbiór unormowanych wektorów własnych \(\{\phi_i\}\) jest bazą
ortonormalną przestrzeni \(\mathscr{H}\), czyli dla dowolnego wektora \(\psi \in \mathscr{H}\)
możemy zapisać
\begin{equation*}
    \ket{\psi} = \sum_{i} c_i\ket{\phi_i} = \sum_{i} \braket{\phi_i}{\psi}\ket{\phi_i}\,,
\end{equation*}
gdzie \(c_i = \braket{\phi_i}{\psi} \in \mathbb{C}\) to współrzędne wektora w zadanej bazie.
Działając operatorem \(\oper{A}\) na wektor \(\psi\) mamy
\begin{equation*}
    \begin{split}
        &\oper{A}\ket{\psi} = \sum_{i} \braket{\phi_i}{\psi}\oper{A}\ket{\phi_i} =\\
        &\sum_{i} \braket{\phi_i}{\psi}\lambda_i\ket{\phi_i} = \left(\sum_i\lambda_i\ket{\phi_i}\bra{\phi_i}\right)\ket{\psi} \,.
    \end{split}
\end{equation*}
Całka Stieltjesa z twierdzenia spektralnego przechodzi więc w tym przypadku w sumę (być może
nieskończoną) operatorów rzutowych rzutujących na jednowymiarowe podprzestrzenie rozpięte na
kolejnych wektorach własnych operatora
\begin{equation*}
    \oper{A} = \sum_{i} \lambda_i \ket{\phi_i}\bra{\phi_i}\,.
\end{equation*}

\subsubsection{POSTULATY TEORII KWANTÓW}

Poniżej przedstawiono postulaty ogólnej, abstrakcyjnej teorii kwantów. Postulaty te obowiązują we
wszystkich realizacjach teorii kwantów np. mechanice falowej, czy kwantowej teorii pola, jednak ze
względu na swój ogólny charakter same w sobie nie dostarczają narzędzi do rozwiązywania żadnych
konkretnych problemów fizycznych. Nie należy ich również traktować jako podstaw do aksjomatyzacji
teorii kwantowej. Stanowią one raczej sposób uporządkowania w spójną strukturę wiedzy dotyczącej
konkretnych realizacji teorii kwantów. Podane poniżej postulaty to nieco zmodyfikowane postulaty
pochodzące z książki ,,Teoria kwantów. Mechanika falowa'' I. Białynicki--Birula, M. Cieplak, J.
Kamiński.
\medskip

\begin{enumerate}[label=\Roman*.]
    
    \item \textbf{O modelu matematycznym.} Modelem matematycznym teorii kwantów jest teoria
    przestrzeni Hilberta nad ciałem liczb zespolonych i teoria operatorów liniowych działających w
    tej przestrzeni.

    \item \textbf{O pytaniach elementarnych.} Pytaniem elementarnym nazwiemy pytanie, na które
    odpowiedź może brzmieć jedynie ,,TAK'' lub ,,NIE''. Pytanie elementarne nazwiemy rozstrzygalnym
    w obrębie danej teorii kwantowej, iff istnieje wzajemnie jednoznaczne odwzorowanie tego pytania
    na pewien operator rzutowy \(\oper{P}\). Będziemy wówczas mówili, iż dane pytanie elementarne
    jest reprezentowane przez \(\oper{P}\). Każde pytanie elementarne reprezentowane przez
    \(\oper{P}\) można zanegować otrzymując pytanie reprezentowane przez\(\oper{I} - \oper{P}\)
    (zauważmy, że \((\oper{I} - \oper{P})^2 = \oper{I} - \oper{P}\)), natomiast dwa pytania
    elementarne reprezentowane przez \(\oper{P}_1\) i \(\oper{P}_2\) można połączyć spójnikiem
    \begin{itemize}
        \item ,,I''; otrzymując pytanie reprezentowane przez \(\oper{P}_1\oper{P}_2\), przy czym z
        oczywistych względów musi zachodzić \([\oper{P}_1, \oper{P}_2] = \oper{\mathsf{0}}\)

        \item ,,LUB''; otrzymując pytanie reprezentowane przez \(\oper{P}_1 + \oper{P}_2\), przy
        czym musi zachodzić \(\oper{P}_1\oper{P}_2 = \oper{\mathsf{0}}\) (istotnie \((\oper{P}_1 +
        \oper{P}_2)^2 = \oper{P}_1 + \oper{P}_2 \)).

    \end{itemize}

    \item \textbf{O stanach układu.} Stan prostego układu fizycznego jest reprezentowany przez
    unormowany wektor \(\ket{\Psi}\) w abstrakcyjnej przestrzeni Hilberta \(\mathscr{H} =
    (\mathbb{V}, \braket{\cdot})\), przy czym utożsamiamy ze sobą wektory różniące się jedynie
    globalnym czynnikiem fazowym tj. \(\ket{\Psi} \equiv \e^{\im\alpha}\ket{\Psi}\) dla \(\alpha \in
    \mathbb{R}\).

    \item \textbf{O prawdopodobieństwach.} Teoria kwantowa dostarcza jedynie probabilistycznych
    odpowiedzi na rozstrzygalne pytania elementarne. Prawdopodobieństwo \(p\), iż odpowiedź na
    pytanie elementarne reprezentowane przez \(\oper{P}\) jest twierdząca, dla układu
    reprezentowanego przez \(\Psi\) wynosi
    \begin{equation*}
        p = \bra{\Psi}\oper{P}\ket{\Psi}\,.
    \end{equation*}
    Dla operatora rzutowego \(\ket{\Psi}\bra{\Psi}\) reprezentującego pytanie: \textit{czy układ
    znajduje się w stanie \(\Psi\)?}, prawdopodobieństwo odpowiedzi twierdzącej wynosi 1.

    \item \textbf{O wielkościach fizycznych.} Każda wielkość fizyczna \(A\) występująca w danej
    teorii kwantowej jest reprezentowana przez samosprzężony operator liniowy \(\oper{A}\) i
    stowarzyszoną z nim na mocy twierdzenia spektralnego rodzinę operatorów rzutowych
    \(\oper{P}_A(\lambda)\). Operator rzutowy \(\oper{P}_A(\lambda)\) reprezentuje pytanie:
    \textit{czy wielkość fizyczna \(A\) ma wartość nie większą od \(\lambda\)?}, natomiast operator
    rzutowy \(\oper{I}-\oper{P}_A(\lambda)\): \textit{czy wielkość fizyczna \(A\) ma wartość większą
    od \(\lambda\)?} Na mocy postulatu drugiego możemy skonstruować pytanie: \textit{czy wielkość
    fizyczna \(A\) ma wartość z przedziału \((\lambda_1;\lambda_2]\)?}, reprezentowane przez
    operator 
    \begin{equation*}
        (\oper{I} - \oper{P}_A(\lambda_1))\oper{P}_A(\lambda_2) =
        \oper{P}_A(\lambda_2)-\oper{P}_A(\lambda_1)\,.
    \end{equation*}
    Wartość oczekiwaną wielkości \(A\) dla układu reprezentowanego przez \(\Psi\) obliczamy jako
    \begin{equation*}
        \langle A \rangle = \bra{\Psi}\oper{A}\ket{\Psi}\,.
    \end{equation*}

    \item \textbf{O ewolucji układu w czasie.} Prawdopodobieństwo \(p\) odpowiedzi twierdzącej na
    pytanie \(\oper{P}\) dla układu reprezentowanego przez \(\Psi\) ewoluuje w czasie zgodnie z
    \begin{equation*}
        p(t) = \bra{\Psi(t)}\oper{P}\ket{\Psi(t)}\,,
    \end{equation*}
    gdzie wektory stanu \(\ket{\Psi}\) ewoluują zgodnie z \textit{równaniem Schr\"{o}dingera}
    \begin{equation*}
        \oper{H}\ket{\Psi} = \im\hbar\partial_t\ket{\Psi}\,,
    \end{equation*}
    gdzie w ogólności \(\oper{H} = \oper{H}(t)\) jest operatorem Hamiltona danego układu tworzonym
    wedle określonych reguł w danej realizacji teorii kwantów, natomiast \(\hbar\) to stała fizyczna
    o wymiarze działania
    \begin{equation*}
        \hbar = \frac{h}{2\pi} = 1.054571817\ldots \cdot 10^{-34}\,\text{J}\cdot\text{s}
    \end{equation*}
    zwana \textit{zredukowaną stałą Plancka}.
    
    \item \textbf{O układach złożonych.} Przestrzeń Hilberta \(\mathscr{H}\) układu złożonego ma
    strukturę iloczynu tensorowego przestrzeni Hilberta układów prostych wchodzących w jego skład
    \(\mathscr{H}=\mathscr{H}_1\otimes\mathscr{H}_2\otimes\ldots\) .

\end{enumerate}

\subsubsection{Zasada nieoznaczoności}

Niech \(A\) będzie pewną wielkością fizyczną reprezentowaną przez operator \(\oper{A}\). Zdefiniujmy
odchylenie standardowe \(\sigma_A \geq 0\) wielkości \(A\) dla układu w stanie \(\Psi\) jako
\begin{equation*}
    \sigma_A^2 := \langle (A - \langle A \rangle)^2 \rangle = \braket{(\oper{A}-a)\Psi}{(\oper{A}-a)\Psi}\,,
\end{equation*}
gdzie \(a = \langle A \rangle\) jest wartością oczekiwaną wielkości \(A\). Dla dowolnych dwóch
wielkość fizycznych \(A\) i \(B\) w układzie reprezentowanym przez \(\Psi\) mamy z nierówności
Cauchy'ego--Schwarza
\begin{equation*}
    \sigma_A^2\sigma_B^2 \geq \left|\braket{(\oper{A}-a)\Psi}{(\oper{B}-b)\Psi}\right|^2\,.
\end{equation*}
Jednocześnie dla dowolnego \(z = x + \im y \in\mathbb{C}\) mamy
\begin{equation*}
    |z|^2 = x^2 + y^2 \geq y^2 = \left(\frac{z-z^*}{2\im}\right)^2\,.
\end{equation*}
Z powyższego mamy więc
\begin{equation*}
    \sigma_A^2\sigma_B^2 \geq \left[\frac{1}{2\im}\left(\braket{\oper{\mathcal{A}}\Psi}{\oper{\mathcal{B}}\Psi} - \braket{\oper{\mathcal{B}}\Psi}{\oper{\mathcal{A}}\Psi}\right)\right]^2\,,
\end{equation*}
gdzie \(\oper{\mathcal{A}} := \oper{A} - a\), \(\oper{\mathcal{B}} := \oper{B} - b\). Jednocześnie
\begin{equation*}
    \begin{split}
        &\braket{\oper{\mathcal{A}}\Psi}{\oper{\mathcal{B}}\Psi} - \braket{\oper{\mathcal{B}}\Psi}{\oper{\mathcal{A}}\Psi} =\\
        &\bra{\Psi}(\oper{\mathcal{A}}\oper{\mathcal{B}}\ket{\Psi}-\oper{\mathcal{B}}\oper{\mathcal{A}}\ket{\Psi})=\\
        &\bra{\Psi}[\oper{\mathcal{A}},\oper{\mathcal{B}}]\ket{\Psi} = \bra{\Psi}[\oper{A},\oper{B}]\ket{\Psi}\,.
    \end{split}
\end{equation*}
Ostatecznie otrzymujemy więc
\begin{equation*}
    \boxed{
    \sigma_A^2\sigma_B^2 \geq \left(\frac{1}{2\im}\bra{\Psi}[\oper{A},\oper{B}]\ket{\Psi}\right)^2
    }\quad.
\end{equation*}
Powyższą nierówność nazywamy \textit{uogólnioną zasadą nieoznaczoności}.


\subsubsection{Twierdzenie Ehrenfesta}

Niech \(A\) będzie pewną wielkością fizyczną reprezentowaną przez operator \(\oper{A}\), wówczas
zakładając obraz Schr\"{o}dingera mamy
\begin{equation*}
    \begin{split}
        &\dv{\expectationvalue{A}}{t} = \dv{}{t} \braket{\Psi}{\oper{A}\Psi} \\
        &= \braket{\pdv{\Psi}{t}}{\oper{A}\Psi} + \braket{\Psi}{\oper{A}\pdv{\Psi}{t}} + \braket{\Psi}{\pdv{\oper{A}}{t}\Psi}\,.
    \end{split}
\end{equation*}
Jednocześnie z równania Schr\"{o}dingera mamy \(\oper{H}\Psi=\im\hbar{\partial_t\Psi}\), skąd
\begin{equation*}
    \dv{\expectationvalue{A}}{t} = \frac{\im}{\hbar}\braket{\oper{H}\Psi}{\oper{A}\Psi} - \frac{\im}{\hbar}\braket{\Psi}{\oper{A}\oper{H}\Psi} + \bra{\Psi}\pdv{\oper{A}}{t}\ket{\Psi}\,,
\end{equation*}
ale ze względu na fakt, iż \(\oper{H}\) jest operatorem samosprzężonym mamy
\begin{equation*}
    \boxed{
    \dv{\expectationvalue{A}}{t} = \bra{\Psi}\pdv{\oper{A}}{t}\ket{\Psi} + \frac{\im}{\hbar}\bra{\Psi}[\oper{H},\oper{A}]\ket{\Psi}
    }\quad.
\end{equation*}
Powyższe równanie nazywamy \textit{twierdzeniem Ehrenfesta}.

\newpage
\subsection{Kwantowe układy dwupoziomowe}

Przedstawimy teraz ważną realizację abstrakcyjnej teorii kwantów -- teorię układów dwupoziomowych,
które stanowią podstawę teorii obliczeń kwantowych i kwantowej teorii informacji.

\subsubsection{Model matematyczny}

Modelem matematycznym teorii układów dwupoziomowych jest przestrzeń Hilberta
\begin{equation*}
    \mathscr{H} = ((\mathbb{C}^2,\mathbb{C},+,\cdot), \braket{\cdot})\,,
\end{equation*}
gdzie wektory \(\Psi\in\mathbb{C}^2\) będziemy reprezentować w danej bazie ortonormalnej
\(B=\{\ket{1},\ket{2}\}\) jako macierze kolumnowe o elementach zespolonych
\begin{equation*}
    \ket{\Psi} = \psi_1\ket{1} + \psi_2\ket{2} = \mqty(\psi_1 \\ \psi_2)\,.
\end{equation*}
Jest oczywiste, że struktura \((\mathbb{C}^2,\mathbb{C},+,\cdot)\) jest przestrzenią wektorową.
Iloczyn wewnętrzny wektorów \(\Psi\), \(\Phi\) w \(\mathscr{H}\) definiujemy jako
\begin{equation*}
    \braket{\Phi}{\Psi} := \psi_1\phi_1^* + \psi_2\phi_2^*\,.
\end{equation*}
Z powyższej definicji widoczne jest, czym jest funkcjonał liniowy \(\bra{\Phi}\) odpowiadający
(zgodnie z tw. Riesza) wektorowi \(\ket{\Phi}\)
\begin{equation*}
    \bra{\Phi} = \mqty(\phi_1 \\ \phi_2)^\dagger = \mqty(\phi_1^* & \phi_2^*)\,.
\end{equation*}

Są to zatem macierze wierszowe będące \textit{sprzężeniami hermitowskimi} macierzy kolumnowych
reprezentujących wektory.

\subsubsection{Macierze Pauliego}

Macierze Pauliego definiujemy jako zespolone macierze \(2\times 2\)
\begin{equation*}
    \sigma_x := \mqty(0 & 1 \\ 1 & 0)\,,\quad \sigma_y := \mqty(0 & -\im \\ \im & 0)\,,\quad \sigma_z := \mqty(1 & 0 \\ 0 & -1)\,.
\end{equation*}
Przydatne jest zdefiniowanie \textit{wektora macierzy Pauliego} \(\boldsymbol{\sigma}\)
\begin{equation*}
    \boldsymbol{\sigma} = (\sigma_x, \sigma_y, \sigma_z)\,,
\end{equation*}
dzięki któremu możemy łatwo zapisać sumę przeskalowanych macierzy Pauliego jako
\(\mathbf{c}\cdot\boldsymbol{\sigma}\), gdzie \(\mathbf{c}\in\mathbb{C}^3\) jest pewnym wektorem o
elementach zespolonych. Wybrane własności macierzy Pauliego:

\begin{itemize}
    
    \item \(\det\sigma_x = \det\sigma_y = \det\sigma_z = -1\)

    \item \(\Tr\sigma_x = \Tr\sigma_y = \Tr\sigma_z = 0\)

    \item \(\sigma_x^2 = \sigma_y^2 = \sigma_z^2 = \oper{I}\)

    \item Iloczyny macierzy Pauliego spełniają związki
    \begin{equation*}
        \begin{split}
            &\sigma_x\sigma_y = \im\sigma_z = -\sigma_y\sigma_x \\
            &\sigma_y\sigma_z = \im\sigma_x = -\sigma_z\sigma_y \\
            &\sigma_z\sigma_x = \im\sigma_y = -\sigma_x\sigma_z
        \end{split}\quad.
    \end{equation*}

    \item Komutatory i antykomutatory macierzy Pauliego wynoszą
    \begin{equation*}
        \begin{split}
            [\sigma_x,\sigma_y] = 2\im\sigma_z\,,\quad \{\sigma_x, \sigma_y\} = \oper{\mathsf{0}}\\
            [\sigma_y,\sigma_z] = 2\im\sigma_x\,,\quad \{\sigma_y, \sigma_z\} = \oper{\mathsf{0}}\\
            [\sigma_z,\sigma_x] = 2\im\sigma_y\,,\quad \{\sigma_z, \sigma_x\} = \oper{\mathsf{0}}\\
        \end{split}\quad.
    \end{equation*}

    \item Iloczyn \((\mathbf{a}\cdot\boldsymbol{\sigma})\cdot(\mathbf{b}\cdot\boldsymbol{\sigma})\)
    dla \(\mathbf{a} = (a_x,a_y,a_z)\), \(\mathbf{b}=(b_x,b_y,b_z)\) wynosi
    \begin{equation*}
        \begin{split}
            &(\mathbf{a}\cdot\boldsymbol{\sigma})\cdot(\mathbf{b}\cdot\boldsymbol{\sigma}) = \\
            &= (a_x\sigma_x + a_y\sigma_y + a_z\sigma_z)(b_x\sigma_x + b_y\sigma_y + b_z\sigma_z) =\\
            &= (\mathbf{a}\cdot\mathbf{b})\oper{I} + \im(\mathbf{a}\times\mathbf{b})\cdot\boldsymbol{\sigma}\,.
        \end{split}
    \end{equation*}

    \item Wielkość \(\exp(\im\mathbf{a}\cdot\boldsymbol{\sigma})\) wynosi
    \begin{equation*}
            \exp(\im\mathbf{a}\cdot\boldsymbol{\sigma}) = \oper{I} \cos{|\mathbf{a}|} + \im\left(\frac{\mathbf{a}}{|\mathbf{a}|}\cdot\boldsymbol{\sigma}\right)\sin{|\mathbf{a}|}\,.
    \end{equation*}

\end{itemize}

\subsubsection{Operatory}

Operatory w \(\mathscr{H}\) to w ogólności odwzorowania liniowe
\(\oper{A}:\mathbb{C}^2\mapsto\mathbb{C}^2\), które, jak wiadomo, zawsze można reprezentować (dla
przestrzeni skończenie wymiarowych) przez macierze, której elementy są wyznaczone przez wartości
odwzorowania \(\oper{A}\) na wektorach bazowych, tj. zakładając
\begin{equation*}
    \begin{split}
        \oper{A}\ket{1} = a_{11}\ket{1} + a_{21}\ket{2}\\
        \oper{A}\ket{2} = a_{12}\ket{1} + a_{22}\ket{2}
    \end{split}\quad,
\end{equation*}
macierz endomorfizmu \(\oper{A}\) możemy zapisać jako
\begin{equation*}
    \oper{A} = \mqty(a_{11} & a_{12} \\ a_{21} & a_{22})
\end{equation*}
dla \(a_{ij} \in \mathbb{C}\).

\begin{itemize}
    
    \item Ślad operatora \(\oper{A}\) w przestrzeni \((\mathbb{C}^2,\mathbb{C},+,\cdot)\) jest dany
    przez sumę elementów diagonalnych macierzy \(\oper{A}\)
    \begin{equation*}
        \Trace{\oper{A}} = \bra{1}\oper{A}\ket{1} + \bra{2}\oper{A}\ket{2} = a_{11} + a_{22}\,.
    \end{equation*}

    \item Sprzężeniem operatora reprezentowanego przez macierz
    \begin{equation*}
        \oper{A} = \mqty(a_{11} & a_{12} \\ a_{21} & a_{22})
    \end{equation*}
    jest po prostu sprzężenie hermitowskie tej macierzy, tj.
    \begin{equation*}
        \oper{A}^\dagger = \mqty(a^*_{11} & a^*_{21} \\ a^*_{12} & a^*_{22})\,.
    \end{equation*}

    \item Operatory samosprzężone to operatory \(\oper{A}\), które spełniają równość
    \begin{equation*}
        \oper{A} = \mqty(a_{11} & a_{12} \\ a_{21} & a_{22}) = \mqty(a^*_{11} & a^*_{21} \\ a^*_{12} & a^*_{22}) = \oper{A}^\dagger\,,
    \end{equation*}
    czyli \(a_{11},a_{22}\in\mathbb{R}\) i \(a_{12} = a_{21}^*\). Są to zatem operatory
    reprezentowane przez macierze hermitowskie.

    \item Zauważmy, iż dowolną macierz hermitowską \(\oper{A}\) możemy przedstawić jako
    \begin{equation*}
        \oper{A} = a_0\oper{I} + \mathbf{a}\cdot\boldsymbol{\sigma} =\mqty(a_0+a_z & a_x-\im a_y \\ a_x + \im a_y & a_0 - a_z)
    \end{equation*}
    dla \(a_0\in\mathbb{R}\) i rzeczywistego wektora \(\mathbf{a}\in\mathbb{R}^3\).

    Zauważmy również, że bez straty ogólności zawsze możemy zakładać \(a_0=0\). Istotnie, jeśli
    \(\ket{\Psi}\) jest wektorem własnym operatora \(\oper{A} =
    a_0\oper{I}+\mathbf{a}\boldsymbol{\sigma}\) z wartością własną \(\lambda\), tj. zachodzi
    \begin{equation*}
        a_0\oper{I}\ket{\Psi}+\mathbf{a}\boldsymbol{\sigma}\ket{\Psi} = \lambda\ket{\Psi}\,,
    \end{equation*}
    to \(\ket{\Psi}\) jest wektorem własnym operatora \(\oper{A}' = \mathbf{a}\boldsymbol{\sigma}\)
    z wartością własną \(\lambda-a_0\). Istotnie z założenia
    \begin{equation*}
        \mathbf{a}\boldsymbol{\sigma}\ket{\Psi} = \lambda\ket{\Psi} - a_0\oper{I}\ket{\Psi} = (\lambda-a_0)\ket{\Psi}\,,
    \end{equation*}
    czyli przyjmując \(a_0 = 0\), przesuwamy jedynie widmo operatora o pewną stałą.

\end{itemize}

\subsubsection{Sfera Blocha}

Dowolny operator samosprzężony \(\oper{A}\) możemy przedstawić w trójwymiarowej przestrzeni,
korzystając z reprezentacji Pauliego
\begin{equation*}
    \oper{A} = \mathbf{a} \cdot \boldsymbol{\sigma}
\end{equation*}
dla \(\mathbf{a} \in \mathbb{R}^3\). Zauważmy również, iż w tej samej przestrzeni możemy przedstawić
wektor stanu \(\ket{\Psi}\). Istotnie operator rzutowy \(\ket{\Psi}\bra{\Psi}\) jest operatorem
samosprzężonym, więc istnieje para \((r_0,\mathbf{r})\) taka, że
\begin{equation*}
    \ket{\Psi}\bra{\Psi} = \mqty(\psi_1\psi_1^* & \psi_1\psi_2^* \\ \psi_1^*\psi_2 & \psi_2\psi_2^*) = r_0\oper{I} + \mathbf{r}\cdot\boldsymbol{\sigma}\,.
\end{equation*}
Ze względu na unormowanie \(\braket{\Psi} = 1\) mamy warunek \(r_0 = \frac{1}{2}\), natomiast
wartość wyznacznika nakłada dodatkowy warunek \(\det\ket{\Psi}\bra{\Psi} = 0 = r_0^2 -
|\mathbf{r}|^2\). Dowolny wektor stanu \(\ket{\Psi}\) możemy więc przedstawić w przestrzeni
trójwymiarowej \((x,y,z)\) jako
\begin{equation*}
    \ket{\Psi}\bra{\Psi} = \frac{1}{2}\left(\oper{I} + \mathbf{r}\cdot\boldsymbol{\sigma}\right)\,,
\end{equation*}
przy warunku \(|\mathbf{r}|^2 = 1\), co określa punkt na sferze jednostkowej o środku w punkcie
\((0,0,0)\), którą nazywamy \textit{sferą Blocha}.
\medskip

Możemy również wyprowadzić równanie ewolucji punktu \(\mathbf{r}(t)\) na sferze Blocha dla układu
opisywanego hamiltonianem \(\oper{H} = \mathbf{h}(t)\cdot\boldsymbol{\sigma}\). Istotnie
różniczkując operator rzutowy \(\oper{\rho} = \ket{\Psi}\bra{\Psi}\) mamy
\begin{equation*}
    \begin{split}
        \im\hbar\partial_t\oper{\rho} &= \im\hbar(\partial_t\ket{\Psi})\bra{\Psi} + \im\hbar\ket{\Psi}\partial_t\bra{\Psi} \\
        &= \oper{H}\ket{\Psi}\bra{\Psi} + \im\hbar\ket{\Psi}(\partial_t\ket{\Psi})^\dagger\\
        &= \oper{H}\ket{\Psi}\bra{\Psi} + \im\hbar\ket{\Psi}\left(\frac{1}{\im\hbar}\oper{H}\ket{\Psi}\right)^\dagger\\
        &= \oper{H}\ket{\Psi}\bra{\Psi} - \ket{\Psi}\bra{\Psi}\oper{H} = [\oper{H},\oper{\rho}]\,,
    \end{split}
\end{equation*}
skąd, korzystając z tożsamości na iloczyn
\(\mathbf{a}\boldsymbol{\sigma}\cdot\mathbf{b}\boldsymbol{\sigma}\), otrzymujemy równanie
\begin{equation*}
    \dot{\mathbf{r}} = \frac{1}{\hbar}(\mathbf{h}(t) \times \mathbf{r})\,,
\end{equation*}
które nazywamy \textit{równaniem Blocha}. Ponieważ \(|\mathbf{r}|^2 = 1\), więc wygodnie jest
zapisać powyższe równanie w układzie współrzędnych sferycznych \((r,\theta,\phi)\)
\begin{equation*}
    \begin{cases}
        r = 1\\
        \dot{\theta} = \frac{1}{\hbar}h_\phi(t)\\
        \dot{\phi} = -\frac{1}{\hbar}h_\theta(t)\csc\theta
    \end{cases}\quad,
\end{equation*}
gdzie \((h_r,h_\theta,h_\phi)\) są dane równaniem
\begin{equation*}
    \mqty(h_r\\h_\theta\\h_\phi) = \frac{\partial(r,\theta,\phi)}{\partial(x,y,z)}\mqty(h_x \\ h_y\\h_z)\,.
\end{equation*}
Pozostaje jeszcze pokazać jak zapisać stan \(\ket{\Psi}\) reprezentowany na sferze przez współrzędne
\((r,\theta,\phi)\). Zauważmy wpierw, iż bez straty ogólności możemy przyjąć
\begin{equation*}
    \ket{\Psi(t)} = \mqty(\alpha_1\e^{\im\beta} \\ \alpha_2)\,,
\end{equation*}
dla \(\alpha_1,\alpha_2,\beta\in\mathbb{R}\) (istotnie pamiętajmy, iż utożsamiamy wektory różniące
się jedynie globalnym czynnikiem fazowym). W takim razie mamy
\begin{equation*}
    \begin{split}
        &\alpha_1 = \frac{1}{\sqrt{2}}\sqrt{1+\cos\theta} = \cos\frac{\theta}{2}\\
        &\alpha_2 = \frac{1}{\sqrt{2}}\sqrt{1-\cos\theta} = \sin\frac{\theta}{2}\\
        &\beta = \phi
    \end{split}\quad,
\end{equation*}
skąd
\begin{equation*}
        \ket{\Psi(t)} =\cos\frac{\theta}{2}\e^{\im\phi}\ket{1} + \sin\frac{\theta}{2}\ket{2}\,.
\end{equation*}

\subsubsection{Ewolucja układu dwupoziomowego}

W przypadku hamiltonianu \(\oper{H}=\mathbf{h}\boldsymbol{\sigma}\) niezależnego od czasu możemy
podać ogólne rozwiązanie równania Schr\"{o}dingera postaci
\begin{equation*}
    \begin{split}
        \ket{\Psi(t)} &= \e^{-\im\mathbf{h}\boldsymbol{\sigma}t/\hbar}\ket{\Psi(0)} \\
        &= \left[\oper{I}\cos\left(\frac{|\mathbf{h}|}{\hbar}t\right) - \im \mathbf{\hat{h}}\boldsymbol{\sigma}\sin\left(\frac{|\mathbf{h}|}{\hbar}t\right)\right]\ket{\Psi(0)}\,.
    \end{split}
\end{equation*}
\medskip

W przypadku hamiltonianów zależnych od czasu możemy znaleźć rozwiązanie przybliżone, korzystając z
rachunku zaburzeń (najczęściej 1-go lub 2-go rzędu). Istnieje jednak ważny układ dwupoziomowy z
hamiltonianem zależnym od czasu, dla którego istnieje ścisłe rozwiązanie analityczne.

\noindent\rule{\columnwidth}{0.5pt}\\
\textcolor{blue}{\textbf{Oscylacje Rabiego.} Rozważmy układ opisany hamiltonianem
\begin{equation*}
    \oper{H} = -\frac{1}{2}\gamma\hbar\mathbf{B}(t)\boldsymbol{\sigma}\,,
\end{equation*}    
gdzie
\begin{equation*}
    \mathbf{B}(t) = B_0(\sin\beta\cos\Omega t,\sin\beta\sin\Omega t, \cos\beta)\,,
\end{equation*}
dla pewnych stałych \(\gamma\), \(B_0\), \(\beta\), \(\Omega\), który opisuje interakcję spinu z
zewnętrznym polem magnetycznym. Chcemy rozwiązać równanie Schr\"{o}dingera
\begin{equation*}
    \mqty(\dot{\psi}_1 \\ \dot{\psi}_2) = \frac{\im\gamma B_0}{2}\mqty(\cos\beta & \e^{-\im\Omega t}\sin\beta \\ \e^{+\im\Omega t}\sin\beta & \cos\beta) \mqty(\psi_1 \\ \psi_2)\,.
\end{equation*}
Poszukajmy rozwiązań postaci
\begin{equation*}
    \mqty(\psi_1 \\ \psi_2) = \mqty(\phi_1\e^{-\frac{\im}{2}\Omega t} \\ \phi_2 \e^{+\frac{\im}{2}\Omega t})\,,
\end{equation*}
skąd otrzymujemy
\begin{equation*}
    \mqty(\dot{\phi}_1 \\ \dot{\phi}_2) = \frac{\im}{2} \mqty(\Omega +\omega_0\cos\beta & \omega_0\sin\beta \\ \omega_0\sin\beta & -(\Omega +\omega_0\cos\beta)) \mqty(\phi_1 \\ \phi_2)\,,
\end{equation*}
gdzie \(\omega_0 := \gamma B_0\). Wartości własne powyższej macierzy to
\begin{equation*}
    \lambda = \pm\frac{\im}{2}\sqrt{\omega_0^2 + \Omega^2 + 2\omega_0\Omega\cos\beta} = \pm\frac{\im}{2}\Lambda\,,
\end{equation*}
natomiast wektory własne mają postać
\begin{equation*}
    \alpha_\pm \mqty( -\omega_0\sin\beta \\ \Omega + \omega_0\cos\beta \mp \Lambda)\,,
\end{equation*}
dla pewnych niezerowych stałych \(\alpha_+\), \(\alpha_-\). Rozwiązanie na wektor \(\ket{\phi(t)}\)
ma więc postać
\begin{equation*}
    \begin{split}
        \ket{\phi(t)} &= \alpha_+\mqty(-\omega_0\sin\beta \\ \Omega+\omega_0\cos\beta - \Lambda)\e^{\im\Lambda t/2} \\
        &+ \alpha_- \mqty(-\omega_0\sin\beta \\ \Omega + \omega_0\cos\beta+\Lambda)\e^{-\Lambda t/2}\,.
    \end{split}
\end{equation*}
Zakładając, iż w stanie początkowym \(\ket{\Psi(0)} = \ket{2}\) możemy obliczyć prawdopodobieństwo
przejścia do stanu \(\ket{1}\)
\begin{equation*}
    p_{2\to1}(t) = |\braket{1}{\Psi}|^2 = \frac{\omega_0^2\sin^2\beta}{\Lambda^2}\sin^2\left(\frac{\Lambda t}{2}\right)\,.
\end{equation*}
Zauważmy, że prawdopodobieństwo przejścia oscyluje z amplitudą zależną od częstości wymuszenia
\begin{equation*}
    p_\text{max}(\Omega) = \frac{\sin^2\beta}{1 + \left(\frac{\Omega}{\gamma B_0}\right)^2 + 2\left(\frac{\Omega}{\gamma B_0}\right) \cos\beta} \,.
\end{equation*}
Amplituda ta przyjmuje wartość maksymalną dla częstości wymuszenia równej \(|\Omega_\text{rez}| =
|\omega_0 \cos\beta|\). Zauważmy również, iż niezerowa szerokość połówkowa nie wynika z żadnych
procesów dyssypatywnych, jak ma to miejsce w np. w przypadku klasycznego oscylatora harmonicznego z
tłumieniem i wymuszeniem, tylko z samej teorii kwantowej. }\noindent\rule{\columnwidth}{0.5pt}
\newpage
\subsection{OBLICZENIA KWANTOWE}









\end{document}
