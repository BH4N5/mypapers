\documentclass{myclass}
\usepackage[polish]{babel}

\numberwithin{equation}{section}

\begin{document}

\tableofcontents
\newpage

\section{Wstęp}

Celem tych notatek jest zwięzłe przedstawienie kompletu zagadnień związanych z szeroko pojętym
uczeniem głębokim jako podejściem do Sztucznej Inteligencji (SI). Zaczynamy od minimalnego zbioru
wymaganych tematów z zakresu rachunku prawdopodobieństwa i statystyki matematycznej. Następnie
opisujemy podstawowe metody uczenia maszynowego z probabilistycznego punktu widzenia. W końcu
przechodzimy do zasadniczej części związanej z uczeniem głębokim i sieciami neuronowymi. W każdej
części staramy się przedstawiać opisywane tematy w sposób minimalistyczny, skupiając się głównie na
matematycznej i ideowej, a nie implementacyjnej stronie zagadnień. Liczymy, iż takie podejście
zapewni odpowiednio głębokie zrozumienie tematu, dzięki któremu dalsze studiowanie całej gamy
specyficznych technicznych tematów nie sprawi żadnego problemu.

\subsection{Notacja}

W dalszej części tekstu będziemy stosować przedstawioną tutaj pokrótce notację. Wektory, które
traktujemy jako elementy przestrzeni \(\mathbb{R}^d\) ze standardowo zdefiniowanymi operacjami
dodawania i mnożenia przez skalar będziemy oznaczali wytłuszczonymi małymi literami np.
\(\mathbf{x}, \mathbf{w}, \boldsymbol{\phi}\). Wielkość \(\mathbf{x}_i\) z indeksem dolnym będącym
literą alfabetu łacińskiego będzie oznaczać dany element wektora (w tym przypadku \(i\)--ty element
\(\mathbf{x}\)). Wielkość \(\mathbf{x}^\mu\) z indeksem górnym będącym literą alfabetu greckiego
będzie oznaczać pewien (w tym przypadku \(\mu\)--ty) element pewnego zbioru wektorów. Macierze oraz
wielowymiarowe tablice (zwane również niefortunnie tensorami) będziemy oznaczać wytłuszczonymi
wielkimi literami np. \(\mathbf{X}, \mathbf{W}, \boldsymbol{\Phi}\). Analogicznie jak w przypadku
wektorów przez \(\mathbf{X}_{i_1 i_2 \ldots i_k}\) będziemy oznaczać \((i_1,i_2,\ldots,i_k)\)
element \(k\)--wymiarowej tablicy \(\mathbf{X}\), natomiast \(\mathbf{X}^\mu\) będzie oznaczać
\(\mu\)--ty element pewnego zbioru tablic.

\subsection{Uczenie nadzorowane}

Uczenie nadzorowane jest jednym z dwóch podstawowych (pomijając tzw. uczenie ze wzmocnieniem)
paradygmatów w uczeniu maszynowym, którego ogólną ideą jest zdefiniowanie pewnego modelu
odwzorowującego dane wejściowe na wyjściowe predykcje. Zakładamy w nim, iż mamy dostępny zbiór
obserwacji w postaci uporządkowanych par \(\mathcal{X} = \{(\mathbf{x}^\mu, t_\mu)\}_{\mu=1}^n\),
gdzie \(\mathbf{x} \in \mathbb{R}^d\) nazywamy wektorem cech a \(t\) jest prawidłową wartością
odpowiedzi dla tych cech. Dwa najbardziej podstawowe przypadki zagadnień tego rodzaju to regresja
oraz klasyfikacja. W przypadku regresji zmienna \(t\) przyjmuje wartości z przedziału liczb
rzeczywistych. W przypadku klasyfikacji zmienna \(t\) przyjmuje wartości ze skończonego zbioru
kategorii, przy czym wartości z tego zbioru nie powinny posiadać naturalnej tj. wynikającej z natury
problemu, relacji porządku; gdy tak nie jest mamy do czynienia z regresją/klasyfikacją porządkową (z
ang. \textit{ordinal regression/classification}).

W jaki sposób tworzymy wspomniany model odwzorowujący \(\mathbf{x}\) na \(t\)? W dalszych
paragrafach poznamy różne metody, ale najczęściej (nie wchodząc teraz w modelowanie
probabilistyczne) modelem jest pewna rodzina funkcji postaci \(\phi(\mathbf{x}; \mathbf{w})\)
parametryzowana skończoną liczbą parametrów, które możemy łącznie zapisać jako pewien wektor
\(\mathbf{w} \in \mathbb{R}^m\). Aby znaleźć parametry \(\mathbf{w}\), dzięki którym dla konkretnego
zagadnienia model będzie zadowalająco odwzorowywał cechy na predykcje (innymi słowy aby nauczyć
model) wprowadzamy dodatkowo funkcjonał kosztu (z ang. \textit{loss function})
\(L[\phi(\mathbf{x};\mathbf{w}), t]\), który kwantyfikuje odpowiedzi modelu \(\phi\) w stosunku do
znanych prawidłowych odpowiedzi \(t\). Trening modelu polega wówczas na znalezieniu parametrów
\(\mathbf{w}^*\), które minimalizują sumę wartości funkcji kosztu dla przykładów w zbiorze
treningowym \(\mathcal{X}\)
\begin{equation}
    \mathbf{w}^* = \arg\min_{\mathbf{w}} \sum_{\mu=1}^n L[\phi(\mathbf{x}^\mu;\mathbf{w}), t_\mu]\,.
\end{equation}
Zauważmy, że takie podejście ma jedną zasadniczą wadę -- istotnie nie interesuje nas tak naprawdę,
jak model radzi sobie na zbiorze treningowym, tylko jak będzie radził sobie na nowych, niewidzianych
wcześniej danych. Sytuację, w której model bardzo dobrze modeluje dane w zbiorze treningowym, ale
słabo radzi sobie na nowych danych nazywamy przeuczeniem lub nadmiernym dopasowaniem (z ang.
\textit{overfitting}). Sytuację, w której model słabo radzi sobie zarówno na zbiorze treningowym,
jak i na nowych danych nazywamy niedouczeniem lub niedopasowaniem (z ang. \textit{underfitting}).
Występowanie overfittingu i underfittingu jest powiązane z pojemnością (z ang. \textit{capacity})
modelu. Złożony model o dużej pojemności potrafi dopasować się do bardzo skomplikowanych obserwacji,
ale istnieje ryzyko jego przeuczenia (mówimy wówczas o \textit{high variance}). Dla prostego modelu
o małej pojemności istnieje z kolei ryzyko, iż nie ma on wystarczająco ekspresywności (mówimy
wówczas o \textit{high bias}).
 
\subsection{Uczenie nienadzorowane}

W przypadku uczenia nienadzorowanego naszym celem nie jest znalezienie modelu odwzorowującego cechy
na predykcje. Chcemy raczej zrozumieć wewnętrzną strukturę danych. Modele tego rodzaju znajdują
zastosowanie w analizie biznesowej, gdzie pozwalają, chociażby na analizę ważności poszczególnych
wskaźników, czy wizualizację wysoko-wymiarowych danych. Nas będą interesować natomiast szczególnie
generatywne modele nienadzorowane, za pomocą których modelujemy rozkład prawdopodobieństwa danych,
czy to wprost poprzez funkcję gęstości prawdopodobieństwa, czy też jedynie jako model, z którego
możemy próbkować nowe przykłady.

\subsection{Praktyka uczenia maszynowego}

W dalszej części nie będziemy skupiać się na detalach implementacyjnych. W naszej ocenie jednak
dziedzina uczenia maszynowego jest przede wszystkim \enquote{nauką eksperymentalną} dlatego bardzo
ważne jest empiryczne sprawdzenie różnych metod przed wybraniem ostatecznego modelu dla danego
zagadnienia. W tym paragrafie opisujemy standardowe praktyki stosowane przy treningu modeli uczenia
maszynowego (głównie nadzorowanych operujących na danych tabelarycznych).

\subsubsection{Przygotowanie danych}

Kluczem do uzyskania dobrych wyników przy korzystaniu z algorytmów uczenia maszynowego jest
odpowiednie przygotowanie danych (z ang. \textit{preprocessing}). Typowo preprocessing składa się z:

\begin{itemize}
   
    \item wczytania danych;
    
    \item eksploracji danych oraz wstępnego czyszczenia, w szczególności usunięcia jawnych wartości
    odstających (z ang. \textit{outliers}) oraz cech posiadających zbyt dużo wartości brakujących;

    \item analizy rozkładu zmiennej docelowej oraz ewentualnej transformacji logarytmicznej, która
    poprawia stabilność numeryczną, gdy przewidywane wartości są dużymi dodatnimi liczbami
    rzeczywistymi, zmienia dziedzinę zmiennej objaśnianej z \(\mathbb{R}_+\) na \(\mathbb{R}\) oraz
    dodatkowo jest przykładem transformacji stabilizującej wariancję;
    
    \item \textbf{podziału zbioru na część treningową oraz testową};

    \item dokonania skalowania i imputacji brakujących wartości cech (metody \texttt{.fit()}
    wywołujemy jedynie dla zbioru treningowego);

    \item usunięcia silnie skorelowanych cech;

    \item zakodowania wartości kategorycznych za pomocą tzw. \textit{one--hot encoding} pamiętając o
    \textit{dummy variable trap} -- jedną z \(k\) kategorii kodujemy za pomocą wektora
    \textit{one--hot} długości \(n-1\), aby uniknąć zależności liniowej między cechami (opcja
    \texttt{drop="first"} w \texttt{OneHotEncoder} w scikit-learn);

    \item wykonania feature engineering -- dodania wielomianów cech do naszych danych lub
    skonstruowania innych cech (np. cech określających miesiąc, dzień itp.);

\end{itemize}

Podział zbioru na część treningową i testową jest najważniejszym etapem preprocessingu. Zbiór
testowy wydzielamy, aby po wytrenowaniu modelu sprawdzić, jak poradzi on sobie na nowych,
niewidzianych wcześniej danych. Powinniśmy go traktować jako dane, które będziemy w przyszłości
dostawać po wdrożeniu modelu do realnego systemu. Takie dane również będziemy musieli przeskalować,
zakodować itp., ale parametry potrzebne do wykonania tych transformacji możemy wziąć jedynie z
dostępnego wcześniej zbioru treningowego. Wykorzystanie danych testowych w procesie treningu to
\textbf{błąd wycieku danych} (z ang. \textit{data leakage}). Skutkuje on niepoprawnym, nadmiernie
optymistycznym oszacowaniem jakości modelu.

\subsubsection{Tuning hiperparametrów i walidacja skrośna}

Praktycznie wszystkie modele uczenia maszynowego mają hiperparametry, często liczne, które w
zauważalny sposób wpływają na wyniki, a szczególnie na underfitting i overfitting. Ich wartości
trzeba dobrać zatem dość dokładnie. Proces doboru hiperparametrów nazywa się tuningiem
hiperparametrów (z ang. \textit{hyperparameter tuning}).

Istnieje na to wiele sposobów. Większość z nich polega na tym, że trenuje się za każdym razem model
z nowym zestawem hiperparametrów i wybiera się ten zestaw, który pozwala uzyskać najlepsze wyniki.
Metody głównie różnią się między sobą sposobem doboru kandydujących zestawów hiperparametrów.
Najprostsze i najpopularniejsze to:
\begin{itemize}

    \item pełne przeszukiwanie (z ang. \textit{grid search}) -- definiujemy możliwe wartości dla
    różnych hiperparametrów, a metoda sprawdza ich wszystkie możliwe kombinacje (czyli siatkę),

    \item losowe przeszukiwanie (z ang. \textit{randomized search}) -- definiujemy możliwe wartości
    jak w pełnym przeszukiwaniu, ale sprawdzamy tylko ograniczoną liczbę losowo wybranych
    kombinacji.

\end{itemize}

Jak ocenić, jak dobry jest jakiś zestaw hiperparametrów? Nie możemy sprawdzić tego na zbiorze
treningowym -- wyniki byłyby zbyt optymistyczne. Nie możemy wykorzystać zbioru testowego --
mielibyśmy wyciek danych, bo wybieralibyśmy model explicite pod nasz zbiór testowy. Trzeba zatem
osobnego zbioru, na którym będziemy na bieżąco sprawdzać jakość modeli dla różnych hiperparametrów.
Jest to zbiór walidacyjny (z ang. \textit{validation set}). Zbiór taki wycina się ze zbioru
treningowego.

Jednorazowy podział zbioru na części nazywa się \textit{split validation} lub \textit{holdout}.
Używamy go, gdy mamy sporo danych, i 10-20\% zbioru jako dane walidacyjne czy testowe to dość dużo,
żeby mieć przyzwoite oszacowanie. Zbyt mały zbiór walidacyjny czy testowy da nam mało wiarygodne
wyniki -- nie da się nawet powiedzieć, czy zbyt pesymistyczne, czy optymistyczne. W praktyce
niestety często mamy mało danych. Trzeba zatem jakiejś magicznej metody, która stworzy nam więcej
zbiorów walidacyjnych z tej samej ilości danych. Taką metodą jest walidacja skrośna (z ang.
\textit{cross validation}, CV). Polega na tym, że dzielimy zbiór treningowy na \(K\) równych
podzbiorów, tzw. foldów. Każdy podzbiór po kolei staje się zbiorem walidacyjnym, a pozostałe łączymy
w zbiór treningowy. Trenujemy zatem \(K\) modeli dla tego samego zestawu hiperparametrów i każdy
testujemy na zbiorze walidacyjnym. Mamy \(K\) wyników dla zbiorów walidacyjnych, które możemy
uśrednić (i ewentualnie obliczyć odchylenie standardowe). Takie wyniki są znacznie bardziej
wiarygodne.

\section{Rachunek prawdopodobieństwa i statystyka matematyczna}

\subsection{Przestrzeń probabilistyczna}

Pojęciem pierwotnym w rachunku prawdopodobieństwa jest pojęcie \textbf{przestrzeni zdarzeń
elementarnych}, którą oznaczamy \(\Omega\). W przypadku doświadczeń losowych przestrzeń zdarzeń
elementarnych jest zbiorem wszystkich niepodzielnych wyników obserwacji.

\begin{definition}[Rodziny zdarzeń]

Niech \(\Omega\) będzie przestrzenią zdarzeń elementarnych. Rodziną zdarzeń nazwiemy rodzinę zbiorów
\(\mathcal{F}\) taką, że
\begin{enumerate}
    
    \item \(\Omega \in \mathcal{F}\).
    
    \item Jeśli \(A \in \mathcal{F}\) to \(\Omega \setminus A \in \mathcal{F}\).
    
    \item Jeśli \(A_1,A_2,\ldots \in \mathcal{F}\) to \(\bigcup_{i=1}^\infty A_i \in \mathcal{F}\).

\end{enumerate}

Rodzinę zdarzeń \(\mathcal{F}\) nazywamy \(\sigma\)--ciałem zdarzeń losowych.

\end{definition}

\begin{definition}[Zdarzenia losowego]
    Zdarzeniem losowym nazywamy dowolny zbiór należący do rodziny zdarzeń \(\mathcal{F}\). W
    szczególności \(\Omega\) nazwiemy zdarzeniem pewnym, a \(\emptyset\) -- zdarzeniem niemożliwym.
\end{definition}

\begin{definition}[Rozkładu prawdopodobieństwa]
Niech dana będzie przestrzeń zdarzeń elementarnych \(\Omega\) i rodzina zdarzeń \(\mathcal{F}\).
Rozkładem prawdopodobieństwa nazwiemy funkcję \(P: \mathcal{F} \mapsto [0;1]\) spełniającą
\begin{enumerate}
    
    \item Dla każdego \(A \in \mathcal{F}\) zachodzi \(P(A) \geq 0\).
    
    \item \(P(\Omega) = 1\).
    
    \item Dla każdego ciągu zdarzeń parami rozłącznych \(A_1,A_2,\ldots\), \(\forall i \neq j : A_i
    \cap A_j = \emptyset\) zachodzi
    \begin{equation*}
        P\left(\bigcup_{i=1}^\infty A_i\right) = \sum_{i=1}^\infty P(A_i)\,.
    \end{equation*}

\end{enumerate}
\end{definition}

\begin{theorem}
Niech \(P\) będzie rozkładem prawdopodobieństwa w rodzinie zdarzeń \(\mathcal{F}\), wówczas

\begin{enumerate}
    
    \item \(P(\emptyset) = 1\).
    
    \item (Addytywność) Dla dowolnych zdarzeń \(A_1,\ldots,A_n\) parami rozłącznych zachodzi
    \begin{equation*}
        P\left(\bigcup_{i=1}^n A_i\right) = \sum_{i=1}^n P(A_i)\,.
    \end{equation*}

    \item Dla dowolnego zdarzenia \(A\) zachodzi
    \begin{equation*}
        P(A') = 1 - P(A)\,.
    \end{equation*}

    \item Dla dowolnych zdarzeń \(A, B\) zachodzi
    \begin{equation*}
        P(A \cup B) = P(A) + P(B) - P(A \cap B)\,.
    \end{equation*}

    \item Jeśli \(A \subset B\) to \(P(A) \leq P(B)\).

    \item Dla każdego zdarzenia \(A\) zachodzi \(P(A) \leq 1\).

\end{enumerate}
\end{theorem}

\begin{definition}[Przestrzeni probabilistycznej]
Przestrzenią probabilistyczną nazywamy uporządkowaną trójkę \((\Omega, \mathcal{F}, P)\), gdzie
\(\Omega\) jest przestrzenią zdarzeń elementarnych, \(\mathcal{F}\) -- rodziną zdarzeń określoną na
\(\Omega\), a \(P\) -- rozkładem prawdopodobieństwa w \(\mathcal{F}\).
\end{definition}

\subsection{Prawdopodobieństwo warunkowe i niezależność zdarzeń}

\begin{definition}[Prawdopodobieństwa warunkowego]
Jeśli \(P(B) > 0\) to prawdopodobieństwem \(P(A \mid B)\) zdarzenia \(A\) pod warunkiem zdarzenia
\(B\) nazywamy iloraz prawdopodobieństw

\begin{equation*}
    P(A \mid B) := \frac{P(A \cap B)}{ P(B)}\,.
\end{equation*}
\end{definition}

Można pokazać, iż prawdopodobieństwo warunkowe jako funkcja zdarzenia \(A\) przy ustalonym zdarzeniu
\(B\) spełnia wszystkie aksjomaty rozkładu prawdopodobieństwa.

\begin{theorem}[O wielokrotnym warunkowaniu]
Dla dowolnych zdarzeń \(A_1,\ldots,A_n\) takich, że \(P(A_1,\ldots,A_n) > 0\) zachodzi
\begin{equation*}
    \begin{split}
        &P(A_1,\ldots,A_n) =\\
        &P(A_n \mid A_{n-1},\ldots,A_1) \ldots P(A_2 \mid A_1) P(A_1)\,,
    \end{split}
\end{equation*}
gdzie dla uproszczenia zapisu używamy \(P(A,B)\) w znaczeniu \(P(A \cap B)\).
\end{theorem}

\begin{definition}[Niezależności pary zdarzeń]
Zdarzenia \(A, B\) nazwiemy niezależnymi jeśli
\begin{equation*}
    P(A, B) = P(A) P(B)\,.
\end{equation*}
\end{definition}

Zauważmy, iż w przypadku \(P(B) > 0\) powyższa definicja jest równoważna bardziej intuicyjnej
wynikającej z definicji prawdopodobieństwa warunkowego, mianowicie zdarzenia \(A,B\) nazywamy
niezależnymi jeśli \(P(A \mid B) = P(A)\).

\begin{definition}[Niezależności zdarzeń]
Zdarzenia \(A_1,\ldots,A_n\) nazwiemy niezależnymi, jeśli dla dowolnych wskaźników
\(k_1,\ldots,k_s\), gdzie \(1 \leq k_1 < \ldots < k_s \leq n\) zachodzi
\begin{equation*}
    P(A_{k_1},\ldots,A_{k_s}) = P(A_{k_1}) \ldots P(A_{k_s})\,.
\end{equation*}
\end{definition}

\begin{theorem}[O łącznym prawdopodobieństwie niezależnych zdarzeń]
Jeśli zdarzenia \(A_1,\ldots,A_n\) są niezależne to
\begin{equation*}
    P(A_1,\ldots,A_n) = P(A_1) \ldots P(A_n)\,.
\end{equation*}
\end{theorem}

\begin{definition}[Warunkowej niezależności pary zdarzeń]
Zdarzenia \(A,B\) są warunkowo niezależne względem \(C\) jeśli
\begin{equation*}
    P(A,B \mid C) = P(A \mid C) P(B \mid C)\,.
\end{equation*}
\end{definition}

\begin{definition}[Warunkowej niezależności zdarzeń]
Zdarzenia \(A_1,\ldots,A_n\) są warunkowo niezależne względem \(B\) jeśli dla dowolnych wskaźników
\(k_1,\ldots,k_s\), gdzie \(1 \leq k_1 < \ldots < k_s \leq n\) zachodzi
\begin{equation*}
    P(A_{k_1},\ldots,A_{k_s} \mid B) = P(A_{k_1} \mid B) \ldots P(A_{k_s} \mid B)\,.
\end{equation*}
\end{definition}

\begin{theorem}
Jeśli zdarzenia \(A_1,\ldots,A_n\) są warunkowo niezależne względem \(B\) to
\begin{equation*}
    P(A_1,\ldots,A_n \mid B) = P(A_1 \mid B) \ldots P(A_n \mid B)\,.
\end{equation*}
\end{theorem}

\subsection{Prawdopodobieństwo całkowite i wzór Bayesa}

\begin{definition}[Układu zupełnego zdarzeń]
Jeśli zdarzenia \(A_1,A_2\ldots,\) są parami rozłączne oraz \(\bigcup_{i=1}^\infty A_i = \Omega\) to
zbiór zdarzeń \(\{A_i\}\) nazywamy układem zupełnym.
    
\end{definition}

\begin{theorem}[O prawdopodobieństwie całkowitym]
Jeśli zdarzenia \(A_i\) (gdzie \(i\) przebiega przeliczalny zbiór wartości) tworzą układ zupełny
zdarzeń oraz \(P(A_i) > 0\) dla każdego \(i\), to dla dowolnego zdarzenia \(B\) zachodzi
\begin{equation*}
    P(B) = \sum_{i} P(B \mid A_i) P(A_i)\,.
\end{equation*}
\end{theorem}

\begin{theorem}[Bayesa]
Jeśli zdarzenia \(A_i\) spełniają założenia tw. o prawdopodobieństwie całkowitym oraz \(P(B) > 0\),
to dla każdego zdarzenia \(A_j\) z rozpatrywanego układu zdarzeń zachodzi
\begin{equation*}
    P(A_j \mid B) = \frac{P(B \mid A_j) P(A_j)}{P(B)} = \frac{P(B \mid A_j) P(A_j)}{\sum_{i} P(B \mid A_i) P(A_i)}\,.
\end{equation*}
Prawdopodobieństwa \(P(A_j)\) nazywamy prawdopodobieństwami \textbf{a priori}, a \(P(A_j \mid
B)\) -- \textbf{a posteriori}. Prawdopodobieństwo \(P(B \mid A_j)\) nazywamy
\textbf{wiarygodnością}.   
\end{theorem}

\subsection{Zmienne losowe}

\begin{definition}[Zmiennej losowej]
Niech \((\Omega, \mathcal{F}, P)\) będzie przestrzenią probabilistyczną. Niech dany będzie również
drugi zbiór \(\mathcal{X}\), w którym wyróżniamy \(\sigma\)--ciało \(\mathcal{F}_\mathcal{X}\).
Zmienną losową \(X\) nazywamy odwzorowanie
\begin{equation*}
    X: \Omega \mapsto \mathcal{X}
\end{equation*}
takie, że dla każdego \(A \in \mathcal{F}_\mathcal{X}\) zachodzi warunek
\begin{equation*}
    \{\omega \in \Omega : X(\omega) \in A \} \in \mathcal{F}\,.
\end{equation*}
\end{definition}

W szczególności jeśli \(\mathcal{X} = \mathbb{R}\) to zmienną losową \(X\) nazywamy
\textbf{zmienną losową rzeczywistą}, natomiast jeśli \(\mathcal{X} = \mathbb{R}^n\) to zmienną
losową \(\mathbf{X}\) nazywamy \textbf{zmienną losową \(n\)--wymiarową}.

\begin{definition}[Rozkładu prawdopodobieństwa zmiennej losowej]
Niech \(X: \Omega \mapsto \mathcal{X}\) będzie zmienną losową. Funkcję \(P_X:
\mathcal{F}_\mathcal{X}\mapsto [0;1]\) określoną jako
\begin{equation*}
    P_X(A) = P(\{\omega \in \Omega : X(\omega) \in A\})
\end{equation*}
nazywamy rozkładem prawdopodobieństwa zmiennej losowej \(X\).
\end{definition}

W dalszym ciągu będziemy stosować notację uproszczoną pomijając indeks dolny \(X\) tj.
\begin{equation*}
    P(X \in A) := P(\{\omega \in \Omega : X(\omega) \in A\})\,.
\end{equation*}

Znajomość rozkładu zmiennej losowej \(X\) pozwala badać własności tej zmiennej bez znajomości
przestrzeni probabilistycznej \((\Omega, \mathcal{F}, P)\), na której owa zmienna jest określona.
Możemy mianowicie zawsze rozpatrywać tę zmienną jako określoną na przestrzeni probabilistycznej
\((\mathcal{X}, \mathcal{F}_\mathcal{X}, P_X)\).

\begin{definition}[Dystrybuanty zmiennej losowej rzeczywistej]
Niech \(X: \Omega \mapsto \mathbb{R}\) będzie zmienną losową rzeczywistą. Dystrybuantą zmiennej
losowej \(X\) nazywamy funkcję \(F: \mathbb{R} \mapsto [0;1]\) zdefiniowaną jako
\begin{equation*}
    F(x) = P(X \leq x)\,.
\end{equation*}
\end{definition}

\begin{theorem}[Własności dystrybuanty zmiennej rzeczywistej]
Niech \(F\) będzie dystrybuantą zmiennej losowej rzeczywistej \(X\), wówczas
\begin{enumerate}
    
    \item Jeśli \(a < b\) to \(F(b) - F(a) = P(X \in (a;b] )\).
    
    \item Funkcja \(F\) jest niemalejąca.

    \item \(\lim_{x \to -\infty} F(x) = 0\) oraz \(\lim_{x \to +\infty} F(x) = 1\).

    \item \(F\) jest prawostronnie ciągła.

    \item \(F\) jest ciągła w \(x_0\) wtedy i tylko wtedy, gdy \(P(X = x_0) = 0\).

\end{enumerate}
\end{theorem}

\begin{definition}[Rozkładu dyskretnego zmiennej losowej rzeczywistej]
Mówimy, że zmienna losowa rzeczywista \(X\) ma rozkład dyskretny jeśli istnieje skończony lub
przeliczalny zbiór \(\mathcal{S} \subset \mathbb{R}\) taki, że
\begin{equation*}
    P( X \in \mathcal{S} ) = 1\,.
\end{equation*}
Dla takiej zmiennej określa się funkcję prawdopodobieństwa (z ang. \textit{probability mass
function}, pmf)
\begin{equation*}
    p(x) = P(X = x)\,,\quad x \in \mathcal{S}\,.
\end{equation*}
\end{definition}

Zauważmy, że jeśli zmienna \(X\) ma rozkład dyskretny to zbiór \(\mathcal{S}\) ma postać
\(\{x_1,\ldots,x_k\}\) dla pewnych \(x_i \in \mathbb{R}\).

\begin{definition}[Rozkładu ciągłego zmiennej losowej rzeczywistej]
Mówimy, że zmienna losowa rzeczywista \(X\) ma rozkład ciągły jeśli istnieje funkcja \(p: \mathbb{R}
\mapsto [0; +\infty)\) taka, że dla dowolnego przedziału \((a;b)\) zachodzi
\begin{equation*}
    P(X \in (a;b)) = \int\limits_a^b p(x) \dd{x}\,.
\end{equation*}
Funkcję \(p\) nazywamy gęstością rozkładu prawdopodobieństwa (z ang. \textit{probability density
function}, pdf).
\end{definition}

Jeśli \(X\) jest zmienną losową rzeczywistą o rozkładzie ciągłym to wartość dystrybuanty jest dana
przez
\begin{equation}
    F(x) = \int\limits_{-\infty}^x p(x') \dd{x'}\,.
\end{equation}
W szczególności dystrybuanta \(F\) jest ciągła oraz dla każdego \(x\) zachodzi \(P(X = x) = 0\).

\begin{definition}[Wartości oczekiwanej zmiennej losowej rzeczywistej]
Wartością oczekiwaną zmiennej losowej rzeczywistej \(X\) nazywamy liczbę \(m\) określoną wzorem
\begin{equation*}
    m = \int\limits_{-\infty}^{+\infty} x p(x) \dd{x}
\end{equation*}
dla rozkładu ciągłego oraz
\begin{equation*}
    m = \sum_{x \in \mathcal{S}} x p(x)
\end{equation*}
dla rozkładu dyskretnego. Stosujemy dodatkowo oznaczenie \(\mathbb{E}[X] := m\).
\end{definition}

\begin{definition}[Wariancji zmiennej losowej rzeczywistej]
Wariancją zmiennej losowej rzeczywistej \(X\) nazywamy liczbę \(\sigma\) określoną wzorem
\begin{equation*}
    \sigma^2 = \mathbb{E}[(X - m)^2]\,,
\end{equation*} 
gdzie \(m = \mathbb{E}[X]\). Stosujemy dodatkowo oznaczenie \(\mathbb{V}[X] := \sigma^2\).
\textbf{Odchyleniem standardowym} zmiennej losowej \(X\) nazywamy pierwiastek jej wariancji
\begin{equation*}
    \sigma = \sqrt{\mathbb{V}[X]}\,.
\end{equation*} 
\end{definition}

\begin{theorem}[Własności wariancji]
Niech \(X\) będzie zmienną losową rzeczywistą, wówczas
\begin{enumerate}
    
    \item \(\mathbb{V}[X] = \mathbb{E}[X^2] - (\mathbb{E}[X])^2\)\,.

    \item Jeśli zmienna \(X\) ma skończoną wariancję, to dla dowolnych \(a,b \in \mathbb{R}\)
    zachodzi
    \begin{equation*}
        \mathbb{V}[aX + b] = a^2 \mathbb{V}[X]\,.
    \end{equation*}

    \item \(\mathbb{V}[X] = 0\) wtedy i tylko wtedy, gdy istnieje \(x_0 \in \mathbb{R}\) takie, że
    \begin{equation*}
        P(X \neq x_0) = 0\,.
    \end{equation*}

\end{enumerate}
\end{theorem}

\begin{definition}[Zmiennej standaryzowanej]
Zmienną losową o wartości oczekiwanej 0 i wariancji 1 nazywamy zmienną standaryzowaną. Jeśli \(X\)
jest dowolną zmienną o niezerowej wariancji, to
\begin{equation*}
    Z := \frac{X - m}{\sigma}
\end{equation*}
jest zmienną standaryzowaną, ponieważ
\begin{equation*}
    \mathbb{E}[Z] = \frac{1}{\sigma}\mathbb{E}[X - m] = \frac{1}{\sigma}(m - m) = 0
\end{equation*}
oraz
\begin{equation*}
    \mathbb{V}[Z] = \frac{1}{\sigma^2} \mathbb{V}[X] = \frac{\sigma^2}{\sigma^2} = 1\,.
\end{equation*}
\end{definition}

\begin{theorem}[Nierówność Czebyszewa]
Jeśli zmienna losowa \(X\) ma skończoną wartość średnią \(m\) i wariancję \(\sigma^2\), to dla
dowolnego \(\epsilon > 0\) zachodzi
\begin{equation*}
    P(|X - m| \geq \epsilon\sigma) \leq \frac{1}{\epsilon^2}\,.
\end{equation*}
\end{theorem}

\begin{definition}[Kwantyla]
Kwantylem rzędu \(p \in (0;1)\) zmiennej losowej o dystrybuancie \(F\) nazywamy dowolną liczbę
\(q_p\) taką, że
\begin{equation*}
    F(q_p^-) \leq p \leq F(q_p)\,.
\end{equation*}
Kwantyl rzędu 0.5 nazywamy \textbf{medianą}, kwantyl rzędu 0.25 -- \textbf{dolnym kwartylem},
a kwantyl rzędu 0.75 -- \textbf{górnym kwartylem}. Jeśli \(X\) ma rozkład ciągły, to kwantylem
rzędu \(p\) jest dowolne rozwiązanie równania
\begin{equation*}
    F(q_p) = p\,.
\end{equation*}
\end{definition}

\begin{definition}[Mody]
Modą zmiennej losowej o rozkładzie dyskretnym nazywa się dowolne maksimum funkcji prawdopodobieństwa
tego rozkładu.

Modą zmiennej losowej o rozkładzie ciągłym nazywa się dowolne maksimum lokalne gęstości tego
rozkładu.
\end{definition}

\subsection{Ważne rozkłady jednowymiarowe}

\begin{definition}[Rozkładu jednopunktowego]
Jeśli \(X\) jest zmienną losową rzeczywistą o rozkładzie dyskretnym i \(\mathcal{S} = \{x_0\}\), to
mówimy, że \(X\) ma rozkład jednopunktowy, wówczas
\begin{equation*}
    m = x_0,\quad \sigma^2 = 0\,.
\end{equation*}
\end{definition}

\begin{definition}[Rozkładu dwupunktowego]
Jeśli \(X\) jest zmienną losową rzeczywistą o rozkładzie dyskretnym i \(\mathcal{S} = \{x_1, x_2\}\)
oraz \(p(x_1) = p\), to mówimy, że \(X\) ma rozkład dwupunktowy z parametrem \(p\), wówczas
\begin{equation*}
    m = x_1 p + x_2 (1 - p)\,,\quad \sigma^2 = p (1 - p) (x_1 - x_2)^2\,.
\end{equation*}
Jeśli \(x_1 = 1\) i \(x_2 = 0\) to taki rozkład dwupunktowy nazywamy \textbf{rozkładem
zero--jedynkowym} (lub rozkładem Bernoulliego) i oznaczamy jako \(X \sim \mathrm{Ber}(p)\).
\end{definition}

\begin{definition}[Schematu dwumianowego]
Rozważmy doświadczenie losowe o dwu możliwych wynikach: sukces osiągamy z prawdopodobieństwem \(p\),
porażkę z prawdopodobieństwem \(1-p\). Doświadczenie tego rodzaju nazywamy \textbf{próbą
Bernoulliego}. Doświadczenie takie jest modelowane zmienną losową o rozkładzie dwupunktowym z
parametrem \(p\). Schematem dwumianowym (lub schematem Bernoulliego) nazywamy doświadczenie
polegające na \(n\)--krotnym powtórzeniu próby Bernoulliego, przy założeniu, iż poszczególne próby
są od siebie niezależne.
\end{definition}

\begin{definition}[Rozkładu dwumianowego]
Niech \(X\) będzie zmienną losową taką, że \(X\) jest liczbą sukcesów w schemacie dwumianowym
długości \(n\) z prawdopodobieństwem sukcesu w każdej próbie równym \(p\). Wówczas
\begin{equation*}
    P(X = k) = {n \choose k} p^k (1 - p)^{n-k}\,.
\end{equation*}
Rozkład prawdopodobieństwa określony powyższym wzorem nazywam się rozkładem dwumianowym o
parametrach \(n,p\). Jeśli zmienna \(X\) ma rozkład dwumianowy to stosujemy notację \(X \sim
\mathrm{Bin}(n,p)\). Jeśli \(X \sim \mathrm{Bin}(n,p)\) to
\begin{equation*}
    m = np\,,\quad \sigma^2 = np(1-p)\,.
\end{equation*}
\end{definition}

\begin{definition}[Rozkładu geometrycznego]
Mówimy, że zmienna losowa \(X\) ma rozkład geometryczny z parametrem \(p \in (0;1)\), tj. \(X \sim
\mathrm{Geo}(p)\), jeśli \(\mathcal{S} = \mathbb{N} \setminus \{0\}\), a funkcja prawdopodobieństwa
ma postać
\begin{equation*}
    p(x) = (1-p)^{x-1}p\,.
\end{equation*}
Zmienna \(X\) opisuje czas oczekiwania na pierwszy sukces w schemacie dwumianowym o nieskończonej
długości. Jeśli \(X \sim \mathrm{Geo}(p)\), to
\begin{equation*}
    m = p^{-1}\,,\quad \sigma^2 = \frac{1 - p}{p^2}\,.
\end{equation*}
\end{definition}

\begin{definition}[Rozkładu Poissona]
Jeśli zmienna \(X\) o wartościach w \(\mathbb{N}\) opisuje liczbę wystąpień pewnego powtarzalnego
zdarzenia w przedziale czasowym \([0;t]\), przy czym spełnione są następujące założenia:
\begin{itemize}

    \item powtórzenia zdarzenia występują niezależnie od siebie;
    
    \item \enquote{intensywność} wystąpień \(r\) jest stała;

    \item w danej chwili (rozumianej jako odpowiednio mały przedział) może zajść co najwyżej jedno
    zdarzenie

\end{itemize}

to zmienna ta ma rozkład Poissona z parametrem \(\lambda = rt\), tj. \(X \sim
\mathrm{Pos}(\lambda)\). Jeśli \(X \sim \mathrm{Pos}(\lambda)\), to
\begin{equation*}
    P(X = k) = \frac{\e^{-\lambda} \lambda^k}{k!}\,.
\end{equation*}
Ponadto
\begin{equation*}
    m = \lambda\,,\quad \sigma^2 = \lambda\,.
\end{equation*}
\end{definition}

\begin{theorem}[Poissona]
Niech \((X_n)\) będzie ciągiem zmiennych losowych takich, że \(X_n \sim \mathrm{Bin}(n, p_n)\), gdzie
\((p_n)\) jest ciągiem takim, że
\begin{equation*}
    \lim_{n \to \infty} n p_n = \lambda
\end{equation*}
dla pewnej liczby \(\lambda > 0\). Wówczas
\begin{equation*}
    \lim_{n \to \infty} P(X_n = k) = \frac{\e^{-\lambda} \lambda^k}{k!}\,.
\end{equation*}
\end{theorem}

\begin{definition}[Rozkładu jednostajnego]
Mówimy, że zmienna \(X\) o rozkładzie ciągłym ma rozkład jednostajny na przedziale \([a;b]\) tzn.
\(X \sim \mathcal{U}(a,b)\) jeśli jej gęstość wyraża się wzorem
\begin{equation*}
    p(x) = \begin{cases}
        \frac{1}{b - a}\,, &x\in[a;b]\\
        0\,,&x\notin[a;b]
    \end{cases}\quad.
\end{equation*}
Jeśli \(X \sim \mathcal{U}(a,b)\), to
\begin{equation*}
    m = \frac{a+b}{2}\,,\quad \sigma^2 = \frac{(b-a)^2}{12}\,.
\end{equation*}
\end{definition}

\begin{definition}[Rozkładu wykładniczego]
Niech \(T\) będzie zmienną modelującą czas oczekiwania na pierwsze zdarzenie w ciągu zdarzeń takim,
że czas wystąpienia każdego z nich w przedziale \([0;t]\) jest opisany przez zmienną \(X \sim
\mathrm{Pos}(\lambda t)\). wtedy
\begin{equation*}
    P(T > t) = P(X = 0) = \e^{-\lambda t}
\end{equation*}
oraz
\begin{equation*}
    P(T > 0) = 1\,.
\end{equation*}
Mówimy wtedy, że \(T\) ma rozkład wykładniczy z parametrem \(\lambda\), tzn. \(T \sim
\mathrm{Exp}(\lambda)\). Gęstość rozkładu wykładniczego ma postać
\begin{equation*}
    p(t) = \begin{cases}
        0\,,&t \leq 0\\
        \lambda \e^{-\lambda t}\,,&t > 0
    \end{cases}\quad.
\end{equation*}
Jeśli \(T \sim \mathrm{Exp}(\lambda)\), to
\begin{equation*}
    m = \lambda^{-1}\,,\quad \sigma^2 = \lambda^{-2}\,.
\end{equation*}
\end{definition}

\begin{definition}[Rozkładu normalnego]
Mówimy, że zmienna losowa \(X\) o gęstości \(\phi(x;\mu,\sigma^2)\) ma rozkład normalny z
parametrami \(\mu \in \mathbb{R} , \sigma^2 \in [0;+\infty)\), tzn. \(X \sim
\mathcal{N}(\mu,\sigma^2)\), jeśli    
\begin{equation*}
    \phi(x; \mu, \sigma^2) = \frac{1}{\sqrt{2 \pi \sigma^2}} \exp\left(-\frac{(x-\mu)^2}{2\sigma^2}\right)\,.
\end{equation*}
Jeśli \(X \sim \mathcal{N}(\mu,\sigma^2)\), to
\begin{equation*}
    m = \mu\,,\quad \mathbb{V}[X] = \sigma^2\,.
\end{equation*}
\end{definition}

Jeśli \(X \sim \mathcal{N}(\mu,\sigma^2)\), to
\begin{equation}
    \begin{split}        
        &P(|X - m| > 2\sigma) \approx 0.0455003\,,\\
        &P(|X - m| > 3\sigma) \approx 0.0026998\,.
    \end{split}
\end{equation}
Obserwacje te nazywamy odpowiednio regułą 5\% oraz regułą \(3\sigma\). Rozkład \(\mathcal{N}(0,1)\)
nazywamy \textbf{standardowym rozkładem normalnym}. Jeśli \(X \sim \mathcal{N}(\mu,\sigma^2)\),
to zmienna standaryzowana \(Z = (X - \mu) / \sigma\) ma standardowy rozkład normalny.


\subsection{Zmienne losowe wielowymiarowe}

\begin{definition}[Dystrybuanty zmiennej losowej wielowymiarowej]
Niech \(\mathbf{X}: \Omega \mapsto \mathbb{R}^n\) będzie zmienną losową \(n\)--wymiarową.
Dystrybuantą zmiennej losowej \(\mathbf{X}\) nazywamy funkcję \(F: \mathbb{R}^n \mapsto [0;1]\)
zdefiniowaną jako
\begin{equation*}
    F(\mathbf{x}) = P(\mathbf{X}_1 \leq \mathbf{x}_1, \ldots, \mathbf{X}_n \leq \mathbf{x}_n)
\end{equation*}
\end{definition}

\begin{definition}[Rozkładu dyskretnego zmiennej losowej wielowymiarowej]
Mówimy, że \(n\)--wymiarowa zmienna losowa \(\mathbf{X}\) ma rozkład dyskretny jeśli istnieje zbiór
skończony lub przeliczalny \(\mathcal{S} \subset \mathbb{R}^n\) taki, że
\begin{equation*}
    P(\mathbf{X} \in \mathcal{S}) = 1\,.
\end{equation*}
Dla takiej zmiennej określa się funkcję prawdopodobieństwa (z ang. \textit{probability mass
function}, pmf)
\begin{equation*}
    p(\mathbf{x}) = P(\mathbf{X}_1 = \mathbf{x}_1, \ldots, \mathbf{X}_n = \mathbf{x}_n)\,,\quad \mathbf{x} \in \mathcal{S}\,.
\end{equation*}
\end{definition}

Zauważmy, że jeśli zmienna \(\mathbf{X}\) ma rozkład dyskretny to zbiór \(\mathcal{S}\) ma postać
\(\{\mathbf{x}^1,\ldots,\mathbf{x}^k\}\) dla pewnych \(\mathbf{x}^i \in \mathbb{R}\).

\begin{definition}[Rozkładu ciągłego zmiennej losowej wielowymiarowej]
Mówimy, że \(n\)--wymiarowa zmienna losowa \(\mathbf{X}\) ma rozkład ciągły jeśli istnieje funkcja
\(p: \mathbb{R}^n \mapsto [0;+\infty)\) taka, że dla dowolnych przedziałów \((a_i;b_i)\) zachodzi
\begin{equation*}
    P(\mathbf{X}_1 \in (a_1;b_1), \ldots, \mathbf{X}_n \in (a_n;b_n)) = \int\limits_{a_1}^{b_1}\ldots\int\limits_{a_n}^{b_n}p(\mathbf{x}) \dd[n]{\mathbf{x}}\,.
\end{equation*}
Funkcję \(p\) nazywamy gęstością rozkładu prawdopodobieństwa (z ang. \textit{probability density
function}, pdf).
\end{definition}

Zauważmy, że jeśli \(\mathbf{X}\) jest zmienną losową wielowymiarową o rozkładzie ciągłym to wartość
dystrybuanty jest związana z gęstością rozkładu prawdopodobieństwa poprzez
\begin{equation}
    F(\mathbf{x}) = \int\limits_{-\infty}^{\mathbf{x}_1}\ldots\int\limits_{-\infty}^{\mathbf{x}_n} p(\mathbf{x}') \dd[n]{\mathbf{x}'}\,.
\end{equation}

\begin{definition}[Wartości oczekiwanej zmiennej losowej wielowymiarowej]
Wartością oczekiwaną zmiennej losowej wielowymiarowej nazywamy \(n\)--elementowy wektor rzeczywisty
\(\mathbf{m}\), którego elementy są określone wzorem
\begin{equation*}
    \mathbf{m} = \int\limits_{\mathbb{R}^n} \mathbf{x} p(\mathbf{x}) \dd[n]{\mathbf{x}}
\end{equation*}
dla rozkładu ciągłego oraz
\begin{equation*}
    \mathbf{m} = \sum_{\mathbf{x} \in \mathcal{S}} \mathbf{x} p(\mathbf{x})
\end{equation*}
dla rozkładu dyskretnego. Stosujemy dodatkowo oznaczenie \(\mathbb{E}[\mathbf{X}] := \mathbf{m}\).
\end{definition}

\begin{definition}[Rozkładu brzegowego]
Niech \(\mathbf{X}\) będzie \(n\)--wymiarową zmienną losową. Weźmy dwa zbiory indeksów
\(\{i_1,\ldots,i_k\} \subset \{1,\ldots,n\}\) i \(\{j_1,\ldots,j_{n-k}\} = \{1,\ldots,n\} \setminus
\{i_1,\ldots,i_k\}\). \((n-k)\)--wymiarowym rozkładem brzegowym zmiennej \(\mathbf{X}\) (względem
zmiennych \(j_1,\ldots,j_{n-k}\)) nazywamy rozkład prawdopodobieństwa na przestrzeni
\(\mathbb{R}^{n-k}\) określony wzorem
\begin{equation*}
    \begin{split}
        &P_{\mathbf{X}_{j_1},\ldots,\mathbf{X}_{j_{n-k}}}(A) \\
        &= P\left(\mqty[\mathbf{X}_{j_1},\ldots,\mathbf{X}_{j_{n-k}}] \in A\right)\\
        &= P\left(\mqty[\mathbf{X}_{j_1},\ldots,\mathbf{X}_{j_{n-k}}] \in A, \mathbf{X}_{i_1} \in \mathbb{R}, \ldots, \mathbf{X}_{i_k} \in \mathbb{R}\right)
    \end{split}
\end{equation*}    
\end{definition}

Niech \(\mathbf{X}\) będzie \(n\)--wymiarową zmienną losową o dystrybuancie \(F\). Dystrybuanta
rozkładu brzegowego \(\mathbf{X}\) względem zmiennych
\(\mathbf{X}_{j_1},\ldots,\mathbf{X}_{j_{n-k}}\) spełnia równość
\begin{equation}
    F_{\mathbf{X}_{j_1},\ldots,\mathbf{X}_{j_{n-k}}}(\mathbf{x}_{j_1},\ldots,\mathbf{x}_{j_{n-k}}) = \lim_{\mathbf{x}_{i_1} \to \infty,\ldots,\mathbf{x}_{i_k} \to \infty} F(\mathbf{x}_1,\ldots,\mathbf{x}_n)\,.
\end{equation}
Dystrybuanta ta nosi nazwę \((n-k)\)--wymiarowej dystrybuanty brzegowej.

Jeśli \(\mathbf{X}\) jest \(n\)--wymiarową zmienną losową o rozkładzie ciągłym, to rozkłady brzegowe
są także rozkładami ciągłymi o gęstościach
\begin{equation}
    p_{\mathbf{X}_{j_1},\ldots,\mathbf{X}_{j_{n-k}}}(\mathbf{x}_{j_1},\ldots,\mathbf{x}_{j_{n-k}}) = \int\limits_{\mathbb{R}^k} p(\mathbf{x}') \dd{\mathbf{x}_{i_1}}\ldots\dd{\mathbf{x}_{i_k}}\,.
\end{equation}
Jeśli z kolei \(\mathbf{X}\) ma rozkład dyskretny, to rozkłady brzegowe także są rozkładami
dyskretnymi z funkcjami prawdopodobieństwa
\begin{equation}
    p_{\mathbf{X}_{j_1},\ldots,\mathbf{X}_{j_{n-k}}}(\mathbf{x}_{j_1},\ldots,\mathbf{x}_{j_{n-k}}) = \sum_{i_1,\ldots,i_k}p(\mathbf{x})\,.
\end{equation}

\begin{theorem}
Niech zmienna \(n\)--wymiarowa \(\mathbf{X}\) ma rozkład ciągły o gęstości \(p_\mathbf{X}\) i niech
\(\mathbf{Y}_i = \boldsymbol{\varphi}_i(\mathbf{X})\) dla \(i=1,\ldots,n\). Jeśli odwzorowanie
\(\boldsymbol{\varphi}\) jest różniczkowalne i odwracalne, przy czym odwzorowanie odwrotne
\(\boldsymbol{\psi} = \boldsymbol{\varphi}^{-1}\) jest różniczkowalne, to \(n\)--wymiarowa zmienna
\(\mathbf{Y}\) ma rozkład o gęstości
\begin{equation*}
    p_\mathbf{Y}(\mathbf{y}) = |J| p_\mathbf{X}(\boldsymbol{\psi}(\mathbf{y}))\,,
\end{equation*}
gdzie \(J := \det \left[\pdv{\boldsymbol{\psi}_j}{\mathbf{y}_i}\right]\) jest jakobianem
odwzorowania \(\boldsymbol{\psi}\).
\end{theorem}

\subsection{Niezależność zmiennych losowych}

\begin{definition}[Niezależności zmiennych losowych]
Niech \(X_i : \Omega \mapsto \mathcal{X}\) będą zmiennymi losowymi. Zmienne \(X_1,\ldots,X_n\)
nazywamy niezależnymi jeżeli dla dowolnych \(A_1,\ldots,A_n \in \mathcal{F}_\mathcal{X}\) zachodzi
równość
\begin{equation*}
    P(X_1 \in A_1, \ldots, X_n \in A_n) = P(X_1 \in A_1) \ldots P(X_n \in A_n)\,.
\end{equation*}
\end{definition}

\begin{theorem}
Zmienne losowe wielowymiarowe \(\mathbf{X}, \mathbf{Y}\) o rozkładzie dyskretnym lub ciągłym z
funkcją lub gęstością prawdopodobieństwa \(p(\mathbf{x},\mathbf{y})\) są niezależne wtedy i tylko
wtedy, gdy dla dowolnych \(\mathbf{x}, \mathbf{y}\) zachodzi równość
\begin{equation*}
    p(\mathbf{x}, \mathbf{y}) = p_\mathbf{X}(\mathbf{x}) p_\mathbf{Y}(\mathbf{y})\,.    
\end{equation*}
\end{theorem}

\begin{theorem}[O niezależności funkcji zmiennych losowych]
Niech \(X_1,\ldots,X_n\) będą zmiennymi losowymi o wartościach \(\mathcal{X}\), a \(g_1,\ldots,g_n:
\mathcal{X} \mapsto \mathcal{Y}\), \(\varphi: \mathcal{X}^m \mapsto \mathcal{Y}\) i \(\psi:
\mathcal{X}^{n-m}\mapsto\mathcal{Y}\) funkcjami oraz \(m < n\). Wówczas jeśli \(X_1,\ldots,X_n\) są
niezależne to:
\begin{enumerate}
    
    \item zmienne losowe \(g_1(X_1),\ldots,g_n(X_n)\) są niezależne;

    \item zmienne losowe \(\varphi(X_1,\ldots,X_m)\) i \(\psi(X_{m+1},\ldots,X_n)\) są niezależne.

\end{enumerate}
\end{theorem}

Jeśli \((\mathbf{X}, \mathbf{Y})\) jest zmienną losową o rozkładzie ciągłym z gęstością
\(p(\mathbf{x}, \mathbf{y})\) i gęstość brzegowa \(p_\mathbf{Y}\) jest funkcją dodatnią, to dla
każdego \(\mathbf{y}\) rozkład warunkowy \(\mathbf{X}\) pod warunkiem \(\mathbf{Y} = \mathbf{y}\)
także jest ciągły z gęstością
\begin{equation}
    p(\mathbf{x} \mid \mathbf{y}) = \frac{p(\mathbf{x}, \mathbf{y})}{p_\mathbf{Y}(\mathbf{y})}\,,
\end{equation}
którą nazywamy gęstością warunkową \(\mathbf{X}\) pod warunkiem \(\mathbf{Y} = \mathbf{y}\).

Jeśli \((\mathbf{X}, \mathbf{Y})\) jest zmienną losową o rozkładzie dyskretnym z funkcją
prawdopodobieństwa \(p(\mathbf{x}, \mathbf{y})\) i brzegowa funkcja prawdopodobieństwa
\(p_\mathbf{Y}\) jest funkcją dodatnią, to dla każdego \(\mathbf{y}\) rozkład warunkowy
\(\mathbf{X}\) pod warunkiem \(\mathbf{Y} = \mathbf{y}\) także jest dyskretny z funkcją
prawdopodobieństwa
\begin{equation}
    p(\mathbf{x} \mid \mathbf{y}) = \frac{p(\mathbf{x}, \mathbf{y})}{p_\mathbf{Y}(\mathbf{y})}\,,
\end{equation}
którą nazywamy warunkową funkcją prawdopodobieństwa \(\mathbf{X}\) pod warunkiem \(\mathbf{Y} =
\mathbf{y}\).

\begin{definition}[Warunkowej wartości oczekiwanej]
Warunkową wartość oczekiwaną \(\mathbf{X}\) pod warunkiem \(\mathbf{Y} = \mathbf{y}\) nazywamy
wielkość
\begin{equation*}
    \mathbb{E}[\mathbf{X} \mid \mathbf{y}] = \int\limits_{\mathbb{R}^n} \mathbf{x} p(\mathbf{x} \mid \mathbf{y}) \dd[n]{\mathbf{x}}
\end{equation*}
\end{definition}

\begin{definition}[Warunkowej niezależności]
Niech \(X_i : \Omega \mapsto \mathcal{X}\) oraz \(Y\) będą zmiennymi losowymi. Mówimy, że zmienne
\(X_1,\ldots,X_n\) są warunkowo niezależne względem \(Y\), jeśli dla dowolnych \(A_i \in
\mathcal{F}_\mathcal{X}\) i dowolnej wartości \(y\) zachodzi
\begin{equation*}
    \begin{split}
        &P(X_1\in A_1,\ldots,X_n\in A_n \mid Y = y) \\
        &= P(X_1 \in A_1 \mid Y = y)\ldots P(X_n \in A_n \mid Y = y)\,.
    \end{split}
\end{equation*}  
\end{definition}

Dla rozkładów ciągłych lub dyskretnych z gęstością lub funkcją prawdopodobieństwa \(p\) warunkowa
niezależność \(\mathbf{X}, \mathbf{Y}\) względem \(\mathbf{Z}\) jest równoważna
\begin{equation*}
    p(\mathbf{x},\mathbf{y} \mid \mathbf{z}) = p_\mathbf{X}(\mathbf{x} \mid \mathbf{z})p_\mathbf{Y}(\mathbf{y} \mid \mathbf{z})\,.
\end{equation*}

\subsection{Kowariancja i współczynnik korelacji}

\begin{definition}[Kowariancji]
Kowariancją zmiennych losowych rzeczywistych \(X, Y\) o nazywamy liczbę
\begin{equation*}
    \mathrm{Cov}(X, Y) = \mathbb{E}[(X - m_X)(Y - m_Y)]\,,
\end{equation*}
gdzie wartości oczekiwane są liczone względem łącznego rozkładu prawdopodobieństwa \(p(x,y)\).
\textbf{Współczynnikiem korelacji zmiennych \(X, Y\)} nazywamy liczbę
\begin{equation*}
    \rho_{X,Y} = \frac{\mathrm{Cov}(X, Y)}{\sigma_X \sigma_Y}\,.
\end{equation*}
\end{definition}

\begin{theorem}[Własności kowariancji]
Niech \(X, Y\) będą zmiennymi losowymi rzeczywistymi, wówczas

\begin{enumerate}

    \item \(\mathrm{Cov}(X, Y) = \mathbb{E}[XY] - m_X m_Y\).
    
    \item Jeśli \(X\) i \(Y\) są niezależne oraz istnieje \(\mathbb{E}(XY)\), to \(\mathrm{Cov}(X,Y)
    = 0\).
\end{enumerate}
\end{theorem}

\begin{definition}[Macierzy kowariancji]
Macierzą kowariancji \(\mathbf{K}\) \(n\)--wymiarowej zmiennej losowej \(\mathbf{X}\) nazywamy
macierz \(n \times n\) zdefiniowaną jako
\begin{equation*}
    \mathbf{K} = \mathbb{E}[(\mathbf{X} - \mathbf{m})(\mathbf{X} - \mathbf{m})^T]
\end{equation*}
lub równoważnie
\begin{equation*}
    \mathbf{K}_{ij} = \begin{cases}
        \mathbb{V}[\mathbf{X}_i]\,,&i=j\\
        \mathrm{Cov}(\mathbf{X}_i, \mathbf{X}_j)\,,&i\neq j
    \end{cases}\quad.
\end{equation*}
\end{definition}

\begin{theorem}[Własności macierzy kowariancji]
Niech \(\mathbf{K}\) będzie macierzą kowariancji zmiennej \(\mathbf{X}\), wówczas
\begin{enumerate}

    \item Macierz kowariancji jest symetryczna.

    \item Macierz kowariancji jest nieujemnie określona.

    \item Jeśli zmienne \(\mathbf{X}_1,\ldots,\mathbf{X}_n\) są niezależne, to macierz kowariancji
    jest diagonalna.
    
\end{enumerate}
\end{theorem}

\subsection{Wielowymiarowy rozkład normalny}

\begin{definition}[Standardowego wielowymiarowego rozkładu normalnego]
Zmienna losowa \(\mathbf{X}\) ma standardowy \(n\)--wymiarowy rozkład normalny jeśli jej składowe są
niezależne i dla każdego \(i=1,\ldots,n\) \(\mathbf{X}_i \sim \mathcal{N}(0,1)\). Jest to rozkład
ciągły o gęstości
\begin{equation*}
    p(\mathbf{x}) = \frac{1}{\sqrt{(2\pi)^n}}\exp\left(-\frac{1}{2}\mathbf{x}^T\mathbf{x}\right)\,.
\end{equation*}
\end{definition}

\begin{definition}[Wielowymiarowego rozkładu normalnego]
Zmienna losowa \(\mathbf{X}\) ma \(n\)--wymiarowy rozkład normalny (z ang. \textit{Multivariate
Normal Distribution}, MVN), tzn. \(\mathbf{X} \sim \mathcal{N}(\boldsymbol{\mu},
\boldsymbol{\Sigma})\) jeśli istnieje \(k\)--wymiarowa zmienna losowa \(\mathbf{Z}\) o standardowym
rozkładzie normalnym dla pewnego \(k \leq n\) oraz istnieje \(\boldsymbol{\mu} \in \mathbb{R}^n\) i
macierz \(\mathbf{A} \in \mathbb{R}^{n \times k}\) takie, że \(\boldsymbol{\Sigma} =
\mathbf{A}\mathbf{A}^T\) oraz
\begin{equation*}
    \mathbf{X} = \mathbf{A}\mathbf{Z} + \boldsymbol{\mu}\,.
\end{equation*}
Jeśli macierz \(\boldsymbol{\Sigma}\) jest dodatnio określona, to rozkład
\(\mathcal{N}(\boldsymbol{\mu}, \boldsymbol{\Sigma})\) jest ciągły, a jego gęstość jest dana przez
\begin{equation*}
    p(\mathbf{x}) = \frac{1}{\sqrt{(2\pi)^n\det\boldsymbol{\Sigma}}}\exp\left(-\frac{1}{2}(\mathbf{x}-\boldsymbol{\mu})^T\boldsymbol{\Sigma}^{-1}(\mathbf{x}-\boldsymbol{\mu})\right)
\end{equation*}
Macierz \(\boldsymbol{\Sigma}^{-1}\) nazywa się \textbf{macierzą precyzji}.
\end{definition}

Poziomice gęstości niezdegenerowanego wielowymiarowego rozkładu normalnego są elipsoidami, których
półosie są skierowane wzdłuż wektorów własnych macierzy \(\boldsymbol{\Sigma}\) i mają długości
proporcjonalne do pierwiastka z wartości własnych.

\begin{theorem}[Własności niezdegenerowanego rozkładu normalnego]
Niech \(\mathbf{X} \sim \mathcal{N}(\boldsymbol{\mu}, \boldsymbol{\Sigma})\) dla dodatnio określonej
macierzy \(\boldsymbol{\Sigma}\), wówczas
\begin{enumerate}

    \item Wszystkie rozkłady brzegowe i warunkowe \(\mathbf{X}\) są rozkładami normalnymi.

    \item Zmienne składowe \(\mathbf{X}_1,\ldots,\mathbf{X}_n\) są niezależne wtedy i tylko wtedy,
    gdy \(\boldsymbol{\Sigma}\) jest macierzą diagonalną.
    
\end{enumerate}
\end{theorem}

\subsection{Rodzaje zbieżności zmiennych losowych}

Niech \(X_1, X_2, \ldots\) i \(X\) będą zmiennymi losowymi o wartościach w tej samej przestrzeni
\(\mathcal{X}\).

\begin{definition}[Zbieżności stochastycznej]
Ciag \((X_n)\) jest zbieżny do \(X\) według prawdopodobieństwa (lub stochastycznie) jeśli dla
każdego \(\epsilon > 0\)
\begin{equation*}
    \lim_{n \to \infty} P(|X_n - X| < \epsilon) = 1\,.
\end{equation*}
\end{definition}

\begin{definition}[Zbieżności z prawdopodobieństwem 1]
Ciąg \((X_n)\) jest zbieżny do \(X\) z prawdopodobieństwem 1 jeśli
\begin{equation*}
    P\left(\lim_{n \to \infty} X_n = X\right) = 1\,.
\end{equation*}
\end{definition}

\begin{definition}[Zbieżności według dystrybuant]
Załóżmy, że \(\mathcal{X} = \mathbb{R}^k\) i niech \(F_{\mathbf{X}^n}\) będzie dystrybuantą
\(\mathbf{X}^n\), \(F_\mathbf{X}\) dystrybuantą \(\mathbf{X}\). Ciąg \((\mathbf{X}^n)\) jest zbieżny
do \(\mathbf{X}\) według dystrybuant jeśli dla każdego \(\mathbf{x} \in \mathbb{R}^k\) takiego, że
\(F_\mathbf{X}\) jest ciągła w \(\mathbf{x}\) zachodzi
\begin{equation*}
    \lim_{n \to \infty} F_{\mathbf{X}^n}(\mathbf{x}) = F_\mathbf{X}(\mathbf{x})\,.
\end{equation*}
\end{definition}

\begin{theorem}[Zależności między rodzajami zbieżności]
Niech \((X_n)\) będzie ciągiem zmiennych losowych. Wówczas
\begin{enumerate}

    \item Jeśli \(X_n \to X\) z prawdopodobieństwem 1, to \(X_n \to X\) według prawdopodobieństwa.

    \item Jeśli \(X_n \to X\) według prawdopodobieństwa, to istnieje podciąg \(X_{n_k}\) taki, że
    \(X_{n_k} \to X\) z prawdopodobieństwem 1.

    \item Jeśli \(X_n \to X\) według prawdopodobieństwa, to \(X_n \to X\) według dystrybuant.

\end{enumerate}
\end{theorem}

\subsection{Wnioskowanie statystyczne}

Niech zmienna losowa \(X\) określa model rozkładu pewnej cechy (cech) w ustalonym zbiorze instancji
(tzw. \textbf{populacji generalnej}). Innymi słowy, przyjmujemy, że wartości cech zachowują się
jakby zostały wybrane losowo zgodnie z rozkładem zmiennej \(X\). Do podstawowych zagadnień
wnioskowania statystycznego należą:
\begin{itemize}

    \item oszacowanie wielkości charakteryzujących rozkład \(X\) (np. wartości średniej albo
    wariancji);

    \item weryfikacja hipotez dotyczących rozkładu \(X\) (tym nie będziemy się zajmować).
    
\end{itemize}

\begin{definition}[Modelu statystycznego]
Modelem statystycznym nazywamy parę \((\mathcal{P}, \mathcal{X})\), gdzie \(\mathcal{P}\) jest
rodziną rozkładów prawdopodobieństwa na zbiorze \(\mathcal{X}\). Zazwyczaj przyjmuje się
\begin{equation*}
    \mathcal{P} = \{P_{\boldsymbol{\theta}} : \boldsymbol{\theta} \in \Theta \} 
\end{equation*}
dla pewnego zbioru parametrów \(\Theta\). Model statystyczny nazywamy \textbf{parametrycznym} jeśli
\(\Theta \subset \mathbb{R}^k\).
\end{definition}

\begin{definition}[Prostej próby losowej]
Prostą próbą losową o liczności \(n\) nazywamy ciąg niezależnych zmiennych losowych
\(X_1,\ldots,X_n\) o wartościach w \(\mathcal{X}\) i o tym samym rozkładzie
\(P_{\boldsymbol{\theta}} \in \mathcal{P}\) (z ang. \textit{independent and identically distributed,
i.i.d}).
\end{definition}

\begin{definition}[Realizacji prostej próby losowej]
Ciąg wartości \(x_1,\ldots,x_n \in \mathcal{X}\) takich, żeby
\begin{equation*}
    X_1(\omega) = x_1,\ldots,X_n(\omega) = x_n
\end{equation*}
dla pewnego \(\omega\) nazywamy realizacją prostej próby losowej \(X_1,\ldots,X_n\).
\end{definition}

\begin{definition}[Statystyki]
Statystyką nazywa się zmienną losową będącą funkcją prostej próby losowej \(T(X_1,\ldots,X_n)\).
\end{definition}
Średnią w prostej próbie losowej nazywa się statystykę
\begin{equation}
    \overline{X} = \frac{1}{n}\sum_{i=1}^n X_i\,.
\end{equation}
Wariancją w prostej próbie losowej nazywa się statystykę
\begin{equation}
    S^2 = \frac{1}{n - 1}\sum_{i=1}^n (X_i - \overline{X})^2\,.
\end{equation}

\subsection{Prawa wielkich liczb i CTG}

\begin{definition}
Mówimy, że dla ciągu zmiennych losowych \(X_1,X_2,\ldots\) zachodzi \textbf{słabe prawo wielkich
liczb} jeżeli
\begin{equation*}
    \frac{1}{n} \sum_{k=1}^n\left(X_k - \mathbb{E}[X_k]\right) \to 0
\end{equation*}
według prawdopodobieństwa. Jeżeli powyższy ciąg jest zbieżny z prawdopodobieństwem 1, to mówimy, że
dla ciągu \((X_n)\) zachodzi \textbf{mocne prawo wielkich liczb}.
\end{definition}

\begin{theorem}
Słabe prawo wielkich liczb zachodzi, jeśli ciąg \((X_n)\) spełnia \textbf{jeden} z podanych
warunków:
\begin{itemize}

    \item \(\frac{1}{n^2}\mathbb{V}\left[\sum_{k=1}^n X_k\right] \to 0\) (prawo wielkich liczb
    Markowa).

    \item \(X_n\) są niezależne i ich wariancje są wspólnie ograniczone (prawo wielkich liczb
    Czebyszewa).

    \item \(X_n\) są niezależne i mają ten sam rozkład o skończonej wartości średniej (prawo
    wielkich liczb Chinczyna).
    
\end{itemize}
\end{theorem}

\begin{theorem}[Prawo wielkich liczb Kołmogorowa]
Jeśli \(X_n\) są niezależne, mają skończone wariancje oraz
\begin{equation*}
    \sum_{n=1}^\infty \frac{\mathbb{V}[X_n]}{n^2} < +\infty\,,
\end{equation*}
to dla \(X_n\) zachodzi mocne prawo wielkich liczb (prawo wielkich liczb Kołmogorowa).
\end{theorem}

\begin{theorem}[Centralne Twierdzenie Graniczne]
Niech \(X_n\) będzie ciągiem niezależnych zmiennych losowych o tym samym rozkładzie ze skończoną
wartością średnią \(m\) i wariancją \(\sigma^2 > 0\). Wtedy ciąg \(Z_n\)
\begin{equation*}
    Z_n := \frac{\frac{1}{n}\sum_{k=1}^n X_k}{\frac{\sigma}{\sqrt{n}}}
\end{equation*}
jest zbieżny według dystrybuant do zmiennej losowej \(Z\) o standardowym rozkładzie normalnym.
\end{theorem}

\subsection{Estymatory punktowe}

\begin{definition}[Estymatora]
Estymatorem nazywa się statystykę \(\hat{\theta}(X_1,\ldots,X_n)\) służącą do oszacowania wartości
parametru \(\theta\). Liczbę \(\hat{\theta}(x_1,\ldots,x_n)\) dla konkretnej realizacji prostej
próby losowej nazywa się wartością estymatora albo estymatą.
\end{definition}

\begin{definition}[Obciążenia estymatora]
Liczbę
\begin{equation*}
    \mathrm{B}(\hat{\theta}) = \mathbb{E}[\hat{\theta}(X_1,\ldots,X_n)] - \theta
\end{equation*}
nazywamy obciążeniem estymatora \(\hat\theta\). Jeśli \(\mathrm{B}(\hat{\theta}) = 0\) to estymator
nazywamy \textbf{nieobciążonym}.
\end{definition}

\begin{definition}[Błędu średniokwadratowego]
Błędem średniokwadratowym (z ang. \textit{Mean Squared Error, MSE}) estymatora \(\hat{\theta}\)
parametru \(\theta\) nazywamy liczbę
\begin{equation*}
    \mathrm{MSE}(\hat{\theta}) = \mathbb{E}[(\hat{\theta} - \theta)^2]\,.
\end{equation*}
\end{definition}

\begin{theorem}
Dla dowolnego estymatora \(\hat{\theta}\) zachodzi
\begin{equation*}
    \mathrm{MSE}(\hat{\theta}) = \mathbb{V}[\hat{\theta}] + (\mathrm{B}(\hat{\theta}))^2\,.
\end{equation*}
\end{theorem}

\begin{definition}[Funkcji wiarygodności]
Funkcją wiarygodności (z ang. \textit{likelihood function}) dla modelu \(\mathcal{P} =
\{P_{\boldsymbol{\theta}} : \boldsymbol{\theta} \in \Theta\}\) nazywamy funkcję
\begin{equation*}
    \mathcal{L}: \mathbb{R}^n \times \Theta \ni (\mathbf{x},\boldsymbol{\theta}) \mapsto \mathcal{L}(\mathbf{x};\boldsymbol{\theta}) \in [0; +\infty)
\end{equation*}
wyznaczającą rozkład łączny obserwowanych danych jako funkcję parametru \(\boldsymbol{\theta}\).    
\end{definition}
Niech \(\mathbf{X}^1,\ldots,\mathbf{X}^n\) będzie prostą próbą losową. Jeśli
\(P_{\boldsymbol{\theta}}\) są rozkładami ciągłymi lub dyskretnymi o gęstościach lub funkcjach
prawdopodobieństwa \(p(\cdot;\boldsymbol{\theta})\), to
\begin{equation}
    \mathcal{L}(\mathbf{x}^1,\ldots,\mathbf{x}^n;\boldsymbol{\theta}) = \prod_{i=1}^n p(\mathbf{x}^i;\boldsymbol{\theta})\,.
\end{equation}
Dla wygody obliczeń często rozważa się tzw. zanegowaną logarytmiczną funkcję wiarygodności (z ang.
\textit{Negated Log-Likelihood function, NLL}), tzn.
\begin{equation}
    L(\mathbf{x};\boldsymbol{\theta}) = - \log \mathcal{L}(\mathbf{x};\boldsymbol{\theta})\,.
\end{equation}
Wówczas dla realizacji prostej próby losowej mamy
\begin{equation}
    L(\mathbf{x}^1,\ldots,\mathbf{x}^n;\boldsymbol{\theta}) = -\sum_{i=1}^n \log p(\mathbf{x}^i;\boldsymbol{\theta})\,.
\end{equation}

\begin{definition}[Estymatora największej wiarygodności]
Estymatorem największej wiarygodności (z ang. \textit{Maximum Likelihood Estimator, MLE}) nazywamy
funkcję \(\boldsymbol{\hat{\theta}}\), która przy ustalonych wartościach obserwacji (realizacji
prostej próby losowej) \(\{\mathbf{x}^1,\ldots,\mathbf{x}^n\}\) maksymalizuje wartość funkcji
wiarygodności lub, co równoważne, minimalizuje wartość zanegowanej logarytmicznej funkcji
wiarygodności tj.
\begin{equation*}
    \boldsymbol{\hat \theta}(\mathbf{x}^1,\ldots,\mathbf{x}^n) = \arg \min_{\boldsymbol{\theta} \in \Theta} \left[- \sum_{i=1}^n \log p(\mathbf{x}^i ; \boldsymbol{\theta})\right]\,.
\end{equation*}
\end{definition}
    
Jeśli funkcja wiarygodności jest różniczkowalna względem \(\boldsymbol{\theta}\) dla dowolnych
\(\mathbf{x}^i\), to MLE można czasem wyznaczyć analitycznie korzystając z warunku koniecznego
optymalności, tzn. rozwiązując układ równań
\begin{equation}
    \pdv{L(\mathbf{x}^1,\ldots,\mathbf{x}^n; \boldsymbol{\theta})}{\boldsymbol{\theta}} = 0\,.
\end{equation}
Jeśli MLE nie da się wyliczyć analitycznie, wyznacza się je przy użyciu algorytmów optymalizacji
numerycznej. Estymatory MLE są asymptotycznie nieobciążone.

\section{Elementy statystycznego uczenia maszynowego}

% 1.  Podstawy wnioskowania Bayesowskiego; Bayesowski wybór modeli
% 2.  Modele Gaussowskie; Liniowe modele Gaussowskie
% 3.  Bayesowska regresja linowa
% 4.  Regularyzacja
% 5.  Robust regression
% 6.  Procesy Gaussowskie
% 7.  Estymator jądrowy gęstości
% 8.  Wieloklasowa regresja logistyczna
% 9.  Metryki do oceny klasyfikacji
% 10. Klasyfikacja ekstremalnie niezbalansowana; Anomaly detection 
% 11. Wnioskowanie metodami Monte Carlo: Gibbs sampling, Importance sampling, Metropolis-Hastings
% 12. Metody oparte o sąsiedztwo
% 13. Naiwny klasyfikator bayesowski


\section{Uczenie głębokie i sieci neuronowe}

% 1.  Matematyczny model neuronu; Perceptron wielowarstwowy
% 2.  Universal Approximation Theorem
% 3.  Loss functions: MLE, univariate regression, multiclass classification, cross-entropy loss
% 4.  Aspekty optymalizacyjne (gradient descent, stochastic gradient descent, momentum, adam)
% 5.  Algorytm wstecznej propagacji błędu, automatyczne różniczkowanie
% 6.  Regularyzacja i inicjalizacja w sieciach głębokich
% 7.  Measuring performance
% 8.  Convolutional networks
% 9.  Residual networks
% 10. Transformers
% 11. Intro to deep unsupervised learning
% 12. Energy Based Models; RBMs; algorytm (Persistent) Contrastive Divergence
% 13. Diffusion models
    
\end{document}