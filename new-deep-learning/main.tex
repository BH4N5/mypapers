\documentclass{myclass}
\usepackage[polish]{babel}

\numberwithin{equation}{section}

\begin{document}

\tableofcontents
\newpage

\section{Wstęp}

Celem tych notatek jest zwięzłe przedstawienie kompletu zagadnień związanych z szeroko pojętym
uczeniem głębokim jako podejściem do Sztucznej Inteligencji (SI). Zaczynamy od minimalnego zbioru
wymaganych tematów z zakresu rachunku prawdopodobieństwa i statystyki matematycznej. Następnie
opisujemy podstawowe metody uczenia maszynowego z probabilistycznego punktu widzenia. W końcu
przechodzimy do zasadniczej części związanej z uczeniem głębokim i sieciami neuronowymi. W każdej
części staramy się przedstawiać opisywane tematy w sposób minimalistyczny, skupiając się głównie na
matematycznej i ideowej, a nie implementacyjnej stronie zagadnień. Liczymy, iż takie podejście
zapewni odpowiednio głębokie zrozumienie tematu, dzięki któremu dalsze studiowanie całej gamy
specyficznych technicznych tematów nie sprawi żadnego problemu.


\subsection{Notacja}

W dalszej części tekstu będziemy stosować przedstawioną tutaj pokrótce notację. Wektory, które
traktujemy jako elementy przestrzeni \(\mathbb{R}^d\) ze standardowo zdefiniowanymi mathbfacjami
dodawania i mnożenia przez skalar będziemy oznaczali wytłuszczonymi małymi literami np.
\(\mathbf{x}, \mathbf{w}, \boldsymbol{\phi}\). Wielkość \(\mathbf{x}_i\) będzie oznaczać dany
element wektora (w tym przypadku \(i\)--ty element \(\mathbf{x}\)). Wielkość \(\mathbf{x}^\mu\)
będzie oznaczać pewien (w tym przypadku \(\mu\)--ty) element pewnego zbioru wektorów. Macierze oraz
wielowymiarowe tablice (zwane również niefortunnie tensorami) będziemy oznaczać wytłuszczonymi
wielkimi literami np. \(\mathbf{X}, \mathbf{W}, \boldsymbol{\Phi}\). Analogicznie jak w przypadku
wektorów przez \(\mathbf{X}_{i_1 i_2 \ldots i_k}\) będziemy oznaczać \((i_1,i_2,\ldots,i_k)\)
element \(k\)--wymiarowej tablicy \(\mathbf{X}\), natomiast \(\mathbf{X}^\mu\) będzie oznaczać
\(\mu\)--ty element pewnego zbioru tablic.


\subsection{Uczenie nadzorowane}

Uczenie nadzorowane jest jednym z dwóch podstawowych (pomijając tzw. uczenie ze wzmocnieniem)
paradygmatów w uczeniu maszynowym, którego ogólną ideą jest zdefiniowanie pewnego modelu
odwzorowującego dane wejściowe na wyjściowe predykcje. Zakładamy w nim, iż mamy dostępny zbiór
obserwacji w postaci uporządkowanych par \(\mathcal{X} = \{(\mathbf{x}^\mu, y_\mu)\}_{\mu=1}^n\),
gdzie \(\mathbf{x} \in \mathbb{R}^d\) nazywamy wektorem cech a \(y\) jest prawidłową wartością
odpowiedzi dla tych cech. Dwa najbardziej podstawowe przypadki zagadnień tego rodzaju to regresja
oraz klasyfikacja. W przypadku regresji zmienna \(y\) przyjmuje wartości z przedziału liczb
rzeczywistych. W przypadku klasyfikacji zmienna \(y\) przyjmuje wartości ze skończonego zbioru
kategorii, przy czym wartości z tego zbioru nie powinny posiadać naturalnej tj. wynikającej z natury
problemu, relacji porządku; gdy tak nie jest mamy do czynienia z regresją/klasyfikacją porządkową (z
ang. \textit{ordinal regression/classification}).

W jaki sposób tworzymy wspomniany model odwzorowujący \(\mathbf{x}\) na \(y\)? W dalszych
paragrafach poznamy różne metody, ale najczęściej (nie wchodząc teraz w modelowanie
probabilistyczne) modelem jest pewna rodzina funkcji postaci \(\hat{y}(\mathbf{x}; \mathbf{w})\)
parametryzowana skończoną liczbą parametrów, które możemy łącznie zapisać jako pewien wektor
\(\mathbf{w} \in \mathbb{R}^m\). Aby znaleźć parametry \(\mathbf{w}\), dzięki którym dla konkretnego
zagadnienia model będzie zadowalająco odwzorowywał cechy na predykcje (innymi słowy aby nauczyć
model) wprowadzamy dodatkowo funkcjonał kosztu (z ang. \textit{loss function})
\(\ell[\hat{y}(\mathbf{x};\mathbf{w}), y]\), który kwantyfikuje odpowiedzi modelu \(\phi\) w
stosunku do znanych prawidłowych odpowiedzi \(y\). Trening modelu polega wówczas na znalezieniu
parametrów \(\mathbf{w}^*\), które minimalizują średnią z wartości funkcji kosztu dla przykładów w
zbiorze treningowym \(\mathcal{X}\)
\begin{equation}
    \mathbf{w}^* = \arg\min_{\mathbf{w}} \frac{1}{n}L(\mathcal{X};\mathbf{w}) = \arg\min_{\mathbf{w}} \frac{1}{n}\sum_{\mu=1}^n \ell[\hat{y}(\mathbf{x}^\mu;\mathbf{w}), y_\mu]\,.
\end{equation}
Zauważmy, że takie podejście ma jedną zasadniczą wadę -- istotnie nie interesuje nas tak naprawdę,
jak model radzi sobie na zbiorze treningowym, tylko jak będzie radził sobie na nowych, niewidzianych
wcześniej danych. Sytuację, w której model bardzo dobrze modeluje dane w zbiorze treningowym, ale
słabo radzi sobie na nowych danych nazywamy przeuczeniem lub nadmiernym dopasowaniem (z ang.
\textit{overfitting}). Sytuację, w której model słabo radzi sobie zarówno na zbiorze treningowym,
jak i na nowych danych nazywamy niedouczeniem lub niedopasowaniem (z ang. \textit{underfitting}).
Występowanie overfittingu i underfittingu jest powiązane z pojemnością (z ang. \textit{capacity})
modelu. Złożony model o dużej pojemności potrafi dopasować się do bardzo skomplikowanych obserwacji,
ale istnieje ryzyko jego przeuczenia (mówimy wówczas o \textit{high variance}). Dla prostego modelu
o małej pojemności istnieje z kolei ryzyko, iż nie ma on wystarczająco ekspresywności (mówimy
wówczas o \textit{high bias}).
 

\subsection{Uczenie nienadzorowane}

W przypadku uczenia nienadzorowanego naszym celem nie jest znalezienie modelu odwzorowującego cechy
na predykcje. Chcemy raczej zrozumieć wewnętrzną strukturę danych. Modele tego rodzaju znajdują
zastosowanie w analizie biznesowej, gdzie pozwalają, chociażby na analizę ważności poszczególnych
wskaźników, czy wizualizację wysoko-wymiarowych danych. Nas będą interesować natomiast szczególnie
generatywne modele nienadzorowane, za pomocą których modelujemy rozkład prawdopodobieństwa danych,
czy to wprost poprzez funkcję gęstości prawdopodobieństwa, czy też jedynie jako model, z którego
możemy próbkować nowe przykłady.


\subsection{Praktyka uczenia maszynowego}

W dalszej części nie będziemy skupiać się na detalach implementacyjnych. W naszej ocenie jednak
dziedzina uczenia maszynowego jest przede wszystkim \enquote{nauką eksperymentalną} dlatego bardzo
ważne jest empiryczne sprawdzenie różnych metod przed wybraniem ostatecznego modelu dla danego
zagadnienia. W tym paragrafie opisujemy standardowe praktyki stosowane przy treningu modeli uczenia
maszynowego (głównie nadzorowanych mathbfujących na danych tabelarycznych).


\subsubsection{Przygotowanie danych}

Kluczem do uzyskania dobrych wyników przy korzystaniu z algorytmów uczenia maszynowego jest
odpowiednie przygotowanie danych (z ang. \textit{preprocessing}). Typowo preprocessing składa się z:

\begin{itemize}
   
    \item wczytania danych;
    
    \item eksploracji danych oraz wstępnego czyszczenia, w szczególności usunięcia jawnych wartości
    odstających (z ang. \textit{outliers}) oraz cech posiadających zbyt dużo wartości brakujących;

    \item analizy rozkładu zmiennej docelowej oraz ewentualnej transformacji logarytmicznej, która
    poprawia stabilność numeryczną, gdy przewidywane wartości są dużymi dodatnimi liczbami
    rzeczywistymi, zmienia dziedzinę zmiennej objaśnianej z \(\mathbb{R}_+\) na \(\mathbb{R}\) oraz
    dodatkowo jest przykładem transformacji stabilizującej wariancję;
    
    \item \textbf{podziału zbioru na część treningową oraz testową};

    \item dokonania skalowania i imputacji brakujących wartości cech (metody \texttt{.fit()}
    wywołujemy jedynie dla zbioru treningowego);

    \item usunięcia silnie skorelowanych cech;

    \item zakodowania wartości kategorycznych za pomocą tzw. \textit{one--hot encoding} pamiętając o
    \textit{dummy variable trap} -- jedną z \(k\) kategorii kodujemy za pomocą wektora
    \textit{one--hot} długości \(n-1\), aby uniknąć zależności liniowej między cechami (opcja
    \texttt{drop="first"} w \texttt{OneHotEncoder} w scikit-learn);

    \item wykonania feature engineering -- dodania wielomianów cech do naszych danych lub
    skonstruowania innych cech (np. cech określających miesiąc, dzień itp.);

\end{itemize}

Podział zbioru na część treningową i testową jest najważniejszym etapem preprocessingu. Zbiór
testowy wydzielamy, aby po wytrenowaniu modelu sprawdzić, jak poradzi on sobie na nowych,
niewidzianych wcześniej danych. Powinniśmy go traktować jako dane, które będziemy w przyszłości
dostawać po wdrożeniu modelu do realnego systemu. Takie dane również będziemy musieli przeskalować,
zakodować itp., ale parametry potrzebne do wykonania tych transformacji możemy wziąć jedynie z
dostępnego wcześniej zbioru treningowego. Wykorzystanie danych testowych w procesie treningu to
\textbf{błąd wycieku danych} (z ang. \textit{data leakage}). Skutkuje on niepoprawnym, nadmiernie
optymistycznym oszacowaniem jakości modelu.

\subsubsection{Tuning hiperparametrów i walidacja skrośna}

Praktycznie wszystkie modele uczenia maszynowego mają hiperparametry, często liczne, które w
zauważalny sposób wpływają na wyniki, a szczególnie na underfitting i overfitting. Ich wartości
trzeba dobrać zatem dość dokładnie. Proces doboru hiperparametrów nazywa się tuningiem
hiperparametrów (z ang. \textit{hyperparameter tuning}).

Istnieje na to wiele sposobów. Większość z nich polega na tym, że trenuje się za każdym razem model
z nowym zestawem hiperparametrów i wybiera się ten zestaw, który pozwala uzyskać najlepsze wyniki.
Metody głównie różnią się między sobą sposobem doboru kandydujących zestawów hiperparametrów.
Najprostsze i najpopularniejsze to:
\begin{itemize}

    \item pełne przeszukiwanie (z ang. \textit{grid search}) -- definiujemy możliwe wartości dla
    różnych hiperparametrów, a metoda sprawdza ich wszystkie możliwe kombinacje (czyli siatkę),

    \item losowe przeszukiwanie (z ang. \textit{randomized search}) -- definiujemy możliwe wartości
    jak w pełnym przeszukiwaniu, ale sprawdzamy tylko ograniczoną liczbę losowo wybranych
    kombinacji.

\end{itemize}

Jak ocenić, jak dobry jest jakiś zestaw hiperparametrów? Nie możemy sprawdzić tego na zbiorze
treningowym -- wyniki byłyby zbyt optymistyczne. Nie możemy wykorzystać zbioru testowego --
mielibyśmy wyciek danych, bo wybieralibyśmy model explicite pod nasz zbiór testowy. Trzeba zatem
osobnego zbioru, na którym będziemy na bieżąco sprawdzać jakość modeli dla różnych hiperparametrów.
Jest to zbiór walidacyjny (z ang. \textit{validation set}). Zbiór taki wycina się ze zbioru
treningowego.

Jednorazowy podział zbioru na części nazywa się \textit{split validation} lub \textit{holdout}.
Używamy go, gdy mamy sporo danych, i 10-20\% zbioru jako dane walidacyjne czy testowe to dość dużo,
żeby mieć przyzwoite oszacowanie. Zbyt mały zbiór walidacyjny czy testowy da nam mało wiarygodne
wyniki -- nie da się nawet powiedzieć, czy zbyt pesymistyczne, czy optymistyczne. W praktyce
niestety często mamy mało danych. Trzeba zatem jakiejś magicznej metody, która stworzy nam więcej
zbiorów walidacyjnych z tej samej ilości danych. Taką metodą jest walidacja skrośna (z ang.
\textit{cross validation}, CV). Polega na tym, że dzielimy zbiór treningowy na \(K\) równych
podzbiorów, tzw. foldów. Każdy podzbiór po kolei staje się zbiorem walidacyjnym, a pozostałe łączymy
w zbiór treningowy. Trenujemy zatem \(K\) modeli dla tego samego zestawu hiperparametrów i każdy
testujemy na zbiorze walidacyjnym. Mamy \(K\) wyników dla zbiorów walidacyjnych, które możemy
uśrednić (i ewentualnie obliczyć odchylenie standardowe). Takie wyniki są znacznie bardziej
wiarygodne.



\section{Rachunek prawdopodobieństwa i statystyka matematyczna}


\subsection{Przestrzeń probabilistyczna}

Pojęciem pierwotnym w rachunku prawdopodobieństwa jest pojęcie \textbf{przestrzeni zdarzeń
elementarnych}, którą oznaczamy \(\Omega\). W przypadku doświadczeń losowych przestrzeń zdarzeń
elementarnych jest zbiorem wszystkich niepodzielnych wyników obserwacji.

\begin{definition}[Rodziny zdarzeń]

Niech \(\Omega\) będzie przestrzenią zdarzeń elementarnych. Rodziną zdarzeń nazwiemy rodzinę zbiorów
\(\mathcal{F}\) taką, że
\begin{enumerate}
    
    \item \(\Omega \in \mathcal{F}\).
    
    \item Jeśli \(A \in \mathcal{F}\) to \(\Omega \setminus A \in \mathcal{F}\).
    
    \item Jeśli \(A_1,A_2,\ldots \in \mathcal{F}\) to \(\bigcup_{i=1}^\infty A_i \in \mathcal{F}\).

\end{enumerate}

Rodzinę zdarzeń \(\mathcal{F}\) nazywamy \(\sigma\)--ciałem zdarzeń losowych.

\end{definition}

\begin{definition}[Zdarzenia losowego]
    Zdarzeniem losowym nazywamy dowolny zbiór należący do rodziny zdarzeń \(\mathcal{F}\). W
    szczególności \(\Omega\) nazwiemy zdarzeniem pewnym, a \(\emptyset\) -- zdarzeniem niemożliwym.
\end{definition}

\begin{definition}[Rozkładu prawdopodobieństwa]
Niech dana będzie przestrzeń zdarzeń elementarnych \(\Omega\) i rodzina zdarzeń \(\mathcal{F}\).
Rozkładem prawdopodobieństwa nazwiemy funkcję \(P: \mathcal{F} \mapsto [0;1]\) spełniającą
\begin{enumerate}
    
    \item Dla każdego \(A \in \mathcal{F}\) zachodzi \(P(A) \geq 0\).
    
    \item \(P(\Omega) = 1\).
    
    \item Dla każdego ciągu zdarzeń parami rozłącznych \(A_1,A_2,\ldots\), \(\forall i \neq j : A_i
    \cap A_j = \emptyset\) zachodzi
    \begin{equation*}
        P\left(\bigcup_{i=1}^\infty A_i\right) = \sum_{i=1}^\infty P(A_i)\,.
    \end{equation*}

\end{enumerate}
\end{definition}

\begin{theorem}
Niech \(P\) będzie rozkładem prawdopodobieństwa w rodzinie zdarzeń \(\mathcal{F}\), wówczas

\begin{enumerate}
    
    \item \(P(\emptyset) = 1\).
    
    \item (Addytywność) Dla dowolnych zdarzeń \(A_1,\ldots,A_n\) parami rozłącznych zachodzi
    \begin{equation*}
        P\left(\bigcup_{i=1}^n A_i\right) = \sum_{i=1}^n P(A_i)\,.
    \end{equation*}

    \item Dla dowolnego zdarzenia \(A\) zachodzi
    \begin{equation*}
        P(A') = 1 - P(A)\,.
    \end{equation*}

    \item Dla dowolnych zdarzeń \(A, B\) zachodzi
    \begin{equation*}
        P(A \cup B) = P(A) + P(B) - P(A \cap B)\,.
    \end{equation*}

    \item Jeśli \(A \subset B\) to \(P(A) \leq P(B)\).

    \item Dla każdego zdarzenia \(A\) zachodzi \(P(A) \leq 1\).

\end{enumerate}
\end{theorem}

\begin{definition}[Przestrzeni probabilistycznej]
Przestrzenią probabilistyczną nazywamy uporządkowaną trójkę \((\Omega, \mathcal{F}, P)\), gdzie
\(\Omega\) jest przestrzenią zdarzeń elementarnych, \(\mathcal{F}\) -- rodziną zdarzeń określoną na
\(\Omega\), a \(P\) -- rozkładem prawdopodobieństwa w \(\mathcal{F}\).
\end{definition}


\subsection{Prawdopodobieństwo warunkowe i niezależność zdarzeń}

\begin{definition}[Prawdopodobieństwa warunkowego]
Jeśli \(P(B) > 0\) to prawdopodobieństwem \(P(A \mid B)\) zdarzenia \(A\) pod warunkiem zdarzenia
\(B\) nazywamy iloraz prawdopodobieństw

\begin{equation*}
    P(A \mid B) := \frac{P(A \cap B)}{ P(B)}\,.
\end{equation*}
\end{definition}

Można pokazać, iż prawdopodobieństwo warunkowe jako funkcja zdarzenia \(A\) przy ustalonym zdarzeniu
\(B\) spełnia wszystkie aksjomaty rozkładu prawdopodobieństwa.

\begin{theorem}[O wielokrotnym warunkowaniu]
Dla dowolnych zdarzeń \(A_1,\ldots,A_n\) takich, że \(P(A_1,\ldots,A_n) > 0\) zachodzi
\begin{equation*}
    \begin{split}
        &P(A_1,\ldots,A_n) =\\
        &P(A_n \mid A_{n-1},\ldots,A_1) \ldots P(A_2 \mid A_1) P(A_1)\,,
    \end{split}
\end{equation*}
gdzie dla uproszczenia zapisu używamy \(P(A,B)\) w znaczeniu \(P(A \cap B)\).
\end{theorem}

\begin{definition}[Niezależności pary zdarzeń]
Zdarzenia \(A, B\) nazwiemy niezależnymi jeśli
\begin{equation*}
    P(A, B) = P(A) P(B)\,.
\end{equation*}
\end{definition}

Zauważmy, iż w przypadku \(P(B) > 0\) powyższa definicja jest równoważna bardziej intuicyjnej
wynikającej z definicji prawdopodobieństwa warunkowego, mianowicie zdarzenia \(A,B\) nazywamy
niezależnymi jeśli \(P(A \mid B) = P(A)\).

\begin{definition}[Niezależności zdarzeń]
Zdarzenia \(A_1,\ldots,A_n\) nazwiemy niezależnymi, jeśli dla dowolnych wskaźników
\(k_1,\ldots,k_s\), gdzie \(1 \leq k_1 < \ldots < k_s \leq n\) zachodzi
\begin{equation*}
    P(A_{k_1},\ldots,A_{k_s}) = P(A_{k_1}) \ldots P(A_{k_s})\,.
\end{equation*}
\end{definition}

\begin{theorem}[O łącznym prawdopodobieństwie niezależnych zdarzeń]
Jeśli zdarzenia \(A_1,\ldots,A_n\) są niezależne to
\begin{equation*}
    P(A_1,\ldots,A_n) = P(A_1) \ldots P(A_n)\,.
\end{equation*}
\end{theorem}

\begin{definition}[Warunkowej niezależności pary zdarzeń]
Zdarzenia \(A,B\) są warunkowo niezależne względem \(C\) jeśli
\begin{equation*}
    P(A,B \mid C) = P(A \mid C) P(B \mid C)\,.
\end{equation*}
\end{definition}

\begin{definition}[Warunkowej niezależności zdarzeń]
Zdarzenia \(A_1,\ldots,A_n\) są warunkowo niezależne względem \(B\) jeśli dla dowolnych wskaźników
\(k_1,\ldots,k_s\), gdzie \(1 \leq k_1 < \ldots < k_s \leq n\) zachodzi
\begin{equation*}
    P(A_{k_1},\ldots,A_{k_s} \mid B) = P(A_{k_1} \mid B) \ldots P(A_{k_s} \mid B)\,.
\end{equation*}
\end{definition}

\begin{theorem}
Jeśli zdarzenia \(A_1,\ldots,A_n\) są warunkowo niezależne względem \(B\) to
\begin{equation*}
    P(A_1,\ldots,A_n \mid B) = P(A_1 \mid B) \ldots P(A_n \mid B)\,.
\end{equation*}
\end{theorem}


\subsection{Prawdopodobieństwo całkowite i wzór Bayesa}

\begin{definition}[Układu zupełnego zdarzeń]
Jeśli zdarzenia \(A_1,A_2\ldots,\) są parami rozłączne oraz \(\bigcup_{i=1}^\infty A_i = \Omega\) to
zbiór zdarzeń \(\{A_i\}\) nazywamy układem zupełnym.
    
\end{definition}

\begin{theorem}[O prawdopodobieństwie całkowitym]
Jeśli zdarzenia \(A_i\) (gdzie \(i\) przebiega przeliczalny zbiór wartości) tworzą układ zupełny
zdarzeń oraz \(P(A_i) > 0\) dla każdego \(i\), to dla dowolnego zdarzenia \(B\) zachodzi
\begin{equation*}
    P(B) = \sum_{i} P(B \mid A_i) P(A_i)\,.
\end{equation*}
\end{theorem}

\begin{theorem}[Bayesa]
Jeśli zdarzenia \(A_i\) spełniają założenia tw. o prawdopodobieństwie całkowitym oraz \(P(B) > 0\),
to dla każdego zdarzenia \(A_j\) z rozpatrywanego układu zdarzeń zachodzi
\begin{equation*}
    P(A_j \mid B) = \frac{P(B \mid A_j) P(A_j)}{P(B)} = \frac{P(B \mid A_j) P(A_j)}{\sum_{i} P(B \mid A_i) P(A_i)}\,.
\end{equation*}
Prawdopodobieństwa \(P(A_j)\) nazywamy prawdopodobieństwami \textbf{a priori}, a \(P(A_j \mid
B)\) -- \textbf{a posteriori}. Prawdopodobieństwo \(P(B \mid A_j)\) nazywamy
\textbf{wiarygodnością}.   
\end{theorem}


\subsection{Zmienne losowe}

\begin{definition}[Zmiennej losowej]
Niech \((\Omega, \mathcal{F}, P)\) będzie przestrzenią probabilistyczną. Niech dany będzie również
drugi zbiór \(\mathcal{X}\), w którym wyróżniamy \(\sigma\)--ciało \(\mathcal{F}_\mathcal{X}\).
Zmienną losową \(X\) nazywamy odwzorowanie
\begin{equation*}
    X: \Omega \mapsto \mathcal{X}
\end{equation*}
takie, że dla każdego \(A \in \mathcal{F}_\mathcal{X}\) zachodzi warunek
\begin{equation*}
    \{\omega \in \Omega : X(\omega) \in A \} \in \mathcal{F}\,.
\end{equation*}
\end{definition}

W szczególności jeśli \(\mathcal{X} = \mathbb{R}\) to zmienną losową \(X\) nazywamy
\textbf{zmienną losową rzeczywistą}, natomiast jeśli \(\mathcal{X} = \mathbb{R}^n\) to zmienną
losową \(\mathbf{X}\) nazywamy \textbf{zmienną losową \(n\)--wymiarową}.

\begin{definition}[Rozkładu prawdopodobieństwa zmiennej losowej]
Niech \(X: \Omega \mapsto \mathcal{X}\) będzie zmienną losową. Funkcję \(P_X:
\mathcal{F}_\mathcal{X}\mapsto [0;1]\) określoną jako
\begin{equation*}
    P_X(A) = P(\{\omega \in \Omega : X(\omega) \in A\})
\end{equation*}
nazywamy rozkładem prawdopodobieństwa zmiennej losowej \(X\).
\end{definition}

W dalszym ciągu będziemy stosować notację uproszczoną pomijając indeks dolny \(X\) tj.
\begin{equation*}
    P(X \in A) := P(\{\omega \in \Omega : X(\omega) \in A\})\,.
\end{equation*}

Znajomość rozkładu zmiennej losowej \(X\) pozwala badać własności tej zmiennej bez znajomości
przestrzeni probabilistycznej \((\Omega, \mathcal{F}, P)\), na której owa zmienna jest określona.
Możemy mianowicie zawsze rozpatrywać tę zmienną jako określoną na przestrzeni probabilistycznej
\((\mathcal{X}, \mathcal{F}_\mathcal{X}, P_X)\).

\begin{definition}[Dystrybuanty zmiennej losowej rzeczywistej]
Niech \(X: \Omega \mapsto \mathbb{R}\) będzie zmienną losową rzeczywistą. Dystrybuantą zmiennej
losowej \(X\) nazywamy funkcję \(F: \mathbb{R} \mapsto [0;1]\) zdefiniowaną jako
\begin{equation*}
    F(x) = P(X \leq x)\,.
\end{equation*}
\end{definition}

\begin{theorem}[Własności dystrybuanty zmiennej rzeczywistej]
Niech \(F\) będzie dystrybuantą zmiennej losowej rzeczywistej \(X\), wówczas
\begin{enumerate}
    
    \item Jeśli \(a < b\) to \(F(b) - F(a) = P(X \in (a;b] )\).
    
    \item Funkcja \(F\) jest niemalejąca.

    \item \(\lim_{x \to -\infty} F(x) = 0\) oraz \(\lim_{x \to +\infty} F(x) = 1\).

    \item \(F\) jest prawostronnie ciągła.

    \item \(F\) jest ciągła w \(x_0\) wtedy i tylko wtedy, gdy \(P(X = x_0) = 0\).

\end{enumerate}
\end{theorem}

\begin{definition}[Rozkładu dyskretnego zmiennej losowej rzeczywistej]
Mówimy, że zmienna losowa rzeczywista \(X\) ma rozkład dyskretny jeśli istnieje skończony lub
przeliczalny zbiór \(\mathcal{S} \subset \mathbb{R}\) taki, że
\begin{equation*}
    P( X \in \mathcal{S} ) = 1\,.
\end{equation*}
Dla takiej zmiennej określa się funkcję prawdopodobieństwa (z ang. \textit{probability mass
function}, pmf)
\begin{equation*}
    p(x) = P(X = x)\,,\quad x \in \mathcal{S}\,.
\end{equation*}
\end{definition}

Zauważmy, że jeśli zmienna \(X\) ma rozkład dyskretny to zbiór \(\mathcal{S}\) ma postać
\(\{x_1,\ldots,x_k\}\) dla pewnych \(x_i \in \mathbb{R}\).

\begin{definition}[Rozkładu ciągłego zmiennej losowej rzeczywistej]
Mówimy, że zmienna losowa rzeczywista \(X\) ma rozkład ciągły jeśli istnieje funkcja \(p: \mathbb{R}
\mapsto [0; +\infty)\) taka, że dla dowolnego przedziału \((a;b)\) zachodzi
\begin{equation*}
    P(X \in (a;b)) = \int\limits_a^b p(x) \dd{x}\,.
\end{equation*}
Funkcję \(p\) nazywamy gęstością rozkładu prawdopodobieństwa (z ang. \textit{probability density
function}, pdf).
\end{definition}

Jeśli \(X\) jest zmienną losową rzeczywistą o rozkładzie ciągłym to wartość dystrybuanty jest dana
przez
\begin{equation}
    F(x) = \int\limits_{-\infty}^x p(x') \dd{x'}\,.
\end{equation}
W szczególności dystrybuanta \(F\) jest ciągła oraz dla każdego \(x\) zachodzi \(P(X = x) = 0\).

\begin{definition}[Wartości oczekiwanej zmiennej losowej rzeczywistej]
Wartością oczekiwaną zmiennej losowej rzeczywistej \(X\) nazywamy liczbę \(m\) określoną wzorem
\begin{equation*}
    m = \int\limits_{-\infty}^{+\infty} x p(x) \dd{x}
\end{equation*}
dla rozkładu ciągłego oraz
\begin{equation*}
    m = \sum_{x \in \mathcal{S}} x p(x)
\end{equation*}
dla rozkładu dyskretnego. Stosujemy dodatkowo oznaczenie \(\mathbb{E}[X] := m\).
\end{definition}

\begin{definition}[Wariancji zmiennej losowej rzeczywistej]
Wariancją zmiennej losowej rzeczywistej \(X\) nazywamy liczbę \(\sigma\) określoną wzorem
\begin{equation*}
    \sigma^2 = \mathbb{E}[(X - m)^2]\,,
\end{equation*} 
gdzie \(m = \mathbb{E}[X]\). Stosujemy dodatkowo oznaczenie \(\mathbb{V}[X] := \sigma^2\).
\textbf{Odchyleniem standardowym} zmiennej losowej \(X\) nazywamy pierwiastek jej wariancji
\begin{equation*}
    \sigma = \sqrt{\mathbb{V}[X]}\,.
\end{equation*} 
\end{definition}

\begin{theorem}[Własności wariancji]
Niech \(X\) będzie zmienną losową rzeczywistą, wówczas
\begin{enumerate}
    
    \item \(\mathbb{V}[X] = \mathbb{E}[X^2] - (\mathbb{E}[X])^2\)\,.

    \item Jeśli zmienna \(X\) ma skończoną wariancję, to dla dowolnych \(a,b \in \mathbb{R}\)
    zachodzi
    \begin{equation*}
        \mathbb{V}[aX + b] = a^2 \mathbb{V}[X]\,.
    \end{equation*}

    \item \(\mathbb{V}[X] = 0\) wtedy i tylko wtedy, gdy istnieje \(x_0 \in \mathbb{R}\) takie, że
    \begin{equation*}
        P(X \neq x_0) = 0\,.
    \end{equation*}

\end{enumerate}
\end{theorem}

\begin{definition}[Zmiennej standaryzowanej]
Zmienną losową o wartości oczekiwanej 0 i wariancji 1 nazywamy zmienną standaryzowaną. Jeśli \(X\)
jest dowolną zmienną o niezerowej wariancji, to
\begin{equation*}
    Z := \frac{X - m}{\sigma}
\end{equation*}
jest zmienną standaryzowaną, ponieważ
\begin{equation*}
    \mathbb{E}[Z] = \frac{1}{\sigma}\mathbb{E}[X - m] = \frac{1}{\sigma}(m - m) = 0
\end{equation*}
oraz
\begin{equation*}
    \mathbb{V}[Z] = \frac{1}{\sigma^2} \mathbb{V}[X] = \frac{\sigma^2}{\sigma^2} = 1\,.
\end{equation*}
\end{definition}

\begin{theorem}[Nierówność Czebyszewa]
Jeśli zmienna losowa \(X\) ma skończoną wartość średnią \(m\) i wariancję \(\sigma^2\), to dla
dowolnego \(\epsilon > 0\) zachodzi
\begin{equation*}
    P(|X - m| \geq \epsilon\sigma) \leq \frac{1}{\epsilon^2}\,.
\end{equation*}
\end{theorem}

\begin{definition}[Kwantyla]
Kwantylem rzędu \(p \in (0;1)\) zmiennej losowej o dystrybuancie \(F\) nazywamy dowolną liczbę
\(q_p\) taką, że
\begin{equation*}
    F(q_p^-) \leq p \leq F(q_p)\,.
\end{equation*}
Kwantyl rzędu 0.5 nazywamy \textbf{medianą}, kwantyl rzędu 0.25 -- \textbf{dolnym kwartylem},
a kwantyl rzędu 0.75 -- \textbf{górnym kwartylem}. Jeśli \(X\) ma rozkład ciągły, to kwantylem
rzędu \(p\) jest dowolne rozwiązanie równania
\begin{equation*}
    F(q_p) = p\,.
\end{equation*}
\end{definition}

\begin{definition}[Mody]
Modą zmiennej losowej o rozkładzie dyskretnym nazywa się dowolne maksimum funkcji prawdopodobieństwa
tego rozkładu.

Modą zmiennej losowej o rozkładzie ciągłym nazywa się dowolne maksimum lokalne gęstości tego
rozkładu.
\end{definition}


\subsection{Ważne rozkłady jednowymiarowe}

\begin{definition}[Rozkładu jednopunktowego]
Jeśli \(X\) jest zmienną losową rzeczywistą o rozkładzie dyskretnym i \(\mathcal{S} = \{x_0\}\), to
mówimy, że \(X\) ma rozkład jednopunktowy, wówczas
\begin{equation*}
    m = x_0,\quad \sigma^2 = 0\,.
\end{equation*}
\end{definition}

\begin{definition}[Rozkładu dwupunktowego]
Jeśli \(X\) jest zmienną losową rzeczywistą o rozkładzie dyskretnym i \(\mathcal{S} = \{x_1, x_2\}\)
oraz \(p(x_1) = p\), to mówimy, że \(X\) ma rozkład dwupunktowy z parametrem \(p\), wówczas
\begin{equation*}
    m = x_1 p + x_2 (1 - p)\,,\quad \sigma^2 = p (1 - p) (x_1 - x_2)^2\,.
\end{equation*}
Jeśli \(x_1 = 1\) i \(x_2 = 0\) to taki rozkład dwupunktowy nazywamy \textbf{rozkładem
zero--jedynkowym} (lub rozkładem Bernoulliego) i oznaczamy jako \(X \sim \mathrm{Ber}(p)\).
\end{definition}

\begin{definition}[Schematu dwumianowego]
Rozważmy doświadczenie losowe o dwu możliwych wynikach: sukces osiągamy z prawdopodobieństwem \(p\),
porażkę z prawdopodobieństwem \(1-p\). Doświadczenie tego rodzaju nazywamy \textbf{próbą
Bernoulliego}. Doświadczenie takie jest modelowane zmienną losową o rozkładzie dwupunktowym z
parametrem \(p\). Schematem dwumianowym (lub schematem Bernoulliego) nazywamy doświadczenie
polegające na \(n\)--krotnym powtórzeniu próby Bernoulliego, przy założeniu, iż poszczególne próby
są od siebie niezależne.
\end{definition}

\begin{definition}[Rozkładu dwumianowego]
Niech \(X\) będzie zmienną losową taką, że \(X\) jest liczbą sukcesów w schemacie dwumianowym
długości \(n\) z prawdopodobieństwem sukcesu w każdej próbie równym \(p\). Wówczas
\begin{equation*}
    P(X = k) = {n \choose k} p^k (1 - p)^{n-k}\,.
\end{equation*}
Rozkład prawdopodobieństwa określony powyższym wzorem nazywam się rozkładem dwumianowym o
parametrach \(n,p\). Jeśli zmienna \(X\) ma rozkład dwumianowy to stosujemy notację \(X \sim
\mathrm{Bin}(n,p)\). Jeśli \(X \sim \mathrm{Bin}(n,p)\) to
\begin{equation*}
    m = np\,,\quad \sigma^2 = np(1-p)\,.
\end{equation*}
\end{definition}

\begin{definition}[Rozkładu geometrycznego]
Mówimy, że zmienna losowa \(X\) ma rozkład geometryczny z parametrem \(p \in (0;1)\), tj. \(X \sim
\mathrm{Geo}(p)\), jeśli \(\mathcal{S} = \mathbb{N} \setminus \{0\}\), a funkcja prawdopodobieństwa
ma postać
\begin{equation*}
    p(x) = (1-p)^{x-1}p\,.
\end{equation*}
Zmienna \(X\) opisuje czas oczekiwania na pierwszy sukces w schemacie dwumianowym o nieskończonej
długości. Jeśli \(X \sim \mathrm{Geo}(p)\), to
\begin{equation*}
    m = p^{-1}\,,\quad \sigma^2 = \frac{1 - p}{p^2}\,.
\end{equation*}
\end{definition}

\begin{definition}[Rozkładu Poissona]
Jeśli zmienna \(X\) o wartościach w \(\mathbb{N}\) opisuje liczbę wystąpień pewnego powtarzalnego
zdarzenia w przedziale czasowym \([0;t]\), przy czym spełnione są następujące założenia:
\begin{itemize}

    \item powtórzenia zdarzenia występują niezależnie od siebie;
    
    \item \enquote{intensywność} wystąpień \(r\) jest stała;

    \item w danej chwili (rozumianej jako odpowiednio mały przedział) może zajść co najwyżej jedno
    zdarzenie

\end{itemize}

to zmienna ta ma rozkład Poissona z parametrem \(\lambda = rt\), tj. \(X \sim
\mathrm{Pos}(\lambda)\). Jeśli \(X \sim \mathrm{Pos}(\lambda)\), to
\begin{equation*}
    P(X = k) = \frac{\e^{-\lambda} \lambda^k}{k!}\,.
\end{equation*}
Ponadto
\begin{equation*}
    m = \lambda\,,\quad \sigma^2 = \lambda\,.
\end{equation*}
\end{definition}

\begin{theorem}[Poissona]
Niech \((X_n)\) będzie ciągiem zmiennych losowych takich, że \(X_n \sim \mathrm{Bin}(n, p_n)\), gdzie
\((p_n)\) jest ciągiem takim, że
\begin{equation*}
    \lim_{n \to \infty} n p_n = \lambda
\end{equation*}
dla pewnej liczby \(\lambda > 0\). Wówczas
\begin{equation*}
    \lim_{n \to \infty} P(X_n = k) = \frac{\e^{-\lambda} \lambda^k}{k!}\,.
\end{equation*}
\end{theorem}

\begin{definition}[Rozkładu jednostajnego]
Mówimy, że zmienna \(X\) o rozkładzie ciągłym ma rozkład jednostajny na przedziale \([a;b]\) tzn.
\(X \sim \mathcal{U}(a,b)\) jeśli jej gęstość wyraża się wzorem
\begin{equation*}
    p(x) = \begin{cases}
        \frac{1}{b - a}\,, &x\in[a;b]\\
        0\,,&x\notin[a;b]
    \end{cases}\quad.
\end{equation*}
Jeśli \(X \sim \mathcal{U}(a,b)\), to
\begin{equation*}
    m = \frac{a+b}{2}\,,\quad \sigma^2 = \frac{(b-a)^2}{12}\,.
\end{equation*}
\end{definition}

\begin{definition}[Rozkładu wykładniczego]
Niech \(T\) będzie zmienną modelującą czas oczekiwania na pierwsze zdarzenie w ciągu zdarzeń takim,
że czas wystąpienia każdego z nich w przedziale \([0;t]\) jest opisany przez zmienną \(X \sim
\mathrm{Pos}(\lambda t)\). wtedy
\begin{equation*}
    P(T > t) = P(X = 0) = \e^{-\lambda t}
\end{equation*}
oraz
\begin{equation*}
    P(T > 0) = 1\,.
\end{equation*}
Mówimy wtedy, że \(T\) ma rozkład wykładniczy z parametrem \(\lambda\), tzn. \(T \sim
\mathrm{Exp}(\lambda)\). Gęstość rozkładu wykładniczego ma postać
\begin{equation*}
    p(t) = \begin{cases}
        0\,,&t \leq 0\\
        \lambda \e^{-\lambda t}\,,&t > 0
    \end{cases}\quad.
\end{equation*}
Jeśli \(T \sim \mathrm{Exp}(\lambda)\), to
\begin{equation*}
    m = \lambda^{-1}\,,\quad \sigma^2 = \lambda^{-2}\,.
\end{equation*}
\end{definition}

\begin{definition}[Rozkładu normalnego]
Mówimy, że zmienna losowa \(X\) o gęstości \(\phi(x;\mu,\sigma^2)\) ma rozkład normalny z
parametrami \(\mu \in \mathbb{R} , \sigma^2 \in [0;+\infty)\), tzn. \(X \sim
\mathcal{N}(\mu,\sigma^2)\), jeśli    
\begin{equation*}
    \phi(x; \mu, \sigma^2) = \frac{1}{\sqrt{2 \pi \sigma^2}} \exp\left(-\frac{(x-\mu)^2}{2\sigma^2}\right)\,.
\end{equation*}
Jeśli \(X \sim \mathcal{N}(\mu,\sigma^2)\), to
\begin{equation*}
    m = \mu\,,\quad \mathbb{V}[X] = \sigma^2\,.
\end{equation*}
\end{definition}

Jeśli \(X \sim \mathcal{N}(\mu,\sigma^2)\), to
\begin{equation}
    \begin{split}        
        &P(|X - m| > 2\sigma) \approx 0.0455003\,,\\
        &P(|X - m| > 3\sigma) \approx 0.0026998\,.
    \end{split}
\end{equation}
Obserwacje te nazywamy odpowiednio regułą 5\% oraz regułą \(3\sigma\). Rozkład \(\mathcal{N}(0,1)\)
nazywamy \textbf{standardowym rozkładem normalnym}. Jeśli \(X \sim \mathcal{N}(\mu,\sigma^2)\),
to zmienna standaryzowana \(Z = (X - \mu) / \sigma\) ma standardowy rozkład normalny.


\subsection{Zmienne losowe wielowymiarowe}

\begin{definition}[Dystrybuanty zmiennej losowej wielowymiarowej]
Niech \(\mathbf{X}: \Omega \mapsto \mathbb{R}^n\) będzie zmienną losową \(n\)--wymiarową.
Dystrybuantą zmiennej losowej \(\mathbf{X}\) nazywamy funkcję \(F: \mathbb{R}^n \mapsto [0;1]\)
zdefiniowaną jako
\begin{equation*}
    F(\mathbf{x}) = P(\mathbf{X}_1 \leq \mathbf{x}_1, \ldots, \mathbf{X}_n \leq \mathbf{x}_n)
\end{equation*}
\end{definition}

\begin{definition}[Rozkładu dyskretnego zmiennej losowej wielowymiarowej]
Mówimy, że \(n\)--wymiarowa zmienna losowa \(\mathbf{X}\) ma rozkład dyskretny jeśli istnieje zbiór
skończony lub przeliczalny \(\mathcal{S} \subset \mathbb{R}^n\) taki, że
\begin{equation*}
    P(\mathbf{X} \in \mathcal{S}) = 1\,.
\end{equation*}
Dla takiej zmiennej określa się funkcję prawdopodobieństwa (z ang. \textit{probability mass
function}, pmf)
\begin{equation*}
    p(\mathbf{x}) = P(\mathbf{X}_1 = \mathbf{x}_1, \ldots, \mathbf{X}_n = \mathbf{x}_n)\,,\quad \mathbf{x} \in \mathcal{S}\,.
\end{equation*}
\end{definition}

Zauważmy, że jeśli zmienna \(\mathbf{X}\) ma rozkład dyskretny to zbiór \(\mathcal{S}\) ma postać
\(\{\mathbf{x}^1,\ldots,\mathbf{x}^k\}\) dla pewnych \(\mathbf{x}^i \in \mathbb{R}\).

\begin{definition}[Rozkładu ciągłego zmiennej losowej wielowymiarowej]
Mówimy, że \(n\)--wymiarowa zmienna losowa \(\mathbf{X}\) ma rozkład ciągły jeśli istnieje funkcja
\(p: \mathbb{R}^n \mapsto [0;+\infty)\) taka, że dla dowolnych przedziałów \((a_i;b_i)\) zachodzi
\begin{equation*}
    P(\mathbf{X}_1 \in (a_1;b_1), \ldots, \mathbf{X}_n \in (a_n;b_n)) = \int\limits_{a_1}^{b_1}\ldots\int\limits_{a_n}^{b_n}p(\mathbf{x}) \dd[n]{\mathbf{x}}\,.
\end{equation*}
Funkcję \(p\) nazywamy gęstością rozkładu prawdopodobieństwa (z ang. \textit{probability density
function}, pdf).
\end{definition}

Zauważmy, że jeśli \(\mathbf{X}\) jest zmienną losową wielowymiarową o rozkładzie ciągłym to wartość
dystrybuanty jest związana z gęstością rozkładu prawdopodobieństwa poprzez
\begin{equation}
    F(\mathbf{x}) = \int\limits_{-\infty}^{\mathbf{x}_1}\ldots\int\limits_{-\infty}^{\mathbf{x}_n} p(\mathbf{x}') \dd[n]{\mathbf{x}'}\,.
\end{equation}

\begin{definition}[Wartości oczekiwanej zmiennej losowej wielowymiarowej]
Wartością oczekiwaną zmiennej losowej wielowymiarowej nazywamy \(n\)--elementowy wektor rzeczywisty
\(\mathbf{m}\), którego elementy są określone wzorem
\begin{equation*}
    \mathbf{m} = \int\limits_{\mathbb{R}^n} \mathbf{x} p(\mathbf{x}) \dd[n]{\mathbf{x}}
\end{equation*}
dla rozkładu ciągłego oraz
\begin{equation*}
    \mathbf{m} = \sum_{\mathbf{x} \in \mathcal{S}} \mathbf{x} p(\mathbf{x})
\end{equation*}
dla rozkładu dyskretnego. Stosujemy dodatkowo oznaczenie \(\mathbb{E}[\mathbf{X}] := \mathbf{m}\).
\end{definition}

\begin{definition}[Rozkładu brzegowego]
Niech \(\mathbf{X}\) będzie \(n\)--wymiarową zmienną losową. Weźmy dwa zbiory indeksów
\(\{i_1,\ldots,i_k\} \subset \{1,\ldots,n\}\) i \(\{j_1,\ldots,j_{n-k}\} = \{1,\ldots,n\} \setminus
\{i_1,\ldots,i_k\}\). \((n-k)\)--wymiarowym rozkładem brzegowym zmiennej \(\mathbf{X}\) (względem
zmiennych \(j_1,\ldots,j_{n-k}\)) nazywamy rozkład prawdopodobieństwa na przestrzeni
\(\mathbb{R}^{n-k}\) określony wzorem
\begin{equation*}
    \begin{split}
        &P_{\mathbf{X}_{j_1},\ldots,\mathbf{X}_{j_{n-k}}}(A) \\
        &= P\left(\mqty[\mathbf{X}_{j_1},\ldots,\mathbf{X}_{j_{n-k}}] \in A\right)\\
        &= P\left(\mqty[\mathbf{X}_{j_1},\ldots,\mathbf{X}_{j_{n-k}}] \in A, \mathbf{X}_{i_1} \in \mathbb{R}, \ldots, \mathbf{X}_{i_k} \in \mathbb{R}\right)
    \end{split}
\end{equation*}    
\end{definition}

Niech \(\mathbf{X}\) będzie \(n\)--wymiarową zmienną losową o dystrybuancie \(F\). Dystrybuanta
rozkładu brzegowego \(\mathbf{X}\) względem zmiennych
\(\mathbf{X}_{j_1},\ldots,\mathbf{X}_{j_{n-k}}\) spełnia równość
\begin{equation}
    F_{\mathbf{X}_{j_1},\ldots,\mathbf{X}_{j_{n-k}}}(\mathbf{x}_{j_1},\ldots,\mathbf{x}_{j_{n-k}}) = \lim_{\mathbf{x}_{i_1} \to \infty,\ldots,\mathbf{x}_{i_k} \to \infty} F(\mathbf{x}_1,\ldots,\mathbf{x}_n)\,.
\end{equation}
Dystrybuanta ta nosi nazwę \((n-k)\)--wymiarowej dystrybuanty brzegowej.

Jeśli \(\mathbf{X}\) jest \(n\)--wymiarową zmienną losową o rozkładzie ciągłym, to rozkłady brzegowe
są także rozkładami ciągłymi o gęstościach
\begin{equation}
    p_{\mathbf{X}_{j_1},\ldots,\mathbf{X}_{j_{n-k}}}(\mathbf{x}_{j_1},\ldots,\mathbf{x}_{j_{n-k}}) = \int\limits_{\mathbb{R}^k} p(\mathbf{x}') \dd{\mathbf{x}_{i_1}}\ldots\dd{\mathbf{x}_{i_k}}\,.
\end{equation}
Jeśli z kolei \(\mathbf{X}\) ma rozkład dyskretny, to rozkłady brzegowe także są rozkładami
dyskretnymi z funkcjami prawdopodobieństwa
\begin{equation}
    p_{\mathbf{X}_{j_1},\ldots,\mathbf{X}_{j_{n-k}}}(\mathbf{x}_{j_1},\ldots,\mathbf{x}_{j_{n-k}}) = \sum_{i_1,\ldots,i_k}p(\mathbf{x})\,.
\end{equation}

\begin{theorem}
Niech zmienna \(n\)--wymiarowa \(\mathbf{X}\) ma rozkład ciągły o gęstości \(p_\mathbf{X}\) i niech
\(\mathbf{Y}_i = \boldsymbol{\varphi}_i(\mathbf{X})\) dla \(i=1,\ldots,n\). Jeśli odwzorowanie
\(\boldsymbol{\varphi}\) jest różniczkowalne i odwracalne, przy czym odwzorowanie odwrotne
\(\boldsymbol{\psi} = \boldsymbol{\varphi}^{-1}\) jest różniczkowalne, to \(n\)--wymiarowa zmienna
\(\mathbf{Y}\) ma rozkład o gęstości
\begin{equation*}
    p_\mathbf{Y}(\mathbf{y}) = |J| p_\mathbf{X}(\boldsymbol{\psi}(\mathbf{y}))\,,
\end{equation*}
gdzie \(J := \det \left[\pdv{\boldsymbol{\psi}_j}{\mathbf{y}_i}\right]\) jest jakobianem
odwzorowania \(\boldsymbol{\psi}\).
\end{theorem}


\subsection{Niezależność zmiennych losowych}

\begin{definition}[Niezależności zmiennych losowych]
Niech \(X_i : \Omega \mapsto \mathcal{X}\) będą zmiennymi losowymi. Zmienne \(X_1,\ldots,X_n\)
nazywamy niezależnymi jeżeli dla dowolnych \(A_1,\ldots,A_n \in \mathcal{F}_\mathcal{X}\) zachodzi
równość
\begin{equation*}
    P(X_1 \in A_1, \ldots, X_n \in A_n) = P(X_1 \in A_1) \ldots P(X_n \in A_n)\,.
\end{equation*}
\end{definition}

\begin{theorem}
Zmienne losowe wielowymiarowe \(\mathbf{X}, \mathbf{Y}\) o rozkładzie dyskretnym lub ciągłym z
funkcją lub gęstością prawdopodobieństwa \(p(\mathbf{x},\mathbf{y})\) są niezależne wtedy i tylko
wtedy, gdy dla dowolnych \(\mathbf{x}, \mathbf{y}\) zachodzi równość
\begin{equation*}
    p(\mathbf{x}, \mathbf{y}) = p_\mathbf{X}(\mathbf{x}) p_\mathbf{Y}(\mathbf{y})\,.    
\end{equation*}
\end{theorem}

\begin{theorem}[O niezależności funkcji zmiennych losowych]
Niech \(X_1,\ldots,X_n\) będą zmiennymi losowymi o wartościach \(\mathcal{X}\), a \(g_1,\ldots,g_n:
\mathcal{X} \mapsto \mathcal{Y}\), \(\varphi: \mathcal{X}^m \mapsto \mathcal{Y}\) i \(\psi:
\mathcal{X}^{n-m}\mapsto\mathcal{Y}\) funkcjami oraz \(m < n\). Wówczas jeśli \(X_1,\ldots,X_n\) są
niezależne to:
\begin{enumerate}
    
    \item zmienne losowe \(g_1(X_1),\ldots,g_n(X_n)\) są niezależne;

    \item zmienne losowe \(\varphi(X_1,\ldots,X_m)\) i \(\psi(X_{m+1},\ldots,X_n)\) są niezależne.

\end{enumerate}
\end{theorem}

Jeśli \((\mathbf{X}, \mathbf{Y})\) jest zmienną losową o rozkładzie ciągłym z gęstością
\(p(\mathbf{x}, \mathbf{y})\) i gęstość brzegowa \(p_\mathbf{Y}\) jest funkcją dodatnią, to dla
każdego \(\mathbf{y}\) rozkład warunkowy \(\mathbf{X}\) pod warunkiem \(\mathbf{Y} = \mathbf{y}\)
także jest ciągły z gęstością
\begin{equation}
    p(\mathbf{x} \mid \mathbf{y}) = \frac{p(\mathbf{x}, \mathbf{y})}{p_\mathbf{Y}(\mathbf{y})}\,,
\end{equation}
którą nazywamy gęstością warunkową \(\mathbf{X}\) pod warunkiem \(\mathbf{Y} = \mathbf{y}\).

Jeśli \((\mathbf{X}, \mathbf{Y})\) jest zmienną losową o rozkładzie dyskretnym z funkcją
prawdopodobieństwa \(p(\mathbf{x}, \mathbf{y})\) i brzegowa funkcja prawdopodobieństwa
\(p_\mathbf{Y}\) jest funkcją dodatnią, to dla każdego \(\mathbf{y}\) rozkład warunkowy
\(\mathbf{X}\) pod warunkiem \(\mathbf{Y} = \mathbf{y}\) także jest dyskretny z funkcją
prawdopodobieństwa
\begin{equation}
    p(\mathbf{x} \mid \mathbf{y}) = \frac{p(\mathbf{x}, \mathbf{y})}{p_\mathbf{Y}(\mathbf{y})}\,,
\end{equation}
którą nazywamy warunkową funkcją prawdopodobieństwa \(\mathbf{X}\) pod warunkiem \(\mathbf{Y} =
\mathbf{y}\).

\begin{definition}[Warunkowej wartości oczekiwanej]
Warunkową wartość oczekiwaną \(\mathbf{X}\) pod warunkiem \(\mathbf{Y} = \mathbf{y}\) nazywamy
wielkość
\begin{equation*}
    \mathbb{E}[\mathbf{X} \mid \mathbf{y}] = \int\limits_{\mathbb{R}^n} \mathbf{x} p(\mathbf{x} \mid \mathbf{y}) \dd[n]{\mathbf{x}}
\end{equation*}
\end{definition}

\begin{definition}[Warunkowej niezależności]
Niech \(X_i : \Omega \mapsto \mathcal{X}\) oraz \(Y\) będą zmiennymi losowymi. Mówimy, że zmienne
\(X_1,\ldots,X_n\) są warunkowo niezależne względem \(Y\), jeśli dla dowolnych \(A_i \in
\mathcal{F}_\mathcal{X}\) i dowolnej wartości \(y\) zachodzi
\begin{equation*}
    \begin{split}
        &P(X_1\in A_1,\ldots,X_n\in A_n \mid Y = y) \\
        &= P(X_1 \in A_1 \mid Y = y)\ldots P(X_n \in A_n \mid Y = y)\,.
    \end{split}
\end{equation*}  
\end{definition}

Dla rozkładów ciągłych lub dyskretnych z gęstością lub funkcją prawdopodobieństwa \(p\) warunkowa
niezależność \(\mathbf{X}, \mathbf{Y}\) względem \(\mathbf{Z}\) jest równoważna
\begin{equation*}
    p(\mathbf{x},\mathbf{y} \mid \mathbf{z}) = p_\mathbf{X}(\mathbf{x} \mid \mathbf{z})p_\mathbf{Y}(\mathbf{y} \mid \mathbf{z})\,.
\end{equation*}


\subsection{Kowariancja i współczynnik korelacji}

\begin{definition}[Kowariancji]
Kowariancją zmiennych losowych rzeczywistych \(X, Y\) o nazywamy liczbę
\begin{equation*}
    \mathrm{Cov}(X, Y) = \mathbb{E}[(X - m_X)(Y - m_Y)]\,,
\end{equation*}
gdzie wartości oczekiwane są liczone względem łącznego rozkładu prawdopodobieństwa \(p(x,y)\).
\textbf{Współczynnikiem korelacji zmiennych \(X, Y\)} nazywamy liczbę
\begin{equation*}
    \rho_{X,Y} = \frac{\mathrm{Cov}(X, Y)}{\sigma_X \sigma_Y}\,.
\end{equation*}
\end{definition}

\begin{theorem}[Własności kowariancji]
Niech \(X, Y\) będą zmiennymi losowymi rzeczywistymi, wówczas

\begin{enumerate}

    \item \(\mathrm{Cov}(X, Y) = \mathbb{E}[XY] - m_X m_Y\).
    
    \item Jeśli \(X\) i \(Y\) są niezależne oraz istnieje \(\mathbb{E}(XY)\), to \(\mathrm{Cov}(X,Y)
    = 0\).
\end{enumerate}
\end{theorem}

\begin{definition}[Macierzy kowariancji]\label{def:covmat}
Macierzą kowariancji \(\mathbf{K}\) \(n\)--wymiarowej zmiennej losowej \(\mathbf{X}\) nazywamy
macierz \(n \times n\) zdefiniowaną jako
\begin{equation*}
    \mathbf{K} = \mathbb{E}[(\mathbf{X} - \mathbf{m})(\mathbf{X} - \mathbf{m})^T]
\end{equation*}
lub równoważnie
\begin{equation*}
    \mathbf{K}_{ij} = \begin{cases}
        \mathbb{V}[\mathbf{X}_i]\,,&i=j\\
        \mathrm{Cov}(\mathbf{X}_i, \mathbf{X}_j)\,,&i\neq j
    \end{cases}\quad.
\end{equation*}
\end{definition}

\begin{theorem}[Własności macierzy kowariancji]\label{th:covmat}
Niech \(\mathbf{K}\) będzie macierzą kowariancji zmiennej \(\mathbf{X}\), wówczas
\begin{enumerate}

    \item Macierz kowariancji jest symetryczna.

    \item Macierz kowariancji jest nieujemnie określona.

    \item Jeśli zmienne \(\mathbf{X}_1,\ldots,\mathbf{X}_n\) są niezależne, to macierz kowariancji
    jest diagonalna.
    
\end{enumerate}
\end{theorem}


\subsection{Wielowymiarowy rozkład normalny}

\begin{definition}[Standardowego wielowymiarowego rozkładu normalnego]
Zmienna losowa \(\mathbf{X}\) ma standardowy \(n\)--wymiarowy rozkład normalny jeśli jej składowe są
niezależne i dla każdego \(i=1,\ldots,n\) \(\mathbf{X}_i \sim \mathcal{N}(0,1)\). Jest to rozkład
ciągły o gęstości
\begin{equation*}
    p(\mathbf{x}) = \frac{1}{\sqrt{(2\pi)^n}}\exp\left(-\frac{1}{2}\mathbf{x}^T\mathbf{x}\right)\,.
\end{equation*}
\end{definition}

\begin{definition}[Wielowymiarowego rozkładu normalnego]
Zmienna losowa \(\mathbf{X}\) ma \(n\)--wymiarowy rozkład normalny (z ang. \textit{Multivariate
Normal Distribution}, MVN), tzn. \(\mathbf{X} \sim \mathcal{N}(\boldsymbol{\mu},
\boldsymbol{\Sigma})\) jeśli istnieje \(k\)--wymiarowa zmienna losowa \(\mathbf{Z}\) o standardowym
rozkładzie normalnym dla pewnego \(k \leq n\) oraz istnieje \(\boldsymbol{\mu} \in \mathbb{R}^n\) i
macierz \(\mathbf{A} \in \mathbb{R}^{n \times k}\) takie, że \(\boldsymbol{\Sigma} =
\mathbf{A}\mathbf{A}^T\) oraz
\begin{equation*}
    \mathbf{X} = \mathbf{A}\mathbf{Z} + \boldsymbol{\mu}\,.
\end{equation*}
Jeśli macierz \(\boldsymbol{\Sigma}\) jest dodatnio określona, to rozkład
\(\mathcal{N}(\boldsymbol{\mu}, \boldsymbol{\Sigma})\) jest ciągły, a jego gęstość jest dana przez
\begin{equation*}
    p(\mathbf{x}) = \frac{1}{\sqrt{(2\pi)^n\det\boldsymbol{\Sigma}}}\exp\left(-\frac{1}{2}(\mathbf{x}-\boldsymbol{\mu})^T\boldsymbol{\Sigma}^{-1}(\mathbf{x}-\boldsymbol{\mu})\right)
\end{equation*}
Macierz \(\boldsymbol{\Sigma}^{-1}\) nazywa się \textbf{macierzą precyzji}.
\end{definition}

Poziomice gęstości niezdegenerowanego wielowymiarowego rozkładu normalnego są elipsoidami, których
półosie są skierowane wzdłuż wektorów własnych macierzy \(\boldsymbol{\Sigma}\) i mają długości
proporcjonalne do pierwiastka z wartości własnych.

\begin{theorem}[Własności niezdegenerowanego rozkładu normalnego]\label{th:mvn}
Niech \(\mathbf{X} \sim \mathcal{N}(\boldsymbol{\mu}, \boldsymbol{\Sigma})\) dla dodatnio określonej
macierzy \(\boldsymbol{\Sigma}\), wówczas
\begin{enumerate}

    \item Wszystkie rozkłady brzegowe i warunkowe \(\mathbf{X}\) są rozkładami normalnymi.

    \item Zmienne składowe \(\mathbf{X}_1,\ldots,\mathbf{X}_n\) są niezależne wtedy i tylko wtedy,
    gdy \(\boldsymbol{\Sigma}\) jest macierzą diagonalną.
    
\end{enumerate}
\end{theorem}


\subsection{Rodzaje zbieżności zmiennych losowych}

Niech \(X_1, X_2, \ldots\) i \(X\) będą zmiennymi losowymi o wartościach w tej samej przestrzeni
\(\mathcal{X}\).

\begin{definition}[Zbieżności stochastycznej]
Ciag \((X_n)\) jest zbieżny do \(X\) według prawdopodobieństwa (lub stochastycznie) jeśli
\begin{equation*}
    \forall \epsilon > 0, \delta > 0 : \exists N \in \mathbb{N} : \forall n > N : P\left(|X_n - X| > \epsilon\right) < \delta
\end{equation*}
\end{definition}

\begin{definition}[Zbieżności z prawdopodobieństwem 1]
Ciąg \((X_n)\) jest zbieżny do \(X\) z prawdopodobieństwem 1 (lub prawie na pewno) jeśli dla każdego
\(\epsilon\)
\begin{equation*}
    \forall \epsilon > 0, \delta > 0 : \exists N \in \mathbb{N} : \forall n > N : P\left(\sup_{m \geq n} |X_n - X| >  \epsilon\right) < \delta
\end{equation*}
\end{definition}

\begin{definition}[Zbieżności według dystrybuant]
Załóżmy, że \(\mathcal{X} = \mathbb{R}^k\) i niech \(F_{\mathbf{X}^n}\) będzie dystrybuantą
\(\mathbf{X}^n\), \(F_\mathbf{X}\) dystrybuantą \(\mathbf{X}\). Ciąg \((\mathbf{X}^n)\) jest zbieżny
do \(\mathbf{X}\) według dystrybuant jeśli dla każdego \(\mathbf{x} \in \mathbb{R}^k\) takiego, że
\(F_\mathbf{X}\) jest ciągła w \(\mathbf{x}\) zachodzi
\begin{equation*}
    \lim_{n \to \infty} F_{\mathbf{X}^n}(\mathbf{x}) = F_\mathbf{X}(\mathbf{x})\,.
\end{equation*}
\end{definition}

\begin{theorem}[Zależności między rodzajami zbieżności]
Niech \((X_n)\) będzie ciągiem zmiennych losowych. Wówczas
\begin{enumerate}

    \item Jeśli \(X_n \to X\) z prawdopodobieństwem 1, to \(X_n \to X\) według prawdopodobieństwa.

    \item Jeśli \(X_n \to X\) według prawdopodobieństwa, to istnieje podciąg \(X_{n_k}\) taki, że
    \(X_{n_k} \to X\) z prawdopodobieństwem 1.

    \item Jeśli \(X_n \to X\) według prawdopodobieństwa, to \(X_n \to X\) według dystrybuant.

\end{enumerate}
\end{theorem}


\subsection{Wnioskowanie statystyczne}

Niech zmienna losowa \(X\) określa model rozkładu pewnej cechy (cech) w ustalonym zbiorze instancji
(tzw. \textbf{populacji generalnej}). Innymi słowy, przyjmujemy, że wartości cech zachowują się
jakby zostały wybrane losowo zgodnie z rozkładem zmiennej \(X\). Do podstawowych zagadnień
wnioskowania statystycznego należą:
\begin{itemize}

    \item oszacowanie wielkości charakteryzujących rozkład \(X\) (np. wartości średniej albo
    wariancji);

    \item weryfikacja hipotez dotyczących rozkładu \(X\) (tym nie będziemy się zajmować).
    
\end{itemize}

\begin{definition}[Modelu statystycznego]
Modelem statystycznym nazywamy parę \((\mathcal{P}, \mathcal{X})\), gdzie \(\mathcal{P}\) jest
rodziną rozkładów prawdopodobieństwa na zbiorze \(\mathcal{X}\). Zazwyczaj przyjmuje się
\begin{equation*}
    \mathcal{P} = \{P_{\boldsymbol{\theta}} : \boldsymbol{\theta} \in \Theta \} 
\end{equation*}
dla pewnego zbioru parametrów \(\Theta\). Model statystyczny nazywamy \textbf{parametrycznym} jeśli
\(\Theta \subset \mathbb{R}^k\).
\end{definition}

\begin{definition}[Prostej próby losowej]
Prostą próbą losową o liczności \(n\) nazywamy ciąg niezależnych zmiennych losowych
\(X_1,\ldots,X_n\) o wartościach w \(\mathcal{X}\) i o tym samym rozkładzie
\(P_{\boldsymbol{\theta}} \in \mathcal{P}\) (z ang. \textit{independent and identically distributed,
i.i.d}).
\end{definition}

\begin{definition}[Realizacji prostej próby losowej]
Ciąg wartości \(x_1,\ldots,x_n \in \mathcal{X}\) takich, żeby
\begin{equation*}
    X_1(\omega) = x_1,\ldots,X_n(\omega) = x_n
\end{equation*}
dla pewnego \(\omega\) nazywamy realizacją prostej próby losowej \(X_1,\ldots,X_n\).
\end{definition}

\begin{definition}[Statystyki]
Statystyką nazywa się zmienną losową będącą funkcją prostej próby losowej \(T(X_1,\ldots,X_n)\).
\end{definition}
Średnią w prostej próbie losowej nazywa się statystykę
\begin{equation}
    \overline{X} = \frac{1}{n}\sum_{i=1}^n X_i\,.
\end{equation}
Wariancją w prostej próbie losowej nazywa się statystykę
\begin{equation}
    S^2 = \frac{1}{n - 1}\sum_{i=1}^n (X_i - \overline{X})^2\,.
\end{equation}


\subsection{Prawa wielkich liczb i CTG}

\begin{definition}
Mówimy, że dla ciągu zmiennych losowych \(X_1,X_2,\ldots\) zachodzi \textbf{słabe prawo wielkich
liczb} jeżeli
\begin{equation*}
    \frac{1}{n} \sum_{k=1}^n\left(X_k - \mathbb{E}[X_k]\right) \to 0
\end{equation*}
według prawdopodobieństwa. Jeżeli powyższy ciąg jest zbieżny z prawdopodobieństwem 1, to mówimy, że
dla ciągu \((X_n)\) zachodzi \textbf{mocne prawo wielkich liczb}.
\end{definition}

\begin{theorem}
Słabe prawo wielkich liczb zachodzi, jeśli ciąg \((X_n)\) spełnia \textbf{jeden} z podanych
warunków:
\begin{itemize}

    \item \(\frac{1}{n^2}\mathbb{V}\left[\sum_{k=1}^n X_k\right] \to 0\) (prawo wielkich liczb
    Markowa).

    \item \(X_n\) są niezależne i ich wariancje są wspólnie ograniczone (prawo wielkich liczb
    Czebyszewa).

    \item \(X_n\) są niezależne i mają ten sam rozkład o skończonej wartości średniej (prawo
    wielkich liczb Chinczyna).
    
\end{itemize}
\end{theorem}

\begin{theorem}[Prawo wielkich liczb Kołmogorowa]
Jeśli \(X_n\) są niezależne, mają skończone wariancje oraz
\begin{equation*}
    \sum_{n=1}^\infty \frac{\mathbb{V}[X_n]}{n^2} < +\infty\,,
\end{equation*}
to dla \(X_n\) zachodzi mocne prawo wielkich liczb (prawo wielkich liczb Kołmogorowa).
\end{theorem}

\begin{theorem}[Centralne Twierdzenie Graniczne]
Niech \(X_n\) będzie ciągiem niezależnych zmiennych losowych o tym samym rozkładzie ze skończoną
wartością średnią \(m\) i wariancją \(\sigma^2 > 0\). Wtedy ciąg \(Z_n\)
\begin{equation*}
    Z_n := \frac{\frac{1}{n}\sum_{k=1}^n X_k - m}{\frac{\sigma}{\sqrt{n}}}
\end{equation*}
jest zbieżny według dystrybuant do zmiennej losowej \(Z\) o standardowym rozkładzie normalnym.
\end{theorem}


\subsection{Estymatory punktowe}

\begin{definition}[Estymatora]
Estymatorem nazywa się statystykę \(\hat{\theta}(X_1,\ldots,X_n)\) służącą do oszacowania wartości
parametru \(\theta\). Liczbę \(\hat{\theta}(x_1,\ldots,x_n)\) dla konkretnej realizacji prostej
próby losowej nazywa się wartością estymatora albo estymatą.
\end{definition}

\begin{definition}[Obciążenia estymatora]
Liczbę
\begin{equation*}
    \mathrm{B}(\hat{\theta}) = \mathbb{E}[\hat{\theta}(X_1,\ldots,X_n)] - \theta
\end{equation*}
nazywamy obciążeniem estymatora \(\hat\theta\). Jeśli \(\mathrm{B}(\hat{\theta}) = 0\) to estymator
nazywamy \textbf{nieobciążonym}.
\end{definition}

\begin{definition}[Błędu średniokwadratowego]
Błędem średniokwadratowym (z ang. \textit{Mean Squared Error, MSE}) estymatora \(\hat{\theta}\)
parametru \(\theta\) nazywamy liczbę
\begin{equation*}
    \mathrm{MSE}(\hat{\theta}) = \mathbb{E}[(\hat{\theta} - \theta)^2]\,.
\end{equation*}
\end{definition}

\begin{theorem}
Dla dowolnego estymatora \(\hat{\theta}\) zachodzi
\begin{equation*}
    \mathrm{MSE}(\hat{\theta}) = \mathbb{V}[\hat{\theta}] + (\mathrm{B}(\hat{\theta}))^2\,.
\end{equation*}
\end{theorem}

\begin{definition}[Funkcji wiarygodności]
Funkcją wiarygodności (z ang. \textit{likelihood function}) dla modelu \(\mathcal{P} =
\{P_{\boldsymbol{\theta}} : \boldsymbol{\theta} \in \Theta\}\) nazywamy funkcję
\begin{equation*}
    \mathcal{L}: \mathbb{R}^n \times \Theta \ni (\mathbf{x},\boldsymbol{\theta}) \mapsto \mathcal{L}(\mathbf{x};\boldsymbol{\theta}) \in [0; +\infty)
\end{equation*}
wyznaczającą rozkład łączny obserwowanych danych jako funkcję parametru \(\boldsymbol{\theta}\).    
\end{definition}
Niech \(\mathbf{X}^1,\ldots,\mathbf{X}^n\) będzie prostą próbą losową. Jeśli
\(P_{\boldsymbol{\theta}}\) są rozkładami ciągłymi lub dyskretnymi o gęstościach lub funkcjach
prawdopodobieństwa \(p(\cdot;\boldsymbol{\theta})\), to
\begin{equation}
    \mathcal{L}(\mathbf{x}^1,\ldots,\mathbf{x}^n;\boldsymbol{\theta}) = \prod_{i=1}^n p(\mathbf{x}^i;\boldsymbol{\theta})\,.
\end{equation}
Dla wygody obliczeń często rozważa się tzw. zanegowaną logarytmiczną funkcję wiarygodności (z ang.
\textit{Negated Log-Likelihood function, NLL}), tzn.
\begin{equation}
    L(\mathbf{x};\boldsymbol{\theta}) = - \log \mathcal{L}(\mathbf{x};\boldsymbol{\theta})\,.
\end{equation}
Wówczas dla realizacji prostej próby losowej mamy
\begin{equation}
    L(\mathbf{x}^1,\ldots,\mathbf{x}^n;\boldsymbol{\theta}) = -\sum_{i=1}^n \log p(\mathbf{x}^i;\boldsymbol{\theta})\,.
\end{equation}

\begin{definition}[Estymatora największej wiarygodności]
Estymatorem największej wiarygodności (z ang. \textit{Maximum Likelihood Estimator, MLE}) nazywamy
funkcję \(\boldsymbol{\hat{\theta}}\), która przy ustalonych wartościach obserwacji (realizacji
prostej próby losowej) \(\{\mathbf{x}^1,\ldots,\mathbf{x}^n\}\) maksymalizuje wartość funkcji
wiarygodności lub, co równoważne, minimalizuje wartość zanegowanej logarytmicznej funkcji
wiarygodności tj.
\begin{equation*}
    \boldsymbol{\hat \theta}(\mathbf{x}^1,\ldots,\mathbf{x}^n) = \arg \min_{\boldsymbol{\theta} \in \Theta} \left[- \sum_{i=1}^n \log p(\mathbf{x}^i ; \boldsymbol{\theta})\right]\,.
\end{equation*}
\end{definition}
    
Jeśli funkcja wiarygodności jest różniczkowalna względem \(\boldsymbol{\theta}\) dla dowolnych
\(\mathbf{x}^i\), to MLE można czasem wyznaczyć analitycznie korzystając z warunku koniecznego
optymalności, tzn. rozwiązując układ równań
\begin{equation}
    \pdv{L(\mathbf{x}^1,\ldots,\mathbf{x}^n; \boldsymbol{\theta})}{\boldsymbol{\theta}} = 0\,.
\end{equation}
Jeśli MLE nie da się wyliczyć analitycznie, wyznacza się je przy użyciu algorytmów optymalizacji
numerycznej. Estymatory MLE są asymptotycznie nieobciążone.


\section{Elementy statystycznego uczenia maszynowego}

Przechodzimy teraz do zagadnień uczenia maszynowego, w których wykorzystamy przedstawioną wcześniej
teorię rachunku prawdopodobieństwa (w szczególności teorię zmiennych losowych) oraz wnioskowania
statystycznego.


\subsection{Podstawy wnioskowania bayesowskiego}

Niech \(\mathcal{X} = \{\mathbf{x}^1,\ldots,\mathbf{x}^n\}\) będzie realizacją prostej próby
losowej, czyli inaczej zbiorem obserwacji i.i.d. Zakładamy, że obserwacje te pochodzą z pewnego
parametrycznego modelu statystycznego \(\mathcal{P}\) z parametrami \(\boldsymbol{\theta} \in
\Theta\). Wcześniej \(\boldsymbol{\theta}\) miał jedynie rangę parametru. We wnioskowaniu
bayesowskim uznajemy, że parametry \(\boldsymbol{\theta}\) są również zmiennymi losowymi, a model
statystyczny modeluje warunkowy rozkład prawdopodobieństwa obserwacji pod warunkiem parametru. Mamy
zatem rodzinę gęstości prawdopodobieństwa \(p(\mathbf{x} \mid \boldsymbol{\theta})\) i chcemy
wnioskować o parametrze \(\boldsymbol{\theta}\) na podstawie obserwacji \(\mathcal{X}\). Jeśli znamy
rozkład a priori (inaczej \textbf{prior}) parametru \(\boldsymbol{\theta}\) opisany przez
\(p(\boldsymbol{\theta})\), to z twierdzenia Bayesa rozkład a posteriori (\textbf{posterior}) jest
dany przez
\begin{equation}
    p(\boldsymbol{\theta} \mid \mathcal{X}) = \frac{p(\mathcal{X} \mid \boldsymbol{\theta}) p(\boldsymbol{\theta})}{p(\mathcal{X})} = \frac{p(\mathcal{X} \mid \boldsymbol{\theta}) p(\boldsymbol{\theta})}{\sum_{\boldsymbol{\theta}' \in \Theta} p(\mathcal{X} \mid \boldsymbol{\theta}') p(\boldsymbol{\theta}')}
\end{equation}
Nie jesteśmy tutaj zbyt formalni z notacją, gdyż używamy \(p\) bardziej jakby był to rozkład
prawdopodobieństwa (miara probabilistyczna), a w rzeczywistości jest to gęstość lub funkcja
prawdopodobieństwa, więc powinniśmy używać indeksów dolnych określających zmienną losową, której
rozkład jest opisany danym \(p\), aby rozróżnić człony od siebie. Zapis taki jest jednak niezwykle
wygodny i dość czytelny. Należy jedynie pamiętać, iż nazwa argumentu funkcji \(p\) określa teraz z
jakiego wzoru powinniśmy skorzystać aby obliczyć jej wartość. Człon w mianowniku postaci
\begin{equation}
     Z = \sum_{\boldsymbol{\theta}' \in \Theta} p(\mathcal{X} \mid \boldsymbol{\theta}') p (\boldsymbol{\theta}') \approxeq \int\limits_\Theta p(\mathcal{X} \mid \boldsymbol{\theta}') p (\boldsymbol{\theta}') \dd[n]{\boldsymbol{\theta}'}
\end{equation}
jest tzw. \textbf{czynnikiem normalizacyjnym} (czyli po prostu liczbą, często oznaczaną przez
\(Z\)), który zapewnia, iż \(p(\boldsymbol{\theta} \mid \mathcal{X})\) sumuje / całkuje się do
\(1\).

Ponieważ założyliśmy, iż obserwacje ze zbioru \(\mathcal{X}\) są warunkowo niezależne względem
parametru \(\boldsymbol{\theta}\) oraz pochodzą z tego samego rozkładu opisanego przez
\(p(\mathbf{x} \mid \boldsymbol{\theta})\), więc człon \(p(\mathcal{X} \mid \boldsymbol{\theta})\)
zwany \textbf{wiarygodnością} możemy zapisać jako
\begin{equation}
    p(\mathcal{X} \mid \boldsymbol{\theta}) = \prod_{i=1}^n p(\mathbf{x}^i \mid \boldsymbol{\theta})\,.    
\end{equation}
Całe wnioskowanie bayesowskie opiera się na wyznaczeniu rozkładu a posteriori dla danego zbioru
obserwacji \(\mathcal{X}\), który wyraża naszą wiedzę o estymowanym parametrze
\(\boldsymbol{\theta}\). Na podstawie tego rozkładu możemy wyznaczyć estymatę punktową MAP
maksymalizującą gęstość prawdopodobieństwa a posteriori, 
\begin{equation}
    \boldsymbol{\hat \theta}_\mathrm{MAP}(\mathcal{X}) = \arg \max_{\boldsymbol{\theta} \in \Theta} p(\boldsymbol{\theta} \mid \mathcal{X})\,,
\end{equation}
jak również niepewność związaną z wyznaczeniem tej estymaty np. poprzez wyznaczenie przedziału
wiarygodności (nie należy mylić z przedziałem ufności). Możemy również skonstruować rozkład
predykcyjny (z ang. \textit{posterior predictive distribution}) określający prawdopodobieństwo
uzyskania nowej obserwacji \(\mathbf{t}\)\
\begin{equation}
    p(\mathbf{t} \mid \mathcal{X}) = \int\limits_\Theta p(\mathbf{t} \mid \boldsymbol{\theta}) p(\boldsymbol{\theta} \mid \mathcal{X}) \dd[n]{\boldsymbol{\theta}}\,.
\end{equation}

Znając rozkład a posteriori estymowanego parametru \(\boldsymbol{\theta}\) możemy nie tylko
wyznaczyć estymaty punktowe, wartości oczekiwane i przedziały wiarygodności, ale również znaleźć
estymator Bayesa (z ang. \textit{Bayes estimator}), który minimalizuje wartość oczekiwaną pewnej
funkcji straty (z ang. \textit{loss function}) \(L(\boldsymbol{\theta}, \boldsymbol{\hat \theta})\)
po wszystkich estymatorach \(\boldsymbol{\hat \theta}\)
\begin{equation}
    \boldsymbol{\hat \theta}_\mathrm{Bayes}(\mathcal{X}) = \arg\min_{\boldsymbol{\hat \theta} \in \Theta} \int\limits_{\Theta} L(\boldsymbol{\theta}, \boldsymbol{\hat \theta}) p(\boldsymbol{\theta} \mid \mathcal{X}) \dd[n]{\boldsymbol{\theta}}\,.
\end{equation}
Całkę w powyższym wzorze nazywa się również funkcją ryzyka (z ang. \textit{risk function})
\(R(\boldsymbol{\hat \theta})\), która określa oczekiwaną stratę spowodowaną wykorzystaniem danego
estymatora parametru.

Są dwa zasadnicze problemy we wnioskowaniu bayesowskim: pierwszym jest potrzeba znania rozkładu a
priori estymowanego parametru, drugim -- problem z obliczeniem czynnika normalizującego, który może
być skomplikowaną całką lub sumą po wykładniczo-wielu elementach. Oba te problemy można czasami
rozwiązać wprowadzając tzw. \textbf{prior sprzężony do wiarygodności}, tzn. zakładamy taki rozkład a
priori, aby dla danej wiarygodności rozkład a posteriori miał znaną postać (np. rozkładu normalnego,
rozkładu beta), wówczas nie musimy obliczać czynnika normalizującego, gdyż jest on po prostu znany.


\subsection{Modele gaussowskie i liniowe modele gaussowskie}

Zajmiemy się teraz wnioskowaniem bayesowskim w modelach, w których potrafimy analitycznie znaleźć
postać rozkład a posteriori. Jak już wspomnieliśmy (Tw. \ref{th:mvn}) gdy zmienna losowa
\(\mathbf{X}\) ma wielowymiarowy rozkład normalny z wartością oczekiwaną \(\boldsymbol{\mu}\) i
dodatnio określoną macierzą kowariancji \(\boldsymbol{\Sigma}\) to wszystkie rozkłady warunkowe i
brzegowe są rozkładami normalnymi. Poniżej wyznaczymy parametry tych rozkładów.

\begin{theorem}\label{th:gauss_model}
Niech zmienne losowe \(\mathbf{x} \in \mathbb{R}^{n-k}\) i \(\mathbf{y} \in \mathbb{R}^k\) mają
łącznie wielowymiarowy rozkład normalny
\begin{equation*}
    \mqty[\mathbf{x} \\ \mathbf{y}] \sim \mathcal{N}\left(\mqty[\boldsymbol{\mu}_\mathbf{x} \\ \boldsymbol{\mu}_\mathbf{y}], \mqty[\boldsymbol{\Sigma}_{\mathbf{xx}} & \boldsymbol{\Sigma}_{\mathbf{xy}} \\ \boldsymbol{\Sigma}_{\mathbf{yx}} & \boldsymbol{\Sigma}_{\mathbf{yy}}] \right)\,.
\end{equation*} 
Wówczas
\begin{enumerate}
    
    \item Zmienne losowe \(\mathbf{x}\) i \(\mathbf{y}\) mają odpowiednio rozkłady
    \begin{equation*}
        \mathbf{x} \sim \mathcal{N}(\boldsymbol{\mu}_\mathbf{x}, \boldsymbol{\Sigma}_{\mathbf{xx}})\,,\quad \mathbf{y} \sim \mathcal{N}(\boldsymbol{\mu}_\mathbf{y}, \boldsymbol{\Sigma}_{\mathbf{yy}})\,.
    \end{equation*}

    \item Rozkład warunkowy \(\mathbf{x} \mid \mathbf{y}\) jest rozkładem normalnym \(\mathbf{x}
    \mid \mathbf{y} \sim \mathcal{N}(\boldsymbol{\mu}_{\mathbf{x}\mid\mathbf{y}},
    \boldsymbol{\Sigma}_{\mathbf{x}\mid\mathbf{y}})\) o parametrach
    \begin{equation*}
        \boldsymbol{\mu}_{\mathbf{x}\mid\mathbf{y}} = \boldsymbol{\mu}_\mathbf{x} + \boldsymbol{\Sigma}_\mathbf{xy}\boldsymbol{\Sigma}_\mathbf{yy}^{-1}(\mathbf{y} - \boldsymbol{\mu}_\mathbf{y})\,,\quad \boldsymbol{\Sigma}_{\mathbf{x}\mid\mathbf{y}} = \boldsymbol{\Sigma}_\mathbf{xx} - \boldsymbol{\Sigma}_\mathbf{xy}\boldsymbol{\Sigma}_\mathbf{yy}^{-1}\boldsymbol{\Sigma}_\mathbf{yx}\,.
    \end{equation*}
\end{enumerate}
\end{theorem}
Powyższe własności normalnych rozkładów łącznych pozwalają jawnie wnioskować w tzw. liniowych
modelach gaussowskich (z ang. \textit{Linear Gaussian Models}). Załóżmy, iż nasze obserwacje są
modelowane zmienną losową \(\mathbf{y}\) o rozkładzie normalnym z estymowanym parametrem
\(\mathbf{x}\) i znanymi parametrami \(\mathbf{A}, \mathbf{b}, \boldsymbol{\Sigma}_\mathbf{y}\) tak,
że wiarygodność ma postać
\begin{equation}
    \mathbf{y} \mid \mathbf{x} \sim \mathcal{N}(\mathbf{A}\mathbf{x} + \mathbf{b}, \boldsymbol{\Sigma}_\mathbf{y})\,,
\end{equation}
gdzie \(\mathbf{A}\) jest w ogólności macierzą prostokątną. Jeśli jako prior na \(\mathbf{x}\)
przyjmiemy również rozkład normalny
\begin{equation}
    \mathbf{x} \sim \mathcal{N}(\boldsymbol{\mu}_\mathbf{x}, \boldsymbol{\Sigma}_\mathbf{x})
\end{equation}
to rozkład a posteriori jest również rozkładem normalnym. W szczególności załóżmy, że mamy zbiór
obserwacji i.i.d. \(\mathcal{X} = \{\mathbf{y}^1, \ldots, \mathbf{y}^n\}\). Wówczas wiarygodność ma
postać
\begin{equation}
    p(\mathcal{X} \mid \mathbf{x}) \propto \prod_{i=1}^n \exp\left(-\frac{1}{2}(\mathbf{y}^i - (\mathbf{Ax}+\mathbf{b}))^T\boldsymbol{\Sigma}_\mathbf{y}^{-1}(\mathbf{y}^i - (\mathbf{Ax}+\mathbf{b}))\right)\,,
\end{equation}
a rozkład a posteriori
\begin{equation}
    p(\mathbf{x} \mid \mathcal{X}) \propto p(\mathcal{X} \mid \mathbf{x}) \exp\left(-\frac{1}{2}(\mathbf{x} - \boldsymbol{\mu}_\mathbf{x})^T\boldsymbol{\Sigma}_\mathbf{x}^{-1}(\mathbf{x} - \boldsymbol{\mu}_\mathbf{x})\right)\,.
\end{equation}
Rozpisując wszystkie czynniki i pomijając czynnik stałe otrzymujemy
\begin{equation}
    \begin{split}
        \log p(\mathbf{x} \mid \mathcal{X}) &\propto -\frac{1}{2}\mathbf{x}^T\left(\boldsymbol{\Sigma}_\mathbf{x}^{-1} + n\mathbf{A}^T\boldsymbol{\Sigma}_\mathbf{y}^{-1}\mathbf{A}\right)\mathbf{x}\\
        & + \mathbf{x}^T\left(\boldsymbol{\Sigma}_\mathbf{x}^{-1}\boldsymbol{\mu}_\mathbf{x} + \mathbf{A}^T\boldsymbol{\Sigma}_\mathbf{y}^{-1}\left[\sum_{i=1}^n\mathbf{y}^i - n\mathbf{b}\right]\right)
    \end{split}
\end{equation}
skąd widzimy, iż rozkład \(\mathbf{x} \mid \mathcal{X}\) jest rozkładem normalnym o parametrach
\begin{equation}\label{eq:linear_gauss}\boxed{
    \begin{split}
        &\boldsymbol{\mu}_{\mathbf{x} \mid \mathcal{X}} = \boldsymbol{\Sigma}_{\mathbf{x} \mid \mathcal{X}} \left(\boldsymbol{\Sigma}_\mathbf{x}^{-1}\boldsymbol{\mu}_\mathbf{x} + \mathbf{A}^T\boldsymbol{\Sigma}_\mathbf{y}^{-1}\left[\sum_{i=1}^n\mathbf{y}^i - n\mathbf{b}\right]\right) \,,\\
        &\boldsymbol{\Sigma}_{\mathbf{x} \mid \mathcal{X}} = \left(\boldsymbol{\Sigma}_\mathbf{x}^{-1} + n\mathbf{A}^T\boldsymbol{\Sigma}_\mathbf{y}^{-1}\mathbf{A}\right)^{-1}\,.
    \end{split}
}\end{equation}


\subsection{Regresja liniowa}

Załóżmy, iż modelujemy obserwacje postaci \((y, \mathbf{x})\), gdzie \(y\) to skalar zwany
\textbf{zmienną objaśnianą}, którego wartość obserwujemy, a \(\mathbf{x}\) to wektor zmiennych
objaśniających, co do którego zakładamy, iż dla danego pomiaru \(y\) jest on znany dokładnie.
Dodatkowo załóżmy liniowy model \(\hat{y}(\mathbf{x};\mathbf{w})\) postaci
\begin{equation}
    \hat{y}(\mathbf{x};\mathbf{w}) = \mathbf{x}^T\mathbf{w}\,,
\end{equation}
w którym mamy dodatkowo \textbf{błąd losowy} \(\epsilon \sim \mathcal{N}(0, \sigma^2)\) z nieznanym
\(\sigma\). Możemy zatem napisać model statystyczny postaci
\begin{equation}
    y(\mathbf{x}) \mid \mathbf{w}, \sigma \sim \mathcal{N}(\hat{y}(\mathbf{x}; \mathbf{w}), \sigma^2)\,,
\end{equation}
gdzie \(\mathbf{w}\), \(\sigma\) to estymowane parametry. Powiedzmy, iż mamy zbiór obserwacji i.i.d.
\(\mathcal{X} = \{y_1(\mathbf{x}^1), \ldots, y_n(\mathbf{x}^n)\}\). Wiarygodność ma zatem postać
\begin{equation}\boxed{
    \mathcal{L}(\mathcal{X} ; \mathbf{w}, \sigma) = \frac{1}{(2\pi\sigma^2)^{n/2}} \prod_{i=1}^n \exp\left(-\frac{1}{2\sigma^2}\left[y_i - \hat{y}(\mathbf{x}^i; \mathbf{w})\right]^2\right)\,.
}\end{equation}
W przypadku regresji liniowej zamiast pełnego wnioskowania bayesowskiego często stosuje się prostsze
podejście polegające na ograniczeniu się do znalezienia estymaty punktowej MLE. Zanegowana
logarytmiczna funkcja wiarygodności, którą będziemy również nazywać \textbf{funkcją kosztu} ma
postać (pomijamy człony stałe, gdyż nie są one istotne przy dalszej minimalizacji)
\begin{equation}\boxed{
    L(\mathcal{X}; \mathbf{w}, \sigma) = n\log\sigma + \frac{1}{2\sigma^2}\sum_{i=1}^n \left[y_i - \hat{y}(\mathbf{x}^i; \mathbf{w})\right]^2 + \mathrm{const.}
}\end{equation}
Minimalizując funkcję \(L\) względem \(\mathbf{w}\) i \(\sigma\) otrzymamy estymaty MLE tych
parametrów. Dla ustalonego stałego \(\sigma\) otrzymana funkcja \(L\) ma postać formy kwadratowej i
otrzymany przy takim uproszczeniu problem optymalizacyjny nazywamy metodą najmniejszych kwadratów (z
ang. \textit{Ordinary Least Squares, OLS}). W przypadku modelu liniowego estymatory można znaleźć
analitycznie rozwiązując układ równań
\begin{equation}
    \begin{split}
        &\pdv{L}{\mathbf{w}_j} = \frac{1}{2\sigma^2}\pdv{~}{\mathbf{w}_j}\sum_{i=1}^n\left[y_i - \mathbf{w}^T\mathbf{x}^i\right]^2 = -\frac{1}{\sigma^2}\sum_{i=1}^n (y_i - \mathbf{w}^T\mathbf{x}^i)\mathbf{x}^i_j = 0\,,\\
        &\pdv{L}{\sigma} = \frac{n}{\sigma} - \frac{1}{\sigma^3}\sum_{i=1}^n \left[y_i - \hat{y}(\mathbf{x}^i; \mathbf{w})\right]^2 = 0\,.
    \end{split}
\end{equation}
Z powyższego
\begin{equation}
    \begin{split}
        &\sum_{i=1}^n y_i\mathbf{x}^i_j - \mathbf{w}_\mathrm{MLE}^T\sum_{i=1}^n\mathbf{x}^i\mathbf{x}^i_j = 0\,,\\
        &\sigma_\mathrm{MLE}^2 = \frac{1}{n} \sum_{i=1}^n \left[y_i - \hat{y}(\mathbf{x}^i; \mathbf{w}_\mathrm{MLE})\right]^2\,.\\
    \end{split}
\end{equation}
Wprowadzając wektor \(\mathbf{y}_i := y_i\) oraz macierz \(\mathbf{X}_{ij} := \mathbf{x}^i_j\) możemy
zapisać pierwsze równanie jako
\begin{equation}
    -\mathbf{y}^T\mathbf{X} + \mathbf{w}^T_\mathrm{MLE} \mathbf{X}^T\mathbf{X} = 0\,,
\end{equation}
skąd
\begin{equation}\boxed{
    \begin{split}
        &\mathbf{w}_\mathrm{MLE} = \left(\mathbf{X}^T\mathbf{X}\right)^{-1}\mathbf{X}^T\mathbf{y} = \mathbf{X}^+\mathbf{y}\,,\\
        &\sigma_\mathrm{MLE}^2 = \frac{1}{n}\left(\mathbf{y} - \mathbf{X}\mathbf{w}_\mathrm{MLE}\right)^T\left(\mathbf{y} - \mathbf{X}\mathbf{w}_\mathrm{MLE}\right)\,,
    \end{split}
}\end{equation}
gdzie \(\mathbf{X}^+\) oznacza \textbf{pseudoodwrotność Moore'a--Penrose'a}, którą można efektywnie
obliczyć korzystając z rozkładu SVD macierzy \(\mathbf{X}\).


\subsection{Regularyzacja}

Regularyzacją  nazywamy proces polegający na wprowadzeniu ad hoc do zagadnienia optymalizacji
dodatkowych członów tak, aby rozwiązanie było regularne (prostsze, nieosobliwe, jednoznaczne). W
przypadku funkcji kosztu \(L\) najczęściej dodajemy człon penalizujący rozwiązania o dużej normie
estymowanego parametru tj. człon postaci \(\gamma \norm{\mathbf{w}}\) dla pewnej normy
\(\norm{\cdot}\) i hiperparametru \(\gamma\). W kontekście bayesowskim regularyzację można również
rozumieć jako pewną \enquote{niechęć} (tłumienie, zachowawczość) modelu do zmiany rozkładu a priori
estymowanego parametru.

W przypadku regresji liniowej jeśli zamiast poszukiwania estymaty MLE będziemy poszukiwać estymaty
MAP (z ang. \textit{Maximum a Posteriori estimate}) z rozkładem a priori na parametr \(\mathbf{w}\)
danym przez \(\mathbf{w} \sim \mathcal{N}(\mathbf{0}, \tau^{2}\mathbf{1})\), to logarytm gęstości
rozkładu a posteriori (który również będziemy nazywać zregularyzowaną funkcją kosztu) ma postać
(tutaj zakładamy, że \(\sigma\) jest znaną stałą)
\begin{equation}\boxed{
    L(\mathcal{X};\mathbf{w}) = \frac{1}{2\sigma^2}\sum_{i=1}^n\left[y_i - \hat{y}(\mathbf{x}^i;\mathbf{w})\right]^2 + \frac{1}{2\tau^2}\mathbf{w}^T\mathbf{w} + \mathrm{const.}
}\end{equation}
skąd możemy bez problemy wyznaczyć estymatę punktową MAP parametru \(\mathbf{w}\)
\begin{equation}
    \mathbf{w}_\mathrm{MAP} = \left(\gamma \mathbf{1} + \mathbf{X}^T\mathbf{X}\right)^{-1}\mathbf{X}^T\mathbf{y}\,,
\end{equation}
gdzie \(\gamma = \frac{\sigma^2}{\tau^2}\) nazywamy \textbf{siłą regularyzacji}. Zagadnienie
minimalizacji funkcji kosztu będącej formą kwadratową z dodanym członem regularyzującym w postaci
sumy kwadratów współrzędnych wektora \(\mathbf{w}\) (normy L2 wektora) nazywamy \textbf{regresją
grzbietową} (z ang. \textit{ridge regression}), natomiast taką postać członu regularyzującego --
regularyzacją L2. Zauważmy, że im większa jest wartość \(\gamma\) (mniejsza niepewność związana z
rozkładem a priori) tym drugi człon w nawiasie staje się mniej istotny.

Innym przykładem regularyzacji jest tzw. regularyzacja L1, która polega na dodaniu do funkcji kosztu
członu postaci \(\gamma \sum_{j=1}^d |\mathbf{w}_j|\) tj. normy L1 wektora wag. Zagadnie
optymalizacji formy kwadratowej z członem regularyzującym L1 nazywamy \textbf{regresją LASSO}. W
takim przypadku nie da się prosto analitycznie znaleźć estymaty punktowej MAP i trzeba używać
algorytmów optymalizacji numerycznej. W ogólności można połączyć regularyzacje L1 i L2 tj. rozważać
zregularyzowaną funkcję kosztu postaci (tutaj parametr \(\sigma\) nie występuje, tj. ta funkcja
kosztu nie ma bezpośredniej interpretacji probabilistycznej jako funkcja wiarygodności)
\begin{equation}\boxed{
    L(\mathcal{X};\mathbf{w}) = \frac{1}{2}\sum_{i=1}^n \left[y_i - \hat{y}(\mathbf{x}^i;\mathbf{w})\right]^2 + \frac{\gamma_1}{2}\norm{\mathbf{w}}_1 +  \frac{\gamma_2}{2}\norm{\mathbf{w}}_2^2\,.
}\end{equation}
Zagadnienie minimalizacji takiej funkcji kosztu nazywamy ElasticNet i tak jak w przypadku LASSO
musimy korzystać z algorytmów optymalizacji numerycznej. Często wykorzystuje się tutaj algorytmy
bezgradientowe np. coordinate descent.

\subsection{Robust regression}

Zastanówmy się wpierw na czym tak naprawdę polega modelowanie rozkładu warunkowego \(y(\mathbf{x})
\mid \mathbf{w}\) za pomocą określonego rozkładu prawdopodobieństwa. Można by było pomyśleć, iż
takie podejście wprowadza bardzo silne założenia, a co za tym idzie ograniczenia w stosowaniu
naszego modelu. Zauważmy jednak, iż w przypadku podejścia typu \textit{likelihood} rozkład jest
niejako wybierany w taki sposób, aby jego parametry były użyteczne. Istotnie w przypadku regresji
zwykle nie ma jak zweryfikować rzeczywistego rozkładu \(y(\mathbf{x})\), gdyż mamy tylko po jednej
wartości \(y\) dla danego \(\mathbf{x}\). Modelując \(y(\mathbf{x})\) rozkłxadem normalnym chodzi nam
zatem raczej o to, że w takim modelu chcemy znaleźć parametr (prostą) taką, że punkty w zbiorze są
po odpowiednich stronach owej prostej zgodnie z gęstością prawdopodobieństwa, która jest symetryczna
i ma właśnie kształt dzwonu, tj. dużo masy prawdopodobieństwa jest zebrane w niewielkiej odległości
od estymowanej prostej.

W takim ujęciu możemy zakładać różne inne rozkłady na \(y(\mathbf{x})\) jeśli interesują nas proste,
które inaczej mają rozdzielać masę prawdopodobieństwa między punkty. Problemem w przypadku rozkładu
normalnego jest jego czułość na wartości odstające, gdyż w rozkładzie normalnym ogony tego rozkładu
mają stosunkowo niewielką masę prawdopodobieństwa. Chcielibyśmy zatem rozkład z tzw. ciężkimi
ogonami (z ang. \textit{heavy tails}). Dodatkowo chcielibyśmy mieć rozkład, który pozwala znaleźć
prostą, która nie rozkłada masy po równo, ale np. tak, że 90\% masy prawdopodobieństwa jest pod nią.
Oba te problemy możemy rozwiązać modelując rozkład \(y(\mathbf{x})\) przez tzw. \textbf{asymetryczny
rozkład Laplace'a} (z ang. \textit{Asymmetric Laplace Distribution, ALD}).

\begin{definition}[Asymetrycznego Rozkładu Laplace'a]
Mówimy, iż zmienna losowa rzeczywista \(X\) ma asymetryczny rozkład Laplace'a, tzn. \(X \sim
\mathrm{ALD}(m, \lambda, q)\) jeśli jej gęstość wyraża się wzorem
\begin{equation*}
    p(x) = \frac{q(1-q)}{\lambda}\begin{cases}
        \e^{}
        
    \end{cases}

\end{equation*}
\end{definition}


\subsection{Procesy gaussowskie}

Jak już wspomnieliśmy (Def. \ref{def:covmat}) macierz kowariancji \(n\)--wymiarowej zmiennej losowej
\(\mathbf{x}\) o wartości oczekiwanej \(\boldsymbol{\mu}\) jest zdefiniowana jako
\begin{equation}
    \mathbf{\Sigma} = \mathbb{E}\left[(\mathbf{x} - \boldsymbol{\mu})(\mathbf{x} - \boldsymbol{\mu})^T\right]\,.
\end{equation}
Wiemy również (Tw. \ref{th:covmat}), iż macierz ta jest nieujemnie określona. Pokażemy teraz, iż dla
każdej nieujemnie określonej macierzy symetrycznej \(\mathbf{K}\) wymiaru \(n\times n\) istnieje
\(n\)--wymiarowa zmienna losowa o wielowymiarowym rozkładzie normalnym, dla której \(\mathbf{K}\)
jest macierzą kowariancji. Istotnie dla każdej nieujemnie określonej macierzy symetrycznej istnieje
macierz \(\mathbf{L}\) taka, że
\begin{equation}
    \mathbf{K} = \mathbf{L}\mathbf{L}^T\,,
\end{equation}
jest to tzw. \textbf{dekompozycja Choleskiego}. Niech \(\mathbf{z} \sim \mathcal{N}(\mathbf{0},
\mathbf{1})\), wówczas zmienna losowa \(\mathbf{L}\mathbf{z}\) ma rozkład o zerowej wartości
oczekiwanej i macierzy kowariancji
\begin{equation}
    \mathbb{E}\left[(\mathbf{L}\mathbf{z})(\mathbf{L}\mathbf{z})^T\right] = \mathbb{E}\left[\mathbf{L}\mathbf{z}\mathbf{z}^T\mathbf{L}^T\right] = \mathbf{L}\mathbb{E}[\mathbf{z}\mathbf{z}^T]\mathbf{L}^T = \mathbf{L}\mathbf{1}\mathbf{L}^T = \mathbf{K}\,.
\end{equation}
Powyższe własności wskazują, iż macierze kowariancji można w pewnym sensie utożsamiać z nieujemnie
określonymi macierzami symetrycznymi.

\begin{definition}[Funkcji kowariancji]
Funkcję \(k: \mathbb{R}^n\times\mathbb{R}^n\mapsto\mathbb{R}\) taką, że \(\forall m\in\mathbb{N} :
\forall X = \{\mathbf{x}^1,\ldots,\mathbf{x}^m\} \subset \mathbb{R}^n\) macierz
\begin{equation*}
    k(X,X) = \mqty[k(\mathbf{x}^1, \mathbf{x}^1) & k(\mathbf{x}^1, \mathbf{x}^2) & \cdots & k(\mathbf{x}^1, \mathbf{x}^m)\\
    k(\mathbf{x}^2, \mathbf{x}^1) & k(\mathbf{x}^2, \mathbf{x}^2) & \cdots & k(\mathbf{x}^2, \mathbf{x}^m)\\
    \vdots & \vdots & \ddots & \vdots\\
    k(\mathbf{x}^m, \mathbf{x}^1) & k(\mathbf{x}^m, \mathbf{x}^2) & \cdots & k(\mathbf{x}^m, \mathbf{x}^m)\\]
\end{equation*}
jest dodatnio określoną macierzą symetryczną nazywamy funkcją kowariancji, jądrem dodatnio
określonym (z ang. \textit{positive definite kernel}) lub \textbf{jądrem Mercera}.
\end{definition}

Dla dwóch zbiorów punktów \(X = \{\mathbf{x}^1,\ldots,\mathbf{x}^m\} \subset \mathbb{R}^n\) i \(Y =
\{\mathbf{y}^1,\ldots,\mathbf{y}^s\} \subset \mathbb{R}^n\) i funkcji kowariancji \(k\) wprowadzimy
oznaczenie
\begin{equation}
    k(X,Y) := \mqty[k(\mathbf{x}^1, \mathbf{y}^1) & k(\mathbf{x}^1, \mathbf{y}^2) & \cdots & k(\mathbf{x}^1, \mathbf{y}^s)\\
    k(\mathbf{x}^2, \mathbf{y}^1) & k(\mathbf{x}^2, \mathbf{y}^2) & \cdots & k(\mathbf{x}^2, \mathbf{y}^s)\\
    \vdots & \vdots & \ddots & \vdots\\
    k(\mathbf{x}^m, \mathbf{y}^1) & k(\mathbf{x}^m, \mathbf{y}^2) & \cdots & k(\mathbf{x}^m, \mathbf{y}^s)\\]\,.
\end{equation}
Poniżej podajemy kilka przykładów funkcji kowariancji
\begin{itemize}
    \item \textit{Gaussian kernel} dla normy \(\norm{\cdot}\) i hiper-parametrów \(a,l\) (amplituda
    i skala długości)
    \begin{equation}
        k(\mathbf{x}, \mathbf{y}) = a^2\exp\left\{-\frac{1}{2l^2}\norm{\mathbf{x} - \mathbf{y}}^2\right\}
    \end{equation}
    
    \item \textit{Periodic kernel} dla normy \(\norm{\cdot}\) i hiper-parametrów \(a, l, p\)
    (amplituda, skala długości, okres zmienności)
    \begin{equation}
        k(\mathbf{x},\mathbf{y}) = a^2\exp\left\{-\frac{2}{l^2}\sin^2\left(\frac{\pi}{p}\norm{\mathbf{x} - \mathbf{y}}\right)\right\}
    \end{equation}

    \item \textit{White noise kernel} dla hiper-parametru \(\sigma\)
    \begin{equation}
        k(\mathbf{x},\mathbf{y}) = \sigma^2 \delta_{\mathbf{x},\mathbf{y}}
    \end{equation}

    \item \textit{Mat\'ern kernel} dla normy \(\norm{\cdot}\) i hiper-parametrów \(a, l, \nu\)
    (amplituda, skala długości, regularność)
    \begin{equation}
        k(\mathbf{x},\mathbf{y}) = a^2 \frac{2^{1-\nu}}{\Gamma(\nu)}\left(\frac{\sqrt{2\nu}}{l}\norm{\mathbf{x} - \mathbf{y}}\right)^\nu K_\nu\left(\frac{\sqrt{2\nu}}{l}\norm{\mathbf{x} - \mathbf{y}}\right)\,,
    \end{equation}
    gdzie \(\Gamma(x)\) to funkcja gamma Eulera, a \(K_\nu(x)\) to zmodyfikowana funkcja Bessela
    2-go rodzaju rzędu \(\nu\).

\end{itemize}

\begin{theorem}[Własności funkcji kowariancji]
Suma lub iloczyn dwóch funkcji kowariancji oraz złożenie funkcji kowariancji z wielomianem o
nieujemnych współczynnikach jest również funkcją kowariancji.
\end{theorem}

\begin{definition}[Procesu gaussowskiego]
Procesem Gaussowskim (z ang. \textit{Gaussian Process}) nazywamy rodzinę skalarnych zmiennych
losowych indeksowanych przez punkty \(\mathbf{x} \in \mathbb{R}^n\)
\begin{equation*}
    \mathcal{GP} = \left\{f_\mathbf{x} \mid \mathbf{x} \in \mathbb{R}^n\right\}
\end{equation*}
taką że każdy skończony podzbiór \(\mathcal{GP}\) ma łącznie wielowymiarowy rozkład normalny tj. dla
dowolnego zbioru \(X = \{\mathbf{x}^1, \ldots, \mathbf{x}^m\} \subset \mathbb{R}^n\) zachodzi
\begin{equation*}
    \mqty[f_{\mathbf{x}^1} \\ \vdots \\ f_{\mathbf{x}^m}] \sim \mathcal{N}(\boldsymbol{\mu}_X, \mathbf{\Sigma}_{X})\,.
\end{equation*}
\end{definition}

Zauważmy, iż proces Gaussowski możemy jednoznacznie zdefiniować podając przepisy na parametry
\(\boldsymbol{\mu}_X\) i \(\mathbf{\Sigma}_X\) dla dowolnego zbioru \(X\). W praktyce często
przyjmujemy \(\boldsymbol{\mu}_X = \mathbf{0}\), natomiast przepisem na macierz kowariancji może być
zdefiniowana wyżej funkcja kowariancji \(k(X,X)\) tj.
\begin{equation}
    \mqty[f_{\mathbf{x}^1} \\ \vdots \\ f_{\mathbf{x}^m}] \sim \mathcal{N}(\mathbf{0}, k(X,X))\,.
\end{equation}
Process Gaussowski daje nam w praktyce rozkład prawdopodobieństwa nad funkcjami
\(f:\mathbb{R}^n\mapsto\mathbb{R}\), których charakter jest określony przez jądro \(k\) (np. funkcja
gładka dla jądra Gaussowskiego, okresowa dla jądra periodycznego, itp.). Zauważmy, że nie
wnioskujemy tu o parametrach konkretnej rodziny funkcji (jak w przypadku regresji liniowej);
interesuje nas jedynie rozkład predykcyjny. Załóżmy, iż w dokładnie znanych przez nas punktach \(X =
\{\mathbf{x}^1,\mathbf{x}^2,\ldots,\mathbf{x}^m\}\) zaobserwowaliśmy wartości pewnej funkcji, o
których zakładamy, iż pochodzą z procesu Gaussowskiego zadanego jądrem \(k\), które wyraża nasze
założenia a priori co do charakteru badanej funkcji
\begin{equation}
    \mathbf{f}_X = \mqty[f_{\mathbf{x}^1} \\ \vdots \\ f_{\mathbf{x}^m}] \sim \mathcal{N}(\mathbf{0}, k(X,X))\,.
\end{equation}
Powiedzmy, iż chcemy znać wartości \(\mathbf{f}_Y\) tej funkcji w zadanych punktach \(Y =
\{\mathbf{y}^1,\mathbf{y}^2,\ldots,\mathbf{y}^s\}\). Ponieważ założyliśmy, iż wartości funkcji
pochodzą z procesu Gaussowskiego, więc rozkład łączny \(\mathbf{f}_X\) i \(\mathbf{f}_Y\) jest
rozkładem normalnym
\begin{equation}
    \mqty[\mathbf{f}_X \\ \mathbf{f}_Y] \sim \mathcal{N}\left(\mathbf{0}, \mqty[k(X,X) & k(X,Y) \\ k(Y,X) & k(Y,Y)]\right)\,.
\end{equation}
Zauważmy, iż jest to instancja modelu gaussowskiego (Tw. \ref{th:gauss_model}), więc rozkład
warunkowy \(\mathbf{f}_Y\mid \mathbf{f}_X\) jest również rozkładem normalnym o parametrach
\begin{equation}
    \begin{split}
        &\boldsymbol{\mu} = k(Y,X)k^{-1}(X,X)\mathbf{f}_X\\
        &\mathbf{\Sigma} = k(Y,Y) - k(Y,X)k^{-1}(X,X)k(X,Y)
    \end{split}\quad.
\end{equation}
Dodatkową niepewność związaną z pomiarem wartości \(\mathbf{f}_X\) możemy uchwycić zmieniając postać
jądra 
\begin{equation}
    k(\mathbf{x},\mathbf{y}) \leftarrow k(\mathbf{x},\mathbf{y}) + \mathcal{I}_X(\mathbf{x})\sigma^2\delta_{\mathbf{x},\mathbf{y}}\,,
\end{equation}
gdzie \(\sigma\) jest hiper-parametrem określającym precyzję pomiaru. Oczywiście \(k\) jest dalej
funkcją kowariancji, gdyż takie podstawienie powoduje jedynie dodanie dodatnich członów do pewnych
elementów diagonalnych macierzy kowariancji, więc macierz ta jest nadal symetryczna i dodatnio
określona. Wówczas rozkład predykcyjny ma parametry
\begin{equation}
    \begin{split}
        &\boldsymbol{\mu} = k(Y,X)\left[k(X,X) + \sigma^2\mathbf{1}\right]^{-1}\mathbf{f}_X\\
        &\mathbf{\Sigma} = k(Y,Y) - k(Y,X)\left[k(X,X) + \sigma^2\mathbf{1}\right]^{-1}k(X,Y)
    \end{split}\quad.
\end{equation}


\subsection{Metryki do oceny regresji i klasyfikacji}

Wcześniej nie wspominaliśmy nic o metodach sprawdzenia jakości modelu po wytrenowaniu. Zasadniczo,
aby ocenić predykcje modelu używamy odpowiednich metryk, których wartości określają jak dobry jest
model.

% TODO:...

\subsubsection{Regresja}

\subsubsection{Prognozowanie szeregów czasowych}

\subsubsection{Klasyfikacja binarna}

\subsubsection{Klasyfikacja wieloklasowa}


\subsection{Klasyfikator najbliższych sąsiadów}

Do tej pory zajmowaliśmy się problemami regresji ciągłej zmiennej skalarnej. Przechodzimy teraz do
metod uczenia maszynowego wykorzystywanych dla problemów klasyfikacji. Jako pierwszy rozważmy jeden
z najprostszych (ale bardzo mocnych) modeli -- klasyfikator \(k\) najbliższych sąsiadów (z ang.
\textit{k Nearest Neighbors, kNN}). Załóżmy, iż mamy zbiór obserwacji i.i.d. postaci \(\mathcal{X} =
\{y_1(\mathbf{x}^1), \ldots, y_n(\mathbf{x}^n)\}\) przy czym \(y\) może być zarówno skalarną zmienną
ciągłą jak i elementem skończonego zbioru klas. Zakładamy, że wektory cech \(\mathbf{x} \in
\mathbb{R}^d\), a \(\mathsf{d}: \mathbb{R}^d \times \mathbb{R}^d \mapsto I \subseteq \mathbb{R}\)
jest metryką, półmetryką lub pewną miarą podobieństwa między punktami \(\mathbb{R}^d\). Reguła
decyzyjna klasyfikatora \(k\) najbliższych sąsiadów polega na znalezieniu dla nowego wektora cech
\(\mathbf{t}\), \(k\) najbliższych względem funkcji \(\mathsf{d}\) punktów ze zbioru
\(\{\mathbf{x}^1,\ldots,\mathbf{x}^n\}\) i zwróceniu najczęstszej klasy dla tych sąsiadów. Metodę
najbliższych sąsiadów można również wykorzystać do regresji, gdzie zwracamy wartość średnią
arytmetyczną wartości \(y\) dla znalezionych \(k\) najbliższych sąsiadów (powoduje to, że model taki
potrafi tylko interpolować wartości, więc nie jest dobrym modelem dla regresji). W przypadku
klasyfikacji możemy natomiast zwracać również rozkład prawdopodobieństwa klas dla nowego wektora
cech \(\mathbf{t}\) przez podanie stosunków występowania danej klasy wśród \(k\) najbliższych
sąsiadów do \(k\).

Klasyfikator kNN jest bardzo elastycznym modelem z nieliniową granicą decyzyjną. Jakość klasyfikacji
silnie zależy od lokalnej gęstości punktów w przestrzeni \(\mathbb{R}^d\) oraz wybranej wartości
\(k\), będącej hiperparametrem tego modelu. Ogólnie niskie \(k\) powoduje, że kNN ma duży variance i
dość \enquote{poszarpaną} granicę decyzyjną, natomiast wysokie \(k\) powoduje, że kNN ma duży bias i
\enquote{gładką} granicę decyzyjną. Typowo wykorzystywane funkcje \(\mathsf{d}\) to
\begin{itemize}

    \item metryka euklidesowa \(\mathsf{d}(\mathbf{x},\mathbf{y}) = \sqrt{(\mathbf{x} -
    \mathbf{y})^T(\mathbf{x} - \mathbf{y})}\);

    \item pół-metryka euklidesowa \(\mathsf{d}(\mathbf{x},\mathbf{y}) = (\mathbf{x} -
    \mathbf{y})^T(\mathbf{x} - \mathbf{y})\);

    \item metryka Manhattan \(\mathsf{d}(\mathbf{x}, \mathbf{y}) = \sum_{i=1}^d |\mathbf{x}_i -
    \mathbf{y}_i|\);
    
    \item podobieństwo cosinusowe \(\mathsf{d}(\mathbf{x}, \mathbf{y}) =
    \frac{\mathbf{x}^T\mathbf{y}}{\norm{\mathbf{x}}_2 \norm{\mathbf{y}}_2}\)
    
\end{itemize}

Jedną z modyfikacji klasyfikatora kNN, który może polepszyć wyniki w przypadku, gdy w naszej
przestrzeni istnieją obszary, w których mamy małą gęstość punktów ze zbioru \(\mathcal{X}\) jest
\textbf{ważenie sąsiadów}, które polega na tym, iż prawdopodobieństwa danej klasy wśród \(k\)
sąsiadów obliczamy teraz jako średnią ważoną, gdzie wagą jest odwrotność odległości danego sąsiada
od nowego wektora cech \(w(\mathbf{t}, \mathbf{x}) = 1 / \mathsf{d}(\mathbf{t}, \mathbf{x})\).

W przypadku naiwnego kNN podczas treningu zapamiętujemy jedynie zbiór \(\mathcal{X}\) natomiast
naiwna implementacja predykcji ma złożoność czasową \(O(knd)\), przy założeniu, że złożoność
obliczenia funkcji \(\mathsf{d}\) dla pary punktów ma złożoność \(O(d)\). Jest to nieakceptowalna
złożoność, gdyż zwykle chcemy używać klasyfikatora kNN do setek milionów punktów. Dwa podejścia,
które stosuje się zwykle do rozwiązania tego problemu to: 
\begin{itemize}

    \item zbudowanie odpowiedniej struktury danych w fazie treningu, aby w czasie predykcji można
    było szybciej znajdywać najbliższych sąsiadów (np. k-d tree, ball tree);

    \item wykorzystanie algorytmów aproksymacyjnych (z ang. \textit{Approximate Nearest Neighbors,
    ANN}) do znajdowania sąsiadów, którzy niekoniecznie naprawdę są najbliżsi, ale aproksymacja jest
    wystarczająco dobra do praktycznych zastosowań.

\end{itemize}

Obecnie to drugie podejście jest dominujące i wykorzystywane na przykład w przypadku dużych modeli
językowych do wyszukiwania kontekstów dla danych zapytań (z ang. \textit{Retrieval Augmented
Generation}).

\subsection{Naiwny klasyfikator bayesowski}


\subsection{Estymator jądrowy gęstości}


\subsection{Wieloklasowa regresja logistyczna}


\subsection{Klasyfikacja ekstremalnie niezbalansowana i anomaly detection}


\subsection{Próbkowanie Monte Carlo łańcuchami Markowa}

\subsubsection{Algorytm Metropolisa--Hastingsa}

\subsubsection{Próbkowanie Gibbsa}



\section{Uczenie głębokie i sieci neuronowe}

% Matematyczny model neuronu; Perceptron wielowarstwowy
% Universal Approximation Theorem
% Loss functions: MLE, univariate regression, multi-class classification, cross-entropy loss
% Aspekty optymalizacyjne (gradient descent, stochastic gradient descent, momentum, adam)
% Algorytm wstecznej propagacji błędu, automatyczne różniczkowanie
% Regularyzacja i inicjalizacja w sieciach głębokich
% Measuring performance
% Convolutional networks
% Residual networks
% Transformers
    
\end{document}
