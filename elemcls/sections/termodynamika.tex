\documentclass[../main.tex]{subfiles}

\begin{document}

\section{Termodynamika}
\textit{Nature, in providing us with combustibles on all sides, has given us the power to produce,
at all times and in all places, heat and the impelling power which is the result of it. To develop
this power, to appropriate it to our uses, is the object of heat-engines.}\begin{flushright}Nicolas
Léonard Sadi Carnot\end{flushright}

\begin{table}[h]
    \centering
    \begin{tabular}{ *5l }    \toprule \emph{Nazwa} & \emph{Symbol} & \emph{Wartość}  \\\midrule
    stała Avogadra    & \(N_\text{A}\)  & \(6.022\cdot 10^{23}\,\text{mol}^{-1}\)    \\ 
    stała Boltzmanna  & \(k_\text{B}\) & \(1.38\cdot 10^{-23}\,\text{J}\cdot\text{K}^{-1}\)  \\ 
    stała Faradaya  & \(F\) & \(96\,485\,\text{C}\cdot\text{mol}^{-1}\)  \\
    stała gazowa     & \(R\)  & \(8.314\,\text{J}\cdot\text{mol}^{-1}\cdot\text{K}^{-1}\)   \\ 
    stała Stefana-Boltzmanna  & \(\sigma\) & \(5.67\cdot
    10^{-8}\,\text{W}\cdot\text{m}^{-2}\cdot\text{K}^{-4}\)\\ 
    
    \bottomrule
    \hline
\end{tabular}
\caption{Wybrane stałe fizykochemiczne}
\end{table}

\subsection{Termodynamika fenomenologiczna}
Opis fenomenologiczny zajmuje się związkami między makroskopowymi wielkościami \((p,V,T)\)
charakteryzującymi układ jako całość. Na gruncie termodynamiki fenomenologicznej nie wyprowadzamy
opisywanych dalej praw z bardziej podstawowych zasad dynamiki. Traktujemy je jako fakty
doświadczalne.
\subsubsection*{Słowniczek pojęć}
\begin{enumerate}
    \item \textit{funkcja stanu} -- funkcja zależna wyłącznie od aktualnych parametrów układu
    \item \textit{równowaga termodynamiczna} -- stan, w którym wszystkie makroskopowe parametry
    układu oraz funkcje stanu są stałe w czasie
    \item \textit{temperatura} -- dowolna funkcja będąca bijekcją określona na klasie układów
    będących w równowadze termodynamicznej
\end{enumerate}

\subsubsection{Prawa gazowe}
W XIX w. Clapeyron sformułował równanie stanu gazu doskonałego na podstawie 3 empirycznych praw
odkrytych wcześniej.\\

\noindent\fbox{%
    \parbox{\textwidth}{%
        \textbf{Postulaty dotyczące gazu doskonałego}
        \begin{enumerate}[I.]
            \item Cząsteczki gazu są punktami materialnymi.
            \item Pomiędzy cząsteczkami nie występują żadne oddziaływania poza momentami ich zderzeń
            między sobą lub ściankami naczynia.
            \item Zderzenia cząsteczek są elastyczne.
        \end{enumerate}
    }%
}\\

Równanie Clapeyrona stosuje się również dobrze do gazów rzeczywistych w temperaturach wyższych od
temperatury krytycznej i przy niezbyt wysokich ciśnieniach. Wówczas prawdziwe są zależności:
\begin{enumerate}
    \item Prawo Boyle'a \((pV)_{T}=\text{const}\).
    \item Prawo Charlesa \(\left(\frac{p}{T}\right)_{V}=\text{const}\).
    \item Prawo Gay-Lussaca \(\left(\frac{V}{T}\right)_{p}=\text{const}\).
\end{enumerate}
oraz w ogólności spełnione jest \textbf{równanie Clapeyrona} (równanie stanu gazu doskonałego)
\begin{equation*}
    pV=nRT\,.
\end{equation*}

W przypadku mieszaniny \(L\) różnych gazów spełniających RSGD zachodzi
\begin{equation*}
    pV=\sum_{\alpha=1}^Ln_\alpha RT\,,\quad\text{gdzie}
\end{equation*}
\(p,V,T\) oznaczają parametry całej mieszaniny. Jest to równoważne \textbf{prawu Daltona}:
\(p=\sum_\alpha p_\alpha\) tzn. ciśnienie wypadkowe jest sumą ciśnień parcjalnych. Można o tym
myśleć w następujący sposób: ponieważ gazy doskonałe ze sobą nie oddziałują, więc cząsteczki tych
gazów \textit{nie widzą} się nawzajem, a zatem uśredniona siła z jaką cząsteczki danego gazu
uderzają o ścianki naczynia jest taka sama, jaka byłaby w tych samych warunkach przy nieobecności
cząsteczek innych gazów.
\subsubsection{Wzór barometryczny}
Zakładając, że powietrze tworzące atmosferę ziemską możemy traktować jak gaz doskonały o masie
molowej \(\mu\) i stałej temperaturze \(T\) możemy wyprowadzić teoretyczną zależność ciśnienia
atmosferycznego od wysokości \(h\) nad poziomem morza (wzór barometryczny). Z RSGD mamy
\begin{equation*}
    p(h)=\frac{\rho(h)}{\mu}RT\,.
\end{equation*}
Rozpatrzmy warstwę atmosfery na wysokości \(h\) nad poziomem morza o grubości \(\dd{h}\). Ponieważ
warstwa ta znajduje się w równowadze, więc musi zachodzić
\begin{equation*}
    p+\dd{p}+\rho(h) g\dd{h}=p\,,
\end{equation*}
skąd otrzymujemy
\begin{equation*}
    \dd{p}=-\rho(h) g\dd{h}=-\frac{\mu g}{RT}p\dd{h}\,.
\end{equation*}
Wykonując elementarne całkowanie i przyjmując \(p(0)=p_0\) otrzymujemy
\begin{equation*}
    p(h)=p_0\exp\left\{-\frac{\mu g}{RT}h\right\}\,.
\end{equation*}
\subsubsection{Para nasycona i wilgotność}
\textbf{Para nasycona} -- para pozostająca w równowadze termodynamicznej ze swoją cieczą. Jej
ciśnienie ma w danej temperaturze maksymalną wartość i nie zmienia się przy zmianie objętości.
Zwiększenie objętości w stałej temperaturze powoduje parowanie cieczy bez zmiany ciśnienia, a
zmniejszenie objętości skraplanie cieczy bez zmiany ciśnienia. Wrzenie cieczy następuje w takiej
temperaturze, w której ciśnienie pary nasyconej nad tą cieczą jest równe ciśnieniu zewnętrznemu.
Para nasycona nie spełnia praw gazów doskonałych.
\medskip

\noindent\textbf{Wilgotność bezwzględną} \(w_b\) definiujemy jako stosunek masy pary wodnej do
objętości gazu.

\subsubsection{Ciepło właściwe i przemiany fazowej}
\begin{enumerate}
    \item Ciepło właściwe \(c\) definiujemy jako \(c=\frac{\delta Q}{m\dd{T}}\)
    \item Ciepło molowe \(C\) definiujemy jako \(C=\mu c\), gdzie \(\mu\) jest masą molową
    \item Ciepło topnienia (krzepnięcia) definiujemy jako \(l=\pm\frac{\delta Q}{m}\)
    \item Ciepło parowania (skraplania) definiujemy jako  \(r=\pm\frac{\delta Q}{m}\)
\end{enumerate}
Ogrzewanie ciał stałych i cieczy zwykle przeprowadzamy przy ciśnieniu normalnym 1013.25 hPa i
wówczas w dużym zakresie temperatur ciepło właściwe tych substancji jest stałe. Ciepło właściwe
gazów zależy od typu przemiany jakiej podlega gaz.

\subsubsection{Zasada bilansu cieplnego}
Suma ilości ciepła przekazywanego przez ciała układu jest równa energii \(\dd{E}\) przekazywanej z
zewnątrz
\begin{equation*}
    \delta Q= \dd{E}\,.
\end{equation*}

\subsection{Termodynamika statystyczna}
\subsubsection{Podstawowy wzór kinetycznej teorii gazów}
Przyjmując wypisane postulaty dot. gazu doskonałego wyprowadzimy równanie wiążące makroskopowe
parametry z wielkościami mikroskopowymi. Rozpatrzmy sześcienny zbiornik o boku \(\ell\). Wewnątrz
zbiornika znajduje się gaz doskonały składający się z \(N\) cząsteczek, każda o masie \(m\).
Uśredniona po liczbie odbić siła z jaką pojedyncza cząstka działa na ściankę wynosi
\begin{equation*}
    F_\alpha^{(x)}=m\frac{\Delta v_\alpha^{(x)}}{\tau}=\frac{2mv_\alpha^{(x)}}{\tau}\,,
\end{equation*}
gdzie \(\tau\) jest czasem pomiędzy kolejnymi odbiciami tj. \(\tau=\ell/v_\alpha^{(x)}\). Sumując po
wszystkich cząstkach otrzymujemy więc
\begin{equation*}
    F=\frac{m}{\ell}\sum_{\alpha=1}^{N}(v_\alpha^{x})^2=\frac{Nm\langle v_x^2\rangle}{\ell}\,,
\end{equation*}
gdzie wprowadziliśmy \(\langle v_x^2\rangle
=\frac{(v_1^{(x)})^2+(v_2^{(x)})^2+...+(v_N^{(x)})^2}{N}\). Ciśnienie wywierane na ściankę naczynia
wynosi więc
\begin{equation*}
    p=\frac{Nm\langle v_x^2\rangle }{\ell ^3}=\frac{Nm\langle v_x^2\rangle }{V}\,.
\end{equation*}
Ponieważ żaden z kierunków \(x,y,z\) nie jest wyróżniony, więc możemy przyjąć \(\langle
v_x^2\rangle=\langle v_y^2\rangle=\langle v_z^2\rangle\). Jednocześnie zauważmy, że zachodzi
\(\langle v^2\rangle=\langle v_x^2\rangle+\langle v_y^2\rangle+\langle v_z^2\rangle\), zatem
otrzymujemy
\begin{equation*}
    p=\frac{Nm\langle v^2\rangle}{3V}=\frac{2N}{3V}\langle K\rangle\,.
\end{equation*}

\subsubsection{Kinetyczna interpretacja temperatury}
Temperatura bezwzględna \(T\) jest wielkością wprost proporcjonalną do średniej energii kinetycznej
cząstek. Dla gazu doskonałego
\begin{equation*}
    T=\frac{2}{3k_B}\langle K\rangle\,.
\end{equation*}
\subsubsection{Zasada ekwipartycji energii}
Na każdy stopień swobody cząstki przypada taka sama średnia energia równa \(k_BT/2\), skąd całkowita
energia wewnętrzna gazu, którego cząsteczka ma \(i\) stopni swobody wynosi
\begin{equation*}
    U=\frac{Nik_BT}{2}=\frac{inRT}{2}\,.
\end{equation*}

\subsection{Zasady termodynamiki}

\noindent\fbox{%
    \parbox{\textwidth}{%
        \textbf{0 zasada termodynamiki.}\\
        Istnieje funkcja stanu \(\beta\), którą nazywamy temperaturą, taka że jeżeli układ A jest w
    równowadze termicznej z układem B, wówczas wartość funkcji \(\beta\) jest taka sama dla obu tych
    układów. }%
}
\medskip

\noindent\fbox{%
    \parbox{\textwidth}{%
        \textbf{I zasada termodynamiki.}\\
        Istnieje funkcja stanu \(U\), którą nazywamy energią wewnętrzną, taka że różnica
        elementarnego ciepła \(\delta Q\) dostarczonego do układu i elementarnej pracy \(\delta W\)
        wykonanej przez układ jest różniczką zupełną tej funkcji
        \begin{equation*}
            \dd{U}=\delta Q-\delta W\,.
        \end{equation*}
    }%
}
\medskip

\noindent\fbox{%
    \parbox{\textwidth}{%
        \textbf{II zasada termodynamiki.}\\
        Sformułowanie Kelvina-Plancka:\\
        \textit{It is impossible to devise a cyclically operating device, the sole effect of which is to absorb energy in the form of heat from a single thermal reservouir and to deliver an equivalent amount of work.}
    }%
}\\

\subsection{Izoprocesy gazu doskonałego}
\subsubsection{Procesy odwracalne}
Proces nazywamy odwracalnym, gdy za pomocą różniczkowej zmiany otoczenia można wywołać proces do
niego odwrotny tj. przebiegający po tej samej drodze w kierunku przeciwnym.
\subsubsection{Procesy kwazistatyczne}
Mówiąc o izoprocesach gazu doskonałego chcemy opisać dynamikę zmian stanu układu jakim jest gaz
doskonały. Zauważmy jednak, że podane wcześniej na gruncie termodynamiki fenomenologicznej
aksjomatyczne prawa (gazowe i Clapeyrona) opisują stany równowagi termodynamicznej -- statykę układu
jakim jest gaz doskonały. Z tego powodu są raczej domeną \textit{termostatyki} niż termodynamiki.
Zmieniając stan układu nie możemy mieć pewności, że w każdej chwili przy przejściu między stanami
\(A\), \(B\) parametry \((p,V,T)\) są dobrze ustalone (mogą być np. pewnymi polami skalarnymi). Z
pewnością natomiast w punktach \(A\) i \(B\) po odpowiednio długim czasie ustalą się stany równowagi
i wówczas będziemy mogli użyć opisanych praw. Chcąc jednak za ich pomocą opisać dynamikę procesu
musimy przeprowadzać ów proces kwazistatycznie -- bardzo powoli (w granicy nieskończenie wolno), aby
w każdej chwili mógł ustalić się stan równowagi. Takimi procesami zajmujemy się poniżej.
\subsubsection{Izoprocesy}
\begin{enumerate}
    \item \textbf{Przemiana izochoryczna} (\(V=\text{const}\))
    
    \begin{enumerate}
        \item Z I zasady termodynamiki \(\delta Q=\dd{U}\)
        \item Ponieważ \(\Delta V=0\), więc W=0
        \item Definiujemy izochoryczne ciepło molowe 
        \begin{equation*}
            C_V=\frac{\delta Q}{n\dd{T}}=\frac{\dd{U}}{n\dd{T}}\,.
        \end{equation*}
    \end{enumerate}
    
    \item \textbf{Przemiana izobaryczna} (\(p=\text{const}\))
    
    \begin{enumerate}
        \item Z I zasady termodynamiki \(\delta Q=nC_V\dd{T}+p\dd{V}\)
        \item Ponieważ \(p=\text{const}\), więc \( W=\int p\dd{V}=p\Delta V\)
        \item Definiujemy izobaryczne ciepło molowe 
        \begin{equation*}
            C_p=\frac{\delta Q}{n\dd{T}}=\frac{\dd{U}+p\dd{V}}{n\dd{T}}=C_V+\frac{p}{n}\dv{}{T} \left(\frac{nRT}{p}\right)=C_V+R\,.
        \end{equation*}
    \end{enumerate}
    
    \item \textbf{Przemiana izotermiczna} (\(T=\text{const}\))
    
    \begin{enumerate}
        \item Z I zasady termodynamiki \(\delta Q=p\dd{V}\)
        \item Ponieważ \(T=\text{const}\), więc \(p=p(V)=nRT\frac{1}{V}\), skąd
        \begin{equation*}
            W=\int_{V_1}^{V_2}p\dd{V}=nRT\int_{V_1}^{V_2}\frac{\dd{V}}{V}=nRT\ln\frac{V_2}{V_1}\,.
        \end{equation*}
        \item Ponieważ temperatura się nie zmienia nie można zdefiniować analogicznego ciepła
        molowego (w pewnym sensie \(C_T=\infty\)).\\
    \end{enumerate}
    
    \item \textbf{Przemiana adiabatyczna} (\(\delta Q=0\))
    
    \begin{enumerate}
        \item Z I zasady termodynamiki \(\dd{U}=nC_V\dd{T}=-p\dd{V}\). Z RSGD mamy natomiast
        \(p\dd{V}+V\dd{p}=nR\dd{T}\). Z powyższego mamy więc
        \begin{equation*}
            p\dd{V}-nR\dd{T}=p\dd{V}+\frac{R}{C_V}p\dd{V}=-V\dd{p}\,,
        \end{equation*}
        czyli wprowadzając współczynnik Laplace'a \(\kappa=\frac{C_p}{C_V}\) otrzymujemy
        \begin{equation*}
            \kappa \frac{\dd V}{V}=-\frac{\dd{p}}{p}\,,
        \end{equation*}
        skąd uzyskujemy tzw. równanie Poissona
        \begin{equation*}
            pV^\kappa=\text{const}\,.
        \end{equation*}
        Równoważnie możemy napisać \(TV^{\kappa-1}=\text{const}\) lub
        \(p^{1-\kappa}T^\kappa=\text{const}\).
        \item Ponieważ \(\dd{U}=-\delta W\), więc
        \begin{equation*}
            W=-\Delta U=nC_V(T_1-T_2)=\frac{C_V}{R}(p_1V_1-p_2V_2)=\frac{p_1V_1-p_2V_2}{\kappa-1}\,.
        \end{equation*}
        \item Ponieważ układ jest w osłonie adiabatycznej nie można zdefiniować analogicznego ciepła
        właściwego (w pewnym sensie \(C_Q=0\)).
        \item Obliczmy zmianę entropii \(\Delta S\) w przemianie adiabatycznej
        \begin{equation*}
            \Delta S=nC_V\ln\frac{T_2}{T_1}+nR\ln\frac{V_2}{V_1}\,,
        \end{equation*}
        ale z równania Poissona \(T_2/T_1=(V_1/V_2)^{\kappa-1}\), zatem
        \begin{equation*}
            \Delta S=(\kappa -1)nC_V\ln\frac{V_1}{V_2}-nR\ln\frac{V_1}{V_2}=nR\ln\frac{V_1}{V_2}-nR\ln\frac{V_1}{V_2}=0\,,
        \end{equation*}
        z tego powodu przemiana adiabatyczna gazu doskonałego jest nazywana również
        \textit{przemianą izentropową}.
    \end{enumerate}
\end{enumerate}

\subsection{Entropia}
Dla procesów odwracalnych zachodzących quasistatycznie entropię \(S\) definiujemy jako pewną funkcję
stanu, której różniczka zupełna wynosi
\begin{equation*}
    \dd{S}=\frac{\delta Q}{T}\,.
\end{equation*}
\subsubsection{Zmiana entropii w przemianie gazu doskonałego}
Rozpatrzmy pewną przemianę gazu doskonałego przedstawioną w przestrzeni \(T-V\) jako przejście
między punktami \(A(T_A,V_A)\to B(T_B,V_B)\). Dla gazu doskonałego mamy
\begin{equation*}
    \dd{S}=\frac{\dd{U+p\dd{V}}}{T}=nC_V\frac{\dd{T}}{T}+nR\frac{\dd{V}}{V}\,.
\end{equation*}
Zauważmy, że wyrażenie po prawej stronie ma postać
\begin{equation*}
    \left[\frac{nC_V}{T},\frac{nR}{V}\right]\cdot\left[\dd{T},\dd{V}\right]=\mathbf{F}\cdot \dd{\mathbf{l}}\,.
\end{equation*}
Zmiana entropii w rozpatrywanym procesie jest więc równa całce krzywoliniowej w przestrzeni \(T-V\)
\begin{equation*}
    \Delta S=\int_A^B\mathbf{F}\cdot \dd{\mathbf{l}}\,.
\end{equation*}
Łatwo sprawdzić jednak, że \(\nabla\times\mathbf{F}=0\), zatem z twierdzenia dotyczącego pól
bezwirowych wiemy, że owa całka nie zależy od krzywej po jakiej poruszamy się między punktami \(A\)
i \(B\), co jest w pewien sposób oczywiste gdyż \(S\) musi być funckją stanu. Wybierając krzywą
będącą fragmentem prostokąta o bokach równoległych do osi układu współrzędnych mamy
\begin{equation*}
    \Delta S=nC_V\ln\frac{T_B}{T_A}+nR\ln\frac{V_B}{V_A}\,.
\end{equation*}
Powyższy wzór jest prawdziwy dla dowolnej przemiany gazu doskonałego, nawet takiej, której nie da
się jednoznacznie przedstawić (poza oczywiście punktami \(A\), \(B\)) w przestrzeni \(V-T\), gdyż
np. temperatura nie musi być jednakowa w każdym punkcie gazu. W szczególności obliczmy zmianę
entropii podczas rozprężania gazu doskonałego do próżni zakładając, że zwiększył on swoją objętość
\(P\)--krotnie
\begin{equation*}
    \Delta S=nC_V\ln\frac{T_A}{T_A}+nR\ln\frac{PV_A}{V_A}=Nk_B\ln P=k_\text{B}\ln({P^N})\,.
\end{equation*}

\subsection{Procesy cykliczne}
\textbf{Proces cykliczny} -- proces, w którym gaz roboczy po wykonaniu pełnego cyklu wraca do stanu
początkowego. Energia wewnętrzna w takim cyklu nie może się zmienić, gdyż jest ona funkcją stanu.
\medskip

\noindent\textbf{Cykl Carnota}-- idealny teoretyczny proces cykliczny składający się z 4 izoprzemian
gazu doskonałego: rozprężania izotermicznego (1--2) i adiabatycznego (2--3) oraz sprężania
izotermicznego (3--4) i adiabatycznego (4--1). Wyznacza granicę sprawności maszyn cieplnych.

\begin{figure}[ht]
    \centering
\begin{tikzpicture}
[
% Corner style
    corner/.style={ circle, fill,   
        inner sep=1pt},
% Line with Arrow style
    arrowline/.style={ thick, postaction = decorate, decoration = {markings, mark = at position .6
        with \arrow{latex} } },
% Pin style
    every pin/.style = { black!80, inner sep = 1mm, align = center, font = \footnotesize, pin edge =
        {-latex, thin, line to}}, ]
 
% Draw axis with labels
\draw [latex-latex] (0,5) node [right] {\(p\)} |- (5,0) node [below] {\(V\)};
                
% Cycle corners
\node[corner, label = {left:$1$}] (1) at (1,4){} ;
 
\node[corner, label = {right:$2$}] (2) at (3.5,3){} ;
 
\node[corner, label = {right:$3$}] (3) at (4.5,1){} ;
 
\node[corner, label = {below left:$4$}] (4) at (2,1.5){} ;
 
% Curved lines
\draw [arrowline] (1) to [bend right = 10] node [pos = .4] {} (2);
 
\draw [arrowline] (2) to [bend right = 20] node [pos = .4] {}(3);
 
\draw [arrowline] (3) to [bend left  = 10] node [pos = .4]{}(4);
 
\draw [arrowline] (4) to [bend left  = 20] node [pos = .4] {} (1);
 
\end{tikzpicture}
    \caption{Silnik Carnota w przestrzeni \(p-V\)}
    \label{fig:my_label}
\end{figure}
\medskip

\noindent Praca wykonana w całym cyklu wynosi
\begin{equation*}
    W=nRT_1\ln\frac{V_2}{V_1}+nC_V(T_2-T_3)+nRT_3\ln\frac{V_4}{V_3}+nC_V(T_4-T_1)\,,
\end{equation*}
ale \(T_1=T_2\) i \(T_3=T_4\) oraz \(T_2V_2^{\kappa -1}=T_3V_3^{\kappa-1}\) i
\(T_1V_1^{\kappa-1}=T_4V_4^{\kappa-1}\), skąd
\begin{equation*}
    \frac{T_2}{T_3}=\left(\frac{V_3}{V_2}\right)^{\kappa-1}=\frac{T_1}{T_4}=\left(\frac{V_4}{V_1}\right)^{\kappa-1}\,,
\end{equation*}
zatem z powyższego \(V_3/V_2=V_4/V_1\), czyli \(V_2/V_1=V_3/V_4\). Oznaczając \(T_1=T_2=T_H\) i
\(T_3=T_4=T_L\) otrzymujemy
\begin{equation*}
    W=nR(T_H-T_L)\ln\frac{V_2}{V_1}\,.
\end{equation*}
Ciepło przekazywane jest tylko w przemianach (1--2) i (3--4) i wynosi odpowiednio
\begin{equation*}
\begin{split}
    &Q_{1-2}=W_{1-2}=nRT_H\ln\frac{V_2}{V_1}>0\quad\text{gdyż \(V_2>V_1\)}\,,\\
    &Q_{3-4}=W_{3-4}=nRT_L\ln\frac{V_4}{V_3}<0\quad\text{gdyż \(V_4<V_3\)}\,.
\end{split}
\end{equation*}
Definiując sprawność \(\eta\) silnika jako stosunek
\begin{equation*}
    \eta=\frac{W}{Q_\text{in}}\,,
\end{equation*}
gdzie \(Q_\text{in}=\sum_\alpha Q_\alpha : \forall_\alpha Q_\alpha >0\), otrzymujemy
\begin{equation*}
    \eta_\text{Carnot}=\frac{nR(T_H-T_L)\ln\frac{V_2}{V_1}}{nRT_H\ln\frac{V_2}{V_1}}=1-\frac{T_L}{T_H}\,.
\end{equation*}
\noindent\fbox{%
    \parbox{\textwidth}{%
        \textbf{Twierdzenie Carnota}
        \begin{enumerate}[I.]
            \item Sprawność dowolnego silnika cieplnego pobierającego izotermicznie ciepło ze
            zbiornika o temperaturze \(T_H\) i oddającego izotermicznie ciepło do zbiornika o
            temperaturze \(T_L\) nie może być większa od sprawności silnika Carnota pracującego
            między tymi temperaturami.
            
            \item Wszystkie silniki odwracalne (czynnikiem roboczym \textbf{nie musi} być gaz
            doskonały) pobierające izotermicznie ciepło ze zbiornika o temperaturze \(T_H\) i
            oddające izotermicznie ciepło do zbiornika o temperaturze \(T_L\) mają jednakową
            sprawność równą sprawności silnika Carnota pracującego między tymi temperaturami.
           
        \end{enumerate}
    }%
}
\medskip

Cykl Carnota można prowadzić w kierunku przeciwnym i wtedy urządzenie działa jako maszyna chłodząca.
Definiując wówczas wielkości
\begin{equation*}
\begin{split}
    &\xi_\text{cool}=\frac{|Q_\text{cold}|}{|W|}\\
    &\xi_\text{heat}=\frac{|Q_\text{hot}|}{|W|}\,,
\end{split}
\end{equation*}
można pokazać, że analogicznie dla dowolnej chłodziarki zachodzi
\begin{equation*}
    \begin{split}
        &\xi_\text{cool}\leq \frac{T_L}{T_H-T_L}\\
        &\xi_\text{heat}\leq\frac{T_H}{T_H-T_L}\,,
    \end{split}
\end{equation*}
gdzie \(Q_\text{cool}=\sum_\alpha Q_\alpha : \forall_\alpha Q_\alpha >0\), a
\(Q_\text{heat}=\sum_\alpha Q_\alpha : \forall_\alpha Q_\alpha <0\)\,.

\subsection{Sprężyste i termiczne własności ciał stałych i cieczy}
\subsubsection{Współczynniki rozszerzalności}
\begin{enumerate}
    \item Dla ciał stałych definiujemy współczynnik rozszerzalności liniowej jako
\begin{equation*}
    \lambda=\frac{1}{\ell_0}\dv{\ell}{T}\,,
\end{equation*}
gdzie \(\dd\ell/\ell_0\) jest względną zmianą długości spowodowaną przyrostem temperatury o
\(\dd{T}\).
\item Dla ciał stałych i cieczy definiujemy również współczynnik rozszerzalności objętościowej jako
\begin{equation*}
    \alpha=\frac{1}{V_0}\dv{V}{T}\,,
\end{equation*}
gdzie \(\dd V/V_0\) jest względną zmianą objętości spowodowaną przyrostem temperatury o \(\dd{T}\).
\item Dla ciał izotropowych zachodzi \(\alpha=3\lambda\). Istotnie rozpatrzmy prostopadłościan o
krawędziach \(x\), \(y\), \(z\). Jeśli krawędzie wydłużą się odpowiednio o \(\dd x\), \(\dd y\) i
\(\dd z\) to zmiana objętości wyniesie
\begin{equation*}
\begin{split}
    \dd{V}&=(x+\dd x)(y+\dd y)(z+\dd z)-xyz\\
    &=xyz+xy\dd{z}+xz\dd{y}+yz\dd{x}-xyz\,,
\end{split}
\end{equation*}
skąd podstawiając \(\dd{x}=\lambda x\dd{T}\), \(\dd{y}=\lambda y\dd{T}\) i \(\dd{z}=\lambda
z\dd{T}\) otrzymujemy
\begin{equation*}
    \dd{V}=3xyz\lambda \dd{T=3\lambda V_0\dd{T}}\,,
\end{equation*}
skąd \(\alpha=3\lambda\).
\item Dla gazów definiujemy współczynnik rozszerzalności objętościowej przy stałym ciśnieniu
\begin{equation*}
    \alpha_p=\frac{1}{V_0}\left(\dv{V}{T}\right)_p
\end{equation*}
oraz współczynnik prężności termicznej gazu przy stałej objętości
\begin{equation*}
    \beta_V=\frac{1}{p_0}\left(\dv{p}{T}\right)_V\,,
\end{equation*}
gdzie \(p_0\), \(V_0\) oznaczają odpowiednio ciśnienie i objętość danego gazu w temperaturze
\(T_0=273\) K. Dla gazów doskonałych zachodzi
\begin{equation*}
    \left(\dv{V}{T}\right)_p=\frac{nR}{p}=\frac{V_0}{T_0}\quad\text{oraz}\quad \left(\dv{p}{T}\right)_V=\frac{nR}{V}=\frac{p_0}{T_0}\,,
\end{equation*}
skąd
\begin{equation*}
    \alpha_p=\beta_V=\frac{1}{T_0}=\frac{1}{273\,\text{K}}\,.
\end{equation*}
\end{enumerate}
 \subsubsection{Prawo Hooke'a}
 Zależność między wydłużeniem pręta \(\Delta l\), a działającą na niego siłą \(F\) jest dana w
 postaci
 \begin{equation*}
     \Delta l =\frac{Fl_0}{YS}\,,
 \end{equation*}
 gdzie \(l_0\) jest długością swobodną pręta, \(S\) polem przekroju poprzecznego, a \(Y\) to moduł
 Younga. Dla dwóch połączonych prętów o długościach swobodnych \(l_1\), \(l_2\) i przekrojach
 \(S_1\), \(S_2\) oraz modułach Younga \(Y_1\), \(Y_2\) zachodzi
 \begin{equation*}
     \Delta l=F\frac{l_1Y_2S_2+l_2Y_1S_1}{Y_1Y_2S_1S_2}\,.
 \end{equation*}
\subsubsection{Heat equation*}
Przepływ ciepła w nieskończonym ośrodku o współczynniku przewodzenia ciepła \(\kappa\) opisywany
jest równaniem Fouriera
\begin{equation*}
    \oint_\mathcal{S}-\nabla T(\mathbf{r})\cdot \dd{\mathbf{S}}=\frac{1}{\kappa}\pdv{Q}{t}=\frac{P}{\kappa}\,,
\end{equation*}
gdzie \(T(\mathbf{r})\) jest polem skalarnym opisującym rozkład temperatury w przestrzeni, a \(P\)
mocą ciepła przepływającego przez pewną zamkniętą powierzchnię \(\mathcal{S}\). Zauważmy, że jest to
analogiczne równanie do prawa Gaussa w elektrostatyce. Istotnie wprowadzając pole
\(\mathbf{E}=-\nabla T\) mamy oczywiście
\begin{equation*}
    \nabla\times \mathbf{E}=0\quad\text{oraz}\quad \nabla\cdot \mathbf{E}=\frac{\varrho}{\kappa}\,,
\end{equation*}
gdzie \(\varrho\) jest gęstością mocy w \(\text{W}\cdot\text{m}^{-3}\). Pole \(\mathbf{E}\) można
rozumieć jako pole generowane przez \textit{ładunki} -- punktowe źródła ciepła. Wszelkie zagadnienia
dotyczące rozkładu temperatury w przestrzeni dla różnych rozkładów źródeł można zatem rozwiązywać
analogicznie jak w elektrostatyce. W szczególności znając rozwiązanie elektrostatyczne dla danego
problemu możemy podać rozwiązanie jego termodynamicznego odpowiednika. Ciekawe są tutaj problemy
dające się rozwiązać metodą obrazów. Jako przykład możemy podać rozkład temperatury w nieskończonej
przestrzeni o współczynniku przewodzenia \(\kappa\), w której umieściliśmy punktowe źródło ciepła o
mocy \(P\). Mamy
\begin{equation*}
    \mathbf{E}=\frac{P}{4\pi\kappa r^3}\mathbf{r}\,,
\end{equation*}
skąd przyjmując \(T(r\to\infty)=0\) otrzymujemy
\begin{equation*}
    T(\mathbf{r})=\frac{P}{4\pi\kappa r}\,.
\end{equation*}
\subsubsection{Płyny}
Płyn to każda substancja, która charakteryzuje się łatwością zmieniania wzajemnego położenia
elementów nawet dla niewielkich sił.
\medskip

\noindent\fbox{%
    \parbox{\textwidth}{%
        \textbf{Prawo Pascala}\\
        Ciśnienie wywierane na zamknięty płyn jest przekazywane równomierne na wszystkie części
    płynu oraz ścianki naczynia. }%
}
\medskip

\noindent\fbox{%
    \parbox{\textwidth}{%
        \textbf{Prawo Archimedesa}\\
        Na ciało zanurzone w cieczy działa siła wyporu równa co do wartości ciężarowi cieczy
    wypartej przez to ciało. }%
}
\medskip

Ciśnienie hydrostatyczne zależy tylko od wysokości słupa cieczy nad danym punktem i wynosi \(\rho
gh\).
\medskip

Napięcie powierzchniowe \(\sigma\) definiujemy jako stosunek siły stycznej do powierzchni cieczy i
długości konturu tej cieczy
\begin{equation*}
    \sigma=\frac{F}{l}\,.
\end{equation*}
Zakrzywiona powierzchnia cieczy wytwarza ciśnienie \(\Delta p\) dane wzorem Younga-Laplace'a
\begin{equation*}
    \Delta p=\sigma \dv{A}{V}\,,
\end{equation*}
gdzie \(\dd{A}\) jest zmianą powierzchni cieczy spowodowaną zmianą jej objętości o \(\dd{V}\). Dla
sferycznej bańki o promieniu \(R\) i grubości powłoki \(h\ll R\) mamy \(\Delta p=4\sigma/R\).
\medskip

Dla płynu doskonałego (nieściśliwego i nielepkiego), prawdziwe jest, że podczas jego przepływu
\(Sv=\text{const}\) (rów. ciągłości). Jeżeli przepływ płynu doskonałego jest ustalony i bezwirowy to
spełnione jest równanie Bernoulliego
\begin{equation*}
    p_\text{zew}+\rho gh+\frac{1}{2}\rho v^2=\text{const}\,.
\end{equation*}
Jeśli \(E\) oznacza energię mechaniczną układu o zmiennej masie, a \(v_\text{out}\) szybkość
chwilową ubywającej masy \(\dd{m}\) to zachodzi
\begin{equation*}
    \dd{E}+\frac{1}{2}v_\text{out}^2\dd{m}=0\,.
\end{equation*}
\end{document}