\documentclass{myclass}

\begin{document}

\subsection*{Zadanie 8.} Liczby rzeczywiste \(x,y,z,a,b,c\) spełniają równości
\[
\begin{cases}
&a^2 + 2bc = x^2 + 2yz\\
&b^2 + 2ca = y^2 + 2zx\\
&c^2 + 2ab = z^2 + 2xy
\end{cases}\quad.
\]
Wykazać, że \(a^2 + b^2 + c^2 = x^2 + y^2 + z^2\).

\bigskip

\textit{Rozwiązanie.} Zdefiniujmy macierze \(\bm{P}_1, \bm{P}_2, \bm{P_3}\)
\[
\bm{P}_1 := \mqty(1&0&0\\0&0&1\\0&1&0)\,,\quad \bm{P}_2 := \mqty(0&0&1\\0&1&0\\1&0&0)\,,\quad \bm{P}_3 := \mqty(0&1&0\\1&0&0\\0&0&1)
\]
oraz wektory \(\bm{a} := \mqty(a&b&c)^T\), \(\bm{x}:=\mqty(x&y&z)^T\). Zadany w treści układ
równości możemy wówczas zapisać jako
\[
\begin{cases}\label{eq:sys}
&\bm{a}^T\bm{P}_1\bm{a} = \bm{x}^T\bm{P}_1\bm{x}\\
&\bm{a}^T\bm{P}_2\bm{a} = \bm{x}^T\bm{P}_2\bm{x}\\
&\bm{a}^T\bm{P}_3\bm{a} = \bm{x}^T\bm{P}_3\bm{x}
\end{cases}\tag{$\star$}\quad.
\]
Zdefiniujmy macierz \(\bm{R}\) następująco
\[
\bm{R} := \frac{1}{\sqrt{6}}\mqty(\sqrt{2}&\sqrt{2} &\sqrt{2}\\
                                  1       &1        &-2      \\
                                  \sqrt{3}&-\sqrt{3}&0         )
\]
Zauważmy, że zachodzi \(\bm{R}\bm{R}^T = \bm{R}^T\bm{R} = \bm{I}_3\), gdzie \(\bm{I}_3\) oznacza
macierz jednostkową wymiaru \(3\times3\). W takim razie, korzystając z łączności
mnożenia macierzy, układ równań (\ref{eq:sys}) jest równoważny
\[
\begin{cases}
&(\bm{a}^T\bm{R}^T)(\bm{R}\bm{P}_1\bm{R}^T)(\bm{R}\bm{a}) = (\bm{x}^T\bm{R}^T)(\bm{R}\bm{P}_1\bm{R}^T)(\bm{R}\bm{x})\\
&(\bm{a}^T\bm{R}^T)(\bm{R}\bm{P}_2\bm{R}^T)(\bm{R}\bm{a}) = (\bm{x}^T\bm{R}^T)(\bm{R}\bm{P}_2\bm{R}^T)(\bm{R}\bm{x})\\
&(\bm{a}^T\bm{R}^T)(\bm{R}\bm{P}_3\bm{R}^T)(\bm{R}\bm{a}) = (\bm{x}^T\bm{R}^T)(\bm{R}\bm{P}_3\bm{R}^T)(\bm{R}\bm{x})\\
\end{cases}\quad.
\]
Definiując macierze \(\bm{Q}_1,\bm{Q}_2,\bm{Q}_3\) 
\[
\begin{split}
&\bm{Q}_1 := \bm{R}\bm{P}_1\bm{R}^T = \frac{1}{6}\mqty(6&0&0\\0&-3&3\sqrt{3}\\0&3\sqrt{3}&3)\\
&\bm{Q}_2 := \bm{R}\bm{P}_2\bm{R}^T = \frac{1}{6}\mqty(6&0&0\\0&6&0\\0&0&-6)\\
&\bm{Q}_3 := \bm{R}\bm{P}_3\bm{R}^T = \frac{1}{6}\mqty(6&0&0\\0&-3&-3\sqrt{3}\\0&-3\sqrt{3}&3)
\end{split}
\]
oraz wektory \(\bm{f} = \mqty(f&g&h)^T := \bm{Ra}\) i \(\bm{u} = \mqty(u&v&w)^T := \bm{Rx}\), możemy
zapisać powyższy układ równań jako
\[
\begin{cases}
&\bm{f}^T\bm{Q}_1\bm{f} = \bm{u}^T\bm{Q}_1\bm{u}\\
&\bm{f}^T\bm{Q}_2\bm{f} = \bm{u}^T\bm{Q}_2\bm{u}\\
&\bm{f}^T\bm{Q}_3\bm{f} = \bm{u}^T\bm{Q}_3\bm{u}
\end{cases}\quad.
\]
Dodając wszystkie równania stronami i rozpisując pierwsze i drugie równanie otrzymujemy odpowiednio
\[
\begin{cases}
&f^2 = u^2\\
&f^2-\frac{1}{2}g^2 + \frac{1}{2}h^2 + \sqrt{3}gh = u^2-\frac{1}{2}v^2 + \frac{1}{2}w^2 + \sqrt{3}vw\\
&f^2 + g^2 - h^2 = u^2 + v^2 - w^2
\end{cases}\quad,
\]
skąd
\[
\begin{cases}
&f^2 = u^2\\
&g^2 - h^2 = v^2 - w^2\\
&gh = vw
\end{cases}\quad.
\]
Pokażemy, że dwie ostatnie równości implikują \(g^2 = v^2\) i \(h^2 = w^2\). Istotnie z trzeciej
równości wynika \((gh)^2 - (vw)^2 = 0\). Jednocześnie zauważmy, że
\[
2(gh)^2 - 2(vw)^2 = (g^2 - v^2)(h^2 + w^2) + (g^2 + v^2)(h^2 - w^2)\,,
\]
zatem mamy
\[
\begin{cases}
&g^2 - h^2 = v^2 - w^2\\
&(g^2 - v^2)(h^2 + w^2) + (g^2 + v^2)(h^2 - w^2) = 0
\end{cases}\quad,
\]
skąd
\[
(g^2 - v^2)(h^2 +w^2 + g^2 + v^2) = (h^2 - w^2)(h^2 +w^2 + g^2 + v^2) = 0\,,
\]
czyli \(g^2 = v^2\) i \(h^2 = w^2\) (lub \(h=w=g=v=0\), ale wówczas podane równości także zachodzą).
Pokazaliśmy zatem, iż \(f^2 = u^2\), \(g^2 = v^2\), \(h^2 = w^2\), zatem
\[
\bm{f}^T\bm{f} = f^2 + g^2 + h^2 = u^2 + v^2 + w^2 = \bm{u}^T\bm{u}\,,
\]
ale 
\[
\begin{split}
&\bm{f}^T\bm{f} = \bm{a}^T\bm{R}^T\bm{R}\bm{a} = \bm{a}^T\bm{I}_3\bm{a} = \bm{a}^T\bm{a}\\
&\bm{u}^T\bm{u} = \bm{x}^T\bm{R}^T\bm{R}\bm{x} = \bm{x}^T\bm{I}_3\bm{x} = \bm{x}^T\bm{x}   
\end{split}\quad,
\]
skąd
\[
\bm{a}^T\bm{a} = \bm{x}^T\bm{x}\quad\quad\blacksquare
\]

\end{document}