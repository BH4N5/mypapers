\documentclass[../main.tex]{subfiles}

\begin{document}

\subsubsection*{Wybrane tożsamości trygonometryczne}
\begin{itemize}
    \item \(\sin(x\pm y)=\sin x\cos y\pm\sin y\cos x\)
    \item \(\cos(x\pm y)=\cos x\cos y\mp\sin x\sin y\)
    \item \(\tan(x+y)=\frac{\tan x+\tan y}{1-\tan x\tan y}\)
    \item \(\sin 2x=2\sin x\cos x\)
    \item \(\cos 2x=\cos^2x-\sin^2x=2\cos^2x-1=1-2\sin^2x\)
    \item \(\tan 2x=\frac{2\tan x}{1-\tan^2 x}\)
    \item \(\sin x\pm\sin y=2\sin\frac{x\pm y}{2}\cos\frac{x\mp y}{2}\)
    \item \(\cos x+\cos y=2\cos\frac{x+y}{2}\cos\frac{x-y}{2}\)
    \item \(\tan x\pm\tan y=\frac{\sin(x\pm y)}{\cos x\cos y}\)
    \item \(\tan x+\cot y=\frac{\cos(x-y)}{\cos x\sin y}\)
    \item \(\sin^2x\cos^2x=\frac{1}{4}\sin^2 2x\)
    \item \(\sin^2x-\sin^2y=\sin(x+y)\sin(x-y)\)
    \item \(\arctan x+\arctan\frac{1}{x}=\frac{\pi}{2}\)
    \item Podstawienie uniwersalne \(t=\tan\frac{x}{2}\)
    \begin{itemize}
        \item \(\sin x=\frac{2t}{1+t^2}\)
        \item \(\cos x=\frac{1-t^2}{1+t^2}\)
    \end{itemize}
\end{itemize}
\noindent\rule{\textwidth}{.5pt}
\subsubsection*{Wzór Taylora}
\begin{itemize}
    \item \(f(x)\approx
    f(a)+\frac{(x-a)}{1!}f'(a)+\frac{(x-a)^2}{2!}f''(a)+...+\frac{(x-a)^n}{n!}f^{(n)}(a)\)
    \item \(f(x,y)\approx
    f(0,0)+\pdv{f}{x}\,x+\pdv{f}{y}\,y+\frac{1}{2}\pdv[2]{f}{x}\,x^2+\pdv[2]{f}{x}{y}\,xy+\frac{1}{2}\pdv[2]{f}{y}\,y^2\)
\end{itemize}
\noindent\rule{\textwidth}{.5pt}
\begin{multicols}{2}
\subsubsection*{Przydatne przybliżenia dla małych \textit{x}}
\begin{itemize}
    \item \(\sin x\approx x\)
    \item \(\cos x\approx 1-\frac{x^2}{2}\)
    \item \(\tan x\approx \sin x\approx x\)
    \item \(\cot x\approx\frac{1}{x}\)
    \item \((1+x)^n\approx 1+nx\)
    \item \(e^x\approx 1+x\)
\end{itemize}
\subsubsection*{Wzór Eulera}
\begin{itemize}
    \item \(e^{ix}=\cos x+i\sin x\)
    \item \(\sin x=\frac{e^{ix}-e^{-ix}}{2i}\)
    \item \(\cos x=\frac{e^{ix}+e^{-ix}}{2}\)
\end{itemize}
\subsubsection*{Funkcje hiperboliczne}
\begin{itemize}
    \item \(\sinh x=\frac{e^x-e^{-x}}{2}\)
    \item \(\cosh x=\frac{e^x+e^{-x}}{2}\)
    \item \(\tanh x=\frac{\sinh x}{\cosh x}\)
    \item \(\sech x=\frac{1}{\cosh x}\)
    \item \(\csch x=\frac{1}{\sinh x}\)
    \item \(\coth x=\frac{1}{\tanh x}\)
\end{itemize}
\subsubsection*{Średnia całkowa}
\begin{itemize}
    \item \(\mu=\frac{1}{b-a}\int_a^bf(x)\dd{x}\)
\end{itemize}
\subsubsection*{Pochodne}
\begin{itemize}
    \item \((af)'=af'\)
    \item \((f\pm g)'=f'\pm g'\)
    \item \((fg)'=f'g+fg'\)
    \item \(g[f(x)]=\dv[]{g}{f}\dv[]{f}{x}\)
    \item \((f^{-1})'=(f')^{-1}\)
    \item \((\frac{1}{f})'=\frac{-f'}{f^2}\)
    \item \((\frac{f}{g})'=\frac{f'g-fg'}{g^2}\)
\end{itemize}
\subsubsection*{Tożsamości wektorowe}
\begin{itemize}
    \item
    \(\mathbf{A}\cdot(\mathbf{B}\times\mathbf{C})=\mathbf{B}\cdot(\mathbf{C}\times\mathbf{A})\)
    \item
    \(\mathbf{A}\times(\mathbf{B}\times\mathbf{C})=\mathbf{B}(\mathbf{A}\cdot\mathbf{C})-\mathbf{C}(\mathbf{A}\cdot\mathbf{B})\)
\end{itemize}
\subsubsection*{Operator nabla}
\begin{itemize}
    \item \(\nabla(f+g)=\nabla f+\nabla g\)
    \item \(\nabla(kf)=k\nabla f\)
    \item \(\nabla(fg)=g\nabla f+f\nabla g\)
    \item \(\nabla(\frac{f}{g})=\frac{g\nabla f-f\nabla g}{g^2}\)
    \item \(\nabla\cdot(\mathbf{A}+\mathbf{B})=\nabla\cdot\mathbf{A}+\nabla\cdot\mathbf{B}\)
    \item \(\nabla\cdot(k\mathbf{A})=k\nabla\cdot\mathbf{A}\)
    \item \(\nabla\cdot (f\mathbf{A})=f(\nabla\cdot\mathbf{A})+\mathbf{A}\cdot\nabla f\)
    \item \(\nabla\cdot(\frac{\mathbf{A}}{f})=\frac{f(\nabla\cdot\mathbf{A})-\mathbf{A}\cdot\nabla
    f}{f^2}\)
    \item
    \(\nabla\cdot(\mathbf{A}\times\mathbf{B})=\mathbf{B}\cdot(\nabla\times\mathbf{A})-\mathbf{A}\cdot(\nabla\times\mathbf{B})\)
    \item \(\nabla\times(\mathbf{A}+\mathbf{B})=\nabla\times\mathbf{A}+\nabla\times\mathbf{B}\)
    \item \(\nabla\times(k\mathbf{A})=k\nabla\times\mathbf{A}\)
    \item \(\nabla\times(f\mathbf{A})=f(\nabla\times\mathbf{A})-\mathbf{A}\times(\nabla f)\)
    \item \(\nabla\times(\nabla f)=0\)
    \item \(\nabla\cdot(\nabla\times \mathbf{A})=0\)
\end{itemize}
\subsubsection*{Kąt bryłowy}
\begin{itemize}
    \item \(\Omega=2\pi(1-\cos\alpha)\)
\end{itemize}
\end{multicols}
\noindent\rule{\textwidth}{.5pt}
\subsubsection*{Podstawowe twierdzenia analizy wektorowej}
\begin{enumerate}
    \item Podstawowe twierdzenie gradientów
    \begin{equation*}
        \int_a^b\nabla\varphi\cdot\dd{\mathbf{l}}=\varphi(b)-\varphi(a)
    \end{equation*}
    
    \item Twierdzenie Stokesa
    \begin{equation*}
        \int_\mathcal{S}(\nabla\times\mathbf{A})\cdot \dd{\mathbf{S}}=\oint_\ell \mathbf{A}\cdot\dd{\mathbf{l}}
    \end{equation*}
    
    \item Twierdzenie Gaussa--Ostrogradskiego
    \begin{equation*}
        \int_\mathcal{V}(\nabla\cdot \mathbf{A})\dd{V}=\oint_\mathcal{S}\mathbf{A}\cdot\dd{\mathbf{S}}
    \end{equation*}
    
    \item Twierdzenie dotyczące pól bezwirowych
    \begin{itemize}
        \item \(\nabla\times\mathbf{A}=0\) w całej przestrzeni
        \item \(\int_a^b\mathbf{A}\cdot\dd{\mathbf{l}}\) nie zależy od drogi pomiędzy punktami \(a\)
        i \(b\)
        \item \(\oint_\ell \mathbf{A}\cdot\dd{\mathbf{l}}=0\) dla dowolnego zamkniętego konturu
        \(\ell\)
        \item \(\mathbf{A}=-\nabla\varphi\), gdzie \(\varphi\) jest pewnym polem skalarnym
    \end{itemize}
    \item Twierdzenie dotyczące pól o zerowej dywergencji
    \begin{itemize}
        \item \(\nabla\cdot \mathbf{B}=0\) w całej przestrzeni
        \item \(\int_\mathcal{S}\mathbf{B}\cdot\dd{\mathbf{S}}\) nie zależy od wyboru powierzchni
        całkowania \(\mathcal{S}\) przy zadanym konturze
        \item \(\oint_\mathcal{S}\mathbf{B}\cdot\dd{\mathbf{S}}=0\) dla dowolnej powierzchni
        zamkniętej \(\mathcal{S}\)
        \item \(\mathbf{B}=\nabla\times\boldsymbol{\Phi}\), gdzie \(\boldsymbol{\Phi}\) to tzw.
        potencjał wektorowy
    \end{itemize}
\end{enumerate}
\noindent\rule{\textwidth}{.5pt}
\subsubsection*{Różne układy współrzędnych}
\begin{enumerate}
    \item Współrzędne kartezjańskie \((x,y,z)\)
    
    \begin{itemize}
    \item Element długości
    \begin{equation*}
        \dd{\mathbf{l}}=\dd{x}\mathbf{\hat{x}}+\dd{y}\mathbf{\hat{y}}+\dd{z}\mathbf{\hat{z}}
    \end{equation*}
    \item Element objętości
    \begin{equation*}
        \dd{V}=\dd{x}\dd{y}\dd{z}
    \end{equation*}
        \item Gradient
        \begin{equation*}
            \nabla \varphi=\pdv{\varphi}{x}\mathbf{\hat{x}}+\pdv{\varphi}{y}\mathbf{\hat{y}}+\pdv{\varphi}{z}\mathbf{\hat{z}}
        \end{equation*}
        \item Dywergencja
        \begin{equation*}
            \nabla\cdot\mathbf{A}=\pdv{A_x}{x}+\pdv{A_y}{y}+\pdv{A_z}{z}
        \end{equation*}
        \item Rotacja
        \begin{equation*}
        \begin{split}
            \nabla\times\mathbf{A}&=\left(\pdv{A_y}{z}-\pdv{A_z}{y}\right)\mathbf{\hat{x}}+\left(\pdv{A_z}{x}-\pdv{A_x}{z}\right)\mathbf{\hat{y}}\\
            &+\left(\pdv{A_x}{y}-\pdv{A_y}{x}\right)\mathbf{\hat{z}}
        \end{split}
        \end{equation*}
        \item Laplasjan
        \begin{equation*}
            \nabla^2\varphi=\pdv[2]{\varphi}{x}+\pdv[2]{\varphi}{y}+\pdv[2]{\varphi}{z}
        \end{equation*}
    \end{itemize}

    \item Współrzędne cylindryczne \((s,\phi,z)\)
    \begin{itemize}
    \item Element długości
    \begin{equation*}
        \dd{\mathbf{l}}=\dd{s}\mathbf{\hat{s}}+s\dd{\phi}\boldsymbol{\hat{\phi}}+\dd{z}\mathbf{\hat{z}}
    \end{equation*}
    
    \item Element objętości
    \begin{equation*}
        \dd{V}=s\dd{\phi}\dd{s}\dd{z}
    \end{equation*}
        \item Gradient
        \begin{equation*}
            \nabla \varphi =\pdv{\varphi}{s}\mathbf{\hat{s}}+\frac{1}{s}\pdv{\varphi}{\phi}\boldsymbol{\hat{\phi}}+\pdv{\varphi}{z}\mathbf{\hat{z}}
        \end{equation*}
        
        \item Dywergencja
        \begin{equation*}
            \nabla\cdot\mathbf{A}=\frac{1}{s}\pdv{(sA_s)}{s}+\frac{1}{s}\pdv{A_\phi}{\phi}+\pdv{A_z}{z}
        \end{equation*}
        
        \item Rotacja
        \begin{equation*}
        \begin{split}
            \nabla\times\mathbf{A}&=\left(\frac{1}{s}\pdv{A_z}{\phi}-\pdv{A_\phi}{z}\right)\mathbf{\hat{s}}\\
            &+\left(\pdv{A_s}{z}-\pdv{A_z}{s}\right)\boldsymbol{\hat{\phi}}\\
            &+\frac{1}{s}\left(\pdv{(sA_\phi)}{s} -\pdv{A_s}{\phi}\right)\mathbf{\hat{z}}
            \end{split}
        \end{equation*}
        
        \item Laplasjan
        \begin{equation*}
            \nabla^2\varphi =\frac{1}{s}\frac{\partial}{\partial s}\left(s\pdv{\varphi}{s}\right)+\frac{1}{s^2}\pdv[2]{\varphi}{\phi}+\pdv[2]{\varphi}{z}
        \end{equation*}
    \end{itemize}
    \item Współrzędne sferyczne \((r,\theta,\phi)\)
    
    \begin{itemize}
        \item Element długości
        \begin{equation*}
            \dd{\mathbf{l}}=\dd{r}\mathbf{\hat{r}}+r\dd{\theta}\boldsymbol{\hat{\theta}}+r\sin\theta\dd{\phi}\boldsymbol{\hat{\phi}}
        \end{equation*}
        \item Element objętości
        \begin{equation*}
            \dd{V}=r^2\sin\theta \dd{r}\dd{\theta}\dd{\phi}
        \end{equation*}
        \item Gradient
        \begin{equation*}
            \nabla\varphi=\pdv{\varphi}{r}\mathbf{\hat{r}}+\frac{1}{r}\pdv{\varphi}{\theta}\boldsymbol{\hat{\theta}}+\frac{1}{r\sin\theta}\pdv{\varphi}{\phi}\boldsymbol{\hat{\phi}}        
            \end{equation*}
            
            \item Dywergencja
            \begin{equation*}
                \nabla\cdot\mathbf{A}=\frac{1}{r^2}\pdv{(r^2A_r)}{r}+\frac{1}{r\sin\theta}\pdv{(\sin\theta A_\theta)}{\theta}+\frac{1}{r\sin\theta}\pdv{A_\phi}{\phi}
            \end{equation*}
            
            \item Rotacja
            \begin{equation*}
            \begin{split}
                \nabla\times\mathbf{A}&=\frac{1}{r\sin\theta}\left(\pdv{(\sin\theta A_\phi)}{\theta}-\pdv{A_\theta}{\phi}\right)\mathbf{\hat{r}}\\
                &+\frac{1}{r}\left(\frac{1}{\sin\theta}\pdv{A_r}{\phi}-\pdv{(rA_\phi)}{r}\right)\boldsymbol{\hat{\theta}}\\
                &+\frac{1}{r}\left(\pdv{(rA_\theta)}{r}-\pdv{A_r}{\theta}\right)\boldsymbol{\hat{\phi}}
            \end{split}
            \end{equation*}
            
            \item Laplasjan
            \begin{equation*}
            \begin{split}
                \nabla^2\varphi&=\frac{1}{r^2}\frac{\partial}{\partial r}\left(r^2\pdv{\varphi}{r}\right)+\frac{1}{r^2\sin\theta}\frac{\partial}{\partial\theta}\left(\sin\theta\pdv{\varphi}{\theta}\right)\\
                &+\frac{1}{r^2\sin^2\theta}\pdv[2]{\varphi}{\phi}
            \end{split}
            \end{equation*}
    \end{itemize}
\end{enumerate}

\end{document}