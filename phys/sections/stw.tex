  \documentclass[../main.tex]{subfiles}
\begin{document}

  \subsection{Szczególna teoria względności}
        \textit{The views of space and time which I wish to lay before you have sprung from the soil
        of experimental physics, and therein lies their strength. They are radical. Henceforth space
        by itself, and time by itself, are doomed to fade away into mere shadows, and only a kind of
        union of the two will preserve an independent reality.}\begin{flushright}Hermann
        Minkowski\end{flushright}

\subsubsection{Postulaty}
Przed rozważeniem głównych idei szczególnej teorii względności Einsteina--Poincarégo--Lorentza
przedyskutujmy podstawowe idee oraz poglądy ontologiczne, które leżą u fundamentów klasycznej
mechaniki, a które zwykle przyjmujemy intuicyjnie.
\begin{itemize}
    \item \textit{Ciągłość czasu i przestrzeni.} Zarówno w mechanice nierelatywistycznej, jak i
    relatywistycznej zakładamy, iż przestrzeń oraz czas mają budowę ciągłą.
    \item \textit{Względność przestrzeni.} W mechanice nierelatywistycznej zauważamy, iż pojęcie
    położenia danego obiektu \(A\) nie ma sensu, jeżeli nie podamy obiektu  \(B\) względem którego
    określamy owo położenie.
    \item \textit{Absolutność czasu.} W mechanice nierelatywistycznej przyjmujemy, iż czas jest
    absolutnym i uniwersalnym parametrem, który zmienia się jednakowo dla wszystkich obserwatorów
    inercjalnych.
    \item \textit{Jednorodność czasu i przestrzeni.} Zarówno w mechanice nierelatywistycznej, jak i
    relatywistycznej zakładamy zakładamy, iż czas i przestrzeń są jednorodne, tzn. nie istnieją
    wyróżnione położenia, ani chwile czasu.
\end{itemize}
\textit{Zasada względności}. Już Galileusz zapostulował niemożliwość odróżnienia dwóch inercjalnych
układów odniesienia za pomocą jakichkolwiek doświadczeń mechanicznych. Oznacza to, że zakładamy, iż
prawa mechaniki są takie same we wszystkich inercjalnych układach odniesienia w ich zwyczajnej
formie. Einstein rozszerzył owy postulat proponując, iż wszystkie prawa fizyki są takie same we
wszystkich inercjalnych układach odniesienia.\\

Z zasady względności wynika, iż jeśli ciało \(A\) porusza się względem ciała \(B\) z prędkością
\(\mathbf{V}\) to z perspektywy ciała \(A\), ciało \(B\) porusza się z prędkością \(-\mathbf{V}\).
Istotnie gdyby tak nie było, wówczas istniałby pewien wyróżniony kierunek w przestrzeni, za pomocą
którego moglibyśmy odróżnić dwa układy inercjalne. Obserwacja ta jest słuszna zarówno w mechanice
nierelatywistycznej, jak i w STW.\\

Mechanika nierelatywistyczna wraz z wymienionymi wcześniej założenia zakłada również, iż
oddziaływania rozchodzą się w sposób natychmiastowy. Istotnie chociażby siła z jaką przyciągają się
dwa masywne ciała zależy jedynie od aktualnego położenia tych ciał. Z doświadczenia wynika jednak,
iż istnieje graniczna prędkość rozchodzenia się oddziaływań równa prędkości światła w próżni
\begin{equation*}
    c=299\,792\,458\,\frac{\text{m}}{\text{s}}\,.
\end{equation*}
Prędkość ta jest jednakowa we wszystkich inercjalnych układach odniesienia. Z tego względu w STW
postulujemy absolutność prędkości światła w próżni. Szczególna teoria względności opiera się zatem
na dwóch postulatach:
\begin{itemize}
    \item \textit{Zasada względności.} Wszystkie prawa fizyki są nieimiennicze we wszystkich
    inercjalnych układach odniesienia.
    \item \textit{Absolutność prędkości światła.} Graniczna prędkość rozchodzenia się oddziaływań
    jest jednakowa we wszystkich inercjalnych układach odniesienia i równa prędkości światła w
    próżni \(c\).
\end{itemize}

\subsubsection{Układ odniesienia}
\textit{Układem odniesienia} nazywamy punkt \(O\) wybrany w taki sposób, iż jest on nieruchomy
względem pewnego ciała \(A\) wraz z układem współrzędnych kartezjańskich \((x_1,x_2,x_3)\) o
początku \(O\) i sposobem mierzenia czasu \(t\).\\

Zastanówmy się, w jaki sposób w danym układzie odniesienia mierzymy owe parametry. Istotnie potrzeba
nam do tego nieskończenie wielu pomocników wyposażonych w zsynchronizowane zegary. W każdym punkcie
\((x_p,y_p,z_p)\) umieszczamy pomocnika wyposażonego w zegar, który wcześniej został
zsynchronizowany ze wszystkimi innymi zegarami (np. poprzez przesłanie sygnału świetlnego). Jeśli
pewne zdarzenie zajdzie w \((x,y,z)\) to pomocnik umieszczony w tym punkcie zarejestruje je
natychmiast w pewnej chwili \(t\). Porozumiewając się następnie ze wszystkimi pomocnikami możemy
ustalić, iż w punkcie \((x,y,z)\) w chwili \(t\) zaszło dane zdarzenie. Taki sposób pomiaru niweluje
problemy związane ze skończoną prędkością światła, a tym samym opóźnieniem między rzeczywistą chwilą
wystąpienia danego zdarzenia, a chwilą, w której obserwator umieszczony z dala od tego zdarzenia je
zarejestruje.\\

Synchronizację zegarów można również przeprowadzić w inny sposób. Główny obserwator \(O\) wzywa
swoich pomocników do siebie, gdzie wszyscy synchronizują swoje zegary, a następnie każdy z
pomocników udaje się na przydzieloną mu pozycję, poruszając się bardzo wolno (w granicy
nieskończenie wolno). Istotnie wówczas różnica między wskazaniami zegara głównego obserwatora, a
pomocnika oddalonego od niego o \(r\) wynosi
\begin{equation*}
    \Delta t=\lim_{\beta\to0}\frac{r}{\beta c}\left(1-\sqrt{1-\beta^2}\right)=0\,,
\end{equation*}
czyli zegary są zsynchronizowane.
\subsubsection{Czasoprzestrzeń}
\textit{Czasoprzestrzeń} definiujemy jako zbiór wszystkich możliwych punktów \((x_1,x_2,x_3,ct)\)
mierzonych w danym inercjalnym układzie odniesienia \(S\), przy czym będziemy czasami oznaczać
\(x_4:=ct\). Dowolny punkt w czasoprzestrzeni nazywamy \textit{zdarzeniem}. Zdarzenie to dowolny
,,efekt'' występujący w rzeczywistości fizycznej, któremu jednoznacznie można przypisać położenie
\(\mathbf{x}\) i chwilę wystąpienia \(ct\). Wielkość
\begin{equation*}
    x:=(x_1,x_2,x_3,x_4)=(\mathbf{x},x_4)
\end{equation*}
nazywamy \textit{czterowektorem położenia} zdarzenia w czasoprzestrzeni. Krzywą
\(x(t)=(\mathbf{x}(t),ct)\) zakreślaną przez punkt w czasoprzestrzeni nazywamy \textit{linią
świata}.
\subsubsection{Transformacja Galileusza}
Interesującym nas zagadnieniem jest znalezienie transformacji \(\Lambda\) takiej, że
\begin{equation*}
    x'=\Lambda x\,,
\end{equation*}
gdzie \(x'\) jest czterowektorem położenia danego zdarzenia mierzonym względem inercjalnego układu
odniesienia poruszającego się względem \(S\) z prędkością \([V,0,0]\). W mechanice
nierelatywistycznej transformacją tą (dla \textit{standardowego pchnięcia galileuszowskiego}) jest
transformacja Galileusza
\begin{equation*}
    \Lambda_0=\left[\begin{array}{cccc}
         1&0&0&-V  \\
         0&1&0&0  \\
         0&0&1&0  \\
         0&0&0&1  
    \end{array}\right]\,.
\end{equation*}
Jak łatwo się jednak przekonać nie spełnia ona postulatu o absolutności prędkości światła. Okazuje
się, że prawidłową transformacją jest transformacja Lorentza, jednak jak się przekonamy dla \(V\ll
c\) przechodzi ona w transformację Galileusza, zatem dla typowych prędkości na Ziemi transformacja
Galileusza jest wystarczającym i niezwykle dogodnym przybliżeniem.
\subsubsection{Transformacja Lorentza}
Jako punkt wyjścia do STW przyjmiemy wzory transformacyjne transformacji Lorentza. Nie będziemy
tutaj przedstawiać różnorakich możliwych heurystycznych lub innych wyprowadzeń owych wzorów.
Przedstawimy je po prostu jako punkt wyjścia do dalszych rozważań. Niech \((x_1,x_2,x_3,ct)\)
oznaczają współrzędne pewnego \textit{zdarzenia} w pewnym inercjalnym układzie odniesienia \(S\), a
\((x_1',x_2',x_3',ct')\) współrzędne tego zdarzenia w układzie \(S'\). Zakładamy tutaj oczywiście,
zgodnie z zasadą względności, że jeśli dane zdarzenie zaszło w \(S\) to musiało również zajść w
\(S'\).\\

Rozważmy tzw. \textit{standardowe pchnięcie lorentzowskie}, tj. zakładamy, iż osie przestrzenne
układów \(S\) i \(S'\) są równoległe i tak samo skierowane, w chwili \(t=0\) zachodzi \(t'=0\) i
\(\mathbf{x}=\mathbf{x}'=\mathbf{0}\) oraz układ \(S'\) porusza się względem \(S\) wzdłuż osi
\(x_1\) ze stałą prędkością \([V,0,0]\), wówczas zależność między współrzędnymi czasoprzestrzennymi
dowolnego zdarzenia w \(S\) i \(S'\) ma postać
\begin{equation*}
    \left[\begin{array}{c}
         x_1'  \\
         x_2'\\
         x_3'\\
         x_4'
    \end{array}\right]=
    \left[\begin{array}{cccc}
         \gamma(V)&0&0&-\beta(V)\gamma(V)  \\
         0&1&0&0  \\
         0&0&1&0  \\
         -\beta(V)\gamma(V)&0&0&\gamma(V)
    \end{array}\right]\left[\begin{array}{cc}
         x_1  \\
         x_2\\
         x_3\\
         x_4
    \end{array}\right]\,,
\end{equation*}
gdzie zachodzą związki
\begin{equation*}
    \beta(V):=\frac{V}{c}\,,\quad \gamma(V):=\frac{1}{\sqrt{1-\beta^2(V)}}\,.
\end{equation*}
Wynik ten możemy zapisać w bardziej zwarty sposób jeśli wprowadzimy tzw. \textit{macierz
standardowego pchnięcia} \(\Lambda\)
\begin{equation*}
    \Lambda:=\left[\begin{array}{cccc}
         \gamma(V)&0&0&-\beta(V)\gamma(V)  \\
         0&1&0&0  \\
         0&0&1&0  \\
         -\beta(V)\gamma(V)&0&0&\gamma(V)
    \end{array}\right]\,,
\end{equation*}
wówczas \(x'=\Lambda x\). Łatwo sprawdzić, iż dla \(V\ll c\) mamy \(\beta\approx0\) i
\(\gamma\approx 1\) zatem \(\Lambda\approx\Lambda_0\). Okazuje się, że dowolną transformację
Lorentza możemy przedstawić jako złożenie przestrzennego obrotu osi \((x_1,x_2,x_3)\) i
standardowego pchnięcia w pewnym kierunku \(\mathbf{e}\).
\subsubsection{Czterowektory i czteroskalary}
Wprowadziliśmy już czterowektor położenia \(x=(x_1,x_2,x_3,x_4)\). W ogólności czterowektorem \(q\)
nazywamy zespół czterech wielkości \((q_1,q_2,q_3,q_4)\), które przy przechodzeniu do poruszających
się, inercjalnych układów odniesienia transformują się tak jak czterowektor położenia, tj.
\begin{equation*}
    q'=\Lambda q\,.
\end{equation*}
Definiujemy iloczyn dwóch czterowektorów \(p\) i \(q\) jako wielkość
\begin{equation*}
    p\cdot q:=p_1q_1+p_2q_2+p_3q_3-p_4q_4\,.
\end{equation*}
Wielkość, która jest niezmiennicza względem transformacji Lorentza nazywamy \textit{skalarem
lorentzowskim} (lub \textit{czteroskalarem}). Iloczyn dowolnych dwóch czterowektorów jest skalarem
lorentzowskim. Istotnie dowolną transformację Lorentza można przedstawić jako przestrzenny obrót i
standardowe pchnięcie w pewnym kierunku. Przestrzenny obrót oczywiście nie zmienia wartości iloczynu
dwóch czterowektorów, gdyż (jak łatwo sprzwdzić) obrót w przestrzeni 3D nie zmienia iloczynu
skalarnego dwóch wektorów \(\mathbf{p}\) i \(\mathbf{q}\). Pozostaje zatem pokazać, iż standardowe
pchnięcie nie zmienia wartości iloczynu dwóch czterowektorów. Istotnie mamy
\begin{equation*}
\begin{split}
    p'\cdot q'&=\gamma^2(q_1-\beta q_4)(p_1-\beta p_4)+q_2p_2+q_3p_3-\gamma^2(q_4-\beta q_1)(p_4-\beta p_1)\\
    &=p_1q_1+p_2q_2+p_3q_3-p_4q_4=p\cdot q\,.
\end{split}
\end{equation*}
Często, aby udowodnić, iż dana wielkość \(k\) jest czterowektorem korzystamy z \textit{reguły
ilorazowej}:\\
\begin{itemize}
    \item Niech \(k\) oznacza zespół pewnych czterech wielkości \((k_1,k_2,k_3,k_4)\), a
    \(x=(x_1,x_2,x_3,x_4)\) czterowektor położenia. Jeśli \(\psi:=k\cdot x\) jest czteroskalarem, to
    \(k\) jest czterowektorem.
    \item Niech \(p\), \(q\) będą pewnymi czterowektorami. Jeśli w każdym inercjalnym układzie
    odniesiena \(p=\lambda q\) dla pewnej stałej \(\lambda\), to we wszystkich inercjalnych układach
    odniesienia \(\lambda=\lambda'\), czyli \(\lambda\) jest skalarem lorentzowskim.
\end{itemize}
Zauważmy, że dla dowolnego czterowektora \(q\) zachodzi
\begin{equation*}
    \frac{q_1-q_1'}{q_4+q_4'}=\frac{q_1(1-\gamma)+\gamma\beta q_4}{q_4(1+\gamma)-\gamma\beta q_1}=\frac{\beta\gamma}{1+\gamma}=\frac{V}{c+\sqrt{c^2-V^2}}
\end{equation*}
\subsubsection{Transformacja Lorentza jako obrót}
Zauważmy, iż transformacja Lorentza \(\Lambda\) jest pewnego rodzaju obrotem w czasoprzestrzeni
\((x_1,x_2,x_3,x_4)\). Istotnie ponieważ zawsze \(\gamma(V)>1\), więc możemy wprowadzić kąt
\(\psi(V)\) zdefiniowany jako
\begin{equation*}
    \cosh\psi=\gamma(V)\,,
\end{equation*}
skąd od razu mamy
\begin{equation*}
    \sinh\psi=\sqrt{\cosh^2\psi-1}=\sqrt{\gamma^2(V)-1}=\beta(V)\gamma(V)
\end{equation*}
oraz
\begin{equation*}
    \tanh\psi=\frac{\sinh\psi}{\cosh\psi}=\beta(V)\,.
\end{equation*}
Parametr ten nazywamy \textit{pospiesznością} (z ang. -- \textit{rapidity}). Macierz standardowego
pchnięcia możemy zapisać więc jako
\begin{equation*}
    \Lambda=\left[\begin{array}{cccc}
         \cosh\psi&0&0&-\sinh\psi \\
         0&1&0&0  \\
         0&0&1&0  \\
         -\sinh\psi &0&0&\cosh\psi
    \end{array}\right]\,.
\end{equation*}
Dla porównania wypiszmy macierz przestrzennego obrotu \(\mathbf{R}\) osi \((x,z)\) wokół \(y\) o kąt
\(\psi\)
\begin{equation*}
    \mathbf{R}=\left[\begin{array}{ccc}
         \cos\psi&0&\sin\psi  \\
         0&1&0  \\
         -\sin\psi&0&\cos\psi  \\
    \end{array}\right]\,.
\end{equation*}
Widzimy, iż pomiędzy tymi wzorami występuje silna analogia.
\subsubsection{Interwał czasoprzestrzenny}
Podobnie jak zwykłe obroty osi \((x,y,z)\) w trójwymiarowej przestrzeni nie zmieniają
infinitezymalnej długości \(\dd{x}^2+\dd{y}^2+\dd{z}^2\), również w czasoprzestrzeni
\((x_1,x_2,x_3,x_4)\) interwał czasoprzestrzenny \(\dd{s}\)
\begin{equation*}
    \dd{s}^2=\dd{x}_1^2+\dd{x}_2^2+\dd{x}_3^2-\dd{x}_4^2
\end{equation*}
jest czteroskalarem. Istotnie korzystając z transformacji Lorentza mamy
\begin{equation*}
\begin{split}
    \dd{s'}^2&=\dd{x'}_1^2+\dd{x'}_2^2+\dd{x'}_3^2-\dd{x'}_4^2\\
    &=\gamma^2(\dd{x}_1-\beta\dd{x}_4)^2+\dd{x}_2^2+\dd{x}_3^2-\gamma^2(\dd{x}_4-\beta\dd{x}_1)^2\\
    &=\dd{x}_1^2+\dd{x}_2^2+\dd{x}_3^2-\dd{x}_4^2=\dd{s}^2
\end{split}
\end{equation*}
\subsubsection{Dylatacja czasu i skrócenie Lorentza}
\begin{itemize}
    \item \textit{Dylatacja czasu.} Rozważmy zdarzenie \(Z\) w danym punkcie
    \(\mathbf{x}=(x_1,x_2,x_3)\), rozpoczynające się w chwili \(t_p\) i kończące w chwili \(t_k\) w
    układzie odniesienia \(S\). Czas trwania tego zdarzenia w \(S\) wynosi zatem
    \begin{equation*}
        \Delta t_0:=t_k-t_p\,.
    \end{equation*}
    Rozpatrzmy teraz to samo zdarzenie z perspektywy układu \(S'\) powiązanego z \(S\) standardowym
    pchnięciem lorentzowskim. Zgodnie z transformacją Lorentza, z punktu widzenia \(S'\), \(Z\)
    rozpoczęło się w chwili
    \begin{equation*}
        t_p'=\gamma\left(t_p-\frac{Vx_1}{c^2}\right)
    \end{equation*}
    i zakończyło w chwili
    \begin{equation*}
        t_k'=\gamma\left(t_k-\frac{Vx_1}{c^2}\right)\,.
    \end{equation*}
    Względem \(S'\) zdarzenie \(Z\) trwało więc
    \begin{equation*}
        \Delta t':=t_k'-t_p'=\gamma \Delta t_0=\frac{\Delta t_0}{\sqrt{1-\frac{V^2}{c^2}}}>\Delta t_0\,.
    \end{equation*}
    Z tego powodu poruszające się zegary będą się spóźniać.
    \item \textit{Skrócenie Lorentza.} Rozpatrzmy pręt umieszczony na osi \(x_1\) i poruszający się
    względem układu \(S\) z prędkością \(V\). Załóżmy, iż w chwili \(t\) współrzędne końców pręta to
    \((x_1^A,0,0)\) i \((x_1^B,0,0)\). Długość pręta mierzona w układzie \(S\) wynosi więc
    \begin{equation*}
        L:=x_1^B-x_1^A\,.
    \end{equation*}
    Rozważmy teraz układ \(S'\) powiązany z \(S\) standardowym pchnięciem z prędkością \(V\).
    Oczywiście w tym układzie odniesienia pręt jest nieruchomy, zatem w chwili \(t\) współrzędne
    jego końców są dane przez
    \begin{equation*}
        x_1^{A'}=\gamma(x_1^A-Vt)\,,\quad x_1^{B'}=\gamma(x_1^B-Vt)\,.
    \end{equation*}
    Długość spoczywającego pręta wynosi zatem
    \begin{equation*}
        L_0=x_1^{B'}-x_1^{A'}=\gamma L\,,
    \end{equation*}
    skąd otrzymujemy, iż długość poruszającego się pręta wynosi
    \begin{equation*}
        L=\frac{L_0}{\gamma}=L_0\sqrt{1-\frac{V^2}{c^2}}<L_0\,.
    \end{equation*}
    Jak łatwo pokazać jeśli pręt o długości spoczynkowej \(L_0\) porusza się z prędkością \(V\)
    wzdłuż osi \(x_1\) i jest nachylony pod kątem \(\theta_0\) do osi \(x_1\) w układzie
    odniesienia, w którym jest stacjonarny to w układzie odniesienia, w którym się porusza
    \begin{equation*}
        L=L_0\sqrt{\sin^2\theta_0+\frac{1}{\gamma^2}\cos^2\theta_0}=L_0\sqrt{1+\frac{1-\gamma^2}{\gamma^2+\tan^2\theta}}\,,
    \end{equation*}
    przy czym
    \begin{equation*}
        \tan\theta=\gamma\tan\theta_0\,.
    \end{equation*}
    \item\textit{Względność równoczesności.} Rozważmy dwa zdarzenia \(A\), \(B\), które w układzie
    \(S\) zachodzą odpowiednio w punktach \((d,0,0)\) i \((-d,0,0)\) równocześnie w chwili
    \(t=d/c\). Wyznaczmy miejsca tych zdarzeń i chwile ich wystąpienia w układzie \(S'\) powiązanym
    z \(S\) standardowym pchnięciem lorentzowskim z prędkością \(-V\). Zgodnie z transformacją
    Lorentza mamy
    \begin{equation*}
        \begin{split}
            &x'_A=\gamma(-V)\left(d+\frac{Vd}{c}\right)\,,\quad t'_A=\gamma(-V)\left(\frac{d}{c}+\frac{Vd}{c^2}\right)\\
            &x'_B=\gamma(-V)\left(-d+\frac{Vd}{c}\right)\,,\quad t'_B=\gamma(-V)\left(\frac{d}{c}-\frac{Vd}{c^2}\right)
        \end{split}\quad\quad\,.
    \end{equation*}
    Widzimy, iż pomimo faktu równoczesności zdarzeń \(A\) i \(B\) w układzie \(S\), w układzie
    \(S'\) zdarzenia te nie są równoczesne. Jedno poprzedza drugie o
    \begin{equation*}
        \Delta t'=t_A'-t_B'=\frac{2\gamma Vd}{c^2}\,.
    \end{equation*}
\end{itemize}
\subsubsection{Pozorne obrazy szybko poruszających się obiektów}
Rozważmy pręt o długości spoczynkowej \(L_0\) poruszający się wzdłuż osi \(x_1\) z prędkością \(V\).
Niech \(O(\xi,\eta)\) oznacza punkt, w którym znajduje się obserwator. Oczywiście zgodnie ze wzorem
na skrócenie Lorentza \textbf{dokładne pomiary} (wykonane np. przez nieskończenie wielu pomocników
\(O\)) \textbf{pokażą, iż w układzie obserwatora \(O\) długość pręta wynosi \(L_0/\gamma(V)\)}.
Obserwator \textbf{zobaczy} jednak co innego. Istotnie to, co zobaczy to światło docierające do
\(O\). Niech \(\chi_A(t)\) oznacza położenie \(x_1\) końca \(A\) pręta, a \(\chi_B(t)\) położenie
końca \(B\). Oczywiście zachodzi
\begin{equation*}
    \chi_B-\chi_A=\frac{L_0}{\gamma(V)}\,.
\end{equation*}
Obserwator zobaczy pręt o długości \(L_0/\gamma+\Delta L\), gdzie \(\Delta L\) jest dane przez
\begin{equation*}
    \sqrt{\eta^2+(\xi-\chi_A)^2}=\beta^{-1}(V)\Delta L+\sqrt{\eta^2+\left(\xi-\chi_A-\frac{L_0}{\gamma}-\Delta L\right)^2}\,.
\end{equation*}
Dla \(\eta\approx 0\) i pręta zbliżającego się do obserwatora (\(\xi>\chi_B+\Delta L\)) otrzymujemy
\begin{equation*}
    \Delta L=\frac{L_0}{\gamma(V)}\frac{1}{\beta^{-1}(V)-1}\,,
\end{equation*}
skąd
\begin{equation*}
    L_\text{app}=\frac{L_0}{\gamma}+\Delta L=L_0\sqrt{\frac{1+\beta}{1-\beta}}>L_0\,.
\end{equation*}
Z kolei dla pręta oddalającego się od obserwatora mamy
\begin{equation*}
    0=\beta^{-1}\Delta L+\frac{L_0}{\gamma}+\Delta L\,,
\end{equation*}
skąd
\begin{equation*}
    L_\text{rec}=L_0\sqrt{\frac{1-\beta}{1+\beta}}\,.
\end{equation*}

\subsubsection{Relatywistyczny wzór na dodawanie prędkości}
Rozważmy cząstkę poruszającą się w układzie \(S\) z prędkością
\begin{equation*}
    \mathbf{v}=[v_1,v_2,v_3]:=\left[\dv[]{x_1}{t},\dv[]{x_2}{t},\dv[]{x_3}{t}\right]\,.
\end{equation*}
Chcemy znaleźć wyrażenie na prędkość cząstki \(\mathbf{v}'\) w układzie \(S'\) powiązanym z \(S\)
standardowym pchnięciem lorentzowskim z prędkością \(V\). Oczywiście w układzie \(S'\) mamy
\begin{equation*}
    \mathbf{v}'=\left[\dv{x_{1}'}{t'},\dv{x'_2}{t'},\dv{x'_3}{t'}\right]\,,
\end{equation*}
gdzie zgodnie z transformacją Lorentza mamy
\begin{equation*}
    \begin{split}
        &\dd{x_1'}=\gamma(\dd{x_1}-V\dd{t})\\
        &\dd{x_2'}=\dd{x_2}\\
        &\dd{x_3'}=\dd{x_3}\\
        &\dd{t'}=\gamma\left(\dd{t}-\frac{V\dd{x_1}}{c^2}\right)
    \end{split}\quad\quad\,,
\end{equation*}
zatem
\begin{equation*}
    \begin{split}
        &v'_1=\frac{v_1-V}{1-\frac{Vv_1}{c^2}}\\
        &v'_2=\frac{1}{\gamma}\frac{v_2}{1-\frac{Vv_1}{c^2}}\\
        &v'_3=\frac{1}{\gamma}\frac{v_3}{1-\frac{Vv_1}{c^2}}
    \end{split}\quad\quad\,.
\end{equation*}
Zauważmy przy tym, iż w granicy \(c\to\infty\) otrzymujemy klasyczny wzór na składanie prędkości
\begin{equation*}
    \begin{split}
        &v'_1=v_1-V\\
        &v'_2= v_2\\
        &v'_3= v_3
    \end{split}\quad\quad\,.
\end{equation*}
\textbf{Włóczenie Fresnela}
\medskip

Jako niezwykle piękne zastosowanie uzyskanych wzorów na relatywistyczne dodawanie prędkości rozważmy
dwa układy odniesienia \(S\) i \(S'\), przy czym układ \(S'\) jest powiązany z \(S\) standardowym
pchnięciem z prędkością \(\pm V\). Załóżmy, że względem \(S'\) pewien sygnał propaguje z prędkością
\([c/n,0,0]\), gdzie \(n\geq1\). Wówczas, zgodnie z relatywistycznym wzorem na dodawanie prędkości,
względem \(S\) sygnał ten propaguje wzdłuż osi \(x_1\) z szybkością
\begin{equation*}
    v=\frac{\frac{c}{n}\pm V}{1\pm\frac{V}{cn}}=\frac{c}{n}\pm fV\,,
\end{equation*}
gdzie \(f\) nazwiemy historycznie \textit{współczynnikiem włóczenia Fresnela} i oczywiście
\begin{equation*}
    f=\frac{1-\frac{1}{n^2}}{1\pm\frac{\beta}{n}}\,,
\end{equation*}
gdzie \(\beta=V/c\). Dla \(V\ll c\) mamy oczywiście
\begin{equation*}
    f\approx 1-\frac{1}{n^2}\,.
\end{equation*}
Jeśli zidentyfikujemy teraz propagujący sygnał jako światło biegnące w wodzie (o współczynniku
załamania światła \(n\)), która względem układu laboratoryjnego (\(S\)) porusza się z prędkością
\(\pm V\) to otrzymujemy niezwykle proste wyjaśnienie doświadczenia Fizeau.
\medskip

\noindent\textbf{Trzy układy odniesienia i pospieszność}
\medskip

Rozważmy trzy inercjalne układy odniesienia \(S\), \(S'\), \(S''\). Układ \(S'\) jest powiązany z
\(S\) standardowym pchnięciem lorentzowskim z prędkością \(V_1\). Natomiast układ \(S''\) jest
powiązany z \(S'\) standardowym pchnięciem z prędkością \(V_2\). Chcemy znaleźć prędkość \(V\)
standardowego pchnięcia, z jakim powiązany jest układ \(S''\) z \(S\). Z relatywistycznego wzoru na
dodawanie prędkości mamy oczywiście
\begin{equation*}
    V_2=\frac{V-V_1}{1-\frac{VV_1}{c^2}}\,,
\end{equation*}
skąd otrzymujemy
\begin{equation*}
    V=\frac{V_1+V_2}{1+\frac{V_1V_2}{c^2}}\,.
\end{equation*}
Zauważmy, iż jeśli wprowadzimy pospieszności \(\psi\), \(\psi_1\), \(\psi_2\) to mamy
\begin{equation*}
    \tanh\psi=\frac{V}{c}\,,\quad\tanh\psi_1=\frac{V_1}{c}\,,\quad\tanh\psi_2=\frac{V_2}{c}\,,
\end{equation*}
skąd
\begin{equation*}
    \tanh\psi=\frac{\tanh\psi_1+\tanh\psi_2}{1+\tanh\psi_1\tanh\psi_2}=\tanh(\psi_1+\psi_2)\,,
\end{equation*}
zatem
\begin{equation*}
    \psi=\psi_1+\psi_2\,.
\end{equation*}
\subsubsection{Relatywistyczny efekt Dopplera}
Rozpatrzmy źródło \(Z\) poruszające się wzdłuż osi \(x_1\) z prędkością \(V\), które emituje w
kierunku obserwatora \(O\) falę świetlną \(\cos(\mathbf{k}\cdot\mathbf{x}-\omega t)\). Wektor falowy
\(\mathbf{k}\) jest skierowany zgodnie z odcinkiem \(ZO\) i tworzy kąt \(\theta\) z osią \(x_1\).
Wprowadźmy czterowektor falowy
\begin{equation*}
    k:=\left(\mathbf{k},\frac{\omega}{c}\right)\,.
\end{equation*}
Taka wielkość jest czterowektorem, gdyż faza fali \(\phi=k\cdot x\) z całą pewnością jest
czteroskalarem. W układzie odniesienia obserwatora \(S\) mamy zatem
\begin{equation*}
    k_4=\frac{\omega}{c}\,,\quad |\mathbf{k}|=\frac{\omega}{c}\,,\quad k_1=|\mathbf{k}|\cos\theta\,.
\end{equation*}
W układzie \(S'\) powiązanym z \(S\) standardowym pchnięciem z prędkością \(V\) mamy natomiast
\begin{equation*}
    k_4'=\frac{\omega_0}{c}=\gamma(k_4-\beta k_1)=\gamma\omega(1-\beta\cos\theta)\,,
\end{equation*}
skąd
\begin{equation*}
    \omega=\frac{\omega_0}{\gamma(1-\beta\cos\theta)}\,.
\end{equation*}
Wzór ten opisuje relatywistyczny efekt Dopplera dla światła, który objawia się zmianą częstotliwość
(koloru) docierającego do obserwatora światła, gdy źródło się porusza. W szczególnym przypadku dla
\(\cos\theta=\pm 1\) (tj. \(\theta=0\) lub \(\theta=\pi\)) otrzymujemy
\begin{equation*}
    \omega=\omega_0\frac{\sqrt{1-\beta^2}}{1\mp\beta}\,,
\end{equation*}
gdzie minus odpowiada sytuacji, gdy źródło zbliża się do obserwatora, a plus -- gdy się od niego
oddala. Zauważmy, iż relatywistyczny efekt Dopplera występuje nawet dla \(\theta=\frac{\pi}{2}\) w
przeciwieństwie do klasycznego odpowiednika tego efektu. Dla tej wartości kąta i \(V=0.2\,c\)
otrzymujemy
\begin{equation*}
    \Delta\omega:=\omega_0-\omega=\omega_0\frac{\gamma-1}{\gamma}\approx 0.02\,\omega_0
\end{equation*}
\subsubsection{Czas własny cząstki materialnej}
Rozważmy cząstkę poruszającą się względem układu \(S\) z prędkością
\begin{equation*}
    \mathbf{v}=\left[\dv[]{x_1}{t},\dv[]{x_2}{t},\dv[]{x_3}{t}\right]\,.
\end{equation*}
Zdefiniujmy wielkość \(\dd{\tau}\)
\begin{equation*}
    c^2\dd{\tau}^2=\frac{c^2\dd{t}^2}{\gamma^2(|\mathbf{v}|)}=\dd{t}^2\left\{c^2-\left(\dv{x_1}{t}\right)^2-\left(\dv{x_2}{t}\right)^2-\left(\dv{x_3}{t}\right)^2\right\}=-\dd{s}^2\,,
\end{equation*}
ponieważ interwał czasoprzestrzenny \(\dd{s}\) jest czteroskalarem, więc tak zdefiniowana wielkość
również jest czteroskalarem. Wielkość tą nazywamy \textit{czasem własnym cząstki materialnej}.\\

\noindent\textbf{Czteroprędkość.}\\
Czas własny posłuży nam do zdefiniowana czterowektora prędkości (\textit{czteroprędkości}). Istotnie
ponieważ \(\dd x\) jest czterowektorem i \(\dd{\tau}\) jest czteroskalarem, więc (oznaczając
\(|\mathbf{v}|=v\))
\begin{equation*}
    u:=\dv[]{x}{\tau}=(\gamma(v)\mathbf{v},\gamma(v)c)
\end{equation*}
jest czterowektorem. Korzystając zatem z faktu, iż dowolny czterowektor transformuje się zgodnie z
transformacją Lorentza możemy alternatywnie wyprowadzić relatywistyczny wzór na dodawanie prędkości.
Istotnie mamy
\begin{equation*}
    u_4'=\gamma(u_4-\beta u_1)\,,
\end{equation*}
gdzie \(\gamma\) bez argumentu \(\gamma()\) oznacza \(\frac{1}{\sqrt{1-\frac{V^2}{c^2}}}\), skąd
\begin{equation*}
    \gamma(v')=\gamma\gamma(v)\left(1-\frac{Vv_1}{c^2}\right)\,.
\end{equation*}
Z powyższego zatem
\begin{equation*}
    \begin{split}
        &u_1'=\gamma(v')v_1'=\gamma(u_1-\beta u_4)=\gamma\gamma(v)(v_1-V)\\
        &u_2'=\gamma(v')v_2'=u_2=\gamma(v)v_2\\
        &u_3'=\gamma(v')v_3'=u_3=\gamma(v)v_3
    \end{split}\quad\quad\,,
\end{equation*}
skąd otrzymujemy
\begin{equation*}
    \begin{split}
        &v'_1=\frac{v_1-V}{1-\frac{Vv_1}{c^2}}\\
        &v'_2=\frac{1}{\gamma}\frac{v_2}{1-\frac{Vv_1}{c^2}}\\
        &v'_3=\frac{1}{\gamma}\frac{v_3}{1-\frac{Vv_1}{c^2}}
    \end{split}\quad\quad\,,
\end{equation*}
w zgodzie z wcześniej wyprowadzonymi wzorami.\\
Poniżej podamy kilka użytecznych związków. Z definicji czteroprędkości mamy od razu
\begin{equation*}
    u\cdot u=\frac{1}{1-\frac{v_1^2+v_2^2+v_3^2}{c^2}}\left(v_1^2+v_2^2+v_3^2-c^2\right)=-c^2\,.
\end{equation*}
Ponieważ wiemy, że iloczyn dowolnych dwóch czterowektorów jest czteroskalarem, więc w dowolnym
inercjalnym układzie odniesienia \(u\cdot u=-c^2\). Rozpatrzmy teraz iloczyn dwóch różnych
czterowektorów \(u_1=\gamma_1(\mathbf{v},c)\) i \(u_2=\gamma_2(\mathbf{w},c)\)
\begin{equation*}
\begin{split}
    u_1\cdot u_2&=\gamma_1\gamma_2\left(\mathbf{v}\cdot\mathbf{w}-c^2\right)=-c^2\frac{c^2-\mathbf{v}\cdot\mathbf{w}}{\sqrt{(c^2-\mathbf{v}\cdot\mathbf{w})^2-c^2(\mathbf{v}-\mathbf{w})^2}}\\
    &=\frac{-c^2}{\sqrt{1-\frac{1}{c^2}\left(\frac{\mathbf{v}-\mathbf{w}}{1-\frac{\mathbf{v}\cdot\mathbf{w}}{c^2}}\right)^2}}=\frac{-c^2}{\sqrt{1-\frac{v^2_\text{wzg}}{c^2}}}=-\gamma_\text{wzg}c^2\,.
\end{split}
\end{equation*}
\subsubsection{Obrót Thomasa--Wignera}
Rozważmy trzy cząstki \(A\), \(B\), \(C\). Cząstka \(B\) porusza się względem cząstki \(A\) wzdłuż
prostej \(AB\) z prędkością \(v\). Natomiast cząstka \(C\) porusza się względem cząstki \(B\) wzdłuż
prostej prostopadłej do \(AB\) z prędkością \(v'\). Chcemy wyznaczyć prędkość \(\mathbf{u}\) cząstki
\(C\) względem \(A\). Z relatywistycznego wzoru na dodawanie prędkości mamy
\begin{equation*}
\begin{split}
    &u_\parallel'=\frac{u_\parallel-v}{1-\frac{u_\parallel v}{c^2}}=0\\
    &u_\perp '=\frac{1}{\gamma}\frac{u_\perp}{1-\frac{u_\parallel v}{c^2}}=v'
\end{split}\quad\quad\,,
\end{equation*}
skąd otrzymujemy
\begin{equation*}
    u_\parallel=v\,,\quad u_\perp=\frac{v'}{\gamma}\,,
\end{equation*}
gdzie \(\gamma=\frac{1}{\sqrt{1-v^2/c^2}}\). Cząstka \(C\) porusza się zatem względem cząstki \(A\)
z szybkością
\begin{equation*}
    u=\sqrt{v^2+v'^2-\frac{v^2v'^2}{c^2}}\,,
\end{equation*}
wzdłuż prostej \(AC\) nachylonej pod kątem
\begin{equation*}
    \theta=\arctan\frac{v'}{\gamma v}
\end{equation*}
do prostej \(AB\). Wyznaczmy teraz prędkość \(\mathbf{w}\) cząstki \(A\) względem \(C\). Względem
\(C\) cząstka \(B\) porusza się z prędkością \(-v'\) wzdłuż prostej prostopadłej do \(AB\),
natomiast \(A\) porusza się względem \(B\) z prędkością \(-v\) wzdłuż \(AB\). Z relatywistycznego
wzoru na dodawanie prędkości mamy zatem
\begin{equation*}
    \begin{split}
        &w_\parallel '=\frac{1}{\gamma'}\frac{w_\parallel}{1+\frac{w_\perp v'}{c^2}}=-v\\
        &w_\perp'=\frac{w_\perp+v'}{1+\frac{w_\perp v'}{c^2}}=0
    \end{split}\quad\quad\,,
\end{equation*}
skąd otrzymujemy
\begin{equation*}
    w_\parallel=-\frac{v}{\gamma'}\,,\quad w_\perp=-v'\,,
\end{equation*}
gdzie \(\gamma'=\frac{1}{\sqrt{1-v'^2/c^2}}\). Cząstka \(A\) porusza się zatem względem cząstki
\(C\) z szybkością
\begin{equation*}
    w=\sqrt{v^2+v'^2-\frac{v^2v'^2}{c^2}}=u\,,
\end{equation*}
wzdłuż prostej \(CA\) nachylonej pod kątem
\begin{equation*}
    \theta'=\arctan\frac{\gamma'v'}{v}
\end{equation*}
do prostej prostopadłej do \(BC\). Widzimy więc, że w ogólności \(\theta\neq\theta'\), ale prędkość
cząstki \(C\) względem \(A\) mierzona wzdłuż prostej \(AC\) jest równa prędkości cząstki \(A\)
względem \(C\) mierzonej wzdłuż prostej \(CA\). Obrót  o kąt \(\phi\) zdefiniowany jako
\begin{equation*}
    \phi:=\theta'-\theta=\arctan\frac{\gamma'v'}{v}-\arctan\frac{v'}{\gamma v}\,,
\end{equation*}
nazywamy \textit{obrotem Wignera}. Możemy podać bardziej eleganckie wyrażenie na kąt \(\phi\).
Istotnie
\begin{equation*}
    \tan\phi=\frac{\frac{\gamma'\beta'}{\beta}-\frac{\beta'}{\gamma\beta}}{1+\frac{\gamma'\beta'^2}{\gamma\beta^2}}=\frac{\beta\beta'(\gamma\gamma'-1)}{\beta^2\gamma+\beta'^2\gamma'^2}\,,
\end{equation*}
ale \(\beta^2=1-\frac{1}{\gamma^2}\), zatem
\begin{equation*}
    \tan\phi=\frac{\beta\beta'(\gamma\gamma'-1)}{\gamma+\gamma'-\frac{1}{\gamma}-\frac{1}{\gamma'}}=\frac{\beta\beta'\gamma\gamma'}{\gamma+\gamma'}\,.
\end{equation*}
Dla \(v'=\delta v'\ll 1\) mamy
\begin{equation*}
    \tan\theta=\frac{\delta v'}{\gamma v}\approx\theta\,,\quad \tan\theta'=\frac{\delta v'}{v\sqrt{1-\frac{\delta v'^2}{c^2}}}\approx \frac{\delta v'}{v}\approx \theta'\,,
\end{equation*}
skąd
\begin{equation*}
    \delta \phi\approx\frac{\gamma-1}{\gamma}\frac{\delta v'}{v}\,.
\end{equation*}
Jeśli dodatkowo chcemy wyrazić \(\delta\phi\) tylko przez wielkości mierzone w układzie \(A\) to
musimy podstawić \(\delta v'=\gamma\,\delta v\) i otrzymujemy
\begin{equation*}
    \delta\phi=(\gamma-1)\frac{\delta v}{v}=\frac{\gamma^2}{\gamma+1}\frac{v\,\delta v}{c^2}\,.
\end{equation*}
Zauważmy również, że w granicy \(\delta v'\ll 1\) kąt \(\delta \phi\) jest równy różnicy kątów pod
jakimi prosta \(AC\) jest nachylona do prostej \(AB\) odpowiednio względem układu \(A\) i \(B\).
\medskip

\noindent \textbf{Obrót wektora} 
\medskip

Rozważmy teraz cztery cząstki \(A\), $B$, $C$, $D$, przy czym cząstka $D$ również porusza się
względem $B$ prostopadle do $AB$ z prędkością \(v'\) i wektor \(CD\) jest równoległy do prostej
\(AB\) (względem \(B\)). Niech
\begin{equation*}
    \begin{cases}
    &x=0\\
    &y=0\\
    &z=0\\
    &t=t\\
    \end{cases}\quad\,,\quad
    \begin{cases}
     &x=L\\
     &y=0\\
     &z=0\\
     &t=t\\
    \end{cases}
\end{equation*}
będą odpowiednio współrzędnymi czasoprzestrzennymi zdarzeń polegających na przecięciu prostej
\(y=0\) przez cząstki \(C\), \(D\) mierzonych w układzie \(B\) (przyjmujemy tutaj oś \(x\)
pokrywającą się z prostą \(AB\) i oś \(y\) prostopadłą do niej). Ponieważ zakładamy, że wektor
\(CD\) jest równoległy do \(AB\), więc cząstki oczywiście przetną ową prostą w tej samej chwili.
Zgodnie z transformacją Lorentza w układzie \(A\) zdarzenia te zaszły natomiast w 
    \begin{equation*}
    \begin{cases}
    &x'=\gamma vt\\
    &y'=0\\
    &z'=0\\
    &t'=\gamma t\\
    \end{cases}\quad\,,\quad
    \begin{cases}
     &x'=\gamma(L+vt)\\
     &y'=0\\
     &z'=0\\
     &t'=\gamma(t+vL/c^2)\\
    \end{cases}\quad\quad\,.
\end{equation*}
Składowe wektora \(CD\) w układzie \(A\) wynoszą zatem
\begin{equation*}
    L_\parallel=\Delta x'-v\Delta t'=\gamma L\left(1-\frac{v^2}{c^2}\right)=\frac{L}{\gamma}\,,\quad L_\perp=\frac{v'}{\gamma}\Delta t'=\frac{vv'L}{c^2}\,,
\end{equation*}
gdzie skorzystaliśmy z tego, iż prędkości cząstek \(C\), \(D\) względem \(A\) wynoszą
\([v,v'/\gamma]\). Z powyższego widzimy zatem, że względem \(A\) wektor \(CD\) jest nachylony do
prostej \(AB\) pod kątem \(\tilde{\phi}\) równym
\begin{equation*}
    \tan\tilde{\phi}=\frac{L_\perp}{L_\parallel}=\frac{1}{\sqrt{1-\frac{v^2}{c^2}}}\frac{vv'}{c^2}=\gamma\beta\beta'\,,
\end{equation*}
a jego długość wynosi
\begin{equation*}
    L'=\sqrt{L_\perp^2+L_\parallel^2}=L\sqrt{1-\frac{\beta^2}{\gamma'^2}}\,.
\end{equation*}
Z powyższego widzimy, że wektor \(CD\) równoległy do \(AB\) względem układu odniesienia \(B\), nie
jest równoległy do \(AB\) w układzie odniesienia \(A\). Jednakże kąt, o jaki jest obrócony, nie jest
równy kątowi Wignera. Jak łatwo sprawdzić, jeśli w układzie \(B\) wektor \(CD\) tworzy kąt
\(\alpha\) z prostą \(AB\), to w układzie \(A\) tworzy on kąt
\begin{equation*}
    \tan\tilde{\phi}=\gamma\tan\alpha+\gamma\beta\beta'
\end{equation*}
z prostą \(AB\).
\subsubsection{Dynamika relatywistyczna}
Poprzednie rozważania dotyczyły głównie kinematyki relatywistycznej. Teraz przejdziemy do opisu
zagadnień dynamicznych.
\medskip

\noindent\textbf{Pojęcie masy w STW}
\medskip

Masę \(m\) cząstki materialnej definiujemy jako czteroskalar, który w układzie spoczynkowym danej
cząstki jest równy jej klasycznej masie (którą będziemy także nazywać \textit{masą spoczynkową}).
\medskip

\noindent\textbf{Pęd relatywistyczny}
\medskip

Zdefiniowaliśmy już pojęcie masy w STW jako pewnego czteroskalara oraz czteroprędkości \(u\),
naturalnym będzie więc zdefiniowanie \textit{czteropędu} cząstki materialnej, jako iloczynu tych
wielkości (oczywiście odpowiednia reguła ilorazowa gwarantuje nam, że iloczyn czteroskalara i
czterowektora jest czterowektorem)
\begin{equation*}
    p=mu=\left(\gamma(v)m\mathbf{v},\gamma(v)mc\right)\,.
\end{equation*}
Zauważmy tutaj, że składowa przestrzenna czteropędu (będziemy ją również nazywać \textit{trójpędem}
\(\mathbf{p}=\gamma(v)m\mathbf{v}\)) jest równa klasycznemu pędowi \(m\mathbf{v}\) jedynie w granicy
\(\gamma\to 1\). Czteropęd jest niezwykle użytecznym pojęciem, gdyż okazuje się, że w układzie
izolowanym spełniona jest zasada zachowania czteropędu, przy czym całkowity czteropęd \(P\) jest po
prostu sumą czteropędów poszczególnych cząstek materialnych.\\

\noindent\fbox{%
    \parbox{\textwidth}{%
    \textbf{Zasada zachowania czteropędu (czterowektora energii--pędu)}\\
        Jeżeli układ cząstek materialnych jest izolowany, wówczas całkowity czteropęd \(P\) układu
    jest stały (względem danego układu odniesienia). }%
}
\medskip

\noindent\fbox{%
    \parbox{\textwidth}{%
    \textbf{Definicja energii relatywistycznej}\\
        Energię relatywistyczną \(E\) cząstki materialnej o czteropędzie \(p=(\mathbf{p},p_4)\)
        definiujemy jako składową czasową czteropędu \(p\) pomnożoną przez prędkość światła w próżni
        \begin{equation*}
            E:=cp_4=\gamma(v)mc^2
        \end{equation*}
    }%
}
\medskip

\noindent Przy takiej definicji energii relatywistycznej możemy zapisać
\begin{equation*}
    p=(\mathbf{p},\gamma(v)mc)=\left(\gamma(v)m\mathbf{v},\frac{E}{c}\right)\,.
\end{equation*}
Z tego powodu czteropęd nazywamy również \textit{czterowektorem energii--pędu}. Pokażemy teraz, że
tak zdefiniowana \(E\) odpowiada klasycznej energii kinetycznej cząstki o prędkości \(v\). Istotnie
korzystając z rozwinięcia Taylora dla małych \(v/c\) mamy
\begin{equation*}
    E=\frac{mc^2}{\sqrt{1-\frac{v^2}{c^2}}}\approx mc^2+\frac{1}{2}mv^2\,.
\end{equation*}
Człon stały \(mc^2\) nazywamy \textit{energią spoczynkową} cząstki materialnej. Istotnie jeśli
\(v=0\) to zgodnie z definicją \(E_0=mc^2\). W mechanice klasycznej stała ta nie ma znaczenia,
jednak w STW energia jest w pewien sposób określona absolutnie, gdyż tylko dla takiej postaci \(E\)
czterowektor energii--pędu jest czterowektorem, czyli prawo zachowania czteropędu jest niezmiennicze
względem transformacji Lorentza. Często przydatne są następujące użyteczne związki
\begin{equation*}
    p\cdot p=-(mc)^2=\mathbf{p}\cdot\mathbf{p}-\frac{E^2}{c^2}
\end{equation*}
\begin{equation*}
    \boldsymbol{\beta}:=\frac{\mathbf{v}}{c}=\frac{\mathbf{p}c}{E}
\end{equation*}
\begin{equation*}
    E=\sqrt{\mathbf{p}^2c^2+m^2c^4}\,.
\end{equation*}
Dla dowolnych dwóch czterowektorów energii--pędu \(p_a=(\mathbf{p}_a,E_a/c)\) i
\(p_b=(\mathbf{p}_b,E_b/c)\) zachodzi
\begin{equation*}
    p_a\cdot p_b=m_am_bu_a\cdot u_b=-m_am_b\gamma_\text{wzg}c^2\,,
\end{equation*}
gdzie
\begin{equation*}
    \mathbf{v}_\text{wzg}=\frac{\mathbf{v}_a-\mathbf{v}_b}{1-\frac{\mathbf{v}_a\cdot\mathbf{v}_b}{c^2}}\,.
\end{equation*}
Zdefiniujmy również \textit{relatywistyczną energię kinetyczną} \(T\) jako wielkość
\begin{equation*}
    T:=E-E_0=E-mc^2=(\gamma(v)-1)mc^2\,.
\end{equation*}
Korzystając z rozwinięcia Taylora możemy zapisać
\begin{equation*}
    T\approx \frac{1}{2}mc^2\beta^2+\frac{3}{8}mc^2\beta^4+\frac{5}{16}mc^2\beta^6\dotso\,.
\end{equation*}

\subsubsection{Zderzenia}
Zasada zachowania czterowektora energii--pędu pozwala rozwiązywać zagadnienia dotyczące zderzeń
relatywistycznych cząstek materialnych. Najłatwiej takie zadania rozwiązywać w \textit{układzie
środka czteropędu}, który definiujemy jako układ odniesienia, w którym całkowity trójpęd układu
wynosi \(0\), tj.
\begin{equation*}
    P=(\mathbf{0},P_4)\,.
\end{equation*}
Jeśli w pewnym układzie odniesienia \(S\) całkowity czteropęd układu wynosi
\begin{equation*}
    P=(\mathbf{P},P_4)\,,
\end{equation*}
to układ środka czteropędu \(S'\) jest powiązany z \(S\) standardowym pchnięciem z prędkością
\begin{equation*}
    \boldsymbol{\mathfrak{P}}=\frac{\mathbf{P}}{P_4}=\frac{\mathbf{P}c}{E}\,.
\end{equation*}
\textbf{Zderzenie centralne}
\medskip

Jako przykład rozwiążmy zagadnienie sprężystego, centralnego zderzenia dwóch cząstek
relatywistycznych o masach \(m\) i \(M\), przy czym przed zderzeniem w układzie laboratoryjnym \(S\)
cząstka \(M\) spoczywa, a cząstka \(m\) porusza się z prędkością \(\boldsymbol{\beta}=[\beta,0,0]\),
gdzie przyjmujemy oś \(x\) pokrywającą się z prostą łączącą obie cząstki. W układzie \(S\) całkowity
czteropęd układu wynosi
\begin{equation*}
    P=\left(\frac{mc\boldsymbol{\beta}}{\sqrt{1-\beta^2}},\frac{mc}{\sqrt{1-\beta^2}}+Mc\right)\,.
\end{equation*}
Układ środka czteropędu \(S'\) jest powiązany z \(S\) standardowym pchnięciem z prędkością
\begin{equation*}
    \boldsymbol{\mathfrak{P}}=\frac{m\boldsymbol{\beta}}{m+M\sqrt{1-\beta^2}}\,.
\end{equation*}
W tym układzie odniesienia \(\mathbf{P}'=0\), zatem przed zderzeniem cząstki poruszają się z
trójpędami
\begin{equation*}
    \mathbf{p}'_m=\frac{Mc\boldsymbol{\mathfrak{P}}}{\sqrt{1-\mathfrak{P}^2}}=-\mathbf{p}'_M=\mathbf{p}'
\end{equation*}
oraz po zderzeniu \(\mathbf{p}_m''=-\mathbf{p}_M''=\mathbf{p}''\). Jednocześnie z zachowania
czasowej składowej całkowitego czteropędu układu mamy
\begin{equation*}
    \sqrt{\mathbf{p}'^2+m^2c^2}+\sqrt{\mathbf{p}'^2+M^2c^2}=\sqrt{\mathbf{p}''^2+m^2c^2}+\sqrt{\mathbf{p}''^2+M^2c^2}\,.
\end{equation*}
Ponieważ funkcja \(\sqrt{x+a}+\sqrt{x+b}\) jest różnowartościowa, więc wnioskujemy, że
\(|\mathbf{p}'|=|\mathbf{p}''|\). Ponieważ zakładamy, że zderzenie jest centralne, więc z powyższego
wynika \(\mathbf{p}'=-\mathbf{p}''\). Ostatecznie zatem trójpędy cząstek po zderzeniu wynoszą (w
układzie \(S'\))
\begin{equation*}
    \mathbf{p}_m''=-\frac{Mc\boldsymbol{\mathfrak{P}}}{\sqrt{1-\mathfrak{P}^2}}=-\mathbf{p}_M''\,.
\end{equation*}
W układzie \(S'\) czterowektor energii--pędu cząstki \(M\) po zderzeniu wynosi zatem
\begin{equation*}
    p_M''=\left(\frac{Mc\boldsymbol{\mathfrak{P}}}{\sqrt{1-\mathfrak{P}^2}},\frac{Mc}{\sqrt{1-\mathfrak{P}^2}}\right)\,,
\end{equation*}
zatem w układzie laboratoryjnym \(S\) trójpęd cząstki \(M\) po zderzeniu wynosi
\begin{equation*}
    \mathbf{p}_M=\frac{2Mc\boldsymbol{\mathfrak{P}}}{1-\mathfrak{P}^2}\,,
\end{equation*}
co odpowiada prędkości
\begin{equation*}
    \mathbf{v}_M=\frac{2\boldsymbol{\mathfrak{P}}c}{1+\mathfrak{P}^2}\,,
\end{equation*}
gdzie
\begin{equation*}
    \boldsymbol{\mathfrak{P}}=\frac{m\boldsymbol{\beta}}{m+M\sqrt{1-\beta^2}}\,.
\end{equation*}
Dla \(m=M\) mamy
\begin{equation*}
    \boldsymbol{\mathfrak{P}}=\frac{\boldsymbol{\beta}}{1+\sqrt{1-\beta^2}}\,,\quad 1+\mathfrak{P}^2=\frac{2}{1+\sqrt{1-\beta^2}}\,,
\end{equation*}
skąd
\begin{equation*}
    \mathbf{v}_M=\boldsymbol{\beta}c
\end{equation*}
czyli cząstka \(M\) ma po zderzeniu dokładnie prędkość uderzającej w nią cząstki \(m\). Jednocześnie
wówczas z zachowania czteropędu w układzie \(S\) mamy
\begin{equation*}
    \frac{mc\boldsymbol{\beta}}{\sqrt{1-\beta^2}}=\frac{m\mathbf{v}_m}{\sqrt{1-\frac{v_m^2}{c^2}}}+\frac{M\mathbf{v}_M}{\sqrt{1-\frac{v_M^2}{c^2}}}=\frac{m\mathbf{v}_m}{\sqrt{1-\frac{v_m^2}{c^2}}}+\frac{mc\boldsymbol{\beta}}{\sqrt{1-\beta^2}}\,,
\end{equation*}
czyli \(\mathbf{v}_m=\mathbf{0}\).
\medskip

\noindent\textbf{Zderzenie niecentralne}
\medskip

Rozpatrzmy teraz z kolei sprężyste zderzenie niecentralne dwóch cząstek o jednakowej masie \(m\).
Załóżmy, iż przed zderzeniem w układzie odniesienia \(S\) jedna z cząstek spoczywa, a druga porusza
się z prędkością \(\mathbf{v}=[v,0,0]\). Natomiast po zderzeniu cząstki poruszają się po torach
nachylonych do siebie pod kątem \(\Psi\). Znając energie relatywistyczne cząstek \(E\), \(E'\) po
zderzeniu (w układzie \(S\)) chcemy wyznaczyć kąt \(\Psi\). Z zasady zachowania czterowektora
energii--pędu mamy (poniżej kładę \(c=1\))
\begin{equation*}
\begin{split}
    &\frac{m}{\sqrt{1-v^2}}=E+E'-m\\
    &\frac{m\mathbf{v}}{\sqrt{1-v^2}}=\frac{m\mathbf{u}}{\sqrt{1-u^2}}+\frac{m\mathbf{w}}{\sqrt{1-w^2}}
\end{split}
\end{equation*}
Podnosząc drugie równanie obustronnie do kwadratu i korzystając z 
\begin{equation*}
    E=\frac{m}{\sqrt{1-u^2}}\,,\quad E'=\frac{m}{\sqrt{1-w^2}}\,,\quad \frac{m}{\sqrt{1-v^2}}=E+E'-m
\end{equation*}
otrzymujemy
\begin{equation*}
    (E+E'-m)^2-m^2=E^2+E'^2-2m^2+2\sqrt{(E^2-m^2)(E'^2-m^2)}\cos\Psi\,,
\end{equation*}
skąd
\begin{equation*}
    \cos\Psi=\sqrt{\frac{(E-m)(E'-m)}{(E+m)(E'+m)}}\,.
\end{equation*}
Zauważmy, iż w granicy nierelatywistycznej (tj. \(c\to\infty\)) otrzymujemy oczywiście
\(\Psi=\frac{\pi}{2}\), co jest zgodne z klasycznymi rozważaniami.
\subsubsection{Siła}
Po zdefiniowaniu czteropędu (czterowektora energii--pędu) jako wielkości
\begin{equation*}
    p=(\gamma(v)m\mathbf{v},\gamma(v)mc)=(\mathbf{p},E/c)
\end{equation*}
naturalnym sposobem zdefiniowania czterowektora siły (\textit{czterosiły}) \(K\) będzie
\begin{equation*}
    K:=\dv[]{p}{\tau}=\gamma(v)\dv[]{p}{t}=\left(\gamma(v)\dv[]{\mathbf{p}}{t},\frac{\gamma(v)}{c}\dv[]{E}{t}\right)=\left(\gamma(v)\mathbf{F},\frac{\gamma(v)}{c}\dv[]{E}{t}\right)\,,
\end{equation*}
gdzie wielkośc \(\mathbf{F}\) nazywamy \textit{trójsiłą}. Aby wyznaczyć wzory transformacyjne dla
\(\mathbf{F}\) mierzonej w danym punkcie \(\mathbf{x}\), przy standardowym pchnięciu z prędkością
\(V\)zauważmy, że
\begin{equation*}
\begin{split}
    &F_1'=\dv[]{p{'}_1}{t{'}}=\frac{\gamma\dd{p_1}-\gamma\beta\dd{p_4}}{\gamma\dd{t}-\gamma\frac{V\dd{x_1}}{c^2}}=\frac{F_1-\frac{\beta}{c}\dv[]{E}{t}}{1-\beta v_1/c}\\
    &F_2'=\dv[]{p{'}_2}{t{'}}=\frac{\dd{p_2}}{\gamma\dd{t}-\gamma\frac{V\dd{x_1}}{c^2}}=\frac{1}{\gamma(V)}\frac{F_2}{1-\beta v_1/c}\\
    &F_2'=\dv[]{p{'}_3}{t{'}}=\frac{\dd{p_3}}{\gamma\dd{t}-\gamma\frac{V\dd{x_1}}{c^2}}=\frac{1}{\gamma(V)}\frac{F_3}{1-\beta v_1/c}
\end{split}\quad\quad\,.
\end{equation*}


Ważne jest zaznaczenie, iż w przypadku siły Lorentza możemy zapisać
\begin{equation*}
    \mathbf{F}=q(\mathbf{E}+\mathbf{v}\times\mathbf{B})\,,
\end{equation*}
gdzie \(\mathbf{F}\) jest zdefiniowaną powyżej trójsiłą. Oczywiście ponieważ \(\dd{p}\) jest
czterowektorem i czas własny \(\dd{\tau}\) jest czteroskalarem, więc tak zdefiniowana czterosiła z
całą pewnością jest czterowektorem. Zauważmy jednocześnie, że z definicji energii relatywistycznej
mamy
\begin{equation*}
    E^2=\mathbf{p}^2c^2+m^2c^4\,.
\end{equation*}
Różniczkując obustronnie otrzymujemy zatem
\begin{equation*}
    E\dv[]{E}{\tau}=c^2\mathbf{p}\cdot\dv[]{\mathbf{p}}{\tau}+c^4m\dv[]{m}{\tau}\,,
\end{equation*}
co możemy zapisać jako
\begin{equation*}
    -mc^2\dv[]{m}{\tau}=-\gamma(v)mc^2\dv[]{m}{t}=K\cdot p\,.
\end{equation*}
Ponieważ w ogólności \(K\cdot p\neq 0\), więc masa cząstki może zmieniać się w czasie, jeśli na
cząstkę działa pewna czterosiła. Podkreślmy, iż fakt, że masa cząstki może się zmieniać nie ma nic
wspólnego z jej niezmienniczością względem transformacji Lorentza. W szczególności jednak kiedy
\(K\cdot p= 0\) (np. dla sił elektromagnetycznych) wówczas \(\dv[]{m}{t}=0\). Ponieważ \(K\cdot p\)
jest czteroskalarem, więc ma tę samą wartość we wszystkich inercjalnych układach odniesienia, w
szczególności w układzie spoczynkowym cząstki \(p=(\mathbf{0},mc)\), zatem
\begin{equation*}
    K\cdot p=m\dv[]{E}{t}\,,
\end{equation*}
gdzie \(\dv[]{E}{t}\) jest zmianą energii relatywistycznej cząstki w jej spoczynkowym układzie
odniesienia. Dla sił, które nie zmieniają masy cząstki mamy tożsamość
\begin{equation*}
    \dv[]{E}{t}=\frac{\mathbf{p}c^2}{E}\cdot\dv[]{\mathbf{p}}{t}=\mathbf{v}\cdot\dv[]{\mathbf{p}}{t}=\mathbf{v}\cdot\mathbf{F}\,,
\end{equation*}
czyli dla takich sił czterowektor siły ma postać
\begin{equation*}
    K=\left(\gamma(v)\mathbf{F},\gamma(v)\boldsymbol{\beta}\cdot\mathbf{F}\right)\,.
\end{equation*}
W takim przypadku możemy również zapisać
\begin{equation*}
    \mathbf{F}=m\dv[]{}{t}\left(\frac{\mathbf{v}}{\sqrt{1-\frac{v^2}{c^2}}}\right)=\frac{m}{\sqrt{1-\frac{v^2}{c^2}}}\dv[]{\mathbf{v}}{t}+\frac{m\mathbf{v}}{c^2}\frac{v}{\left(1-\frac{v^2}{c^2}\right)^{3/2}}\dv[]{v}{t}\,.
\end{equation*}
Zauważmy, że z powyższego
\begin{equation*}
    \mathbf{F}\cdot\mathbf{v}=\frac{mv}{\sqrt{1-\frac{v^2}{c^2}}}\dv[]{v}{t}\left(1+\frac{\beta^2}{1-\beta^2}\right)=\frac{mv}{\left(1-\frac{v^2}{c^2}\right)^{3/2}}\dv[]{v}{t}\,,
\end{equation*}
zatem możemy również zapisać
\begin{equation*}
    \mathbf{F}=\frac{m}{\sqrt{1-\frac{v^2}{c^2}}}\dv[]{\mathbf{v}}{t}+\frac{\mathbf{v}}{c^2}(\mathbf{F}\cdot\mathbf{v})\,.
\end{equation*}
\noindent\fbox{%
    \parbox{\textwidth}{%
        Dla sił niezmieniających masy cząstki (np. dla sił elektromagnetycznych) zachodzi
        \begin{equation*}
            \mathbf{F}=\gamma m\dv[]{\mathbf{v}}{t}+\gamma^3\beta\boldsymbol{\beta}m\dv[]{v}{t}\,.
        \end{equation*}
    }%
}
\medskip

Zauważmy w szczególności, iż dla ruchu po okręgu o promieniu \(R\) mamy po prostu
\(F_\text{doś}=\gamma mv^2/R\). Zatem dla ruchu elektronu po okręgu w polu magnetycznym \(B\) z
prędkością relatywistyczną promień orbity wynosi
\begin{equation*}
    R=\frac{\gamma mv}{eB}\,.
\end{equation*}
\medskip

Jako przykład rozważmy ruch cząstki materialnej o masie stałej \(m\) w płaszczyźnie \(xy\) pod
wpływem stałej trójsiły \(\mathbf{F}=[F,0,0]\) (może to być np. siła działająca na ładunek w stałym,
jednorodnym polu elektrycznym), jeśli \(x(0)=0\), \(y(0)=0\), \(\mathbf{p}(0)=[p_{xo},p_{yo},0]\).
Ponieważ
\begin{equation*}
    \mathbf{F}=[F,0,0]=\dv[]{\mathbf{p}}{t}=\left[\dv[]{p_x}{t},\dv[]{p_y}{t},\dv[]{p_z}{t}\right]\,,
\end{equation*}
zatem
\begin{equation*}
    p_x(t)=Ft+p_{xo}\,,\quad p_y(t)=p_{yo}\,,\quad p_z(t)=0\,.
\end{equation*}
Jednocześnie mamy
\begin{equation*}
    p_x(t)=\frac{mv_x(t)}{\sqrt{1-\frac{v^2(t)}{c^2}}}\,,\quad p_y(t)=\frac{mv_y(t)}{\sqrt{1-\frac{v^2(t)}{c^2}}}\,,
\end{equation*}
skąd, podnosząc oba wyrażenia do kwadratu i dodając
\begin{equation*}
    \frac{v^2(t)}{c^2}=\frac{p_x^2+p_y^2}{m^2c^2+p_x^2+p_y^2}=\frac{(Ft+p_{xo})^2+p^2_{yo}}{m^2c^2+(Ft+p_{xo})^2+p^2_{yo}}\,.
\end{equation*}
Wprowadzając wielkośc \(E_o^2/c^2:=p_{xo}^2+p_{yo}^2+m^2c^2\) otrzymujemy
\begin{equation*}
\begin{split}
    v_x(t)=\frac{c(Ft+p_{xo})}{\sqrt{\frac{E_o^2}{c^2}+Ft(Ft+2p_{xo})}}\\
    v_y(t)=\frac{p_{yo}c}{\sqrt{\frac{E_o^2}{c^2}+Ft(Ft+2p_{xo})}}
\end{split}\quad\quad\,.
\end{equation*}
W szczególnym przypadku \(p_{xo}=p_{yo}=0\) otrzymujemy
\begin{equation*}
    v_x(t)=\frac{Fct}{\sqrt{m^2c^2+F^2t^2}}\,,
\end{equation*}
zauważmy, że dla \(t\to\infty\) otrzymujemy \(v_x(t)=c\). Z kolei w granicy nierelatywistycznej
(\(c\to\infty\)) mamy \(v_x(t)=Ft/m\). Ruch taki nazywamy \textit{ruchem hiperbolicznym}. Istotnie
całkując mamy
\begin{equation*}
    x(t)=\sqrt{\frac{c^4}{\alpha^2}+c^2t^2}\,,
\end{equation*}\\
gdzie \(\alpha:=F/m\) i przyjąłem \(x(0)=c^2/\alpha\). Równanie to opisuje oczywiście hiperbole w
układzie współrzędnych \((x,t)\).
\subsubsection{Czterotensory}
Czterotensor \(T\) (drugiego rzędu) definiujemy jako zespół szesnastu liczb \(T_{ij}\), które przy
zmianach układu odniesienia transformują się zgodnie z
\begin{equation*}
    T'=\left[\begin{array}{cccc}
         T'_{11}&T'_{12}&T'_{13}&T'_{14}  \\
         T'_{21}&T'_{22}&T'_{23}&T'_{24}  \\
         T'_{31}&T'_{32}&T'_{33}&T'_{34}  \\
         T'_{41}&T'_{42}&T'_{43}&T'_{44}  \\
    \end{array}\right]=\Lambda T\tilde{\Lambda}=\Lambda \left[\begin{array}{cccc}
         T_{11}&T_{12}&T_{13}&T_{14}  \\
         T_{21}&T_{22}&T_{23}&T_{24}  \\
         T_{31}&T_{32}&T_{33}&T_{34}  \\
         T_{41}&T_{42}&T_{43}&T_{44}  \\
    \end{array}\right]\tilde{\Lambda}\,,
\end{equation*}
gdzie \(\Lambda\) jest macierzą standardowego pchnięcia, a \(\tilde{\Lambda}\) jest macierzą
transponowaną do \(\Lambda\), tj. dla elementów macierzy \(\lambda_{ij}\) i \(\tilde{\lambda}_{ji}\)
zachodzi
\begin{equation*}
    \lambda_{ij}=\tilde{\lambda}_{ji}\,.
\end{equation*}
Własności transponowania macierzy
\begin{itemize}
    \item Jeśli \(\Lambda\) jest macierzą symetryczną (tj. \(\lambda_{ij}=\lambda_{ji}\)) wówczas
    \begin{equation*}
        \tilde{\Lambda}=\Lambda\,.
    \end{equation*}
    \item Jeśli \(\Lambda\) jest macierzą kwadratową to 
    \begin{equation*}
        \det \tilde{\Lambda}=\det \Lambda\,.
    \end{equation*}
\end{itemize}

Iloczyn skalarny czterotensora \(T\) i czterowektora \(q\) definiujemy jako
\begin{equation*}
\begin{split}
    T\cdot q&:=\left[\begin{array}{cccc}
         T_{11}&T_{12}&T_{13}&T_{14}  \\
         T_{21}&T_{22}&T_{23}&T_{24}  \\
         T_{31}&T_{32}&T_{33}&T_{34}  \\
         T_{41}&T_{42}&T_{43}&T_{44}  \\
    \end{array}\right]\cdot\left[\begin{array}{c}
         q_1  \\
         q_2  \\
         q_3  \\
         q_4  \\
    \end{array}\right]\\
    &=\left[\begin{array}{c}
         T_{11}q_1+T_{12}q_1+T_{13}q_1-T_{14}q_4 \\
         T_{21}q_1+T_{22}q_1+T_{23}q_1-T_{24}q_4 \\
         T_{31}q_1+T_{32}q_1+T_{33}q_1-T_{34}q_4 \\
         T_{41}q_1+T_{42}q_1+T_{43}q_1-T_{44}q_4  
    \end{array}\right]
\end{split}
\end{equation*}
Okazuje się, że przy takiej definicji \(p=T\cdot q\) jest czterowektorem.

\subsubsection{STW i Elektrodynamika}
Okazuje się, że dla pola elektromagnetycznego możemy zdefiniować \textit{tensor pola
elektromagnetycznego} \(\mathscr{F}\), który transformuje się zgodnie z regułą
\(\mathscr{F}'=\Lambda\mathscr{F}\tilde{\Lambda}\)
\begin{equation*}
    \mathscr{F}=\frac{1}{c}\left[\begin{array}{cccc}
         0&cB_3&-cB_2&-E_1  \\
         -cB_3&0&cB_1&-E_2 \\
         cB_2&-cB_1&0&-E_3 \\
         E_1 &E_2&E_3&0
    \end{array}\right]\,.
\end{equation*}
Zauważmy, że tensor pola elektromagnetycznego jest antysymetryczny
(\(\mathscr{F}_{ij}=-\mathscr{F}_{ji}\)) oraz z powyższego
\begin{equation*}
\begin{split}
    \mathscr{F}\cdot u&=\frac{1}{c}\left[\begin{array}{cccc}
         0&cB_3&-cB_2&-E_1  \\
         -cB_3&0&cB_1&-E_2 \\
         cB_2&-cB_1&0&-E_3 \\
         E_1 &E_2&E_3&0
    \end{array}\right]\cdot \left[\begin{array}{c}
         u_1 \\
         u_2 \\
         u_3 \\
         u_4 
    \end{array}\right]\\
    &=\left[\begin{array}{c}
         \gamma(v)\left\{[\mathbf{v}\times\mathbf{B}]_1+E_1\right\}  \\
         \gamma(v)\left\{[\mathbf{v}\times\mathbf{B}]_2+E_2\right\}  \\
         \gamma(v)\left\{[\mathbf{v}\times\mathbf{B}]_3+E_3\right\}  \\
         \frac{\gamma(v)}{c}\mathbf{v}\cdot\mathbf{E}
    \end{array}\right]=\gamma(v)\left(\mathbf{v}\times\mathbf{B}+\mathbf{E},\frac{\mathbf{v}\cdot\mathbf{E}}{c}\right)\,.
\end{split}
\end{equation*}
Załóżmy, że ładunek cząstki \(q\) jest czteroskalarem, czyli \(q\mathscr{F}\cdot u\) jest
czterowektorem
\begin{equation*}
    q\mathscr{F}\cdot u=\gamma(v)q\left(\mathbf{v}\times\mathbf{B}+\mathbf{E},\frac{\mathbf{v}\cdot\mathbf{E}}{c}\right)\,.
\end{equation*}
Zauważmy, że składowa przestrzenna \(q\mathscr{F}\cdot u\) jest równa wyrażeniu na trójsiłę Lorentza
pomnożonemu przez \(\gamma\), natomiast część przestrzenna jest równa zmianie energii
relatywistycznej cząstki pomnożonej przez \(\gamma/c\). Istotnie, ponieważ z doświadczenia wynika,
iż siła Lorentza nie zmienia masy cząstki, więc zachodzi dla niej
\begin{equation*}
    \dv[]{\mathscr{E}}{t}=\mathbf{v}\cdot\mathbf{F}\,,
\end{equation*}
gdzie przez \(\mathscr{E}\) oznaczam energię relatywistyczną cząstki. Jednocześnie trójsiła wynosi
\(\mathbf{F}=q(\mathbf{v}\times\mathbf{B}+\mathbf{E})\) zatem
\begin{equation*}
    \dv[]{\mathscr{E}}{t}=q\mathbf{v}\cdot\mathbf{E}\,.
\end{equation*}
Widzimy zatem, że czterowektor \(q\mathscr{F}\cdot u\) ma postać czterowektora siły Lorentza
\begin{equation*}
    K=q\mathscr{F}\cdot u=\gamma(v)q\left(\mathbf{v}\times\mathbf{B}+\mathbf{E},\frac{\mathbf{v}\cdot\mathbf{E}}{c}\right)\,.
\end{equation*}
Wyraziliśmy więc jedno z praw elektrodynamiki w postaci relatywistycznie niezmienniczej. Aby
wyznaczyć składowe siły \(K'\) w układzie odniesienia \(S'\) powiązanym z \(S\) standardowym
pchnięciem z prędkością \(V\) musimy znać transformacje pól \(\mathbf{E}'\) i \(\mathbf{B}'\).
Korzystając z tego, że \(\mathscr{F}\) jest czterotensorem mamy
\begin{equation*}
    \mathscr{F}'=\Lambda\mathscr{F}\Lambda\,,
\end{equation*}
gdzie skorzystaliśmy, z własności \(\tilde{\Lambda}=\Lambda\) dla macierzy symetrycznych. Z
powyższego mamy zatem
\begin{equation*}
\begin{split}
   & \left[\begin{array}{cccc}
         0&cB_3'&-cB_2'&-E_1'  \\
         -cB_3'&0&cB_1'&-E_2'  \\
         cB'_2&-cB_1'&0'&-E_3'  \\
         E_1'&E_2'&E_3'&0  
    \end{array}\right]=\\
    &=\left[\begin{array}{cccc}
         0&\gamma cB_3-\beta\gamma E_2&-\gamma cB_2-\beta\gamma E_3&-E_1 \\
         -\gamma cB_3+\beta\gamma E_3&0&cB_1&\beta\gamma cB_3-\gamma E_2\\
         \gamma cB_2+\gamma\beta E_3&-cB_1&0&-\beta\gamma cB_2-\gamma E_3\\
         E_1&-\beta\gamma cB_3+\gamma E_2&\gamma\beta cB_2+\gamma E_3&0
    \end{array}\right]
\end{split}
\end{equation*}
skąd otrzymujemy transformacje pól \(\mathbf{E}\) i \(\mathbf{B}\). Oznaczając (\(x\),\(y\),\(z\))
zamiast \((1,2,3)\) mamy\\

\noindent\fbox{%
    \parbox{\textwidth}{%
        Transformacje pól \(\mathbf{E}=[E_x,E_y,E_z]\) i \(\mathbf{B}=[B_x,B_y,B_z]\) przy
        standardowym pchnięciu lorentzowskim z prędkością \(V\) mają postać
        \begin{equation*}
            \begin{array}{cc}
                 E_x'=E_x& cB_x'=cB_x  \\
                 E_y'=\gamma(E_y-\beta cB_z)& cB_y'=\gamma(cB_y+\beta E_z) \\
                 E_z'=\gamma(E_z+\beta cB_y)& cB_z'=\gamma(cB_z-\beta E_y)
            \end{array}\,,
        \end{equation*}
        gdzie \(\beta=V/c\) i \(\gamma^{-2}=1-\beta^2\). }%
}
\medskip

Podkreślmy tutaj, że transformacje te odnoszą się do pól \(\mathbf{E}\) i \(\mathbf{B}\) mierzonych
w danym punkcie \(\mathbf{x}\) względem \(S\) (tj. w punkcie \(\mathbf{x}'\) względem \(S'\)).\\
Łatwo pokazać prostym rachunkiem, iż \(\mathbf{E}\cdot\mathbf{B}\) oraz \(E^2-c^2B^2\) są
czteroskalarami.
\medskip

Relatywistyczny lagranżjan dla cząstki naładowanej poruszającej się w polach \(\mathbf{E}\),
\(\mathbf{B}\) ma postać
\begin{equation*}
    L=-mc^2\sqrt{1-\frac{\mathbf{v}^2}{c^2}}-q(V-\mathbf{v}\cdot\mathbf{A})\,,
\end{equation*}
gdzie \(m\) jest niezmienniczą i zachowaną masą cząstki.
\medskip


\end{document}