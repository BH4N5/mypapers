\documentclass[../main.tex]{subfiles}

\begin{document}
\section{Optyka}
\textit{It was not observation but theory which led me to this result which experience then
confirmed.}\begin{flushright}Augustin-Jean Fresnel\end{flushright}
\begin{center}
    \pgfspectra[width=10 cm, height=2cm, axis, axis color=white, axis font color=black]
\end{center}
\subsection{Optyka geometryczna}
Bezwzględny współczynnik załamania światła w danym ośrodku \(n\) definiujemy jako
\begin{equation*}
    n=\frac{c}{v}\,,
\end{equation*}
gdzie \(v\) jest prędkością światła w tym ośrodku. Okazuje się, że występuje zależność
\(v=v(\omega)\) (tzw. zjawisko \textit{dyspersji}). Dla typowych materiałów optycznych
\(v\propto1/\omega\).
\subsubsection{Zjawiska odbicia i załamania}
\noindent\fbox{%
    \parbox{\textwidth}{%
        \textbf{Prawo odbicia}\\
        Promień padający, odbity i normalna do powierzchni padania leżą w jednej płaszczyźnie, a kąt
    padania jest równy kątowi odbicia. }%
}
\medskip

\noindent\fbox{%
    \parbox{\textwidth}{%
        \textbf{Prawo Snelliusa}\\
        Promień padający, załamany i normalna do powierzchni padania leżą w jednej płaszczyźnie, a
        stosunek sinusa kąta padania do sinusa kąta załamania jest równy względnemu współczynnikowi
        załamania światła
        \begin{equation*}
            \frac{\sin\alpha}{\sin\beta}=\frac{n_\beta}{n_\alpha}\,.
        \end{equation*}
    }%
}

\subsubsection*{Zjawisko całkowitego wewnętrznego odbicia}
 Z prawa Snella wynika, że jeśli \(n_\beta>n_\alpha\) to \(\alpha>\beta\). Wynika stąd, że istnieje
 taki graniczny kąt \(\beta_\text{gr}\), że \(\alpha=\pi/2\). Z prawa Snella mamy wówczas
 \(n_\beta/n_\alpha=\sin \frac{\pi}{2}/\sin\beta_\text{gr}\), skąd
\begin{equation*}
    \beta_\text{gr}=\arcsin\frac{n_\alpha}{n_\beta}\,.
\end{equation*}
\noindent\fbox{%
    \parbox{\textwidth}{%
        \textbf{Zasada Fermata}\\
        Promień świetlny poruszający się w danym ośrodku między punktami \(A\), \(B\) ma tę
        własność, że droga optyczna
        \begin{equation*}
            ct=\int_A^Bn(\mathbf{r})\dd{l}
        \end{equation*}
        jest stacjonarna. }%
}\\

W powyższym wzorze \(\dd{l}\) to fragment tzw. \textit{drogi geometrycznej}, a \(n(\mathbf{r})\)
jest polem skalarnym opisującym współczynnik załamania światła w danym obszarze. Takie sformułowanie
zasady Fermata pozwala w prosty sposób rozwiązywać zadania polegające na znalezieniu trajektorii
biegu promienia świetlnego w ośrodku o zmiennym współczynniku załamania. Poniżej zamieszczam trzy
zadania.
\medskip

\textbf{Zadania.}
\begin{enumerate}
    \item \textit{Na ośrodek przezroczysty o współczynniku załamania zależnym od \(y\), w punkcie
    \(y=0\), pod kątem prostym pada promień światła. Jaka powinna być postać funkcji \(n(y)\), aby
    wewn. ośrodka promień biegł po paraboli \(y(x)=\alpha x^2\)? Przyjmij \(n(0)=n_0\).}
\medskip

\textit{Rozwiązanie.}
\medskip

Rozpatrzmy warstwę ośrodka na wysokości \(y\) i grubości \(\dd y\). Czas \(\dd t\) potrzebny światłu
na przebycie tej warstwy wynosi
\begin{equation*}
    \dd{t}=\frac{\sqrt{\dd{y}^2+\dd{x}^2}}{c/n(y)}\,.
\end{equation*}
Całkowity czas \(t\) potrzebny światłu na przebycie odległości \(\Delta y=y_B-y_A\) wynosi
\begin{equation*}
    ct=\int_{y_A}^{y_B} n(y)\sqrt{x'(y)^2+1}\,\dd{y}=\int_{y_A}^{y_B} \mathcal{F}(x(y), x'(y), y)\,\dd{y}\,.
\end{equation*}
Zgodnie z zasadą Fermata światło przebywa drogę z A do B po trajektorii \(y(x)\), dla której czas
\(t\) potrzebny na jej przebycie jest stacjonarny. Z rachunku wariacyjnego wiemy, że aby wielkość
dana w postaci powyższej całki była stacjonarna, funkcja podcałkowa \(\mathcal{F}\) musi spełniać
równania Eulera-Lagrange'a
\begin{equation*}
    \pdv{\mathcal{F}}{x}=\dv{}{y}\pdv{\mathcal{F}}{x'}\,.
\end{equation*}
Ponieważ \(\partial\mathcal{F}/\partial x=0\), zatem
\begin{equation*}
\pdv{\mathcal{F}}{x'}=n(y)\frac{x'(y)}{\sqrt{x'(y)^2+1}}=n(y)\frac{\dd{x}}{\sqrt{\dd{x}^2+\dd{y}^2}}=\text{const}=p\,.
\end{equation*}
 Z powyższego równania możemy z jednej strony znaleźć trajektorię \(y(x)\) dla danego \(n(y)\)
 poprzez rozwiązanie równania różniczkowego
\begin{equation*}
    \dv{y}{x}=\sqrt{\frac{n^2(y)}{p^2}-1}\,,
\end{equation*}
lub wyznaczyć funkcję \(n(y)\) dla znanej trajektorii \(y(x)\) przez przekształcenie powyższego
równania
\begin{equation*}
    n(y)=p\sqrt{y'(x)^2+1}\,.
\end{equation*}
W naszym zadaniu \(y'(x)=2\alpha x=2\sqrt{\alpha y}\), zatem otrzymujemy
\begin{equation*}
    n(y)=n_0\sqrt{1+4\alpha y}\,.
\end{equation*}
\subsubsection*{Pewne uogólnienia}
Drogę optyczną \(ct\) możemy również zapisać w innym układzie współrzędnych. Prawdopodobnie
najbardziej przydatnym z nich jest układ współrzędnych biegunowych \((r,\phi)\)
\begin{equation*}
    ct=\int_{A}^{B}n(r,\phi)\sqrt{\dd{r}^2+r^2\dd{\phi}^2}\,.
\end{equation*}
Jeżeli rozpatrujemy przypadek, w którym gęstość optyczna zmienia się radialnie, wówczas
przeprowadzając analogicznie obliczenia, możemy wyznaczyć ogólne rów. różniczkowe, którego
rozwiązanie dla danej postaci \(n(r)\) jest trajektorią światła w tym ośrodku (we wsp. biegunowych)
\begin{equation*}
    \dv[]{r}{\phi}=r\sqrt{\frac{n^2(r)}{P^2}r^2-1}\,.
\end{equation*}

\item \textit{Na ośrodek przezroczysty o współczynniku załamania zależnym od \(y\), w punkcie
\(y=a\), pod kątem prostym pada promień światła. Jaka powinna być postać funkcji \(n(y)\), aby wewn.
ośrodka promień biegł po sinusoidzie \(y(x)=a\cos mx\)? Przyjmij, że \(n(a)\) jest znane.}
\medskip

\textit{Rozwiązanie.}
\medskip

Różniczkując otrzymujemy 
\begin{equation*}
\begin{split}
    y'(x)&=-ma\sin mx\\
    y'(x)^2&=m^2a^2\sin^2mx=m^2a^2(1-\cos^2 mx)=m^2a^2\left(1-\frac{y^2}{a^2}\right)
\end{split}
\end{equation*}
Z poprzednich rozważań mamy zatem
\begin{equation*}
    n(y)=p\sqrt{m^2a^2\left(1-\frac{y^2}{a^2}\right)+1}=n(a)\sqrt{m^2\left(a^2-y^2\right)+1}\,.
\end{equation*}
\begin{equation*}
    \text{Odp.: }\,n(y)=n(a)\sqrt{m^2\left(a^2-y^2\right)+1}\,.
\end{equation*}
\item
\noindent\textit{Dana jest przezroczysta kula o promieniu \(R\) umieszczona w powietrzu \(n_0\approx
1\). Współczynnik załamania światła w tej kuli zmienia się radialnie zgodnie ze wzorem
\begin{equation*}
    n(r)=\frac{R+a}{r+a}\,.
\end{equation*}
Na kulę pod kątem \(\alpha\) pada promień świetlny. Wyznacz najmniejszą odległość \(d\) tego
promienia od środka kuli.}
\medskip

\textit{Rozwiązanie.}
\medskip

Ponieważ \(\dv{r}{\phi}|_{r=d}=0\), zatem 
\begin{equation*}
    0=d\sqrt{\frac{n^2(d)}{P^2}d^2-1}\quad \implies \quad n(d)=\frac{P}{d}\,.
\end{equation*}
Pozostaje wyznaczyć wielkość zachowaną podczas ruchu promienia świetlnego w rozważanym ośrodku --
\(P\) (można mówić o optycznej całce ruchu \(P\)). Z równań Lagrange'a mamy
\begin{equation*}
    \pdv{\mathcal{F}}{\phi'}=n(r)\frac{r^2\phi'(r)}{\sqrt{1+r^2\phi'(r)^2}}=n(r)\frac{r^2\dd{\phi}}{\sqrt{\dd{r}^2+r^2\dd{\phi}^2}}=P\,.
\end{equation*}
Zauważmy, że \(\frac{r\dd{\phi}}{\sqrt{\dd{r}^2+r^2\dd{\phi}^2}}=\sin \beta(r)\), zatem
\begin{equation*}
    n(R)R\sin\beta(R)=p\quad \text{oraz z prawa Snella}\quad \frac{\sin \alpha }{\sin \beta(R)}=\frac{n(R)}{1}\,.
\end{equation*}
Mamy \(n(R)=1\), zatem \(\sin\beta(R)=\sin\alpha\), więc \(P=R\sin\alpha\), skąd otrzymujemy
\begin{equation*}
    \frac{R+a}{d+a}=\frac{R}{d}\sin\alpha\,.
\end{equation*}
\begin{equation*}
    \text{Odp.: }\,d=\frac{aR\sin\alpha}{a+R(1-\sin\alpha)}\,.
\end{equation*}
\end{enumerate}
\subsubsection{Zwierciadła}
W ogólności będziemy oznaczać \(F\) -- ognisko zwierciadła, \(G\) -- środek optyczny zwierciadła,
tj. punkt przecięcia zwierciadła z jego płaszczyzną symetrii. Definiujemy ogniskową zwierciadła
\(f\) jako odległość \(f=|GF|\).
\medskip

\textbf{\textit{Zwierciadła sferyczne.}} Zwierciadłem sferycznym nazywamy wycinek sfery pokryty
materiałem odbijającym promienie świetlne. Jeśli materiał odbijający znajduje się na wewnętrznej
części sfery wówczas jest to \textit{zwierciadło wklęsłe}, w przeciwnym wypadku jest to
\textit{zwierciadło wypukłe}. W konstrukcjach przyjmujemy, że dla zwierciadła sferycznego \(f=\pm
R/2\), gdzie \(R\) jest promieniem sfery, z której zostało wycięte zwierciadło. W rzeczywistości
przybliżenie to jest dobre tylko dla promieni przyosiowych, dla których możemy pominąć wpływ
\textit{aberracji sferycznej}. Istotnie nietrudno pokazać, że promień równoległy do osi optycznej
biegnący w odległości \(d\) od niej po odbiciu przetnie oś optyczną w punkcie \(F'\) takim, że
\begin{equation*}
    |GF'|=f^*=2f-\frac{f}{\sqrt{1-d^2/4f^2}}\,,
\end{equation*}
gdzie \(|f|=R/2\). Dla \(2f\gg d\) otrzymujemy oczywiście \(f^*\approx f\). Równanie zwierciadła
sferycznego ma postać
\begin{equation*}
    \frac{1}{x}+\frac{1}{y}=\frac{1}{f}\,,
\end{equation*}
gdzie \(x\) jest położeniem przedmiotu, \(y\) położeniem obrazu. Powiększenie obrazu \(p\)
definiujemy jako
\begin{equation*}
    p=\left|\frac{y}{x}\right|\,.
\end{equation*}
\textit{Wzór Newtona.} Niech \(x_1\), \(x_2\) oznaczają odpowiednio odległość \(F\) od przedmiotu
oraz obrazu. Wówczas zachodzi
\begin{equation*}
    x_1=x-f\,,\quad x_2=y-f\,,\quad xy=fx+fy\,,
\end{equation*}
skąd otrzymujemy
\begin{equation*}
    x_1x_2=(x-f)(y-f)=xy-xf-yf+f^2=f^2\,.
\end{equation*}
\indent \textbf{\textit{Zwierciadła paraboliczne.}} Zwierciadłem parabolicznym nazywamy paraboloidę
obrotową pokrytą materiałem odbijającym promienie świetlne. W ogólności ogniskowa \(f\) zwierciadła
powstałego przez obrót wokół osi Y krzywej będącej wykresem funkcji parzystej \(y(x)\) wynosi
\begin{equation*}
   f=y(x)+\frac{x}{2}\left(\frac{1}{y'(x)}-y'(x)\right)\,.
\end{equation*}
Dla \(y=ax^2\), otrzymujemy
\begin{equation*}
    f=\frac{1}{4a}\,,
\end{equation*}
zatem widzimy, że dla zwierciadła parabolicznego istotnie wiązka promieni równoległych do osi
optycznej skupia się w jednym punkcie \(\mathcal{F}\).
\medskip

\textit{Zasada odwracalności biegu promieni.} Zawsze możemy odwrócić bieg promienia świetlnego
zamieniając obraz z przedmiotem rzeczywistym.
\subsubsection{Soczewki}
Soczewka to ciało przezroczyste ograniczone dwiema sferami o promieniach \(r_1\) i \(r_2\). Dla
soczewek cienkich i pomijając aberrację sferyczną mamy
\begin{equation*}
    Z=\frac{1}{f}=\left(\frac{n}{n_0}-1\right)\left(\frac{1}{r_1}+\frac{1}{r_2}\right)\,,
\end{equation*}
gdzie analogicznie \(f\) jest odległością między ogniskiem soczewki \(F\), a jej środkiem optycznym
\(\mathcal{G}\). Dla soczewek cienkich i pomijając aberrację sferyczną wiązka promieni równoległych
padających na soczewkę skupiającą pod kątem \(\vartheta\) do osi optycznej jest skupiana w punkcie
\(F'\) leżącym na prostej przechodzącej przez \(F\) i prostopadłej do osi optycznej takim, że
\begin{equation*}
    |FF'|=f\tan\vartheta\,.
\end{equation*}
Równanie soczewki ma postać
\begin{equation*}
    \frac{1}{x}+\frac{1}{y}=\frac{1}{f}\,,
\end{equation*}
gdzie \(x\) jest położeniem przedmiotu, \(y\) położeniem obrazu.

\subsection{Optyka falowa}
Optykę geometryczną możemy stosować w przypadkach, kiedy wymiary wszelkich obiektów (soczewek,
pryzmatów itp.) są o wiele większe od długości fali światła użytego w doświadczeniu. Eksperymentem
ilustrującym falową naturę światła jest doświadczenie Younga.
\subsubsection{Doświadczenie Younga}
W doświadczeniu Younga mamy ekran z dwiema niewielkimi szczelinami \(S_1\) i \(S_2\), w których
znajdują się koherentne i zgodne w fazie źródła \(F_1\) i \(F_2\), które emitują monochromatyczne
światło (w rzeczywistym eksperymencie Young użył dodatkowego ekranu z pojedynczą szczeliną
umieszczonego wcześniej). Na odległym ekranie obserwujemy wówczas tzw. prążki interferencyjne.
Drgania pola elektrycznego fal świetlnej generowanych przez źródła \(F_1\) i \(F_2\) wynoszą
\(E_0\sin\omega t\) z dokładnością do przesunięcia fazowego. Drgania pól elektrycznych fal w punkcie
o współrzędnej kątowej \(\vartheta\) na położonym daleko ekranie wynoszą odpowiednio
\begin{equation*}
    E_1=E_0\sin\omega t\quad\text{i}\quad E_2=E_0\sin(\omega t+k\Delta s)\,,
\end{equation*}
gdzie \(\Delta s\) jest różnicą dróg geometrycznych dwóch promieni. Jeśli ekran jest bardzo daleko
możemy przyjąć
\begin{equation*}
    \Delta s=d\sin\vartheta\,,
\end{equation*}
gdzie \(d\) jest odległością między szczelinami. Maksymalne wzmocnienie fal (interferencja
konstruktywna) zajdzie oczywiście gdy
\begin{equation*}
    (\text{max})\quad \delta=k\Delta s=2\pi m\,,\quad m\in\mathbb{Z}
\end{equation*}
natomiast całkowite wygaszenie fal zajdzie, gdy
\begin{equation*}
    (\text{min})\quad \delta=k\Delta s=2\pi m +\pi\,,\quad m\in\mathbb{Z}\,.
\end{equation*}
Z powyższego otrzymujemy, że położenia kątowe maksimów \(\vartheta_\text{max}\) oraz minimów
\(\vartheta_\text{min}\) są dane wzorami
\begin{equation*}
    \vartheta_\text{max}=\arcsin\frac{m\lambda}{d}\quad\text{oraz}\quad \vartheta_\text{min}=\arcsin\frac{(m+\frac{1}{2})\lambda}{d}\,.
\end{equation*}
\subsubsection{Siatka dyfrakcyjna}
Siatka dyfrakcyjna to układ wielu \(N\) równoległych niewielkich szczelin umieszczonych w równych
odległościach \(d\) od siebie. Położenia kątowe maksimów (ale nie minimów) są takie same jak w
doświadczeniu Younga. Jeśli światło pada na siatkę pod kątem \(\beta\), to wówczas
\begin{equation*}
    \delta =kd(\sin\vartheta+\sin\beta)\,.
\end{equation*}
Obliczmy natężenie światła \(I(\vartheta)\) na daleko położonym ekranie, gdy na siatkę pada
prostopadle światło monochromatyczne. Potrzebne są nam do tego następujące fakty.
\begin{enumerate}
    \item Natężenie światła jest wprost proporcjonalne do średniej z kwadratu wypadkowego pola
    elektrycznego
    \begin{equation*}
        I\propto \langle E_\text{tot}^2\rangle\,.
    \end{equation*}
    
    \item Dla źródeł koherentnych, aby obliczyć wypadkowe natężenie światła obliczamy najpierw
    wypadkowe natężenie pola elektrycznego, a następnie uśredniamy je, zgodnie z punktem pierwszym.
    
    \item Dla źródeł niekoherentych obliczamy najpierw średnie natężenia światła, a następnie je
    sumujemy.
\end{enumerate}
Będziemy oznaczać \(A^2=\langle E_\text{tot}^2\rangle\). Źródła są koherentne, zatem obliczmy
najpierw \(E_\text{tot}\)
\begin{equation*}
    E_\text{tot}=\sum_{\alpha=0}^{N-1}E_0\sin(\omega t+\alpha\delta)=E_0(P\sin\omega t +Q\cos\omega t)\,,
\end{equation*}
gdzie \(\delta =kd\sin\vartheta\) oraz
\begin{equation*}
    P=\sum_{\alpha=0}^{N-1}\cos(\alpha\delta)\quad\text{i}\quad Q=\sum_{\alpha=0}^{N-1}\sin(\alpha\delta)\,.
\end{equation*}
Z powyższego mamy zatem
\begin{equation*}
    A^2=\frac{\omega }{2\pi}\int_0^{2\pi/\omega}E_\text{tot}^2\dd{t}=\frac{1}{2}E_0^2(P^2+Q^2)\,.
\end{equation*}
Dla \(\delta=0\), tj. dla centralnego maksimum \(\vartheta=0\) mamy \(A^2(0)=E_0^2N^2/2\). Dla
\(\vartheta\in (-\pi/2;0)\cup(0;\pi/2)\) mamy z kolei
\begin{equation*}
    A^2(\vartheta)=\frac{1}{2}E_0^2\sin^2\left(\frac{N\delta}{2}\right)\csc^2\left(\frac{\delta}{2}\right)\,,
\end{equation*}
skąd, przyjmując \(I(0)=I_0\), otrzymujemy
\begin{equation*}
    I(\vartheta)=\frac{I_0}{N^2}\sin^2\left(\frac{Nkd\sin\vartheta}{2}\right)\csc^2\left(\frac{kd\sin\vartheta}{2}\right)\,,\quad \vartheta\in\left(-\frac{\pi}{2};\frac{\pi}{2}\right)\,.
\end{equation*}
\subsubsection{Dyfrakcja na pojedynczej szczelinie}
W doświadczeniu Younga i siatce dyfrakcyjnej zakładaliśmy, że szczeliny są niezwykle małe, jednak w
rzeczywistości mają one skończone rozmiary. Rozpatrzmy szczelinę o szerokości \(D\). Możemy
przeanalizować obraz powstający na położonym daleko ekranie, gdy na szczelinę pada prostopadle
światło monochromatyczne korzystając ze wzoru na natężenie światła dla siatki dyfrakcyjnej. Istotnie
podstawiając \(d=D/(N-1)\) mamy 
\begin{equation*}
\begin{split}
    I(\vartheta)&=\lim_{N\to\infty}\frac{I_0}{N^2}\sin^2\left(\frac{Nkd\sin\vartheta}{2}\right)\csc^2\left(\frac{kd\sin\vartheta}{2}\right)\\
    &=I_0\left(\frac{\sin\beta}{\beta}\right)^2\,,
\end{split}
\end{equation*}
gdzie \(\beta(\vartheta)=\frac{1}{2}kD\sin\vartheta\). Z powyższego widzimy, że minima występują dla 
\begin{equation*}
    (\text{min})\quad \beta=m\pi\,,\quad m\in\mathbb{Z}\setminus \{0\}\,,
\end{equation*}
czyli dla punktów o położeniu kątowym
\begin{equation*}
    \vartheta_\text{min}=\arcsin\frac{m\lambda}{D}\,.
\end{equation*}
\subsubsection{Interferencja w cienkich warstwach}
Gdy fala monochromatyczna poruszająca się w pewnym ośrodku \(n\) odbija się od ośrodka o większej
gęstości optycznej \(n'>n\) , wówczas jej faza zmienia się o \(\pi\). Rozpatrzmy światło
monochromatyczne padające prostopadle na cienką warstwę (np. oleju) o grubości \(w\) i gęstości
optycznej \(n\). Będziemy mieli wówczas do czynienia z interferencją. Istotnie jeśli drganie pola
elektrycznego padającego światła wynosi \(E_0\sin\omega t\), to po odbiciu od górnej strony warstwy
mamy \(E_1=E_0\sin(\omega t+\pi)\), natomiast po przejściu przez warstwę i odbiciu od dolnej strony
warstwy mamy \(E_2=E_0\sin(\omega t+2kw)\), gdzie \(k=2\pi n/\lambda\), a \(\lambda=\frac{2\pi
c}{\omega}\). Różnica faz wynosi zatem
\begin{equation*}
    \delta =2kw-\pi\,.
\end{equation*}
Maksymalne wzmocnienie nastąpi oczywiście gdy \(\delta =2m\pi\), czyli
\begin{equation*}
    2w_\text{max}=\left(m+\frac{1}{2}\right)\frac{\lambda}{n}\,,
\end{equation*}
a całkowite wygaszenie, gdy \(\delta=2\pi m+\pi\), czyli 
\begin{equation*}
    2w_\text{min}=m\frac{\lambda}{n}\,.
\end{equation*}
\textbf{Zadanie.} \textit{Na płytkę szklaną o grubości 100.25 \(\lambda\) pada prostopadle promień
światła laserowego, którego długość fali wewnątrz płytki wynosi \(\lambda\). Gdy światło o natężeniu
\(I\) pada na powierzchnię styku powietrze--szkło lub szkło--powietrze, wiązka przechodząca będzie
miała natężenie \(rI\) (dla pewnego ustalonego \(0<r<1\)), zaś odbita: \((1-r)I\). Wyznacz natężenie
wiązki przechodzącej przez płytkę i wiązki odbitej od płytki. Przyjmij, że szkło nie pochłania
światła.}
\medskip

\textit{Rozwiązanie.}
\medskip

Ponieważ źródłem padającego światła jest laser, więc światło jest monochromatyczne. Niech pole
elektryczne padającego światła będzie dane wzorem
\begin{equation*}
    E_\text{pad}=E_0\sin \omega t\,.
\end{equation*}
Pole elektryczne światła odbitego od płytki będzie superpozycją pól elektrycznych promieni
świetlnych odbitych kolejno od górnej i dolnej powierzchni płytki. Ponieważ te drgania (pola
elektrycznego) są koherentne, zatem aby obliczyć wypadkowe natężenie światła musimy najpierw
obliczyć wypadkowe pole elektryczne (nie możemy dodawać natężeń światła). Ponieważ \(I\sim E^2\),
więc amplituda pola elektrycznego promienia po przejściu przez granicę powietrze--szkło wyniesie
\(\sqrt{r}E_0\), a promienia odbitego \(\sqrt{1-r}E_0\). Wprowadźmy parametr \(l=\sqrt{1-r}\). Pole
elektryczne promienia odbitego od górnej powierzchni od strony powietrza wynosi
\begin{equation*}
    lE_0\sin(\omega t+\pi)=lE_0\sin(\phi+\pi)\,.
\end{equation*}
Pole elektryczne \(n\)-tego promienia wychodzącego z płytki po górnej stronie po \(n\) odbiciach od
dolnej powierzchni płytki wynosi
\begin{equation*}
    E_n=rl^{2n-1}E_0\sin(\phi+n\delta)\,,
\end{equation*}
gdzie \(\delta=\frac{4\pi}{\lambda}h\) (\(h\) - grubość płytki). Istotnie amplituda pola
elektrycznego promienia po przejściu przez granicę powietrze--szkło wynosi \(\sqrt{r}E_0\), każde
odbicie mnoży ją przez czynnik \(l\), a odbić tych jest zawsze nieparzyście wiele, dodatkowo przy
przejściu ponownym przez granicę szkło-powietrze należy ją pomnożyć przez czynnik \(\sqrt{r}\).
Wypadkowe pole elektryczne światła odbitego od płytki wynosi więc
\begin{equation*}
    E_\text{tot}=lE_0\sin(\phi+\pi)+rE_0\left[l\sin(\phi+\delta)+l^3\sin(\phi+2\delta)+...\right]\,.
\end{equation*}
Podstawiając \(h=\lambda/4\) otrzymujemy \(\delta=\pi\) zatem
\begin{equation*}
    E_\text{tot}=-lE_0\sin\phi+rE_0\sin\phi\left[(-l-l^5-l^9-...)+(l^3+l^7+...)\right]\,.
\end{equation*}
Wyrażenie w nawiasie kwadratowym jest równe sumie dwóch szeregów geometrycznych, które z łatwością
można obliczyć
\begin{equation*}
    E_\text{tot}=-lE_0\sin\phi+rE_0\sin\phi\frac{-l}{1+l^2}=-E_0\sin\phi \frac{2l}{1+l^2}\,.
\end{equation*}
Amplituda pola elektrycznego światła odbitego wynosi zatem
\begin{equation*}
    E_0\frac{2l}{1+l^2}\,.
\end{equation*}
Natężenie światła odbitego jest więc równe
\begin{equation*}
    I_{o}=I\frac{4l^2}{(1+l^2)^2}=4I\frac{1-r}{(2-r)^2}\,.
\end{equation*}
Oczywiście ze względu na zachowanie energii natężenie światła przechodzącego przez płytkę jest równe
\begin{equation*}
    I_p=I-I_o=I-4I\frac{1-r}{(2-r)^2}\,.
\end{equation*}
\subsubsection{Polaryzacja}
Światło jest falą poprzeczną, tj. kierunki drgań wektorów \(\mathbf{E}\) i \(\mathbf{B}\) są
prostopadłe do wektora falowego \(\mathbf{k}\). Fala płasko spolaryzowana to fala, w której kierunek
drgań \(\mathbf{E}\) leż w każdej chwili w tej samej płaszczyźnie. Typowe źródła światła nie emitują
fal płaskich, tylko tzw. \textit{fale niespolaryzowane}, których płaszczyzny drgań są zorientowane
przypadkowo wokół \(\mathbf{k}\).
\subsubsection*{Płytki polaryzujące}
W płytce polaryzującej istnieje charakterystyczny kierunek polaryzacji. Płytka przepuszcza tylko te
fale, dla których kierunki drgań \(\mathbf{E}\) są równoległe do kierunku polaryzacji. Jeśli \(I_0\)
oznacza natężenie światła spolaryzowanego liniowo padającego na polaryzator w taki sposób, że
płasczyzny polaryzacji fali i polaryzatora tworzą kat \(\theta\) to zgodnie z \textbf{prawem Malusa}
natężenie światła po przejściu przez polaryzator wynosi
\begin{equation*}
    I=I_0\cos^2\theta\,.
\end{equation*}
Jeśli na polaryzator skierujemy światło niespolaryzowane o natężeniu \(I_0\) to niezależnie od
ustawienia osi polaryzatora \(I=I_0/2\).
\subsubsection*{Polaryzacja przez odbicie}
Jeśli na materiał o gęstości optycznej \(n\) skierujemy światło będące złożeniem dwóch fal płaskich
o prostopadłych polaryzacjach: w płaszczyźnie wyznaczonej przez promień padający i normalną do
płaszczyzny padania (tzw. polaryzacja \(p\)) oraz w płaszczyźnie prostopadłej (tzw. polaryzacja
\(s\)), to doświadczalnie stwierdzono, że istnieje kąt \(\alpha_B\) (\textit{kąt Brewstera}), taki,
że promień odbity ma tylko polaryzację \(s\) oraz promień załamany jest prostopadły to promienia
odbitego. Z prawa Snella mamy więc
\begin{equation*}
    \tan\alpha_B=\frac{n}{n_0}\,.
\end{equation*}

\subsection{Promieniowanie termiczne}
Przy rozchodzeniu się fal EM przenoszona jest energia. Natężenie światła izotropowego źródła
punktowego o mocy \(P_s\) w odległości \(r\) od niego wynosi
\begin{equation*}
    I(r)=\frac{P_s}{4\pi r^2}\,.
\end{equation*}
\subsubsection*{Model ciała doskonale czarnego}
Rozgrzane do wysokich temperatur ciała są źródłami światła widzialnego (np. żarówki wolframowe).
Wszystkie ciała emitują promieniowanie termiczne do otoczenia (niekoniecznie w spektrum widzialnym)
oraz je absorbują w każdej temperaturze większej od 0 K. Gdy dwa ciała znajdują się w równowadze
termicznej szybkość emicji \(P_e\) jest równa szybkości absorpcji \(P_a\).
\medskip

Ciało doskonale czarne pochłania całkowicie padające nań promieniowanie. Niewielki otwór dużej
wnęki, której wewnętrzne ścianki pokryto sadzą jest modelem ciała doskonale czarnego.
\medskip

\noindent\fbox{%
    \parbox{\textwidth}{%
        \textbf{Prawo Stefana-Boltzmanna}\\
        Całkowita emisja energetyczna \(R=\int_0^\infty R(\lambda)\dd{\lambda}\) (gdzie
        \(R(\lambda)\) jest mocą na jednostkę powierzchni wypromieniowywaną przez ciało w postaci
        fali EM o długości z przedziału \((\lambda, \lambda+\dd{\lambda})\)) ciała doskonale
        czarnego jest wprost proporcjonalna do czwartej potęgi jego temperatury
        \begin{equation*}
            R=\sigma T^4\,.
        \end{equation*}
    }%
}

\subsubsection*{Prawo Plancka}
Zgodnie z prawem Plancka moc na jednostkę powierzchni wypromieniowywana przez ciało doskonale czarne
w postaci fali EM o długości fali z zakresu \((\lambda, \lambda+\dd{\lambda})\) wynosi
\begin{equation*}
    R(\lambda)=\frac{8\pi c^2h}{\lambda^5}\frac{1}{\e^{\frac{hc}{\lambda kT}}-1}\,.
\end{equation*}
\subsubsection*{Prawo Wiena}
Korzystając z prawa Plancka możemy wyznaczyć długość fali o maksymalnej mocy promieniowania
\begin{equation*}
    \lambda_\text{max}=\frac{b}{T}\,,
\end{equation*}
gdzie \(b\approx2.898\cdot10^{-3}\,\text{m}\cdot\text{K}\) jest tzw. \underline{stałą Wiena}.
\end{document}